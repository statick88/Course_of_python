% Options for packages loaded elsewhere
\PassOptionsToPackage{unicode}{hyperref}
\PassOptionsToPackage{hyphens}{url}
\PassOptionsToPackage{dvipsnames,svgnames,x11names}{xcolor}
%
\documentclass[
  a4paper,
  DIV=11,
  numbers=noendperiod,
  onepage,
  openany]{scrreprt}

\usepackage{amsmath,amssymb}
\usepackage{iftex}
\ifPDFTeX
  \usepackage[T1]{fontenc}
  \usepackage[utf8]{inputenc}
  \usepackage{textcomp} % provide euro and other symbols
\else % if luatex or xetex
  \usepackage{unicode-math}
  \defaultfontfeatures{Scale=MatchLowercase}
  \defaultfontfeatures[\rmfamily]{Ligatures=TeX,Scale=1}
\fi
\usepackage{lmodern}
\ifPDFTeX\else  
    % xetex/luatex font selection
\fi
% Use upquote if available, for straight quotes in verbatim environments
\IfFileExists{upquote.sty}{\usepackage{upquote}}{}
\IfFileExists{microtype.sty}{% use microtype if available
  \usepackage[]{microtype}
  \UseMicrotypeSet[protrusion]{basicmath} % disable protrusion for tt fonts
}{}
\makeatletter
\@ifundefined{KOMAClassName}{% if non-KOMA class
  \IfFileExists{parskip.sty}{%
    \usepackage{parskip}
  }{% else
    \setlength{\parindent}{0pt}
    \setlength{\parskip}{6pt plus 2pt minus 1pt}}
}{% if KOMA class
  \KOMAoptions{parskip=half}}
\makeatother
\usepackage{xcolor}
\usepackage[lmargin=30mm,rmargin=30mm,tmargin=35mm,bmargin=30mm]{geometry}
\setlength{\emergencystretch}{3em} % prevent overfull lines
\setcounter{secnumdepth}{5}
% Make \paragraph and \subparagraph free-standing
\ifx\paragraph\undefined\else
  \let\oldparagraph\paragraph
  \renewcommand{\paragraph}[1]{\oldparagraph{#1}\mbox{}}
\fi
\ifx\subparagraph\undefined\else
  \let\oldsubparagraph\subparagraph
  \renewcommand{\subparagraph}[1]{\oldsubparagraph{#1}\mbox{}}
\fi

\usepackage{color}
\usepackage{fancyvrb}
\newcommand{\VerbBar}{|}
\newcommand{\VERB}{\Verb[commandchars=\\\{\}]}
\DefineVerbatimEnvironment{Highlighting}{Verbatim}{commandchars=\\\{\}}
% Add ',fontsize=\small' for more characters per line
\usepackage{framed}
\definecolor{shadecolor}{RGB}{241,243,245}
\newenvironment{Shaded}{\begin{snugshade}}{\end{snugshade}}
\newcommand{\AlertTok}[1]{\textcolor[rgb]{0.68,0.00,0.00}{#1}}
\newcommand{\AnnotationTok}[1]{\textcolor[rgb]{0.37,0.37,0.37}{#1}}
\newcommand{\AttributeTok}[1]{\textcolor[rgb]{0.40,0.45,0.13}{#1}}
\newcommand{\BaseNTok}[1]{\textcolor[rgb]{0.68,0.00,0.00}{#1}}
\newcommand{\BuiltInTok}[1]{\textcolor[rgb]{0.00,0.23,0.31}{#1}}
\newcommand{\CharTok}[1]{\textcolor[rgb]{0.13,0.47,0.30}{#1}}
\newcommand{\CommentTok}[1]{\textcolor[rgb]{0.37,0.37,0.37}{#1}}
\newcommand{\CommentVarTok}[1]{\textcolor[rgb]{0.37,0.37,0.37}{\textit{#1}}}
\newcommand{\ConstantTok}[1]{\textcolor[rgb]{0.56,0.35,0.01}{#1}}
\newcommand{\ControlFlowTok}[1]{\textcolor[rgb]{0.00,0.23,0.31}{#1}}
\newcommand{\DataTypeTok}[1]{\textcolor[rgb]{0.68,0.00,0.00}{#1}}
\newcommand{\DecValTok}[1]{\textcolor[rgb]{0.68,0.00,0.00}{#1}}
\newcommand{\DocumentationTok}[1]{\textcolor[rgb]{0.37,0.37,0.37}{\textit{#1}}}
\newcommand{\ErrorTok}[1]{\textcolor[rgb]{0.68,0.00,0.00}{#1}}
\newcommand{\ExtensionTok}[1]{\textcolor[rgb]{0.00,0.23,0.31}{#1}}
\newcommand{\FloatTok}[1]{\textcolor[rgb]{0.68,0.00,0.00}{#1}}
\newcommand{\FunctionTok}[1]{\textcolor[rgb]{0.28,0.35,0.67}{#1}}
\newcommand{\ImportTok}[1]{\textcolor[rgb]{0.00,0.46,0.62}{#1}}
\newcommand{\InformationTok}[1]{\textcolor[rgb]{0.37,0.37,0.37}{#1}}
\newcommand{\KeywordTok}[1]{\textcolor[rgb]{0.00,0.23,0.31}{#1}}
\newcommand{\NormalTok}[1]{\textcolor[rgb]{0.00,0.23,0.31}{#1}}
\newcommand{\OperatorTok}[1]{\textcolor[rgb]{0.37,0.37,0.37}{#1}}
\newcommand{\OtherTok}[1]{\textcolor[rgb]{0.00,0.23,0.31}{#1}}
\newcommand{\PreprocessorTok}[1]{\textcolor[rgb]{0.68,0.00,0.00}{#1}}
\newcommand{\RegionMarkerTok}[1]{\textcolor[rgb]{0.00,0.23,0.31}{#1}}
\newcommand{\SpecialCharTok}[1]{\textcolor[rgb]{0.37,0.37,0.37}{#1}}
\newcommand{\SpecialStringTok}[1]{\textcolor[rgb]{0.13,0.47,0.30}{#1}}
\newcommand{\StringTok}[1]{\textcolor[rgb]{0.13,0.47,0.30}{#1}}
\newcommand{\VariableTok}[1]{\textcolor[rgb]{0.07,0.07,0.07}{#1}}
\newcommand{\VerbatimStringTok}[1]{\textcolor[rgb]{0.13,0.47,0.30}{#1}}
\newcommand{\WarningTok}[1]{\textcolor[rgb]{0.37,0.37,0.37}{\textit{#1}}}

\providecommand{\tightlist}{%
  \setlength{\itemsep}{0pt}\setlength{\parskip}{0pt}}\usepackage{longtable,booktabs,array}
\usepackage{calc} % for calculating minipage widths
% Correct order of tables after \paragraph or \subparagraph
\usepackage{etoolbox}
\makeatletter
\patchcmd\longtable{\par}{\if@noskipsec\mbox{}\fi\par}{}{}
\makeatother
% Allow footnotes in longtable head/foot
\IfFileExists{footnotehyper.sty}{\usepackage{footnotehyper}}{\usepackage{footnote}}
\makesavenoteenv{longtable}
\usepackage{graphicx}
\makeatletter
\def\maxwidth{\ifdim\Gin@nat@width>\linewidth\linewidth\else\Gin@nat@width\fi}
\def\maxheight{\ifdim\Gin@nat@height>\textheight\textheight\else\Gin@nat@height\fi}
\makeatother
% Scale images if necessary, so that they will not overflow the page
% margins by default, and it is still possible to overwrite the defaults
% using explicit options in \includegraphics[width, height, ...]{}
\setkeys{Gin}{width=\maxwidth,height=\maxheight,keepaspectratio}
% Set default figure placement to htbp
\makeatletter
\def\fps@figure{htbp}
\makeatother

\KOMAoption{captions}{tableheading}
\makeatletter
\@ifpackageloaded{tcolorbox}{}{\usepackage[skins,breakable]{tcolorbox}}
\@ifpackageloaded{fontawesome5}{}{\usepackage{fontawesome5}}
\definecolor{quarto-callout-color}{HTML}{909090}
\definecolor{quarto-callout-note-color}{HTML}{0758E5}
\definecolor{quarto-callout-important-color}{HTML}{CC1914}
\definecolor{quarto-callout-warning-color}{HTML}{EB9113}
\definecolor{quarto-callout-tip-color}{HTML}{00A047}
\definecolor{quarto-callout-caution-color}{HTML}{FC5300}
\definecolor{quarto-callout-color-frame}{HTML}{acacac}
\definecolor{quarto-callout-note-color-frame}{HTML}{4582ec}
\definecolor{quarto-callout-important-color-frame}{HTML}{d9534f}
\definecolor{quarto-callout-warning-color-frame}{HTML}{f0ad4e}
\definecolor{quarto-callout-tip-color-frame}{HTML}{02b875}
\definecolor{quarto-callout-caution-color-frame}{HTML}{fd7e14}
\makeatother
\makeatletter
\makeatother
\makeatletter
\@ifpackageloaded{bookmark}{}{\usepackage{bookmark}}
\makeatother
\makeatletter
\@ifpackageloaded{caption}{}{\usepackage{caption}}
\AtBeginDocument{%
\ifdefined\contentsname
  \renewcommand*\contentsname{Table of contents}
\else
  \newcommand\contentsname{Table of contents}
\fi
\ifdefined\listfigurename
  \renewcommand*\listfigurename{List of Figures}
\else
  \newcommand\listfigurename{List of Figures}
\fi
\ifdefined\listtablename
  \renewcommand*\listtablename{List of Tables}
\else
  \newcommand\listtablename{List of Tables}
\fi
\ifdefined\figurename
  \renewcommand*\figurename{Figure}
\else
  \newcommand\figurename{Figure}
\fi
\ifdefined\tablename
  \renewcommand*\tablename{Table}
\else
  \newcommand\tablename{Table}
\fi
}
\@ifpackageloaded{float}{}{\usepackage{float}}
\floatstyle{ruled}
\@ifundefined{c@chapter}{\newfloat{codelisting}{h}{lop}}{\newfloat{codelisting}{h}{lop}[chapter]}
\floatname{codelisting}{Listing}
\newcommand*\listoflistings{\listof{codelisting}{List of Listings}}
\makeatother
\makeatletter
\@ifpackageloaded{caption}{}{\usepackage{caption}}
\@ifpackageloaded{subcaption}{}{\usepackage{subcaption}}
\makeatother
\makeatletter
\@ifpackageloaded{tcolorbox}{}{\usepackage[skins,breakable]{tcolorbox}}
\makeatother
\makeatletter
\@ifundefined{shadecolor}{\definecolor{shadecolor}{rgb}{.97, .97, .97}}
\makeatother
\makeatletter
\makeatother
\makeatletter
\makeatother
\makeatletter
\@ifpackageloaded{tikz}{}{\usepackage{tikz}}
\makeatother
        \newcommand*\circled[1]{\tikz[baseline=(char.base)]{
          \node[shape=circle,draw,inner sep=1pt] (char) {{\scriptsize#1}};}}  
                  
\ifLuaTeX
  \usepackage{selnolig}  % disable illegal ligatures
\fi
\IfFileExists{bookmark.sty}{\usepackage{bookmark}}{\usepackage{hyperref}}
\IfFileExists{xurl.sty}{\usepackage{xurl}}{} % add URL line breaks if available
\urlstyle{same} % disable monospaced font for URLs
\hypersetup{
  pdftitle={Curso de Python 2023},
  pdfauthor={Lcdo. Diego Saavedra Mgtr.},
  colorlinks=true,
  linkcolor={blue},
  filecolor={Maroon},
  citecolor={Blue},
  urlcolor={Blue},
  pdfcreator={LaTeX via pandoc}}

\title{Curso de Python 2023}
\author{Lcdo. Diego Saavedra Mgtr.}
\date{Aug 31, 2023}

\begin{document}
\maketitle
\ifdefined\Shaded\renewenvironment{Shaded}{\begin{tcolorbox}[enhanced, interior hidden, frame hidden, borderline west={3pt}{0pt}{shadecolor}, sharp corners, breakable, boxrule=0pt]}{\end{tcolorbox}}\fi

\renewcommand*\contentsname{Table of contents}
{
\hypersetup{linkcolor=}
\setcounter{tocdepth}{2}
\tableofcontents
}
\bookmarksetup{startatroot}

\hypertarget{bienvenida}{%
\chapter{Bienvenida}\label{bienvenida}}

¡Bienvenidos al Curso Completo de Python: Desde Fundamentos hasta
Aplicaciones Prácticas!

\hypertarget{quuxe9-es-este-curso}{%
\section{¿Qué es este Curso?}\label{quuxe9-es-este-curso}}

Este curso exhaustivo te llevará desde los fundamentos básicos de la
programación hasta la creación de aplicaciones prácticas utilizando el
lenguaje de programación Python. A través de una combinación de teoría y
ejercicios prácticos, te sumergirás en los conceptos esenciales de la
programación y avanzarás hacia la construcción de proyectos reales.
Desde la instalación de herramientas hasta la creación de una API con
Django Rest Framework, este curso te proporcionará una comprensión
sólida y práctica de Python y su aplicación en el mundo real.

\hypertarget{a-quiuxe9n-estuxe1-dirigido}{%
\section{¿A quién está dirigido?}\label{a-quiuxe9n-estuxe1-dirigido}}

Este curso está diseñado para principiantes y aquellos con poca o
ninguna experiencia en programación. No importa si eres un estudiante
curioso, un profesional que busca cambiar de carrera o simplemente
alguien que desea aprender a programar: este curso es para ti. Desde
adolescentes hasta adultos, todos son bienvenidos a participar y
explorar el emocionante mundo de la programación a través de Python.

\hypertarget{cuxf3mo-contribuir}{%
\section{¿Cómo contribuir?}\label{cuxf3mo-contribuir}}

Valoramos tu participación en este curso. Si encuentras errores, deseas
sugerir mejoras o agregar contenido adicional, ¡nos encantaría
escucharte! Puedes contribuir a través de nuestra plataforma en línea,
donde puedes compartir tus comentarios y sugerencias. Juntos, podemos
mejorar continuamente este recurso educativo para beneficiar a la
comunidad de estudiantes y entusiastas de la programación.

Este libro ha sido creado con el objetivo de brindar acceso gratuito y
universal al conocimiento. Estará disponible en línea para que
cualquiera, sin importar su ubicación o circunstancias, pueda acceder y
aprender a su propio ritmo.

¡Esperamos que disfrutes este emocionante viaje de aprendizaje y
descubrimiento en el mundo de la programación con Python!

\part{Unidad 1: Introducción a la Programación}

\hypertarget{introducciuxf3n-general-a-la-programaciuxf3n}{%
\chapter{Introducción general a la
Programación}\label{introducciuxf3n-general-a-la-programaciuxf3n}}

\textless\textless\textless\textless\textless\textless\textless{} HEAD

La programación es el proceso de crear secuencias de instrucciones que
le indican a una computadora cómo realizar una tarea específica.

Estas instrucciones se escriben en lenguajes de programación, que son
conjuntos de reglas y símbolos utilizados para comunicarse con la
máquina. La programación es una habilidad esencial en la era digital, ya
que se aplica en una amplia variedad de campos, desde desarrollo de
software y análisis de datos hasta diseño de juegos y automatización.

\hypertarget{conceptos-clave}{%
\section{Conceptos Clave}\label{conceptos-clave}}

\hypertarget{instrucciones}{%
\subsection{Instrucciones}\label{instrucciones}}

Son comandos específicos que le indican a la computadora qué hacer.
Pueden ser simples, como imprimir un mensaje en pantalla, o complejas,
como realizar cálculos matemáticos.

\hypertarget{lenguajes-de-programaciuxf3n}{%
\subsection{Lenguajes de
Programación}\label{lenguajes-de-programaciuxf3n}}

Son sistemas de comunicación entre humanos y máquinas. Cada lenguaje
tiene reglas sintácticas y semánticas que determinan cómo se escriben y
ejecutan las instrucciones.

\hypertarget{algoritmos}{%
\subsection{Algoritmos}\label{algoritmos}}

Son conjuntos ordenados de instrucciones diseñados para resolver un
problema específico. Los algoritmos son la base de la programación y se
utilizan para desarrollar software eficiente.

\hypertarget{depuraciuxf3n}{%
\subsection{Depuración}\label{depuraciuxf3n}}

Es el proceso de identificar y corregir errores en el código. Los
programadores pasan tiempo depurando para asegurarse de que sus
programas funcionen correctamente.

\hypertarget{ejemplo}{%
\section{Ejemplo:}\label{ejemplo}}

\hypertarget{annotated-cell-1}{%
\label{annotated-cell-1}}%
\begin{Shaded}
\begin{Highlighting}[]
\BuiltInTok{print}\NormalTok{(}\StringTok{"Hola, bienvenido al mundo de la programación."}\NormalTok{) }\hspace*{\fill}\NormalTok{\circled{1}}
\end{Highlighting}
\end{Shaded}

\begin{description}
\tightlist
\item[\circled{1}]
Este es un ejemplo sencillo de un programa en Python que imprime un
mensaje en pantalla.
\end{description}

\hypertarget{explicaciuxf3n}{%
\section{Explicación}\label{explicaciuxf3n}}

En Python, los comentarios comienzan con el símbolo \texttt{\#}. No
afectan la ejecución del programa, pero son útiles para documentar el
código.

La línea
\texttt{print("Hola,\ bienvenido\ al\ mundo\ de\ la\ programación.")} es
una instrucción de impresión. La función \texttt{print()} muestra el
texto entre paréntesis en la consola.

\begin{tcolorbox}[enhanced jigsaw, colbacktitle=quarto-callout-important-color!10!white, toprule=.15mm, leftrule=.75mm, titlerule=0mm, opacityback=0, rightrule=.15mm, opacitybacktitle=0.6, breakable, left=2mm, coltitle=black, title=\textcolor{quarto-callout-important-color}{\faExclamation}\hspace{0.5em}{Actividad Práctica}, toptitle=1mm, bottomtitle=1mm, arc=.35mm, bottomrule=.15mm, colback=white, colframe=quarto-callout-important-color-frame]

Escribe un programa que solicite al usuario su nombre y luego imprima un
mensaje de bienvenida personalizado.

\end{tcolorbox}

\hypertarget{explicaciuxf3n-de-la-actividad}{%
\section{Explicación de la
Actividad}\label{explicaciuxf3n-de-la-actividad}}

======= \#\# Introducción general a la Programación

La programación es el proceso de crear secuencias de instrucciones que
le indican a una computadora cómo realizar una tarea específica.

Estas instrucciones se escriben en lenguajes de programación, que son
conjuntos de reglas y símbolos utilizados para comunicarse con la
máquina. La programación es una habilidad esencial en la era digital, ya
que se aplica en una amplia variedad de campos, desde desarrollo de
software y análisis de datos hasta diseño de juegos y automatización.

\hypertarget{conceptos-clave-1}{%
\section{Conceptos Clave}\label{conceptos-clave-1}}

\hypertarget{instrucciones-1}{%
\subsection{Instrucciones}\label{instrucciones-1}}

Son comandos específicos que le indican a la computadora qué hacer.
Pueden ser simples, como imprimir un mensaje en pantalla, o complejas,
como realizar cálculos matemáticos.

\hypertarget{lenguajes-de-programaciuxf3n-1}{%
\subsection{Lenguajes de
Programación}\label{lenguajes-de-programaciuxf3n-1}}

Son sistemas de comunicación entre humanos y máquinas. Cada lenguaje
tiene reglas sintácticas y semánticas que determinan cómo se escriben y
ejecutan las instrucciones.

\hypertarget{algoritmos-1}{%
\subsection{Algoritmos}\label{algoritmos-1}}

Son conjuntos ordenados de instrucciones diseñados para resolver un
problema específico. Los algoritmos son la base de la programación y se
utilizan para desarrollar software eficiente.

\hypertarget{depuraciuxf3n-1}{%
\subsection{Depuración}\label{depuraciuxf3n-1}}

Es el proceso de identificar y corregir errores en el código. Los
programadores pasan tiempo depurando para asegurarse de que sus
programas funcionen correctamente.

\hypertarget{ejemplo-1}{%
\section{Ejemplo:}\label{ejemplo-1}}

\hypertarget{annotated-cell-2}{%
\label{annotated-cell-2}}%
\begin{Shaded}
\begin{Highlighting}[]
\BuiltInTok{print}\NormalTok{(}\StringTok{"Hola, bienvenido al mundo de la programación."}\NormalTok{) }\hspace*{\fill}\NormalTok{\circled{1}}
\end{Highlighting}
\end{Shaded}

\begin{description}
\tightlist
\item[\circled{1}]
Este es un ejemplo sencillo de un programa en Python que imprime un
mensaje en pantalla.
\end{description}

\hypertarget{explicaciuxf3n-1}{%
\section{Explicación}\label{explicaciuxf3n-1}}

En Python, los comentarios comienzan con el símbolo \texttt{\#}. No
afectan la ejecución del programa, pero son útiles para documentar el
código.

La línea
\texttt{print("Hola,\ bienvenido\ al\ mundo\ de\ la\ programación.")} es
una instrucción de impresión. La función \texttt{print()} muestra el
texto entre paréntesis en la consola.

\begin{tcolorbox}[enhanced jigsaw, colbacktitle=quarto-callout-important-color!10!white, toprule=.15mm, leftrule=.75mm, titlerule=0mm, opacityback=0, rightrule=.15mm, opacitybacktitle=0.6, breakable, left=2mm, coltitle=black, title=\textcolor{quarto-callout-important-color}{\faExclamation}\hspace{0.5em}{Actividad Práctica}, toptitle=1mm, bottomtitle=1mm, arc=.35mm, bottomrule=.15mm, colback=white, colframe=quarto-callout-important-color-frame]

Escribe un programa que solicite al usuario su nombre y luego imprima un
mensaje de bienvenida personalizado.

\end{tcolorbox}

\hypertarget{explicaciuxf3n-de-la-actividad-1}{%
\section{Explicación de la
Actividad}\label{explicaciuxf3n-de-la-actividad-1}}

\begin{quote}
\begin{quote}
\begin{quote}
\begin{quote}
\begin{quote}
\begin{quote}
\begin{quote}
e8ed08b1a5bbe1e369719187cfc4de7f7e2a41a9 El programa utilizará la
función \texttt{input()} para recibir la entrada del usuario. Luego,
utilizará la entrada proporcionada para imprimir un mensaje de
bienvenida personalizado.
\end{quote}
\end{quote}
\end{quote}
\end{quote}
\end{quote}
\end{quote}
\end{quote}

\part{Unidad 2: Instalación de Python y más herramientas}

\hypertarget{instalaciuxf3n-de-python}{%
\chapter{Instalación de Python}\label{instalaciuxf3n-de-python}}

La instalación de Python es el primer paso para comenzar a programar en
este lenguaje. Python es un lenguaje de programación versátil y
ampliamente utilizado, conocido por su sintaxis clara y legible. Aquí
aprenderemos cómo instalar Python en diferentes sistemas operativos.

\hypertarget{conceptos-clave-2}{%
\section{Conceptos Clave}\label{conceptos-clave-2}}

\hypertarget{python}{%
\subsection{Python}\label{python}}

Lenguaje de programación de alto nivel que se utiliza para desarrollar
aplicaciones web, científicas, de automatización y más.

\hypertarget{interprete}{%
\subsection{Interprete}\label{interprete}}

Python es un lenguaje interpretado, lo que significa que se ejecuta
línea por línea en tiempo real.

\hypertarget{ide}{%
\subsection{IDE}\label{ide}}

Los entornos de desarrollo integrados (IDE) como Visual Studio Code (VS
Code) o PyCharm brindan herramientas para escribir, depurar y ejecutar
código de manera más eficiente.

\hypertarget{ejemplo-2}{%
\section{Ejemplo}\label{ejemplo-2}}

No se necesita código para esta lección, ya que se trata de
instrucciones para la instalación de Python en diferentes sistemas
operativos.

\hypertarget{explicaciuxf3n-2}{%
\section{Explicación}\label{explicaciuxf3n-2}}

Para instalar Python en sistemas Windows, macOS y Linux, se pueden
seguir las instrucciones detalladas proporcionadas en el sitio web
oficial de Python
\href{https://www.python.org/downloads/}{www.python.org/downloads/}.

La instalación de Python generalmente incluye el intérprete de Python y
una serie de herramientas y bibliotecas estándar que hacen que sea fácil
comenzar a programar.

\begin{tcolorbox}[enhanced jigsaw, colbacktitle=quarto-callout-important-color!10!white, toprule=.15mm, leftrule=.75mm, titlerule=0mm, opacityback=0, rightrule=.15mm, opacitybacktitle=0.6, breakable, left=2mm, coltitle=black, title=\textcolor{quarto-callout-important-color}{\faExclamation}\hspace{0.5em}{Actividad Práctica}, toptitle=1mm, bottomtitle=1mm, arc=.35mm, bottomrule=.15mm, colback=white, colframe=quarto-callout-important-color-frame]

Instala Python en tu sistema operativo siguiendo las instrucciones del
sitio web oficial de Python. Luego, verifica que Python esté
correctamente instalado ejecutando el intérprete y escribiendo el
siguiente código:

\begin{Shaded}
\begin{Highlighting}[]
\BuiltInTok{print}\NormalTok{(}\StringTok{"Python se ha instalado correctamente."}\NormalTok{)}
\end{Highlighting}
\end{Shaded}

\end{tcolorbox}

\hypertarget{explicaciuxf3n-de-la-actividad-2}{%
\section{Explicación de la
Actividad}\label{explicaciuxf3n-de-la-actividad-2}}

Esta actividad permite a los participantes aplicar lo aprendido
instalando Python en su propio sistema y ejecutando un programa sencillo
para confirmar que la instalación fue exitosa.

\part{Unidad 3: Introducción a Python}

\hypertarget{distintas-formas-de-trabajar-con-python}{%
\chapter{Distintas formas de trabajar con
Python}\label{distintas-formas-de-trabajar-con-python}}

Python es un lenguaje de programación versátil que ofrece diferentes
formas de interactuar con él. Aprenderemos las dos formas principales de
trabajar con Python: \textbf{el intérprete interactivo} y los
\textbf{scripts de Python}.

\hypertarget{conceptos-clave-3}{%
\section{Conceptos Clave:}\label{conceptos-clave-3}}

\hypertarget{intuxe9rprete-interactivo}{%
\subsection{Intérprete Interactivo:}\label{intuxe9rprete-interactivo}}

Permite ejecutar instrucciones de Python en tiempo real y ver los
resultados inmediatamente en la consola.

\hypertarget{scripts-de-python}{%
\subsection{Scripts de Python:}\label{scripts-de-python}}

Son archivos que contienen una serie de instrucciones de Python que se
pueden ejecutar en conjunto.

\hypertarget{ambientes-virtuales}{%
\subsection{Ambientes Virtuales}\label{ambientes-virtuales}}

Son entornos aislados que permiten tener instalaciones y bibliotecas de
Python separadas para diferentes proyectos.

\hypertarget{ejemplo-3}{%
\subsection{Ejemplo}\label{ejemplo-3}}

\hypertarget{annotated-cell-3}{%
\label{annotated-cell-3}}%
\begin{Shaded}
\begin{Highlighting}[]
\OperatorTok{\textgreater{}\textgreater{}\textgreater{}} \DecValTok{2} \OperatorTok{+} \DecValTok{3} \hspace*{\fill}\NormalTok{\circled{1}}
\DecValTok{5}

\hspace*{\fill}\NormalTok{\circled{2}}
\NormalTok{numero1 }\OperatorTok{=} \DecValTok{5}
\NormalTok{numero2 }\OperatorTok{=} \DecValTok{7}
\NormalTok{resultado }\OperatorTok{=}\NormalTok{ numero1 }\OperatorTok{+}\NormalTok{ numero2}
\BuiltInTok{print}\NormalTok{(}\StringTok{"El resultado de la suma es:"}\NormalTok{, resultado)}
\hspace*{\fill}\NormalTok{\circled{3}}
\end{Highlighting}
\end{Shaded}

\begin{description}
\tightlist
\item[\circled{1}]
Uso del intérprete interactivo
\item[\circled{2}]
Ejecución de un script de Python
\item[\circled{3}]
Guarda este código en un archivo llamado ``suma.py''
\end{description}

\hypertarget{explicaciuxf3n-3}{%
\section{Explicación}\label{explicaciuxf3n-3}}

El intérprete interactivo permite ejecutar expresiones de Python
directamente en la consola y ver los resultados en tiempo real.

Los scripts de Python son archivos que contienen un conjunto de
instrucciones. En este ejemplo, se muestra cómo crear un script simple
que calcula la suma de dos números y lo imprime en la consola.

\begin{tcolorbox}[enhanced jigsaw, colbacktitle=quarto-callout-important-color!10!white, toprule=.15mm, leftrule=.75mm, titlerule=0mm, opacityback=0, rightrule=.15mm, opacitybacktitle=0.6, breakable, left=2mm, coltitle=black, title=\textcolor{quarto-callout-important-color}{\faExclamation}\hspace{0.5em}{Actividad Práctica}, toptitle=1mm, bottomtitle=1mm, arc=.35mm, bottomrule=.15mm, colback=white, colframe=quarto-callout-important-color-frame]

Abre el intérprete interactivo de Python y realiza algunas operaciones
matemáticas simples.

Crea un archivo llamado ``operaciones.py'' y escribe un programa que
realice operaciones aritméticas básicas y las muestre en la consola.

\end{tcolorbox}

\hypertarget{explicaciuxf3n-de-la-actividad-3}{%
\section{Explicación de la
Actividad}\label{explicaciuxf3n-de-la-actividad-3}}

Esta actividad permite a los participantes experimentar con el
intérprete interactivo de Python y crear su propio script para realizar
operaciones matemáticas.

\hypertarget{las-bases-de-python}{%
\chapter{Las bases de Python}\label{las-bases-de-python}}

\textless\textless\textless\textless\textless\textless\textless{} HEAD

En esta lección, exploraremos las bases fundamentales de Python.
Aprenderemos sobre las variables, tipos de datos y operadores básicos
que forman la base de cualquier programa en Python.

\hypertarget{conceptos-clave-4}{%
\section{Conceptos Clave}\label{conceptos-clave-4}}

\hypertarget{variables}{%
\subsection{Variables}\label{variables}}

Son nombres que se utilizan para almacenar valores en la memoria de la
computadora.

\hypertarget{tipos-de-datos}{%
\subsection{Tipos de Datos}\label{tipos-de-datos}}

Incluyen enteros, flotantes, cadenas, booleanos y más. Cada tipo de dato
tiene sus propias características y operaciones.

\hypertarget{operadores}{%
\subsection{Operadores}\label{operadores}}

Incluyen operadores aritméticos, de comparación y lógicos que se
utilizan para realizar diferentes tipos de cálculos y comparaciones.

\hypertarget{ejemplo-4}{%
\subsection{Ejemplo}\label{ejemplo-4}}

\begin{Shaded}
\begin{Highlighting}[]

\CommentTok{\# Variables y tipos de datos}
\NormalTok{nombre }\OperatorTok{=} \StringTok{"Juan"}
\NormalTok{edad }\OperatorTok{=} \DecValTok{25}
\NormalTok{altura }\OperatorTok{=} \FloatTok{1.75}
\NormalTok{es\_mayor\_de\_edad }\OperatorTok{=} \VariableTok{True}

\CommentTok{\# Operadores aritméticos}
\NormalTok{suma }\OperatorTok{=} \DecValTok{5} \OperatorTok{+} \DecValTok{3}
\NormalTok{resta }\OperatorTok{=} \DecValTok{10} \OperatorTok{{-}} \DecValTok{2}
\NormalTok{multiplicacion }\OperatorTok{=} \DecValTok{4} \OperatorTok{*} \DecValTok{6}
\NormalTok{division }\OperatorTok{=} \DecValTok{15} \OperatorTok{/} \DecValTok{3}

\CommentTok{\# Operadores de comparación}
\NormalTok{igual }\OperatorTok{=} \DecValTok{5} \OperatorTok{==} \DecValTok{5}
\NormalTok{mayor\_que }\OperatorTok{=} \DecValTok{10} \OperatorTok{\textgreater{}} \DecValTok{5}
\NormalTok{menor\_que }\OperatorTok{=} \DecValTok{7} \OperatorTok{\textless{}} \DecValTok{12}

\CommentTok{\# Operadores lógicos}
\NormalTok{and\_logico }\OperatorTok{=} \VariableTok{True} \KeywordTok{and} \VariableTok{False}
\NormalTok{or\_logico }\OperatorTok{=} \VariableTok{True} \KeywordTok{or} \VariableTok{False}
\NormalTok{not\_logico }\OperatorTok{=} \KeywordTok{not} \VariableTok{True}
\end{Highlighting}
\end{Shaded}

\hypertarget{explicaciuxf3n-4}{%
\section{Explicación:}\label{explicaciuxf3n-4}}

Las variables se utilizan para almacenar información, como el nombre, la
edad, la altura y si una persona es mayor de edad.

Los operadores aritméticos se utilizan para realizar cálculos
matemáticos básicos.

Los operadores de comparación se utilizan para comparar valores y
devuelven un valor booleano (verdadero o falso).

Los operadores lógicos se utilizan para combinar expresiones booleanas y
realizar operaciones lógicas.

\begin{tcolorbox}[enhanced jigsaw, colbacktitle=quarto-callout-important-color!10!white, toprule=.15mm, leftrule=.75mm, titlerule=0mm, opacityback=0, rightrule=.15mm, opacitybacktitle=0.6, breakable, left=2mm, coltitle=black, title=\textcolor{quarto-callout-important-color}{\faExclamation}\hspace{0.5em}{Actividad Práctica:}, toptitle=1mm, bottomtitle=1mm, arc=.35mm, bottomrule=.15mm, colback=white, colframe=quarto-callout-important-color-frame]

Crea variables que almacenen información sobre ti, como tu nombre, edad
y altura.

Realiza operaciones aritméticas y utiliza operadores de comparación para
comparar valores.

Combina expresiones booleanas utilizando operadores lógicos y observa
los resultados.

\end{tcolorbox}

\hypertarget{explicaciuxf3n-de-la-actividad-4}{%
\section{Explicación de la
Actividad}\label{explicaciuxf3n-de-la-actividad-4}}

Esta actividad permite a los participantes aplicar los conceptos de
variables, tipos de datos y operadores en ejemplos prácticos. Les ayuda
a comprender cómo trabajar con diferentes tipos de datos y cómo realizar
operaciones básicas en Python.

======= \#\# Las bases de Python

En esta lección, exploraremos las bases fundamentales de Python.
Aprenderemos sobre las variables, tipos de datos y operadores básicos
que forman la base de cualquier programa en Python.

\hypertarget{conceptos-clave-5}{%
\section{Conceptos Clave}\label{conceptos-clave-5}}

\hypertarget{variables-1}{%
\subsection{Variables}\label{variables-1}}

Son nombres que se utilizan para almacenar valores en la memoria de la
computadora.

\hypertarget{tipos-de-datos-1}{%
\subsection{Tipos de Datos}\label{tipos-de-datos-1}}

Incluyen enteros, flotantes, cadenas, booleanos y más. Cada tipo de dato
tiene sus propias características y operaciones.

\hypertarget{operadores-1}{%
\subsection{Operadores}\label{operadores-1}}

Incluyen operadores aritméticos, de comparación y lógicos que se
utilizan para realizar diferentes tipos de cálculos y comparaciones.

\hypertarget{ejemplo-5}{%
\subsection{Ejemplo}\label{ejemplo-5}}

\begin{Shaded}
\begin{Highlighting}[]

\CommentTok{\# Variables y tipos de datos}
\NormalTok{nombre }\OperatorTok{=} \StringTok{"Juan"}
\NormalTok{edad }\OperatorTok{=} \DecValTok{25}
\NormalTok{altura }\OperatorTok{=} \FloatTok{1.75}
\NormalTok{es\_mayor\_de\_edad }\OperatorTok{=} \VariableTok{True}

\CommentTok{\# Operadores aritméticos}
\NormalTok{suma }\OperatorTok{=} \DecValTok{5} \OperatorTok{+} \DecValTok{3}
\NormalTok{resta }\OperatorTok{=} \DecValTok{10} \OperatorTok{{-}} \DecValTok{2}
\NormalTok{multiplicacion }\OperatorTok{=} \DecValTok{4} \OperatorTok{*} \DecValTok{6}
\NormalTok{division }\OperatorTok{=} \DecValTok{15} \OperatorTok{/} \DecValTok{3}

\CommentTok{\# Operadores de comparación}
\NormalTok{igual }\OperatorTok{=} \DecValTok{5} \OperatorTok{==} \DecValTok{5}
\NormalTok{mayor\_que }\OperatorTok{=} \DecValTok{10} \OperatorTok{\textgreater{}} \DecValTok{5}
\NormalTok{menor\_que }\OperatorTok{=} \DecValTok{7} \OperatorTok{\textless{}} \DecValTok{12}

\CommentTok{\# Operadores lógicos}
\NormalTok{and\_logico }\OperatorTok{=} \VariableTok{True} \KeywordTok{and} \VariableTok{False}
\NormalTok{or\_logico }\OperatorTok{=} \VariableTok{True} \KeywordTok{or} \VariableTok{False}
\NormalTok{not\_logico }\OperatorTok{=} \KeywordTok{not} \VariableTok{True}
\end{Highlighting}
\end{Shaded}

\hypertarget{explicaciuxf3n-5}{%
\section{Explicación:}\label{explicaciuxf3n-5}}

Las variables se utilizan para almacenar información, como el nombre, la
edad, la altura y si una persona es mayor de edad.

Los operadores aritméticos se utilizan para realizar cálculos
matemáticos básicos.

Los operadores de comparación se utilizan para comparar valores y
devuelven un valor booleano (verdadero o falso).

Los operadores lógicos se utilizan para combinar expresiones booleanas y
realizar operaciones lógicas.

\begin{tcolorbox}[enhanced jigsaw, colbacktitle=quarto-callout-important-color!10!white, toprule=.15mm, leftrule=.75mm, titlerule=0mm, opacityback=0, rightrule=.15mm, opacitybacktitle=0.6, breakable, left=2mm, coltitle=black, title=\textcolor{quarto-callout-important-color}{\faExclamation}\hspace{0.5em}{Actividad Práctica:}, toptitle=1mm, bottomtitle=1mm, arc=.35mm, bottomrule=.15mm, colback=white, colframe=quarto-callout-important-color-frame]

Crea variables que almacenen información sobre ti, como tu nombre, edad
y altura.

Realiza operaciones aritméticas y utiliza operadores de comparación para
comparar valores.

Combina expresiones booleanas utilizando operadores lógicos y observa
los resultados.

\end{tcolorbox}

\hypertarget{explicaciuxf3n-de-la-actividad-5}{%
\section{Explicación de la
Actividad}\label{explicaciuxf3n-de-la-actividad-5}}

Esta actividad permite a los participantes aplicar los conceptos de
variables, tipos de datos y operadores en ejemplos prácticos. Les ayuda
a comprender cómo trabajar con diferentes tipos de datos y cómo realizar
operaciones básicas en Python.

\begin{quote}
\begin{quote}
\begin{quote}
\begin{quote}
\begin{quote}
\begin{quote}
\begin{quote}
e8ed08b1a5bbe1e369719187cfc4de7f7e2a41a9
\end{quote}
\end{quote}
\end{quote}
\end{quote}
\end{quote}
\end{quote}
\end{quote}

\hypertarget{identaciuxf3n}{%
\chapter{Identación}\label{identaciuxf3n}}

\textless\textless\textless\textless\textless\textless\textless{} HEAD

En Python, la identación (sangría) juega un papel crucial en la
estructura y organización del código. Aprenderemos cómo se utiliza la
identación para delimitar bloques de código y mejorar la legibilidad.

\hypertarget{conceptos-clave-6}{%
\section{Conceptos Clave}\label{conceptos-clave-6}}

\hypertarget{identaciuxf3n-1}{%
\subsection{Identación}\label{identaciuxf3n-1}}

\begin{itemize}
\tightlist
\item
  Espacios o tabulaciones al comienzo de una línea que indican la
  estructura del código.
\item
  Bloques de Código: Conjuntos de instrucciones que se agrupan juntas y
  se ejecutan en conjunto.
\item
  PEP 8: Guía de estilo para la escritura de código en Python que
  recomienda el uso de cuatro espacios para la identación.
\end{itemize}

\hypertarget{ejemplo-6}{%
\subsection{Ejemplo}\label{ejemplo-6}}

\begin{Shaded}
\begin{Highlighting}[]
\CommentTok{\# Uso de la identación en un condicional}
\NormalTok{numero }\OperatorTok{=} \DecValTok{10}

\ControlFlowTok{if}\NormalTok{ numero }\OperatorTok{\textgreater{}} \DecValTok{5}\NormalTok{:}
    \BuiltInTok{print}\NormalTok{(}\StringTok{"El número es mayor que 5"}\NormalTok{)}
\ControlFlowTok{else}\NormalTok{:}
    \BuiltInTok{print}\NormalTok{(}\StringTok{"El número no es mayor que 5"}\NormalTok{)}
\end{Highlighting}
\end{Shaded}

\hypertarget{explicaciuxf3n-6}{%
\section{Explicación}\label{explicaciuxf3n-6}}

\begin{itemize}
\tightlist
\item
  En este ejemplo, la identación se utiliza para delimitar los bloques
  de código dentro de las instrucciones ``if'' y ``else''.
\item
  La identación es crucial para que Python sepa qué instrucciones están
  dentro de un bloque y cuáles están fuera.
\end{itemize}

\begin{tcolorbox}[enhanced jigsaw, colbacktitle=quarto-callout-important-color!10!white, toprule=.15mm, leftrule=.75mm, titlerule=0mm, opacityback=0, rightrule=.15mm, opacitybacktitle=0.6, breakable, left=2mm, coltitle=black, title=\textcolor{quarto-callout-important-color}{\faExclamation}\hspace{0.5em}{Actividad Práctica:}, toptitle=1mm, bottomtitle=1mm, arc=.35mm, bottomrule=.15mm, colback=white, colframe=quarto-callout-important-color-frame]

Escribe un programa que solicite al usuario su edad y muestre un mensaje
según si es mayor de 18 años o no.

Intenta cambiar la identación incorrectamente y observa cómo afecta al
funcionamiento del programa.

\end{tcolorbox}

\hypertarget{explicaciuxf3n-de-la-actividad.}{%
\section{Explicación de la
Actividad.}\label{explicaciuxf3n-de-la-actividad.}}

\hypertarget{esta-actividad-permite-a-los-participantes-comprender-la-importancia-de-la-identaciuxf3n-en-python-al-trabajar-con-bloques-de-cuxf3digo-como-los-condicionales.-les-ayuda-a-desarrollar-el-huxe1bito-de-utilizar-la-identaciuxf3n-adecuada-para-mantener-el-cuxf3digo-organizado-y-legible.}{%
\chapter{Esta actividad permite a los participantes comprender la
importancia de la identación en Python al trabajar con bloques de código
como los condicionales. Les ayuda a desarrollar el hábito de utilizar la
identación adecuada para mantener el código organizado y
legible.}\label{esta-actividad-permite-a-los-participantes-comprender-la-importancia-de-la-identaciuxf3n-en-python-al-trabajar-con-bloques-de-cuxf3digo-como-los-condicionales.-les-ayuda-a-desarrollar-el-huxe1bito-de-utilizar-la-identaciuxf3n-adecuada-para-mantener-el-cuxf3digo-organizado-y-legible.}}

\hypertarget{identaciuxf3n-2}{%
\section{Identación}\label{identaciuxf3n-2}}

En Python, la identación (sangría) juega un papel crucial en la
estructura y organización del código. Aprenderemos cómo se utiliza la
identación para delimitar bloques de código y mejorar la legibilidad.

\hypertarget{conceptos-clave-7}{%
\section{Conceptos Clave}\label{conceptos-clave-7}}

\hypertarget{identaciuxf3n-3}{%
\subsection{Identación}\label{identaciuxf3n-3}}

\begin{itemize}
\tightlist
\item
  Espacios o tabulaciones al comienzo de una línea que indican la
  estructura del código.
\item
  Bloques de Código: Conjuntos de instrucciones que se agrupan juntas y
  se ejecutan en conjunto.
\item
  PEP 8: Guía de estilo para la escritura de código en Python que
  recomienda el uso de cuatro espacios para la identación.
\end{itemize}

\hypertarget{ejemplo-7}{%
\subsection{Ejemplo}\label{ejemplo-7}}

\begin{Shaded}
\begin{Highlighting}[]
\CommentTok{\# Uso de la identación en un condicional}
\NormalTok{numero }\OperatorTok{=} \DecValTok{10}

\ControlFlowTok{if}\NormalTok{ numero }\OperatorTok{\textgreater{}} \DecValTok{5}\NormalTok{:}
    \BuiltInTok{print}\NormalTok{(}\StringTok{"El número es mayor que 5"}\NormalTok{)}
\ControlFlowTok{else}\NormalTok{:}
    \BuiltInTok{print}\NormalTok{(}\StringTok{"El número no es mayor que 5"}\NormalTok{)}
\end{Highlighting}
\end{Shaded}

\hypertarget{explicaciuxf3n-7}{%
\section{Explicación}\label{explicaciuxf3n-7}}

\begin{itemize}
\tightlist
\item
  En este ejemplo, la identación se utiliza para delimitar los bloques
  de código dentro de las instrucciones ``if'' y ``else''.
\item
  La identación es crucial para que Python sepa qué instrucciones están
  dentro de un bloque y cuáles están fuera.
\end{itemize}

\begin{tcolorbox}[enhanced jigsaw, colbacktitle=quarto-callout-important-color!10!white, toprule=.15mm, leftrule=.75mm, titlerule=0mm, opacityback=0, rightrule=.15mm, opacitybacktitle=0.6, breakable, left=2mm, coltitle=black, title=\textcolor{quarto-callout-important-color}{\faExclamation}\hspace{0.5em}{Actividad Práctica:}, toptitle=1mm, bottomtitle=1mm, arc=.35mm, bottomrule=.15mm, colback=white, colframe=quarto-callout-important-color-frame]

Escribe un programa que solicite al usuario su edad y muestre un mensaje
según si es mayor de 18 años o no.

Intenta cambiar la identación incorrectamente y observa cómo afecta al
funcionamiento del programa.

\end{tcolorbox}

\hypertarget{explicaciuxf3n-de-la-actividad.-1}{%
\section{Explicación de la
Actividad.}\label{explicaciuxf3n-de-la-actividad.-1}}

Esta actividad permite a los participantes comprender la importancia de
la identación en Python al trabajar con bloques de código como los
condicionales. Les ayuda a desarrollar el hábito de utilizar la
identación adecuada para mantener el código organizado y legible.
\textgreater\textgreater\textgreater\textgreater\textgreater\textgreater\textgreater{}
e8ed08b1a5bbe1e369719187cfc4de7f7e2a41a9

\hypertarget{comentarios}{%
\chapter{Comentarios}\label{comentarios}}

\textless\textless\textless\textless\textless\textless\textless{} HEAD

Los comentarios son una herramienta importante en la programación para
añadir explicaciones y notas en el código. Aprenderemos cómo agregar
comentarios en Python y cómo pueden mejorar la comprensión del código.

\hypertarget{conceptos-clave-8}{%
\section{Conceptos Clave:}\label{conceptos-clave-8}}

\hypertarget{comentarios-1}{%
\subsection{Comentarios}\label{comentarios-1}}

Son notas en el código que no se ejecutan y se utilizan para explicar el
propósito y funcionamiento de partes del programa.

\hypertarget{comentarios-de-una-luxednea}{%
\subsection{Comentarios de una
línea}\label{comentarios-de-una-luxednea}}

Se crean con el símbolo ``\#'' y abarcan una sola línea.

\hypertarget{comentarios-de-muxfaltiples-luxedneas}{%
\subsection{Comentarios de múltiples
líneas}\label{comentarios-de-muxfaltiples-luxedneas}}

Se crean entre triple comillas (``\,``\,'' o '\,'\,') y pueden abarcar
múltiples líneas.

\hypertarget{ejemplo-8}{%
\section{Ejemplo}\label{ejemplo-8}}

\begin{Shaded}
\begin{Highlighting}[]
\CommentTok{\# Este es un comentario de una línea}

\CommentTok{"""}
\CommentTok{Este es un comentario}
\CommentTok{de múltiples líneas.}
\CommentTok{Puede abarcar varias líneas.}
\CommentTok{"""}

\NormalTok{numero }\OperatorTok{=} \DecValTok{42}  \CommentTok{\# Este comentario está después de una instrucción}
\end{Highlighting}
\end{Shaded}

\hypertarget{explicaciuxf3n-8}{%
\section{Explicación:}\label{explicaciuxf3n-8}}

Los comentarios son ignorados por el intérprete y no afectan la
ejecución del código.

Los comentarios son útiles para documentar el código, explicar partes
difíciles de entender o dejar notas para otros programadores.

\begin{tcolorbox}[enhanced jigsaw, colbacktitle=quarto-callout-important-color!10!white, toprule=.15mm, leftrule=.75mm, titlerule=0mm, opacityback=0, rightrule=.15mm, opacitybacktitle=0.6, breakable, left=2mm, coltitle=black, title=\textcolor{quarto-callout-important-color}{\faExclamation}\hspace{0.5em}{Important}, toptitle=1mm, bottomtitle=1mm, arc=.35mm, bottomrule=.15mm, colback=white, colframe=quarto-callout-important-color-frame]

Actividad Práctica:

Escribe un programa que realice una tarea sencilla y agrega comentarios
para explicar lo que hace cada parte.

Escribe un comentario de múltiples líneas que explique el propósito
general de tu programa.

\end{tcolorbox}

\hypertarget{explicaciuxf3n-de-la-actividad-6}{%
\section{Explicación de la
Actividad}\label{explicaciuxf3n-de-la-actividad-6}}

\hypertarget{esta-actividad-permite-a-los-participantes-practicar-la-adiciuxf3n-de-comentarios-en-su-cuxf3digo-para-mejorar-la-legibilidad-y-la-comprensiuxf3n.-les-ayuda-a-desarrollar-la-habilidad-de-documentar-su-cuxf3digo-de-manera-efectiva-para-ellos-mismos-y-para-otros-programadores.}{%
\chapter{Esta actividad permite a los participantes practicar la adición
de comentarios en su código para mejorar la legibilidad y la
comprensión. Les ayuda a desarrollar la habilidad de documentar su
código de manera efectiva para ellos mismos y para otros
programadores.}\label{esta-actividad-permite-a-los-participantes-practicar-la-adiciuxf3n-de-comentarios-en-su-cuxf3digo-para-mejorar-la-legibilidad-y-la-comprensiuxf3n.-les-ayuda-a-desarrollar-la-habilidad-de-documentar-su-cuxf3digo-de-manera-efectiva-para-ellos-mismos-y-para-otros-programadores.}}

\hypertarget{comentarios-2}{%
\section{Comentarios}\label{comentarios-2}}

Los comentarios son una herramienta importante en la programación para
añadir explicaciones y notas en el código. Aprenderemos cómo agregar
comentarios en Python y cómo pueden mejorar la comprensión del código.

\hypertarget{conceptos-clave-9}{%
\section{Conceptos Clave:}\label{conceptos-clave-9}}

\hypertarget{comentarios-3}{%
\subsection{Comentarios}\label{comentarios-3}}

Son notas en el código que no se ejecutan y se utilizan para explicar el
propósito y funcionamiento de partes del programa.

\hypertarget{comentarios-de-una-luxednea-1}{%
\subsection{Comentarios de una
línea}\label{comentarios-de-una-luxednea-1}}

Se crean con el símbolo ``\#'' y abarcan una sola línea.

\hypertarget{comentarios-de-muxfaltiples-luxedneas-1}{%
\subsection{Comentarios de múltiples
líneas}\label{comentarios-de-muxfaltiples-luxedneas-1}}

Se crean entre triple comillas (``\,``\,'' o '\,'\,') y pueden abarcar
múltiples líneas.

\hypertarget{ejemplo-9}{%
\section{Ejemplo}\label{ejemplo-9}}

\begin{Shaded}
\begin{Highlighting}[]
\CommentTok{\# Este es un comentario de una línea}

\CommentTok{"""}
\CommentTok{Este es un comentario}
\CommentTok{de múltiples líneas.}
\CommentTok{Puede abarcar varias líneas.}
\CommentTok{"""}

\NormalTok{numero }\OperatorTok{=} \DecValTok{42}  \CommentTok{\# Este comentario está después de una instrucción}
\end{Highlighting}
\end{Shaded}

\hypertarget{explicaciuxf3n-9}{%
\section{Explicación:}\label{explicaciuxf3n-9}}

Los comentarios son ignorados por el intérprete y no afectan la
ejecución del código.

Los comentarios son útiles para documentar el código, explicar partes
difíciles de entender o dejar notas para otros programadores.

\begin{tcolorbox}[enhanced jigsaw, colbacktitle=quarto-callout-important-color!10!white, toprule=.15mm, leftrule=.75mm, titlerule=0mm, opacityback=0, rightrule=.15mm, opacitybacktitle=0.6, breakable, left=2mm, coltitle=black, title=\textcolor{quarto-callout-important-color}{\faExclamation}\hspace{0.5em}{Important}, toptitle=1mm, bottomtitle=1mm, arc=.35mm, bottomrule=.15mm, colback=white, colframe=quarto-callout-important-color-frame]

Actividad Práctica:

Escribe un programa que realice una tarea sencilla y agrega comentarios
para explicar lo que hace cada parte.

Escribe un comentario de múltiples líneas que explique el propósito
general de tu programa.

\end{tcolorbox}

\hypertarget{explicaciuxf3n-de-la-actividad-7}{%
\section{Explicación de la
Actividad}\label{explicaciuxf3n-de-la-actividad-7}}

Esta actividad permite a los participantes practicar la adición de
comentarios en su código para mejorar la legibilidad y la comprensión.
Les ayuda a desarrollar la habilidad de documentar su código de manera
efectiva para ellos mismos y para otros programadores.
\textgreater\textgreater\textgreater\textgreater\textgreater\textgreater\textgreater{}
e8ed08b1a5bbe1e369719187cfc4de7f7e2a41a9

\hypertarget{variables-2}{%
\chapter{Variables}\label{variables-2}}

\textless\textless\textless\textless\textless\textless\textless{} HEAD

Las variables son fundamentales en la programación ya que permiten
almacenar y manipular datos. Aprenderemos cómo declarar y utilizar
variables en Python.

\hypertarget{conceptos-clave-10}{%
\section{Conceptos Clave:}\label{conceptos-clave-10}}

\hypertarget{variables-3}{%
\subsection{Variables}\label{variables-3}}

Nombres que representan ubicaciones de memoria donde se almacenan datos.

\hypertarget{declaraciuxf3n-de-variables.}{%
\subsection{Declaración de
Variables.}\label{declaraciuxf3n-de-variables.}}

\begin{itemize}
\tightlist
\item
  Asignación de un valor a un nombre utilizando el operador ``=''.
\item
  Convenciones de Nombres: Siguen reglas para ser descriptivos y seguir
  una estructura (por ejemplo, letras minúsculas y guiones bajos para
  espacios).
\end{itemize}

\hypertarget{ejemplo-10}{%
\section{Ejemplo}\label{ejemplo-10}}

\begin{Shaded}
\begin{Highlighting}[]
\NormalTok{nombre }\OperatorTok{=} \StringTok{"Ana"}
\NormalTok{edad }\OperatorTok{=} \DecValTok{30}
\NormalTok{saldo\_bancario }\OperatorTok{=} \FloatTok{1500.75}
\NormalTok{es\_mayor\_de\_edad }\OperatorTok{=} \VariableTok{True}
\end{Highlighting}
\end{Shaded}

\hypertarget{explicaciuxf3n-10}{%
\section{Explicación:}\label{explicaciuxf3n-10}}

En este ejemplo, se declaran variables para almacenar el nombre de una
persona, su edad, su saldo bancario y un valor booleano que indica si es
mayor de edad.

Los nombres de variables son descriptivos y siguen la convención de
nombres recomendada (letras minúsculas y guiones bajos para espacios).

\begin{tcolorbox}[enhanced jigsaw, colbacktitle=quarto-callout-important-color!10!white, toprule=.15mm, leftrule=.75mm, titlerule=0mm, opacityback=0, rightrule=.15mm, opacitybacktitle=0.6, breakable, left=2mm, coltitle=black, title=\textcolor{quarto-callout-important-color}{\faExclamation}\hspace{0.5em}{Actividad Práctica}, toptitle=1mm, bottomtitle=1mm, arc=.35mm, bottomrule=.15mm, colback=white, colframe=quarto-callout-important-color-frame]

Crea variables para almacenar información personal, como tu ciudad, tu
edad y tu ocupación.

Declara variables para almacenar cantidades numéricas, como el precio de
un producto y la cantidad de unidades disponibles.

\end{tcolorbox}

\hypertarget{explicaciuxf3n-de-la-actividad-8}{%
\section{Explicación de la
Actividad}\label{explicaciuxf3n-de-la-actividad-8}}

Esta actividad permite a los participantes practicar la declaración de
variables en Python y aplicar el concepto de convenciones de nombres.

\hypertarget{les-ayuda-a-comprender-cuxf3mo-almacenar-y-acceder-a-datos-utilizando-variables-descriptivas-y-significativas.}{%
\chapter{Les ayuda a comprender cómo almacenar y acceder a datos
utilizando variables descriptivas y
significativas.}\label{les-ayuda-a-comprender-cuxf3mo-almacenar-y-acceder-a-datos-utilizando-variables-descriptivas-y-significativas.}}

\hypertarget{variables-4}{%
\section{Variables}\label{variables-4}}

Las variables son fundamentales en la programación ya que permiten
almacenar y manipular datos. Aprenderemos cómo declarar y utilizar
variables en Python.

\hypertarget{conceptos-clave-11}{%
\section{Conceptos Clave:}\label{conceptos-clave-11}}

\hypertarget{variables-5}{%
\subsection{Variables}\label{variables-5}}

Nombres que representan ubicaciones de memoria donde se almacenan datos.

\hypertarget{declaraciuxf3n-de-variables.-1}{%
\subsection{Declaración de
Variables.}\label{declaraciuxf3n-de-variables.-1}}

\begin{itemize}
\tightlist
\item
  Asignación de un valor a un nombre utilizando el operador ``=''.
\item
  Convenciones de Nombres: Siguen reglas para ser descriptivos y seguir
  una estructura (por ejemplo, letras minúsculas y guiones bajos para
  espacios).
\end{itemize}

\hypertarget{ejemplo-11}{%
\section{Ejemplo}\label{ejemplo-11}}

\begin{Shaded}
\begin{Highlighting}[]
\NormalTok{nombre }\OperatorTok{=} \StringTok{"Ana"}
\NormalTok{edad }\OperatorTok{=} \DecValTok{30}
\NormalTok{saldo\_bancario }\OperatorTok{=} \FloatTok{1500.75}
\NormalTok{es\_mayor\_de\_edad }\OperatorTok{=} \VariableTok{True}
\end{Highlighting}
\end{Shaded}

\hypertarget{explicaciuxf3n-11}{%
\section{Explicación:}\label{explicaciuxf3n-11}}

En este ejemplo, se declaran variables para almacenar el nombre de una
persona, su edad, su saldo bancario y un valor booleano que indica si es
mayor de edad.

Los nombres de variables son descriptivos y siguen la convención de
nombres recomendada (letras minúsculas y guiones bajos para espacios).

\begin{tcolorbox}[enhanced jigsaw, colbacktitle=quarto-callout-important-color!10!white, toprule=.15mm, leftrule=.75mm, titlerule=0mm, opacityback=0, rightrule=.15mm, opacitybacktitle=0.6, breakable, left=2mm, coltitle=black, title=\textcolor{quarto-callout-important-color}{\faExclamation}\hspace{0.5em}{Actividad Práctica}, toptitle=1mm, bottomtitle=1mm, arc=.35mm, bottomrule=.15mm, colback=white, colframe=quarto-callout-important-color-frame]

Crea variables para almacenar información personal, como tu ciudad, tu
edad y tu ocupación.

Declara variables para almacenar cantidades numéricas, como el precio de
un producto y la cantidad de unidades disponibles.

\end{tcolorbox}

\hypertarget{explicaciuxf3n-de-la-actividad-9}{%
\section{Explicación de la
Actividad}\label{explicaciuxf3n-de-la-actividad-9}}

Esta actividad permite a los participantes practicar la declaración de
variables en Python y aplicar el concepto de convenciones de nombres.

Les ayuda a comprender cómo almacenar y acceder a datos utilizando
variables descriptivas y significativas.
\textgreater\textgreater\textgreater\textgreater\textgreater\textgreater\textgreater{}
e8ed08b1a5bbe1e369719187cfc4de7f7e2a41a9

\hypertarget{muxfaltiples-variables}{%
\chapter{Múltiples Variables}\label{muxfaltiples-variables}}

\textless\textless\textless\textless\textless\textless\textless{} HEAD

En Python, es posible asignar valores a múltiples variables en una sola
línea. Aprenderemos cómo declarar y utilizar múltiples variables de
manera eficiente.

\hypertarget{conceptos-clave-12}{%
\section{Conceptos Clave:}\label{conceptos-clave-12}}

\hypertarget{asignaciuxf3n-muxfaltiple}{%
\subsection{Asignación Múltiple}\label{asignaciuxf3n-muxfaltiple}}

Permite asignar valores a varias variables en una línea.

\hypertarget{desempaquetado-de-valores}{%
\subsection{Desempaquetado de Valores}\label{desempaquetado-de-valores}}

Se pueden asignar valores de una lista o tupla a múltiples variables en
una sola operación.

\hypertarget{intercambio-de-valores}{%
\subsection{Intercambio de Valores}\label{intercambio-de-valores}}

Se pueden intercambiar los valores de dos variables utilizando
asignación múltiple.

\hypertarget{ejemplo-12}{%
\section{Ejemplo}\label{ejemplo-12}}

\begin{Shaded}
\begin{Highlighting}[]
\NormalTok{nombre, edad, altura }\OperatorTok{=} \StringTok{"María"}\NormalTok{, }\DecValTok{28}\NormalTok{, }\FloatTok{1.65}
\NormalTok{productos }\OperatorTok{=}\NormalTok{ (}\StringTok{"Manzanas"}\NormalTok{, }\StringTok{"Peras"}\NormalTok{, }\StringTok{"Uvas"}\NormalTok{)}
\NormalTok{producto1, producto2, producto3 }\OperatorTok{=}\NormalTok{ productos}
\end{Highlighting}
\end{Shaded}

\hypertarget{explicaciuxf3n-12}{%
\section{Explicación:}\label{explicaciuxf3n-12}}

En el primer ejemplo, se utilizó la asignación múltiple para declarar
tres variables en una sola línea.

En el segundo ejemplo, se desempaquetaron los valores de una tupla en
variables individuales.

::: \{.callout-important\} \#\#\# Actividad Práctica:

Crea una lista con los nombres de tus tres colores favoritos.

Utiliza la asignación múltiple para asignar los valores de la lista a
tres variables individuales.

\hypertarget{explicaciuxf3n-de-la-actividad-10}{%
\section{Explicación de la
Actividad}\label{explicaciuxf3n-de-la-actividad-10}}

Esta actividad permite a los participantes practicar la asignación
múltiple y el desempaquetado de valores.

\hypertarget{les-ayuda-a-comprender-cuxf3mo-trabajar-eficientemente-con-muxfaltiples-variables-y-cuxf3mo-aprovechar-estas-tuxe9cnicas-para-simplificar-el-cuxf3digo.}{%
\chapter{Les ayuda a comprender cómo trabajar eficientemente con
múltiples variables y cómo aprovechar estas técnicas para simplificar el
código.}\label{les-ayuda-a-comprender-cuxf3mo-trabajar-eficientemente-con-muxfaltiples-variables-y-cuxf3mo-aprovechar-estas-tuxe9cnicas-para-simplificar-el-cuxf3digo.}}

\hypertarget{muxfaltiples-variables-1}{%
\section{Múltiples Variables}\label{muxfaltiples-variables-1}}

En Python, es posible asignar valores a múltiples variables en una sola
línea. Aprenderemos cómo declarar y utilizar múltiples variables de
manera eficiente.

\hypertarget{conceptos-clave-13}{%
\section{Conceptos Clave:}\label{conceptos-clave-13}}

\hypertarget{asignaciuxf3n-muxfaltiple-1}{%
\subsection{Asignación Múltiple}\label{asignaciuxf3n-muxfaltiple-1}}

Permite asignar valores a varias variables en una línea.

\hypertarget{desempaquetado-de-valores-1}{%
\subsection{Desempaquetado de
Valores}\label{desempaquetado-de-valores-1}}

Se pueden asignar valores de una lista o tupla a múltiples variables en
una sola operación.

\hypertarget{intercambio-de-valores-1}{%
\subsection{Intercambio de Valores}\label{intercambio-de-valores-1}}

Se pueden intercambiar los valores de dos variables utilizando
asignación múltiple.

\hypertarget{ejemplo-13}{%
\section{Ejemplo}\label{ejemplo-13}}

\begin{Shaded}
\begin{Highlighting}[]
\NormalTok{nombre, edad, altura }\OperatorTok{=} \StringTok{"María"}\NormalTok{, }\DecValTok{28}\NormalTok{, }\FloatTok{1.65}
\NormalTok{productos }\OperatorTok{=}\NormalTok{ (}\StringTok{"Manzanas"}\NormalTok{, }\StringTok{"Peras"}\NormalTok{, }\StringTok{"Uvas"}\NormalTok{)}
\NormalTok{producto1, producto2, producto3 }\OperatorTok{=}\NormalTok{ productos}
\end{Highlighting}
\end{Shaded}

\hypertarget{explicaciuxf3n-13}{%
\section{Explicación:}\label{explicaciuxf3n-13}}

En el primer ejemplo, se utilizó la asignación múltiple para declarar
tres variables en una sola línea.

En el segundo ejemplo, se desempaquetaron los valores de una tupla en
variables individuales.

::: \{.callout-important\} \#\#\# Actividad Práctica:

Crea una lista con los nombres de tus tres colores favoritos.

Utiliza la asignación múltiple para asignar los valores de la lista a
tres variables individuales.

\hypertarget{explicaciuxf3n-de-la-actividad-11}{%
\section{Explicación de la
Actividad}\label{explicaciuxf3n-de-la-actividad-11}}

Esta actividad permite a los participantes practicar la asignación
múltiple y el desempaquetado de valores.

Les ayuda a comprender cómo trabajar eficientemente con múltiples
variables y cómo aprovechar estas técnicas para simplificar el código.
\textgreater\textgreater\textgreater\textgreater\textgreater\textgreater\textgreater{}
e8ed08b1a5bbe1e369719187cfc4de7f7e2a41a9

\hypertarget{concatenaciuxf3n}{%
\chapter{Concatenación}\label{concatenaciuxf3n}}

\textless\textless\textless\textless\textless\textless\textless{} HEAD

La concatenación es la unión de cadenas de texto. Aprenderemos cómo
combinar cadenas de texto en Python para crear mensajes más complejos.

\hypertarget{conceptos-clave-14}{%
\section{Conceptos Clave:}\label{conceptos-clave-14}}

\hypertarget{concatenaciuxf3n-1}{%
\subsection{Concatenación:}\label{concatenaciuxf3n-1}}

Operación que combina dos o más cadenas de texto para formar una cadena
más larga.

\hypertarget{operador}{%
\subsection{Operador +}\label{operador}}

Se utiliza para concatenar cadenas de texto.

\hypertarget{conversiuxf3n-a-cadena}{%
\subsection{Conversión a Cadena}\label{conversiuxf3n-a-cadena}}

Es necesario convertir valores no string a cadenas antes de
concatenarlos.

\hypertarget{ejemplo-14}{%
\section{Ejemplo}\label{ejemplo-14}}

\begin{Shaded}
\begin{Highlighting}[]
\NormalTok{nombre }\OperatorTok{=} \StringTok{"Luisa"}
\NormalTok{mensaje }\OperatorTok{=} \StringTok{"Hola, "} \OperatorTok{+}\NormalTok{ nombre }\OperatorTok{+} \StringTok{". ¿Cómo estás?"}
\NormalTok{edad }\OperatorTok{=} \DecValTok{25}
\NormalTok{mensaje\_edad }\OperatorTok{=} \StringTok{"Tienes "} \OperatorTok{+} \BuiltInTok{str}\NormalTok{(edad) }\OperatorTok{+} \StringTok{" años."}
\end{Highlighting}
\end{Shaded}

\hypertarget{explicaciuxf3n-14}{%
\section{Explicación:}\label{explicaciuxf3n-14}}

En este ejemplo, se utilizó el operador ``+'' para concatenar cadenas de
texto.

La variable ``edad'' se convirtió a una cadena utilizando la función
``str()'' antes de concatenarla.

\begin{tcolorbox}[enhanced jigsaw, colbacktitle=quarto-callout-important-color!10!white, toprule=.15mm, leftrule=.75mm, titlerule=0mm, opacityback=0, rightrule=.15mm, opacitybacktitle=0.6, breakable, left=2mm, coltitle=black, title=\textcolor{quarto-callout-important-color}{\faExclamation}\hspace{0.5em}{Actividad Práctica:}, toptitle=1mm, bottomtitle=1mm, arc=.35mm, bottomrule=.15mm, colback=white, colframe=quarto-callout-important-color-frame]

Crea una variable con tu comida favorita. Utiliza la concatenación para
crear un mensaje que incluya tu comida favorita.

\end{tcolorbox}

\hypertarget{explicaciuxf3n-de-la-actividad-12}{%
\section{Explicación de la
Actividad}\label{explicaciuxf3n-de-la-actividad-12}}

======= \#\# Concatenación

La concatenación es la unión de cadenas de texto. Aprenderemos cómo
combinar cadenas de texto en Python para crear mensajes más complejos.

\hypertarget{conceptos-clave-15}{%
\section{Conceptos Clave:}\label{conceptos-clave-15}}

\hypertarget{concatenaciuxf3n-2}{%
\subsection{Concatenación:}\label{concatenaciuxf3n-2}}

Operación que combina dos o más cadenas de texto para formar una cadena
más larga.

\hypertarget{operador-1}{%
\subsection{Operador +}\label{operador-1}}

Se utiliza para concatenar cadenas de texto.

\hypertarget{conversiuxf3n-a-cadena-1}{%
\subsection{Conversión a Cadena}\label{conversiuxf3n-a-cadena-1}}

Es necesario convertir valores no string a cadenas antes de
concatenarlos.

\hypertarget{ejemplo-15}{%
\section{Ejemplo}\label{ejemplo-15}}

\begin{Shaded}
\begin{Highlighting}[]
\NormalTok{nombre }\OperatorTok{=} \StringTok{"Luisa"}
\NormalTok{mensaje }\OperatorTok{=} \StringTok{"Hola, "} \OperatorTok{+}\NormalTok{ nombre }\OperatorTok{+} \StringTok{". ¿Cómo estás?"}
\NormalTok{edad }\OperatorTok{=} \DecValTok{25}
\NormalTok{mensaje\_edad }\OperatorTok{=} \StringTok{"Tienes "} \OperatorTok{+} \BuiltInTok{str}\NormalTok{(edad) }\OperatorTok{+} \StringTok{" años."}
\end{Highlighting}
\end{Shaded}

\hypertarget{explicaciuxf3n-15}{%
\section{Explicación:}\label{explicaciuxf3n-15}}

En este ejemplo, se utilizó el operador ``+'' para concatenar cadenas de
texto.

La variable ``edad'' se convirtió a una cadena utilizando la función
``str()'' antes de concatenarla.

\begin{tcolorbox}[enhanced jigsaw, colbacktitle=quarto-callout-important-color!10!white, toprule=.15mm, leftrule=.75mm, titlerule=0mm, opacityback=0, rightrule=.15mm, opacitybacktitle=0.6, breakable, left=2mm, coltitle=black, title=\textcolor{quarto-callout-important-color}{\faExclamation}\hspace{0.5em}{Actividad Práctica:}, toptitle=1mm, bottomtitle=1mm, arc=.35mm, bottomrule=.15mm, colback=white, colframe=quarto-callout-important-color-frame]

Crea una variable con tu comida favorita. Utiliza la concatenación para
crear un mensaje que incluya tu comida favorita.

\end{tcolorbox}

\hypertarget{explicaciuxf3n-de-la-actividad-13}{%
\section{Explicación de la
Actividad}\label{explicaciuxf3n-de-la-actividad-13}}

\begin{quote}
\begin{quote}
\begin{quote}
\begin{quote}
\begin{quote}
\begin{quote}
\begin{quote}
e8ed08b1a5bbe1e369719187cfc4de7f7e2a41a9 Esta actividad permite a los
participantes practicar la concatenación de cadenas de texto y
comprender cómo construir mensajes más complejos utilizando variables y
texto. Les ayuda a mejorar su capacidad para crear mensajes
personalizados en sus programas.
\end{quote}
\end{quote}
\end{quote}
\end{quote}
\end{quote}
\end{quote}
\end{quote}

\part{Unidad 4: Tipos de Dato}

\hypertarget{string-y-nuxfameros}{%
\chapter{String y Números}\label{string-y-nuxfameros}}

\textless\textless\textless\textless\textless\textless\textless{} HEAD

Los strings y los números son dos tipos de datos fundamentales en
Python. Aprenderemos cómo trabajar con strings (cadenas de texto) y los
diferentes tipos de números en Python.

\hypertarget{conceptos-clave-16}{%
\section{Conceptos Clave:}\label{conceptos-clave-16}}

\hypertarget{string}{%
\subsection{String}\label{string}}

Secuencia de caracteres alfanuméricos. Se pueden definir utilizando
comillas simples o dobles.

\hypertarget{nuxfameros-enteros-int}{%
\subsection{Números Enteros (int)}\label{nuxfameros-enteros-int}}

Representan números enteros positivos o negativos.

\hypertarget{nuxfameros-de-punto-flotante-float}{%
\subsection{Números de Punto Flotante
(float)}\label{nuxfameros-de-punto-flotante-float}}

Representan números con decimales.

\hypertarget{ejemplo-16}{%
\section{Ejemplo}\label{ejemplo-16}}

\begin{Shaded}
\begin{Highlighting}[]
\CommentTok{\# Strings}
\NormalTok{mensaje }\OperatorTok{=} \StringTok{"Hola, bienvenido al curso de Python."}
\NormalTok{nombre }\OperatorTok{=} \StringTok{\textquotesingle{}María\textquotesingle{}}

\CommentTok{\# Números}
\NormalTok{edad }\OperatorTok{=} \DecValTok{25}
\NormalTok{saldo }\OperatorTok{=} \FloatTok{1500.75}
\end{Highlighting}
\end{Shaded}

\hypertarget{explicaciuxf3n-16}{%
\section{Explicación}\label{explicaciuxf3n-16}}

En este ejemplo, se crean variables que almacenan strings y números.

Los strings se definen utilizando comillas simples o dobles.

Los números enteros y de punto flotante se asignan directamente a
variables.

\begin{tcolorbox}[enhanced jigsaw, colbacktitle=quarto-callout-important-color!10!white, toprule=.15mm, leftrule=.75mm, titlerule=0mm, opacityback=0, rightrule=.15mm, opacitybacktitle=0.6, breakable, left=2mm, coltitle=black, title=\textcolor{quarto-callout-important-color}{\faExclamation}\hspace{0.5em}{Actividad Práctica:}, toptitle=1mm, bottomtitle=1mm, arc=.35mm, bottomrule=.15mm, colback=white, colframe=quarto-callout-important-color-frame]

Crea una variable con tu canción favorita.

Asigna tu edad a una variable y tu altura a otra variable.

Combina las variables para crear un mensaje personalizado.

\end{tcolorbox}

\hypertarget{explicaciuxf3n-de-la-actividad-14}{%
\section{Explicación de la
Actividad:}\label{explicaciuxf3n-de-la-actividad-14}}

Esta actividad permite a los participantes practicar la creación de
strings y trabajar con números enteros y de punto flotante.

\hypertarget{les-ayuda-a-comprender-cuxf3mo-almacenar-y-manipular-diferentes-tipos-de-datos-en-python.}{%
\chapter{Les ayuda a comprender cómo almacenar y manipular diferentes
tipos de datos en
Python.}\label{les-ayuda-a-comprender-cuxf3mo-almacenar-y-manipular-diferentes-tipos-de-datos-en-python.}}

\hypertarget{string-y-nuxfameros-1}{%
\section{String y Números}\label{string-y-nuxfameros-1}}

Los strings y los números son dos tipos de datos fundamentales en
Python. Aprenderemos cómo trabajar con strings (cadenas de texto) y los
diferentes tipos de números en Python.

\hypertarget{conceptos-clave-17}{%
\section{Conceptos Clave:}\label{conceptos-clave-17}}

\hypertarget{string-1}{%
\subsection{String}\label{string-1}}

Secuencia de caracteres alfanuméricos. Se pueden definir utilizando
comillas simples o dobles.

\hypertarget{nuxfameros-enteros-int-1}{%
\subsection{Números Enteros (int)}\label{nuxfameros-enteros-int-1}}

Representan números enteros positivos o negativos.

\hypertarget{nuxfameros-de-punto-flotante-float-1}{%
\subsection{Números de Punto Flotante
(float)}\label{nuxfameros-de-punto-flotante-float-1}}

Representan números con decimales.

\hypertarget{ejemplo-17}{%
\section{Ejemplo}\label{ejemplo-17}}

\begin{Shaded}
\begin{Highlighting}[]
\CommentTok{\# Strings}
\NormalTok{mensaje }\OperatorTok{=} \StringTok{"Hola, bienvenido al curso de Python."}
\NormalTok{nombre }\OperatorTok{=} \StringTok{\textquotesingle{}María\textquotesingle{}}

\CommentTok{\# Números}
\NormalTok{edad }\OperatorTok{=} \DecValTok{25}
\NormalTok{saldo }\OperatorTok{=} \FloatTok{1500.75}
\end{Highlighting}
\end{Shaded}

\hypertarget{explicaciuxf3n-17}{%
\section{Explicación}\label{explicaciuxf3n-17}}

En este ejemplo, se crean variables que almacenan strings y números.

Los strings se definen utilizando comillas simples o dobles.

Los números enteros y de punto flotante se asignan directamente a
variables.

\begin{tcolorbox}[enhanced jigsaw, colbacktitle=quarto-callout-important-color!10!white, toprule=.15mm, leftrule=.75mm, titlerule=0mm, opacityback=0, rightrule=.15mm, opacitybacktitle=0.6, breakable, left=2mm, coltitle=black, title=\textcolor{quarto-callout-important-color}{\faExclamation}\hspace{0.5em}{Actividad Práctica:}, toptitle=1mm, bottomtitle=1mm, arc=.35mm, bottomrule=.15mm, colback=white, colframe=quarto-callout-important-color-frame]

Crea una variable con tu canción favorita.

Asigna tu edad a una variable y tu altura a otra variable.

Combina las variables para crear un mensaje personalizado.

\end{tcolorbox}

\hypertarget{explicaciuxf3n-de-la-actividad-15}{%
\section{Explicación de la
Actividad:}\label{explicaciuxf3n-de-la-actividad-15}}

Esta actividad permite a los participantes practicar la creación de
strings y trabajar con números enteros y de punto flotante.

Les ayuda a comprender cómo almacenar y manipular diferentes tipos de
datos en Python.
\textgreater\textgreater\textgreater\textgreater\textgreater\textgreater\textgreater{}
e8ed08b1a5bbe1e369719187cfc4de7f7e2a41a9

\hypertarget{listas}{%
\chapter{Listas}\label{listas}}

\textless\textless\textless\textless\textless\textless\textless{} HEAD

Las listas son estructuras de datos que permiten almacenar varios
elementos en una sola variable. Aprenderemos cómo crear y manipular
listas en Python.

\hypertarget{conceptos-clave-18}{%
\section{Conceptos Clave:}\label{conceptos-clave-18}}

\hypertarget{listas-1}{%
\subsection{Listas}\label{listas-1}}

Secuencias ordenadas de elementos que pueden ser de diferentes tipos.

\hypertarget{uxedndices}{%
\subsection{Índices}\label{uxedndices}}

Números que indican la posición de un elemento en la lista.

\hypertarget{acceso-a-elementos}{%
\subsection{Acceso a Elementos}\label{acceso-a-elementos}}

Se utiliza el índice para acceder a un elemento específico de la lista.

\hypertarget{ejemplo-18}{%
\section{Ejemplo}\label{ejemplo-18}}

\begin{Shaded}
\begin{Highlighting}[]
\NormalTok{frutas }\OperatorTok{=}\NormalTok{ [}\StringTok{"manzana"}\NormalTok{, }\StringTok{"banana"}\NormalTok{, }\StringTok{"naranja"}\NormalTok{, }\StringTok{"uva"}\NormalTok{]}
\NormalTok{primer\_fruta }\OperatorTok{=}\NormalTok{ frutas[}\DecValTok{0}\NormalTok{]}
\NormalTok{segunda\_fruta }\OperatorTok{=}\NormalTok{ frutas[}\DecValTok{1}\NormalTok{]}
\end{Highlighting}
\end{Shaded}

\hypertarget{explicaciuxf3n-18}{%
\section{Explicación}\label{explicaciuxf3n-18}}

En este ejemplo, se crea una lista de frutas y se accede a elementos
individuales utilizando índices.

Los índices comienzan desde 0, por lo que la primera fruta tiene el
índice 0.

\begin{tcolorbox}[enhanced jigsaw, colbacktitle=quarto-callout-important-color!10!white, toprule=.15mm, leftrule=.75mm, titlerule=0mm, opacityback=0, rightrule=.15mm, opacitybacktitle=0.6, breakable, left=2mm, coltitle=black, title=\textcolor{quarto-callout-important-color}{\faExclamation}\hspace{0.5em}{Actividad Práctica:}, toptitle=1mm, bottomtitle=1mm, arc=.35mm, bottomrule=.15mm, colback=white, colframe=quarto-callout-important-color-frame]

Crea una lista con los nombres de tus tres películas favoritas.

Accede al segundo elemento de la lista e imprímelo en la consola.

\end{tcolorbox}

\hypertarget{explicaciuxf3n-de-la-actividad-16}{%
\section{Explicación de la
Actividad:}\label{explicaciuxf3n-de-la-actividad-16}}

Esta actividad permite a los participantes practicar la creación de
listas y el acceso a elementos utilizando índices. Les ayuda a
comprender cómo organizar y acceder a múltiples elementos en una sola
variable.

======= \#\# Listas

Las listas son estructuras de datos que permiten almacenar varios
elementos en una sola variable. Aprenderemos cómo crear y manipular
listas en Python.

\hypertarget{conceptos-clave-19}{%
\section{Conceptos Clave:}\label{conceptos-clave-19}}

\hypertarget{listas-2}{%
\subsection{Listas}\label{listas-2}}

Secuencias ordenadas de elementos que pueden ser de diferentes tipos.

\hypertarget{uxedndices-1}{%
\subsection{Índices}\label{uxedndices-1}}

Números que indican la posición de un elemento en la lista.

\hypertarget{acceso-a-elementos-1}{%
\subsection{Acceso a Elementos}\label{acceso-a-elementos-1}}

Se utiliza el índice para acceder a un elemento específico de la lista.

\hypertarget{ejemplo-19}{%
\section{Ejemplo}\label{ejemplo-19}}

\begin{Shaded}
\begin{Highlighting}[]
\NormalTok{frutas }\OperatorTok{=}\NormalTok{ [}\StringTok{"manzana"}\NormalTok{, }\StringTok{"banana"}\NormalTok{, }\StringTok{"naranja"}\NormalTok{, }\StringTok{"uva"}\NormalTok{]}
\NormalTok{primer\_fruta }\OperatorTok{=}\NormalTok{ frutas[}\DecValTok{0}\NormalTok{]}
\NormalTok{segunda\_fruta }\OperatorTok{=}\NormalTok{ frutas[}\DecValTok{1}\NormalTok{]}
\end{Highlighting}
\end{Shaded}

\hypertarget{explicaciuxf3n-19}{%
\section{Explicación}\label{explicaciuxf3n-19}}

En este ejemplo, se crea una lista de frutas y se accede a elementos
individuales utilizando índices.

Los índices comienzan desde 0, por lo que la primera fruta tiene el
índice 0.

\begin{tcolorbox}[enhanced jigsaw, colbacktitle=quarto-callout-important-color!10!white, toprule=.15mm, leftrule=.75mm, titlerule=0mm, opacityback=0, rightrule=.15mm, opacitybacktitle=0.6, breakable, left=2mm, coltitle=black, title=\textcolor{quarto-callout-important-color}{\faExclamation}\hspace{0.5em}{Actividad Práctica:}, toptitle=1mm, bottomtitle=1mm, arc=.35mm, bottomrule=.15mm, colback=white, colframe=quarto-callout-important-color-frame]

Crea una lista con los nombres de tus tres películas favoritas.

Accede al segundo elemento de la lista e imprímelo en la consola.

\end{tcolorbox}

\hypertarget{explicaciuxf3n-de-la-actividad-17}{%
\section{Explicación de la
Actividad:}\label{explicaciuxf3n-de-la-actividad-17}}

Esta actividad permite a los participantes practicar la creación de
listas y el acceso a elementos utilizando índices. Les ayuda a
comprender cómo organizar y acceder a múltiples elementos en una sola
variable.

\begin{quote}
\begin{quote}
\begin{quote}
\begin{quote}
\begin{quote}
\begin{quote}
\begin{quote}
e8ed08b1a5bbe1e369719187cfc4de7f7e2a41a9
\end{quote}
\end{quote}
\end{quote}
\end{quote}
\end{quote}
\end{quote}
\end{quote}

\hypertarget{tuplas}{%
\chapter{Tuplas}\label{tuplas}}

\textless\textless\textless\textless\textless\textless\textless{} HEAD

Las tuplas son estructuras de datos similares a las listas, pero son
inmutables, lo que significa que no se pueden modificar después de ser
creadas. Aprenderemos cómo trabajar con tuplas en Python.

\hypertarget{conceptos-clave-20}{%
\section{Conceptos Clave:}\label{conceptos-clave-20}}

\hypertarget{tuplas-1}{%
\subsection{Tuplas}\label{tuplas-1}}

Secuencias ordenadas de elementos que, a diferencia de las listas, no se
pueden modificar.

\hypertarget{inmutabilidad}{%
\subsection{Inmutabilidad}\label{inmutabilidad}}

Una vez creada una tupla, no se pueden agregar, modificar o eliminar
elementos.

\hypertarget{acceso-a-elementos-2}{%
\subsection{Acceso a Elementos}\label{acceso-a-elementos-2}}

Se utiliza el índice para acceder a un elemento específico de la tupla.

\hypertarget{ejemplo-20}{%
\section{Ejemplo}\label{ejemplo-20}}

\begin{Shaded}
\begin{Highlighting}[]
\NormalTok{coordenadas }\OperatorTok{=}\NormalTok{ (}\DecValTok{3}\NormalTok{, }\DecValTok{5}\NormalTok{)}
\NormalTok{x }\OperatorTok{=}\NormalTok{ coordenadas[}\DecValTok{0}\NormalTok{]}
\NormalTok{y }\OperatorTok{=}\NormalTok{ coordenadas[}\DecValTok{1}\NormalTok{]}
\end{Highlighting}
\end{Shaded}

\hypertarget{explicaciuxf3n-20}{%
\section{Explicación}\label{explicaciuxf3n-20}}

En este ejemplo, se crea una tupla que almacena coordenadas (x, y) y se
accede a los valores individuales utilizando índices.

\begin{tcolorbox}[enhanced jigsaw, colbacktitle=quarto-callout-important-color!10!white, toprule=.15mm, leftrule=.75mm, titlerule=0mm, opacityback=0, rightrule=.15mm, opacitybacktitle=0.6, breakable, left=2mm, coltitle=black, title=\textcolor{quarto-callout-important-color}{\faExclamation}\hspace{0.5em}{Actividad Práctica:}, toptitle=1mm, bottomtitle=1mm, arc=.35mm, bottomrule=.15mm, colback=white, colframe=quarto-callout-important-color-frame]

Crea una tupla con las estaciones del año.

Intenta modificar un elemento de la tupla y observa el error que se
produce.

\end{tcolorbox}

\hypertarget{explicaciuxf3n-de-la-actividad-18}{%
\section{Explicación de la
Actividad:}\label{explicaciuxf3n-de-la-actividad-18}}

\hypertarget{esta-actividad-permite-a-los-participantes-practicar-la-creaciuxf3n-de-tuplas-y-comprender-la-diferencia-entre-listas-y-tuplas-en-tuxe9rminos-de-inmutabilidad.-les-ayuda-a-comprender-cuxf3mo-utilizar-tuplas-cuando-necesitan-almacenar-datos-que-no-deben-cambiar.}{%
\chapter{Esta actividad permite a los participantes practicar la
creación de tuplas y comprender la diferencia entre listas y tuplas en
términos de inmutabilidad. Les ayuda a comprender cómo utilizar tuplas
cuando necesitan almacenar datos que no deben
cambiar.}\label{esta-actividad-permite-a-los-participantes-practicar-la-creaciuxf3n-de-tuplas-y-comprender-la-diferencia-entre-listas-y-tuplas-en-tuxe9rminos-de-inmutabilidad.-les-ayuda-a-comprender-cuxf3mo-utilizar-tuplas-cuando-necesitan-almacenar-datos-que-no-deben-cambiar.}}

\hypertarget{tuplas-2}{%
\section{Tuplas}\label{tuplas-2}}

Las tuplas son estructuras de datos similares a las listas, pero son
inmutables, lo que significa que no se pueden modificar después de ser
creadas. Aprenderemos cómo trabajar con tuplas en Python.

\hypertarget{conceptos-clave-21}{%
\section{Conceptos Clave:}\label{conceptos-clave-21}}

\hypertarget{tuplas-3}{%
\subsection{Tuplas}\label{tuplas-3}}

Secuencias ordenadas de elementos que, a diferencia de las listas, no se
pueden modificar.

\hypertarget{inmutabilidad-1}{%
\subsection{Inmutabilidad}\label{inmutabilidad-1}}

Una vez creada una tupla, no se pueden agregar, modificar o eliminar
elementos.

\hypertarget{acceso-a-elementos-3}{%
\subsection{Acceso a Elementos}\label{acceso-a-elementos-3}}

Se utiliza el índice para acceder a un elemento específico de la tupla.

\hypertarget{ejemplo-21}{%
\section{Ejemplo}\label{ejemplo-21}}

\begin{Shaded}
\begin{Highlighting}[]
\NormalTok{coordenadas }\OperatorTok{=}\NormalTok{ (}\DecValTok{3}\NormalTok{, }\DecValTok{5}\NormalTok{)}
\NormalTok{x }\OperatorTok{=}\NormalTok{ coordenadas[}\DecValTok{0}\NormalTok{]}
\NormalTok{y }\OperatorTok{=}\NormalTok{ coordenadas[}\DecValTok{1}\NormalTok{]}
\end{Highlighting}
\end{Shaded}

\hypertarget{explicaciuxf3n-21}{%
\section{Explicación}\label{explicaciuxf3n-21}}

En este ejemplo, se crea una tupla que almacena coordenadas (x, y) y se
accede a los valores individuales utilizando índices.

\begin{tcolorbox}[enhanced jigsaw, colbacktitle=quarto-callout-important-color!10!white, toprule=.15mm, leftrule=.75mm, titlerule=0mm, opacityback=0, rightrule=.15mm, opacitybacktitle=0.6, breakable, left=2mm, coltitle=black, title=\textcolor{quarto-callout-important-color}{\faExclamation}\hspace{0.5em}{Actividad Práctica:}, toptitle=1mm, bottomtitle=1mm, arc=.35mm, bottomrule=.15mm, colback=white, colframe=quarto-callout-important-color-frame]

Crea una tupla con las estaciones del año.

Intenta modificar un elemento de la tupla y observa el error que se
produce.

\end{tcolorbox}

\hypertarget{explicaciuxf3n-de-la-actividad-19}{%
\section{Explicación de la
Actividad:}\label{explicaciuxf3n-de-la-actividad-19}}

Esta actividad permite a los participantes practicar la creación de
tuplas y comprender la diferencia entre listas y tuplas en términos de
inmutabilidad. Les ayuda a comprender cómo utilizar tuplas cuando
necesitan almacenar datos que no deben cambiar.
\textgreater\textgreater\textgreater\textgreater\textgreater\textgreater\textgreater{}
e8ed08b1a5bbe1e369719187cfc4de7f7e2a41a9

\hypertarget{range}{%
\chapter{Range}\label{range}}

\textless\textless\textless\textless\textless\textless\textless{} HEAD

El tipo de dato range se utiliza para generar secuencias de números.
Aprenderemos cómo utilizar range en Python para crear secuencias de
números en rangos específicos.

\hypertarget{conceptos-clave-22}{%
\section{Conceptos Clave:}\label{conceptos-clave-22}}

\hypertarget{range-1}{%
\subsection{range}\label{range-1}}

Tipo de dato utilizado para generar secuencias de números en un rango.

\hypertarget{paruxe1metros-de-range}{%
\subsection{Parámetros de range}\label{paruxe1metros-de-range}}

Se pueden especificar el valor inicial, valor final y paso de la
secuencia.

\hypertarget{conversiuxf3n-a-listas}{%
\subsection{Conversión a Listas}\label{conversiuxf3n-a-listas}}

Es posible convertir un objeto range en una lista utilizando la función
list().

\hypertarget{ejemplo-22}{%
\section{Ejemplo}\label{ejemplo-22}}

\begin{Shaded}
\begin{Highlighting}[]
\CommentTok{\# Generación de secuencias de números}
\NormalTok{secuencia1 }\OperatorTok{=} \BuiltInTok{range}\NormalTok{(}\DecValTok{5}\NormalTok{)          }\CommentTok{\# 0, 1, 2, 3, 4}
\NormalTok{secuencia2 }\OperatorTok{=} \BuiltInTok{range}\NormalTok{(}\DecValTok{2}\NormalTok{, }\DecValTok{10}\NormalTok{)      }\CommentTok{\# 2, 3, 4, 5, 6, 7, 8, 9}
\NormalTok{secuencia3 }\OperatorTok{=} \BuiltInTok{range}\NormalTok{(}\DecValTok{1}\NormalTok{, }\DecValTok{11}\NormalTok{, }\DecValTok{2}\NormalTok{)   }\CommentTok{\# 1, 3, 5, 7, 9}

\CommentTok{\# Conversión a lista}
\NormalTok{lista\_secuencia1 }\OperatorTok{=} \BuiltInTok{list}\NormalTok{(secuencia1)}
\end{Highlighting}
\end{Shaded}

\hypertarget{explicaciuxf3n-22}{%
\section{Explicación}\label{explicaciuxf3n-22}}

En este ejemplo, se utilizan diferentes valores para crear secuencias de
números utilizando el tipo de dato range.

La función list() se utiliza para convertir una secuencia de range en
una lista.

\begin{tcolorbox}[enhanced jigsaw, colbacktitle=quarto-callout-important-color!10!white, toprule=.15mm, leftrule=.75mm, titlerule=0mm, opacityback=0, rightrule=.15mm, opacitybacktitle=0.6, breakable, left=2mm, coltitle=black, title=\textcolor{quarto-callout-important-color}{\faExclamation}\hspace{0.5em}{Actividad Práctica:}, toptitle=1mm, bottomtitle=1mm, arc=.35mm, bottomrule=.15mm, colback=white, colframe=quarto-callout-important-color-frame]

Crea una secuencia de números del 10 al 20 con un paso de 2.

Convierte la secuencia de números en una lista y muestra los elementos
en la consola.

\end{tcolorbox}

\hypertarget{explicaciuxf3n-de-la-actividad-20}{%
\section{Explicación de la
Actividad:}\label{explicaciuxf3n-de-la-actividad-20}}

Esta actividad permite a los participantes practicar la creación de
secuencias de números utilizando range y cómo convertirlas en listas
para trabajar con los elementos individualmente. Les ayuda a comprender
cómo generar secuencias de números en diferentes rangos.

======= \#\# Range

El tipo de dato range se utiliza para generar secuencias de números.
Aprenderemos cómo utilizar range en Python para crear secuencias de
números en rangos específicos.

\hypertarget{conceptos-clave-23}{%
\section{Conceptos Clave:}\label{conceptos-clave-23}}

\hypertarget{range-2}{%
\subsection{range}\label{range-2}}

Tipo de dato utilizado para generar secuencias de números en un rango.

\hypertarget{paruxe1metros-de-range-1}{%
\subsection{Parámetros de range}\label{paruxe1metros-de-range-1}}

Se pueden especificar el valor inicial, valor final y paso de la
secuencia.

\hypertarget{conversiuxf3n-a-listas-1}{%
\subsection{Conversión a Listas}\label{conversiuxf3n-a-listas-1}}

Es posible convertir un objeto range en una lista utilizando la función
list().

\hypertarget{ejemplo-23}{%
\section{Ejemplo}\label{ejemplo-23}}

\begin{Shaded}
\begin{Highlighting}[]
\CommentTok{\# Generación de secuencias de números}
\NormalTok{secuencia1 }\OperatorTok{=} \BuiltInTok{range}\NormalTok{(}\DecValTok{5}\NormalTok{)          }\CommentTok{\# 0, 1, 2, 3, 4}
\NormalTok{secuencia2 }\OperatorTok{=} \BuiltInTok{range}\NormalTok{(}\DecValTok{2}\NormalTok{, }\DecValTok{10}\NormalTok{)      }\CommentTok{\# 2, 3, 4, 5, 6, 7, 8, 9}
\NormalTok{secuencia3 }\OperatorTok{=} \BuiltInTok{range}\NormalTok{(}\DecValTok{1}\NormalTok{, }\DecValTok{11}\NormalTok{, }\DecValTok{2}\NormalTok{)   }\CommentTok{\# 1, 3, 5, 7, 9}

\CommentTok{\# Conversión a lista}
\NormalTok{lista\_secuencia1 }\OperatorTok{=} \BuiltInTok{list}\NormalTok{(secuencia1)}
\end{Highlighting}
\end{Shaded}

\hypertarget{explicaciuxf3n-23}{%
\section{Explicación}\label{explicaciuxf3n-23}}

En este ejemplo, se utilizan diferentes valores para crear secuencias de
números utilizando el tipo de dato range.

La función list() se utiliza para convertir una secuencia de range en
una lista.

\begin{tcolorbox}[enhanced jigsaw, colbacktitle=quarto-callout-important-color!10!white, toprule=.15mm, leftrule=.75mm, titlerule=0mm, opacityback=0, rightrule=.15mm, opacitybacktitle=0.6, breakable, left=2mm, coltitle=black, title=\textcolor{quarto-callout-important-color}{\faExclamation}\hspace{0.5em}{Actividad Práctica:}, toptitle=1mm, bottomtitle=1mm, arc=.35mm, bottomrule=.15mm, colback=white, colframe=quarto-callout-important-color-frame]

Crea una secuencia de números del 10 al 20 con un paso de 2.

Convierte la secuencia de números en una lista y muestra los elementos
en la consola.

\end{tcolorbox}

\hypertarget{explicaciuxf3n-de-la-actividad-21}{%
\section{Explicación de la
Actividad:}\label{explicaciuxf3n-de-la-actividad-21}}

Esta actividad permite a los participantes practicar la creación de
secuencias de números utilizando range y cómo convertirlas en listas
para trabajar con los elementos individualmente. Les ayuda a comprender
cómo generar secuencias de números en diferentes rangos.

\begin{quote}
\begin{quote}
\begin{quote}
\begin{quote}
\begin{quote}
\begin{quote}
\begin{quote}
e8ed08b1a5bbe1e369719187cfc4de7f7e2a41a9
\end{quote}
\end{quote}
\end{quote}
\end{quote}
\end{quote}
\end{quote}
\end{quote}

\hypertarget{diccionarios}{%
\chapter{Diccionarios}\label{diccionarios}}

\textless\textless\textless\textless\textless\textless\textless{} HEAD

Los diccionarios son estructuras de datos que permiten almacenar pares
clave-valor. Aprenderemos cómo crear y trabajar con diccionarios en
Python.

\hypertarget{conceptos-clave-24}{%
\section{Conceptos Clave:}\label{conceptos-clave-24}}

\hypertarget{diccionarios-1}{%
\subsection{Diccionarios}\label{diccionarios-1}}

Estructuras de datos que almacenan pares clave-valor.

\hypertarget{claves}{%
\subsection{Claves}\label{claves}}

Son los nombres o etiquetas utilizados para acceder a los valores en el
diccionario.

\hypertarget{valores}{%
\subsection{Valores}\label{valores}}

Son los datos asociados a cada clave en el diccionario.

\hypertarget{ejemplo-24}{%
\section{Ejemplo}\label{ejemplo-24}}

\begin{Shaded}
\begin{Highlighting}[]
\CommentTok{\# Creación de un diccionario}
\NormalTok{persona }\OperatorTok{=}\NormalTok{ \{}
    \StringTok{"nombre"}\NormalTok{: }\StringTok{"Juan"}\NormalTok{,}
    \StringTok{"edad"}\NormalTok{: }\DecValTok{30}\NormalTok{,}
    \StringTok{"ciudad"}\NormalTok{: }\StringTok{"México"}
\NormalTok{\}}

\CommentTok{\# Acceso a valores utilizando claves}
\NormalTok{nombre }\OperatorTok{=}\NormalTok{ persona[}\StringTok{"nombre"}\NormalTok{]}
\NormalTok{edad }\OperatorTok{=}\NormalTok{ persona[}\StringTok{"edad"}\NormalTok{]}
\end{Highlighting}
\end{Shaded}

\hypertarget{explicaciuxf3n-24}{%
\section{Explicación}\label{explicaciuxf3n-24}}

En este ejemplo, se crea un diccionario que almacena información de una
persona, como nombre, edad y ciudad.

Se accede a los valores del diccionario utilizando las claves
correspondientes.

\begin{tcolorbox}[enhanced jigsaw, colbacktitle=quarto-callout-important-color!10!white, toprule=.15mm, leftrule=.75mm, titlerule=0mm, opacityback=0, rightrule=.15mm, opacitybacktitle=0.6, breakable, left=2mm, coltitle=black, title=\textcolor{quarto-callout-important-color}{\faExclamation}\hspace{0.5em}{Actividad Práctica}, toptitle=1mm, bottomtitle=1mm, arc=.35mm, bottomrule=.15mm, colback=white, colframe=quarto-callout-important-color-frame]

Crea un diccionario que almacene información de tus libros favoritos,
incluyendo título y autor.

Accede a los valores del diccionario utilizando las claves y muestra la
información en la consola.

\end{tcolorbox}

\hypertarget{explicaciuxf3n-de-la-actividad-22}{%
\section{Explicación de la
Actividad:}\label{explicaciuxf3n-de-la-actividad-22}}

======= \#\# Diccionarios

Los diccionarios son estructuras de datos que permiten almacenar pares
clave-valor. Aprenderemos cómo crear y trabajar con diccionarios en
Python.

\hypertarget{conceptos-clave-25}{%
\section{Conceptos Clave:}\label{conceptos-clave-25}}

\hypertarget{diccionarios-2}{%
\subsection{Diccionarios}\label{diccionarios-2}}

Estructuras de datos que almacenan pares clave-valor.

\hypertarget{claves-1}{%
\subsection{Claves}\label{claves-1}}

Son los nombres o etiquetas utilizados para acceder a los valores en el
diccionario.

\hypertarget{valores-1}{%
\subsection{Valores}\label{valores-1}}

Son los datos asociados a cada clave en el diccionario.

\hypertarget{ejemplo-25}{%
\section{Ejemplo}\label{ejemplo-25}}

\begin{Shaded}
\begin{Highlighting}[]
\CommentTok{\# Creación de un diccionario}
\NormalTok{persona }\OperatorTok{=}\NormalTok{ \{}
    \StringTok{"nombre"}\NormalTok{: }\StringTok{"Juan"}\NormalTok{,}
    \StringTok{"edad"}\NormalTok{: }\DecValTok{30}\NormalTok{,}
    \StringTok{"ciudad"}\NormalTok{: }\StringTok{"México"}
\NormalTok{\}}

\CommentTok{\# Acceso a valores utilizando claves}
\NormalTok{nombre }\OperatorTok{=}\NormalTok{ persona[}\StringTok{"nombre"}\NormalTok{]}
\NormalTok{edad }\OperatorTok{=}\NormalTok{ persona[}\StringTok{"edad"}\NormalTok{]}
\end{Highlighting}
\end{Shaded}

\hypertarget{explicaciuxf3n-25}{%
\section{Explicación}\label{explicaciuxf3n-25}}

En este ejemplo, se crea un diccionario que almacena información de una
persona, como nombre, edad y ciudad.

Se accede a los valores del diccionario utilizando las claves
correspondientes.

\begin{tcolorbox}[enhanced jigsaw, colbacktitle=quarto-callout-important-color!10!white, toprule=.15mm, leftrule=.75mm, titlerule=0mm, opacityback=0, rightrule=.15mm, opacitybacktitle=0.6, breakable, left=2mm, coltitle=black, title=\textcolor{quarto-callout-important-color}{\faExclamation}\hspace{0.5em}{Actividad Práctica}, toptitle=1mm, bottomtitle=1mm, arc=.35mm, bottomrule=.15mm, colback=white, colframe=quarto-callout-important-color-frame]

Crea un diccionario que almacene información de tus libros favoritos,
incluyendo título y autor.

Accede a los valores del diccionario utilizando las claves y muestra la
información en la consola.

\end{tcolorbox}

\hypertarget{explicaciuxf3n-de-la-actividad-23}{%
\section{Explicación de la
Actividad:}\label{explicaciuxf3n-de-la-actividad-23}}

\begin{quote}
\begin{quote}
\begin{quote}
\begin{quote}
\begin{quote}
\begin{quote}
\begin{quote}
e8ed08b1a5bbe1e369719187cfc4de7f7e2a41a9 Esta actividad permite a los
participantes practicar la creación de diccionarios y acceder a los
valores utilizando las claves. Les ayuda a comprender cómo organizar
datos en pares clave-valor y cómo acceder a la información de manera
eficiente.
\end{quote}
\end{quote}
\end{quote}
\end{quote}
\end{quote}
\end{quote}
\end{quote}

\hypertarget{booleanos}{%
\chapter{Booleanos}\label{booleanos}}

\textless\textless\textless\textless\textless\textless\textless{} HEAD

Los booleanos son un tipo de dato que puede tener dos valores: True
(verdadero) o False (falso). Aprenderemos cómo trabajar con booleanos en
Python y cómo utilizarlos en expresiones lógicas.

\hypertarget{conceptos-clave-26}{%
\section{Conceptos Clave:}\label{conceptos-clave-26}}

\hypertarget{booleanos-1}{%
\subsection{Booleanos}\label{booleanos-1}}

Tipo de dato que representa valores de verdad (True o False).

Expresiones Lógicas: Combinaciones de valores booleanos utilizando
operadores lógicos como and, or y not.

\hypertarget{ejemplo-26}{%
\section{Ejemplo}\label{ejemplo-26}}

\begin{Shaded}
\begin{Highlighting}[]
\CommentTok{\# Variables booleanas}
\NormalTok{es\_mayor\_de\_edad }\OperatorTok{=} \VariableTok{True}
\NormalTok{tiene\_tarjeta }\OperatorTok{=} \VariableTok{False}

\CommentTok{\# Expresiones lógicas}
\NormalTok{puede\_ingresar }\OperatorTok{=}\NormalTok{ es\_mayor\_de\_edad }\KeywordTok{and}\NormalTok{ tiene\_tarjeta}
\end{Highlighting}
\end{Shaded}

\hypertarget{explicaciuxf3n-26}{%
\section{Explicación}\label{explicaciuxf3n-26}}

En este ejemplo, se utilizan variables booleanas para representar si
alguien es mayor de edad y si tiene una tarjeta.

Se utiliza una expresión lógica para evaluar si alguien puede ingresar
basado en ambas condiciones.

\begin{tcolorbox}[enhanced jigsaw, colbacktitle=quarto-callout-important-color!10!white, toprule=.15mm, leftrule=.75mm, titlerule=0mm, opacityback=0, rightrule=.15mm, opacitybacktitle=0.6, breakable, left=2mm, coltitle=black, title=\textcolor{quarto-callout-important-color}{\faExclamation}\hspace{0.5em}{Actividad Práctica:}, toptitle=1mm, bottomtitle=1mm, arc=.35mm, bottomrule=.15mm, colback=white, colframe=quarto-callout-important-color-frame]

Crea variables booleanas que representen si tienes una mascota y si te
gusta el deporte.

Utiliza expresiones lógicas para determinar si puedes llevar a tu
mascota a un lugar que requiere tu atención durante un partido de tu
deporte favorito.

\end{tcolorbox}

\hypertarget{explicaciuxf3n-de-la-actividad-24}{%
\section{Explicación de la
Actividad:}\label{explicaciuxf3n-de-la-actividad-24}}

======= \#\# Booleanos

Los booleanos son un tipo de dato que puede tener dos valores: True
(verdadero) o False (falso). Aprenderemos cómo trabajar con booleanos en
Python y cómo utilizarlos en expresiones lógicas.

\hypertarget{conceptos-clave-27}{%
\section{Conceptos Clave:}\label{conceptos-clave-27}}

\hypertarget{booleanos-2}{%
\subsection{Booleanos}\label{booleanos-2}}

Tipo de dato que representa valores de verdad (True o False).

Expresiones Lógicas: Combinaciones de valores booleanos utilizando
operadores lógicos como and, or y not.

\hypertarget{ejemplo-27}{%
\section{Ejemplo}\label{ejemplo-27}}

\begin{Shaded}
\begin{Highlighting}[]
\CommentTok{\# Variables booleanas}
\NormalTok{es\_mayor\_de\_edad }\OperatorTok{=} \VariableTok{True}
\NormalTok{tiene\_tarjeta }\OperatorTok{=} \VariableTok{False}

\CommentTok{\# Expresiones lógicas}
\NormalTok{puede\_ingresar }\OperatorTok{=}\NormalTok{ es\_mayor\_de\_edad }\KeywordTok{and}\NormalTok{ tiene\_tarjeta}
\end{Highlighting}
\end{Shaded}

\hypertarget{explicaciuxf3n-27}{%
\section{Explicación}\label{explicaciuxf3n-27}}

En este ejemplo, se utilizan variables booleanas para representar si
alguien es mayor de edad y si tiene una tarjeta.

Se utiliza una expresión lógica para evaluar si alguien puede ingresar
basado en ambas condiciones.

\begin{tcolorbox}[enhanced jigsaw, colbacktitle=quarto-callout-important-color!10!white, toprule=.15mm, leftrule=.75mm, titlerule=0mm, opacityback=0, rightrule=.15mm, opacitybacktitle=0.6, breakable, left=2mm, coltitle=black, title=\textcolor{quarto-callout-important-color}{\faExclamation}\hspace{0.5em}{Actividad Práctica:}, toptitle=1mm, bottomtitle=1mm, arc=.35mm, bottomrule=.15mm, colback=white, colframe=quarto-callout-important-color-frame]

Crea variables booleanas que representen si tienes una mascota y si te
gusta el deporte.

Utiliza expresiones lógicas para determinar si puedes llevar a tu
mascota a un lugar que requiere tu atención durante un partido de tu
deporte favorito.

\end{tcolorbox}

\hypertarget{explicaciuxf3n-de-la-actividad-25}{%
\section{Explicación de la
Actividad:}\label{explicaciuxf3n-de-la-actividad-25}}

\begin{quote}
\begin{quote}
\begin{quote}
\begin{quote}
\begin{quote}
\begin{quote}
\begin{quote}
e8ed08b1a5bbe1e369719187cfc4de7f7e2a41a9 Esta actividad permite a los
participantes practicar el uso de variables booleanas y expresiones
lógicas para tomar decisiones basadas en condiciones booleanas. Les
ayuda a comprender cómo trabajar con valores de verdad y cómo
utilizarlos para evaluar situaciones en el código.
\end{quote}
\end{quote}
\end{quote}
\end{quote}
\end{quote}
\end{quote}
\end{quote}

\part{Unidad 5: Control de Flujo}

\hypertarget{introducciuxf3n-a-if}{%
\chapter{Introducción a If}\label{introducciuxf3n-a-if}}

\textless\textless\textless\textless\textless\textless\textless{} HEAD

El control de flujo es fundamental en la programación para tomar
decisiones basadas en condiciones. Aprenderemos cómo utilizar la
estructura if para ejecutar diferentes bloques de código según una
condición.

\hypertarget{conceptos-clave-28}{%
\section{Conceptos Clave:}\label{conceptos-clave-28}}

\hypertarget{control-de-flujo}{%
\subsection{Control de Flujo}\label{control-de-flujo}}

Manejo de la ejecución del código basado en condiciones.

\hypertarget{estructura-if}{%
\subsection{Estructura if}\label{estructura-if}}

Permite ejecutar un bloque de código si una condición es verdadera.

\hypertarget{bloque-de-cuxf3digo}{%
\subsection{Bloque de Código}\label{bloque-de-cuxf3digo}}

Conjunto de instrucciones que se ejecutan si la condición es verdadera.

\hypertarget{ejemplo-28}{%
\section{Ejemplo:}\label{ejemplo-28}}

\begin{Shaded}
\begin{Highlighting}[]
\NormalTok{edad }\OperatorTok{=} \DecValTok{18}

\ControlFlowTok{if}\NormalTok{ edad }\OperatorTok{\textgreater{}=} \DecValTok{18}\NormalTok{:}
    \BuiltInTok{print}\NormalTok{(}\StringTok{"Eres mayor de edad."}\NormalTok{)}
\end{Highlighting}
\end{Shaded}

\hypertarget{explicaciuxf3n-28}{%
\section{Explicación:}\label{explicaciuxf3n-28}}

En este ejemplo, se utiliza la estructura if para verificar si la
variable ``edad'' es mayor o igual a 18.

Si la condición es verdadera, se ejecuta el bloque de código que muestra
un mensaje.

\begin{tcolorbox}[enhanced jigsaw, colbacktitle=quarto-callout-important-color!10!white, toprule=.15mm, leftrule=.75mm, titlerule=0mm, opacityback=0, rightrule=.15mm, opacitybacktitle=0.6, breakable, left=2mm, coltitle=black, title=\textcolor{quarto-callout-important-color}{\faExclamation}\hspace{0.5em}{Actividad Práctica:}, toptitle=1mm, bottomtitle=1mm, arc=.35mm, bottomrule=.15mm, colback=white, colframe=quarto-callout-important-color-frame]

Crea una variable que represente tu puntuación en un juego.

Utiliza una estructura if para mostrar un mensaje diferente según si tu
puntuación es mayor o igual a 100.

\end{tcolorbox}

\hypertarget{explicaciuxf3n-de-la-actividad-26}{%
\section{Explicación de la
Actividad:}\label{explicaciuxf3n-de-la-actividad-26}}

======= \#\# Introducción a If

El control de flujo es fundamental en la programación para tomar
decisiones basadas en condiciones. Aprenderemos cómo utilizar la
estructura if para ejecutar diferentes bloques de código según una
condición.

\hypertarget{conceptos-clave-29}{%
\section{Conceptos Clave:}\label{conceptos-clave-29}}

\hypertarget{control-de-flujo-1}{%
\subsection{Control de Flujo}\label{control-de-flujo-1}}

Manejo de la ejecución del código basado en condiciones.

\hypertarget{estructura-if-1}{%
\subsection{Estructura if}\label{estructura-if-1}}

Permite ejecutar un bloque de código si una condición es verdadera.

\hypertarget{bloque-de-cuxf3digo-1}{%
\subsection{Bloque de Código}\label{bloque-de-cuxf3digo-1}}

Conjunto de instrucciones que se ejecutan si la condición es verdadera.

\hypertarget{ejemplo-29}{%
\section{Ejemplo:}\label{ejemplo-29}}

\begin{Shaded}
\begin{Highlighting}[]
\NormalTok{edad }\OperatorTok{=} \DecValTok{18}

\ControlFlowTok{if}\NormalTok{ edad }\OperatorTok{\textgreater{}=} \DecValTok{18}\NormalTok{:}
    \BuiltInTok{print}\NormalTok{(}\StringTok{"Eres mayor de edad."}\NormalTok{)}
\end{Highlighting}
\end{Shaded}

\hypertarget{explicaciuxf3n-29}{%
\section{Explicación:}\label{explicaciuxf3n-29}}

En este ejemplo, se utiliza la estructura if para verificar si la
variable ``edad'' es mayor o igual a 18.

Si la condición es verdadera, se ejecuta el bloque de código que muestra
un mensaje.

\begin{tcolorbox}[enhanced jigsaw, colbacktitle=quarto-callout-important-color!10!white, toprule=.15mm, leftrule=.75mm, titlerule=0mm, opacityback=0, rightrule=.15mm, opacitybacktitle=0.6, breakable, left=2mm, coltitle=black, title=\textcolor{quarto-callout-important-color}{\faExclamation}\hspace{0.5em}{Actividad Práctica:}, toptitle=1mm, bottomtitle=1mm, arc=.35mm, bottomrule=.15mm, colback=white, colframe=quarto-callout-important-color-frame]

Crea una variable que represente tu puntuación en un juego.

Utiliza una estructura if para mostrar un mensaje diferente según si tu
puntuación es mayor o igual a 100.

\end{tcolorbox}

\hypertarget{explicaciuxf3n-de-la-actividad-27}{%
\section{Explicación de la
Actividad:}\label{explicaciuxf3n-de-la-actividad-27}}

\begin{quote}
\begin{quote}
\begin{quote}
\begin{quote}
\begin{quote}
\begin{quote}
\begin{quote}
e8ed08b1a5bbe1e369719187cfc4de7f7e2a41a9 Esta actividad permite a los
participantes practicar la utilización de la estructura if para tomar
decisiones basadas en condiciones. Les ayuda a comprender cómo ejecutar
diferentes bloques de código según la situación y cómo utilizar el
control de flujo en sus programas.
\end{quote}
\end{quote}
\end{quote}
\end{quote}
\end{quote}
\end{quote}
\end{quote}

\hypertarget{if-y-condicionales}{%
\chapter{If y Condicionales}\label{if-y-condicionales}}

\textless\textless\textless\textless\textless\textless\textless{} HEAD

En esta lección, aprenderemos cómo trabajar con múltiples condiciones
utilizando la estructura if, elif y else. Esto permite ejecutar
diferentes bloques de código según diferentes condiciones.

\hypertarget{conceptos-clave-30}{%
\section{Conceptos Clave}\label{conceptos-clave-30}}

\hypertarget{estructura-elif}{%
\subsection{Estructura elif}\label{estructura-elif}}

Permite verificar una condición adicional si la condición anterior es
falsa.

\hypertarget{estructura-else}{%
\subsection{Estructura else}\label{estructura-else}}

Define un bloque de código que se ejecuta si todas las condiciones
anteriores son falsas.

\hypertarget{anidaciuxf3n-de-estructuras-if}{%
\subsection{Anidación de Estructuras
if}\label{anidaciuxf3n-de-estructuras-if}}

Es posible anidar múltiples estructuras if para manejar situaciones más
complejas.

\hypertarget{ejemplo-30}{%
\section{Ejemplo:}\label{ejemplo-30}}

\begin{Shaded}
\begin{Highlighting}[]
\NormalTok{puntaje }\OperatorTok{=} \DecValTok{85}

\ControlFlowTok{if}\NormalTok{ puntaje }\OperatorTok{\textgreater{}=} \DecValTok{90}\NormalTok{:}
    \BuiltInTok{print}\NormalTok{(}\StringTok{"¡Excelente trabajo!"}\NormalTok{)}
\ControlFlowTok{elif}\NormalTok{ puntaje }\OperatorTok{\textgreater{}=} \DecValTok{70}\NormalTok{:}
    \BuiltInTok{print}\NormalTok{(}\StringTok{"Buen trabajo."}\NormalTok{)}
\ControlFlowTok{else}\NormalTok{:}
    \BuiltInTok{print}\NormalTok{(}\StringTok{"Necesitas mejorar."}\NormalTok{)}
\end{Highlighting}
\end{Shaded}

\hypertarget{explicaciuxf3n-30}{%
\section{Explicación:}\label{explicaciuxf3n-30}}

En este ejemplo, se utiliza la estructura if, elif y else para evaluar
diferentes rangos de puntajes y mostrar mensajes correspondientes.

\begin{tcolorbox}[enhanced jigsaw, colbacktitle=quarto-callout-important-color!10!white, toprule=.15mm, leftrule=.75mm, titlerule=0mm, opacityback=0, rightrule=.15mm, opacitybacktitle=0.6, breakable, left=2mm, coltitle=black, title=\textcolor{quarto-callout-important-color}{\faExclamation}\hspace{0.5em}{Actividad Práctica:}, toptitle=1mm, bottomtitle=1mm, arc=.35mm, bottomrule=.15mm, colback=white, colframe=quarto-callout-important-color-frame]

Crea una variable que represente tu calificación en un examen.

Utiliza una estructura if, elif y else para mostrar mensajes diferentes
según la calificación obtenida.

\end{tcolorbox}

\hypertarget{explicaciuxf3n-de-la-actividad-28}{%
\section{Explicación de la
Actividad:}\label{explicaciuxf3n-de-la-actividad-28}}

Esta actividad permite a los participantes practicar el uso de la
estructura if, elif y else para manejar múltiples condiciones y
decisiones en sus programas. Les ayuda a comprender cómo ejecutar
diferentes bloques de código en función de los resultados de las
pruebas.

======= \#\# If y Condicionales

En esta lección, aprenderemos cómo trabajar con múltiples condiciones
utilizando la estructura if, elif y else. Esto permite ejecutar
diferentes bloques de código según diferentes condiciones.

\hypertarget{conceptos-clave-31}{%
\section{Conceptos Clave}\label{conceptos-clave-31}}

\hypertarget{estructura-elif-1}{%
\subsection{Estructura elif}\label{estructura-elif-1}}

Permite verificar una condición adicional si la condición anterior es
falsa.

\hypertarget{estructura-else-1}{%
\subsection{Estructura else}\label{estructura-else-1}}

Define un bloque de código que se ejecuta si todas las condiciones
anteriores son falsas.

\hypertarget{anidaciuxf3n-de-estructuras-if-1}{%
\subsection{Anidación de Estructuras
if}\label{anidaciuxf3n-de-estructuras-if-1}}

Es posible anidar múltiples estructuras if para manejar situaciones más
complejas.

\hypertarget{ejemplo-31}{%
\section{Ejemplo:}\label{ejemplo-31}}

\begin{Shaded}
\begin{Highlighting}[]
\NormalTok{puntaje }\OperatorTok{=} \DecValTok{85}

\ControlFlowTok{if}\NormalTok{ puntaje }\OperatorTok{\textgreater{}=} \DecValTok{90}\NormalTok{:}
    \BuiltInTok{print}\NormalTok{(}\StringTok{"¡Excelente trabajo!"}\NormalTok{)}
\ControlFlowTok{elif}\NormalTok{ puntaje }\OperatorTok{\textgreater{}=} \DecValTok{70}\NormalTok{:}
    \BuiltInTok{print}\NormalTok{(}\StringTok{"Buen trabajo."}\NormalTok{)}
\ControlFlowTok{else}\NormalTok{:}
    \BuiltInTok{print}\NormalTok{(}\StringTok{"Necesitas mejorar."}\NormalTok{)}
\end{Highlighting}
\end{Shaded}

\hypertarget{explicaciuxf3n-31}{%
\section{Explicación:}\label{explicaciuxf3n-31}}

En este ejemplo, se utiliza la estructura if, elif y else para evaluar
diferentes rangos de puntajes y mostrar mensajes correspondientes.

\begin{tcolorbox}[enhanced jigsaw, colbacktitle=quarto-callout-important-color!10!white, toprule=.15mm, leftrule=.75mm, titlerule=0mm, opacityback=0, rightrule=.15mm, opacitybacktitle=0.6, breakable, left=2mm, coltitle=black, title=\textcolor{quarto-callout-important-color}{\faExclamation}\hspace{0.5em}{Actividad Práctica:}, toptitle=1mm, bottomtitle=1mm, arc=.35mm, bottomrule=.15mm, colback=white, colframe=quarto-callout-important-color-frame]

Crea una variable que represente tu calificación en un examen.

Utiliza una estructura if, elif y else para mostrar mensajes diferentes
según la calificación obtenida.

\end{tcolorbox}

\hypertarget{explicaciuxf3n-de-la-actividad-29}{%
\section{Explicación de la
Actividad:}\label{explicaciuxf3n-de-la-actividad-29}}

Esta actividad permite a los participantes practicar el uso de la
estructura if, elif y else para manejar múltiples condiciones y
decisiones en sus programas. Les ayuda a comprender cómo ejecutar
diferentes bloques de código en función de los resultados de las
pruebas.

\begin{quote}
\begin{quote}
\begin{quote}
\begin{quote}
\begin{quote}
\begin{quote}
\begin{quote}
e8ed08b1a5bbe1e369719187cfc4de7f7e2a41a9
\end{quote}
\end{quote}
\end{quote}
\end{quote}
\end{quote}
\end{quote}
\end{quote}

\hypertarget{if-elif-y-else}{%
\chapter{If, elif y else}\label{if-elif-y-else}}

\textless\textless\textless\textless\textless\textless\textless{} HEAD

En esta lección, continuaremos trabajando con la estructura if, elif y
else, pero esta vez exploraremos cómo anidar estas estructuras para
manejar situaciones más complejas.

\hypertarget{conceptos-clave-32}{%
\section{Conceptos Clave:}\label{conceptos-clave-32}}

\hypertarget{anidaciuxf3n-de-estructuras}{%
\subsection{Anidación de
Estructuras}\label{anidaciuxf3n-de-estructuras}}

Posibilidad de colocar una estructura if, elif y else dentro de otra.

\hypertarget{jerarquuxeda-de-condiciones}{%
\subsection{Jerarquía de
Condiciones}\label{jerarquuxeda-de-condiciones}}

Se evalúan las condiciones de manera secuencial y se ejecuta el primer
bloque de código correspondiente a una condición verdadera.

\hypertarget{ejemplo-32}{%
\section{Ejemplo}\label{ejemplo-32}}

\begin{Shaded}
\begin{Highlighting}[]
\NormalTok{edad }\OperatorTok{=} \DecValTok{16}
\NormalTok{permiso\_padres }\OperatorTok{=} \VariableTok{True}

\ControlFlowTok{if}\NormalTok{ edad }\OperatorTok{\textgreater{}=} \DecValTok{18}\NormalTok{:}
    \BuiltInTok{print}\NormalTok{(}\StringTok{"Eres mayor de edad."}\NormalTok{)}
\ControlFlowTok{else}\NormalTok{:}
    \ControlFlowTok{if}\NormalTok{ permiso\_padres:}
        \BuiltInTok{print}\NormalTok{(}\StringTok{"Eres menor de edad pero tienes permiso de tus padres."}\NormalTok{)}
    \ControlFlowTok{else}\NormalTok{:}
        \BuiltInTok{print}\NormalTok{(}\StringTok{"Eres menor de edad y no tienes permiso de tus padres."}\NormalTok{)}
\end{Highlighting}
\end{Shaded}

\hypertarget{explicaciuxf3n-32}{%
\section{Explicación:}\label{explicaciuxf3n-32}}

En este ejemplo, se anidan estructuras if para manejar diferentes casos
basados en la edad y el permiso de los padres.

\begin{tcolorbox}[enhanced jigsaw, colbacktitle=quarto-callout-important-color!10!white, toprule=.15mm, leftrule=.75mm, titlerule=0mm, opacityback=0, rightrule=.15mm, opacitybacktitle=0.6, breakable, left=2mm, coltitle=black, title=\textcolor{quarto-callout-important-color}{\faExclamation}\hspace{0.5em}{Actividad Práctica:}, toptitle=1mm, bottomtitle=1mm, arc=.35mm, bottomrule=.15mm, colback=white, colframe=quarto-callout-important-color-frame]

Crea una variable que represente si un usuario está registrado en un
sitio web.

Si el usuario está registrado, muestra un mensaje de bienvenida. Si no
está registrado, muestra un mensaje que lo invite a registrarse.

\end{tcolorbox}

\hypertarget{explicaciuxf3n-de-la-actividad-30}{%
\section{Explicación de la
Actividad:}\label{explicaciuxf3n-de-la-actividad-30}}

Esta actividad permite a los participantes practicar la anidación de
estructuras if, elif y else para manejar situaciones con múltiples
condiciones. Les ayuda a comprender cómo trabajar con jerarquías de
condiciones y cómo manejar casos más complejos en sus programas.

======= \#\# If, elif y else

En esta lección, continuaremos trabajando con la estructura if, elif y
else, pero esta vez exploraremos cómo anidar estas estructuras para
manejar situaciones más complejas.

\hypertarget{conceptos-clave-33}{%
\section{Conceptos Clave:}\label{conceptos-clave-33}}

\hypertarget{anidaciuxf3n-de-estructuras-1}{%
\subsection{Anidación de
Estructuras}\label{anidaciuxf3n-de-estructuras-1}}

Posibilidad de colocar una estructura if, elif y else dentro de otra.

\hypertarget{jerarquuxeda-de-condiciones-1}{%
\subsection{Jerarquía de
Condiciones}\label{jerarquuxeda-de-condiciones-1}}

Se evalúan las condiciones de manera secuencial y se ejecuta el primer
bloque de código correspondiente a una condición verdadera.

\hypertarget{ejemplo-33}{%
\section{Ejemplo}\label{ejemplo-33}}

\begin{Shaded}
\begin{Highlighting}[]
\NormalTok{edad }\OperatorTok{=} \DecValTok{16}
\NormalTok{permiso\_padres }\OperatorTok{=} \VariableTok{True}

\ControlFlowTok{if}\NormalTok{ edad }\OperatorTok{\textgreater{}=} \DecValTok{18}\NormalTok{:}
    \BuiltInTok{print}\NormalTok{(}\StringTok{"Eres mayor de edad."}\NormalTok{)}
\ControlFlowTok{else}\NormalTok{:}
    \ControlFlowTok{if}\NormalTok{ permiso\_padres:}
        \BuiltInTok{print}\NormalTok{(}\StringTok{"Eres menor de edad pero tienes permiso de tus padres."}\NormalTok{)}
    \ControlFlowTok{else}\NormalTok{:}
        \BuiltInTok{print}\NormalTok{(}\StringTok{"Eres menor de edad y no tienes permiso de tus padres."}\NormalTok{)}
\end{Highlighting}
\end{Shaded}

\hypertarget{explicaciuxf3n-33}{%
\section{Explicación:}\label{explicaciuxf3n-33}}

En este ejemplo, se anidan estructuras if para manejar diferentes casos
basados en la edad y el permiso de los padres.

\begin{tcolorbox}[enhanced jigsaw, colbacktitle=quarto-callout-important-color!10!white, toprule=.15mm, leftrule=.75mm, titlerule=0mm, opacityback=0, rightrule=.15mm, opacitybacktitle=0.6, breakable, left=2mm, coltitle=black, title=\textcolor{quarto-callout-important-color}{\faExclamation}\hspace{0.5em}{Actividad Práctica:}, toptitle=1mm, bottomtitle=1mm, arc=.35mm, bottomrule=.15mm, colback=white, colframe=quarto-callout-important-color-frame]

Crea una variable que represente si un usuario está registrado en un
sitio web.

Si el usuario está registrado, muestra un mensaje de bienvenida. Si no
está registrado, muestra un mensaje que lo invite a registrarse.

\end{tcolorbox}

\hypertarget{explicaciuxf3n-de-la-actividad-31}{%
\section{Explicación de la
Actividad:}\label{explicaciuxf3n-de-la-actividad-31}}

Esta actividad permite a los participantes practicar la anidación de
estructuras if, elif y else para manejar situaciones con múltiples
condiciones. Les ayuda a comprender cómo trabajar con jerarquías de
condiciones y cómo manejar casos más complejos en sus programas.

\begin{quote}
\begin{quote}
\begin{quote}
\begin{quote}
\begin{quote}
\begin{quote}
\begin{quote}
e8ed08b1a5bbe1e369719187cfc4de7f7e2a41a9
\end{quote}
\end{quote}
\end{quote}
\end{quote}
\end{quote}
\end{quote}
\end{quote}

\hypertarget{and-y-or}{%
\chapter{And y Or}\label{and-y-or}}

\textless\textless\textless\textless\textless\textless\textless{} HEAD

En esta lección, exploraremos los operadores lógicos and y or, que
permiten combinar condiciones para crear expresiones más complejas en
las estructuras if, elif y else.

\hypertarget{conceptos-clave-34}{%
\section{Conceptos Clave:}\label{conceptos-clave-34}}

\hypertarget{operador-and}{%
\subsection{Operador and}\label{operador-and}}

Retorna True si ambas condiciones son verdaderas.

\hypertarget{operador-or}{%
\subsection{Operador or}\label{operador-or}}

Retorna True si al menos una de las condiciones es verdadera.

\hypertarget{combinaciuxf3n-de-condiciones}{%
\subsection{Combinación de
Condiciones}\label{combinaciuxf3n-de-condiciones}}

Los operadores and y or permiten combinar múltiples condiciones en una
sola expresión.

\hypertarget{ejemplo-34}{%
\section{Ejemplo:}\label{ejemplo-34}}

\begin{Shaded}
\begin{Highlighting}[]
\NormalTok{edad }\OperatorTok{=} \DecValTok{20}
\NormalTok{tiene\_permiso }\OperatorTok{=} \VariableTok{True}

\ControlFlowTok{if}\NormalTok{ edad }\OperatorTok{\textgreater{}=} \DecValTok{18} \KeywordTok{and}\NormalTok{ tiene\_permiso:}
    \BuiltInTok{print}\NormalTok{(}\StringTok{"Puedes ingresar."}\NormalTok{)}
\ControlFlowTok{else}\NormalTok{:}
    \BuiltInTok{print}\NormalTok{(}\StringTok{"No puedes ingresar."}\NormalTok{)}
\end{Highlighting}
\end{Shaded}

\hypertarget{explicaciuxf3n-34}{%
\section{Explicación:}\label{explicaciuxf3n-34}}

En este ejemplo, se utiliza el operador and para evaluar si la edad es
mayor o igual a 18 y si el usuario tiene permiso.

Si ambas condiciones son verdaderas, se permite el ingreso.

\begin{tcolorbox}[enhanced jigsaw, colbacktitle=quarto-callout-important-color!10!white, toprule=.15mm, leftrule=.75mm, titlerule=0mm, opacityback=0, rightrule=.15mm, opacitybacktitle=0.6, breakable, left=2mm, coltitle=black, title=\textcolor{quarto-callout-important-color}{\faExclamation}\hspace{0.5em}{Actividad Práctica:}, toptitle=1mm, bottomtitle=1mm, arc=.35mm, bottomrule=.15mm, colback=white, colframe=quarto-callout-important-color-frame]

Crea dos variables que representen si un usuario tiene una cuenta
premium y si su suscripción está activa.

Utiliza una estructura if y el operador and para determinar si el
usuario tiene acceso premium.

\end{tcolorbox}

\hypertarget{explicaciuxf3n-de-la-actividad-32}{%
\section{Explicación de la
Actividad:}\label{explicaciuxf3n-de-la-actividad-32}}

Esta actividad permite a los participantes practicar la combinación de
condiciones utilizando los operadores and y or. Les ayuda a comprender
cómo crear expresiones más complejas para tomar decisiones basadas en
múltiples condiciones en sus programas.

======= \#\# And y Or

En esta lección, exploraremos los operadores lógicos and y or, que
permiten combinar condiciones para crear expresiones más complejas en
las estructuras if, elif y else.

\hypertarget{conceptos-clave-35}{%
\section{Conceptos Clave:}\label{conceptos-clave-35}}

\hypertarget{operador-and-1}{%
\subsection{Operador and}\label{operador-and-1}}

Retorna True si ambas condiciones son verdaderas.

\hypertarget{operador-or-1}{%
\subsection{Operador or}\label{operador-or-1}}

Retorna True si al menos una de las condiciones es verdadera.

\hypertarget{combinaciuxf3n-de-condiciones-1}{%
\subsection{Combinación de
Condiciones}\label{combinaciuxf3n-de-condiciones-1}}

Los operadores and y or permiten combinar múltiples condiciones en una
sola expresión.

\hypertarget{ejemplo-35}{%
\section{Ejemplo:}\label{ejemplo-35}}

\begin{Shaded}
\begin{Highlighting}[]
\NormalTok{edad }\OperatorTok{=} \DecValTok{20}
\NormalTok{tiene\_permiso }\OperatorTok{=} \VariableTok{True}

\ControlFlowTok{if}\NormalTok{ edad }\OperatorTok{\textgreater{}=} \DecValTok{18} \KeywordTok{and}\NormalTok{ tiene\_permiso:}
    \BuiltInTok{print}\NormalTok{(}\StringTok{"Puedes ingresar."}\NormalTok{)}
\ControlFlowTok{else}\NormalTok{:}
    \BuiltInTok{print}\NormalTok{(}\StringTok{"No puedes ingresar."}\NormalTok{)}
\end{Highlighting}
\end{Shaded}

\hypertarget{explicaciuxf3n-35}{%
\section{Explicación:}\label{explicaciuxf3n-35}}

En este ejemplo, se utiliza el operador and para evaluar si la edad es
mayor o igual a 18 y si el usuario tiene permiso.

Si ambas condiciones son verdaderas, se permite el ingreso.

\begin{tcolorbox}[enhanced jigsaw, colbacktitle=quarto-callout-important-color!10!white, toprule=.15mm, leftrule=.75mm, titlerule=0mm, opacityback=0, rightrule=.15mm, opacitybacktitle=0.6, breakable, left=2mm, coltitle=black, title=\textcolor{quarto-callout-important-color}{\faExclamation}\hspace{0.5em}{Actividad Práctica:}, toptitle=1mm, bottomtitle=1mm, arc=.35mm, bottomrule=.15mm, colback=white, colframe=quarto-callout-important-color-frame]

Crea dos variables que representen si un usuario tiene una cuenta
premium y si su suscripción está activa.

Utiliza una estructura if y el operador and para determinar si el
usuario tiene acceso premium.

\end{tcolorbox}

\hypertarget{explicaciuxf3n-de-la-actividad-33}{%
\section{Explicación de la
Actividad:}\label{explicaciuxf3n-de-la-actividad-33}}

Esta actividad permite a los participantes practicar la combinación de
condiciones utilizando los operadores and y or. Les ayuda a comprender
cómo crear expresiones más complejas para tomar decisiones basadas en
múltiples condiciones en sus programas.

\begin{quote}
\begin{quote}
\begin{quote}
\begin{quote}
\begin{quote}
\begin{quote}
\begin{quote}
e8ed08b1a5bbe1e369719187cfc4de7f7e2a41a9
\end{quote}
\end{quote}
\end{quote}
\end{quote}
\end{quote}
\end{quote}
\end{quote}

\hypertarget{introducciuxf3n-a-while}{%
\chapter{Introducción a While}\label{introducciuxf3n-a-while}}

\textless\textless\textless\textless\textless\textless\textless{} HEAD

En esta lección, introduciremos la estructura while, que permite
ejecutar un bloque de código repetidamente mientras se cumpla una
condición.

\hypertarget{conceptos-clave-36}{%
\section{Conceptos Clave:}\label{conceptos-clave-36}}

\hypertarget{estructura-while}{%
\subsection{Estructura while}\label{estructura-while}}

Permite ejecutar un bloque de código mientras una condición sea
verdadera.

\hypertarget{condiciuxf3n}{%
\subsection{Condición}\label{condiciuxf3n}}

La condición se verifica antes de cada iteración. Si es verdadera, se
ejecuta el bloque de código.

\hypertarget{ejemplo-36}{%
\section{Ejemplo:}\label{ejemplo-36}}

\begin{Shaded}
\begin{Highlighting}[]
\NormalTok{contador }\OperatorTok{=} \DecValTok{0}

\ControlFlowTok{while}\NormalTok{ contador }\OperatorTok{\textless{}} \DecValTok{5}\NormalTok{:}
    \BuiltInTok{print}\NormalTok{(}\StringTok{"Contador:"}\NormalTok{, contador)}
\NormalTok{    contador }\OperatorTok{+=} \DecValTok{1}
\end{Highlighting}
\end{Shaded}

\hypertarget{explicaciuxf3n-36}{%
\section{Explicación:}\label{explicaciuxf3n-36}}

En este ejemplo, se utiliza la estructura while para imprimir el valor
del contador mientras sea menor que 5.

Después de cada iteración, se incrementa el contador en 1.

::: \{.callout-important\} \#\#\# Actividad Práctica:

Crea una variable que represente la cantidad de intentos de un usuario
para ingresar una contraseña correcta.

Utiliza una estructura while para pedir al usuario que ingrese su
contraseña hasta que lo haga correctamente.

\hypertarget{explicaciuxf3n-de-la-actividad-34}{%
\section{Explicación de la
Actividad:}\label{explicaciuxf3n-de-la-actividad-34}}

Esta actividad permite a los participantes practicar el uso de la
estructura while para crear bucles que se ejecutan repetidamente
mientras se cumple una condición. Les ayuda a comprender cómo
implementar lógica de repetición en sus programas.

======= \#\# Introducción a While

En esta lección, introduciremos la estructura while, que permite
ejecutar un bloque de código repetidamente mientras se cumpla una
condición.

\hypertarget{conceptos-clave-37}{%
\section{Conceptos Clave:}\label{conceptos-clave-37}}

\hypertarget{estructura-while-1}{%
\subsection{Estructura while}\label{estructura-while-1}}

Permite ejecutar un bloque de código mientras una condición sea
verdadera.

\hypertarget{condiciuxf3n-1}{%
\subsection{Condición}\label{condiciuxf3n-1}}

La condición se verifica antes de cada iteración. Si es verdadera, se
ejecuta el bloque de código.

\hypertarget{ejemplo-37}{%
\section{Ejemplo:}\label{ejemplo-37}}

\begin{Shaded}
\begin{Highlighting}[]
\NormalTok{contador }\OperatorTok{=} \DecValTok{0}

\ControlFlowTok{while}\NormalTok{ contador }\OperatorTok{\textless{}} \DecValTok{5}\NormalTok{:}
    \BuiltInTok{print}\NormalTok{(}\StringTok{"Contador:"}\NormalTok{, contador)}
\NormalTok{    contador }\OperatorTok{+=} \DecValTok{1}
\end{Highlighting}
\end{Shaded}

\hypertarget{explicaciuxf3n-37}{%
\section{Explicación:}\label{explicaciuxf3n-37}}

En este ejemplo, se utiliza la estructura while para imprimir el valor
del contador mientras sea menor que 5.

Después de cada iteración, se incrementa el contador en 1.

::: \{.callout-important\} \#\#\# Actividad Práctica:

Crea una variable que represente la cantidad de intentos de un usuario
para ingresar una contraseña correcta.

Utiliza una estructura while para pedir al usuario que ingrese su
contraseña hasta que lo haga correctamente.

\hypertarget{explicaciuxf3n-de-la-actividad-35}{%
\section{Explicación de la
Actividad:}\label{explicaciuxf3n-de-la-actividad-35}}

Esta actividad permite a los participantes practicar el uso de la
estructura while para crear bucles que se ejecutan repetidamente
mientras se cumple una condición. Les ayuda a comprender cómo
implementar lógica de repetición en sus programas.

\begin{quote}
\begin{quote}
\begin{quote}
\begin{quote}
\begin{quote}
\begin{quote}
\begin{quote}
e8ed08b1a5bbe1e369719187cfc4de7f7e2a41a9
\end{quote}
\end{quote}
\end{quote}
\end{quote}
\end{quote}
\end{quote}
\end{quote}

\hypertarget{while-loop}{%
\chapter{While loop}\label{while-loop}}

\textless\textless\textless\textless\textless\textless\textless{} HEAD

En esta lección, profundizaremos en el uso de la estructura while para
crear bucles que se ejecutan repetidamente mientras se cumpla una
condición, y aprenderemos a utilizar la sentencia break para salir de un
bucle.

\hypertarget{conceptos-clave-38}{%
\section{Conceptos Clave:}\label{conceptos-clave-38}}

\hypertarget{sentencia-break}{%
\subsection{Sentencia break:}\label{sentencia-break}}

Se utiliza para salir de un bucle antes de que la condición sea falsa.

\hypertarget{bucles-infinitos}{%
\subsection{Bucles Infinitos:}\label{bucles-infinitos}}

Si no se maneja adecuadamente, un bucle while puede ejecutarse
infinitamente.

\hypertarget{ejemplo-38}{%
\section{Ejemplo:}\label{ejemplo-38}}

\begin{Shaded}
\begin{Highlighting}[]
\NormalTok{contador }\OperatorTok{=} \DecValTok{0}

\ControlFlowTok{while} \VariableTok{True}\NormalTok{:}
    \BuiltInTok{print}\NormalTok{(}\StringTok{"Contador:"}\NormalTok{, contador)}
\NormalTok{    contador }\OperatorTok{+=} \DecValTok{1}
    \ControlFlowTok{if}\NormalTok{ contador }\OperatorTok{\textgreater{}=} \DecValTok{5}\NormalTok{:}
        \ControlFlowTok{break}
\end{Highlighting}
\end{Shaded}

\hypertarget{explicaciuxf3n-38}{%
\section{Explicación:}\label{explicaciuxf3n-38}}

En este ejemplo, se utiliza un bucle while que se ejecuta infinitamente.

Se utiliza la sentencia break para salir del bucle cuando el contador
llega a 5.

\begin{tcolorbox}[enhanced jigsaw, colbacktitle=quarto-callout-important-color!10!white, toprule=.15mm, leftrule=.75mm, titlerule=0mm, opacityback=0, rightrule=.15mm, opacitybacktitle=0.6, breakable, left=2mm, coltitle=black, title=\textcolor{quarto-callout-important-color}{\faExclamation}\hspace{0.5em}{Actividad Práctica:}, toptitle=1mm, bottomtitle=1mm, arc=.35mm, bottomrule=.15mm, colback=white, colframe=quarto-callout-important-color-frame]

Crea un bucle while que pida al usuario ingresar un número positivo
menor que 10.

Utiliza la sentencia break para salir del bucle una vez que el usuario
ingrese un número válido.

\end{tcolorbox}

\hypertarget{explicaciuxf3n-de-la-actividad-36}{%
\section{Explicación de la
Actividad:}\label{explicaciuxf3n-de-la-actividad-36}}

Esta actividad permite a los participantes practicar el uso de la
sentencia break para controlar la ejecución de un bucle while y evitar
bucles infinitos. Les ayuda a comprender cómo manejar situaciones en las
que es necesario salir de un bucle antes de que la condición sea falsa.

======= \#\# While loop

En esta lección, profundizaremos en el uso de la estructura while para
crear bucles que se ejecutan repetidamente mientras se cumpla una
condición, y aprenderemos a utilizar la sentencia break para salir de un
bucle.

\hypertarget{conceptos-clave-39}{%
\section{Conceptos Clave:}\label{conceptos-clave-39}}

\hypertarget{sentencia-break-1}{%
\subsection{Sentencia break:}\label{sentencia-break-1}}

Se utiliza para salir de un bucle antes de que la condición sea falsa.

\hypertarget{bucles-infinitos-1}{%
\subsection{Bucles Infinitos:}\label{bucles-infinitos-1}}

Si no se maneja adecuadamente, un bucle while puede ejecutarse
infinitamente.

\hypertarget{ejemplo-39}{%
\section{Ejemplo:}\label{ejemplo-39}}

\begin{Shaded}
\begin{Highlighting}[]
\NormalTok{contador }\OperatorTok{=} \DecValTok{0}

\ControlFlowTok{while} \VariableTok{True}\NormalTok{:}
    \BuiltInTok{print}\NormalTok{(}\StringTok{"Contador:"}\NormalTok{, contador)}
\NormalTok{    contador }\OperatorTok{+=} \DecValTok{1}
    \ControlFlowTok{if}\NormalTok{ contador }\OperatorTok{\textgreater{}=} \DecValTok{5}\NormalTok{:}
        \ControlFlowTok{break}
\end{Highlighting}
\end{Shaded}

\hypertarget{explicaciuxf3n-39}{%
\section{Explicación:}\label{explicaciuxf3n-39}}

En este ejemplo, se utiliza un bucle while que se ejecuta infinitamente.

Se utiliza la sentencia break para salir del bucle cuando el contador
llega a 5.

\begin{tcolorbox}[enhanced jigsaw, colbacktitle=quarto-callout-important-color!10!white, toprule=.15mm, leftrule=.75mm, titlerule=0mm, opacityback=0, rightrule=.15mm, opacitybacktitle=0.6, breakable, left=2mm, coltitle=black, title=\textcolor{quarto-callout-important-color}{\faExclamation}\hspace{0.5em}{Actividad Práctica:}, toptitle=1mm, bottomtitle=1mm, arc=.35mm, bottomrule=.15mm, colback=white, colframe=quarto-callout-important-color-frame]

Crea un bucle while que pida al usuario ingresar un número positivo
menor que 10.

Utiliza la sentencia break para salir del bucle una vez que el usuario
ingrese un número válido.

\end{tcolorbox}

\hypertarget{explicaciuxf3n-de-la-actividad-37}{%
\section{Explicación de la
Actividad:}\label{explicaciuxf3n-de-la-actividad-37}}

Esta actividad permite a los participantes practicar el uso de la
sentencia break para controlar la ejecución de un bucle while y evitar
bucles infinitos. Les ayuda a comprender cómo manejar situaciones en las
que es necesario salir de un bucle antes de que la condición sea falsa.

\begin{quote}
\begin{quote}
\begin{quote}
\begin{quote}
\begin{quote}
\begin{quote}
\begin{quote}
e8ed08b1a5bbe1e369719187cfc4de7f7e2a41a9
\end{quote}
\end{quote}
\end{quote}
\end{quote}
\end{quote}
\end{quote}
\end{quote}

\hypertarget{while-break-y-continue}{%
\chapter{While, break y continue}\label{while-break-y-continue}}

\textless\textless\textless\textless\textless\textless\textless{} HEAD

En esta lección, continuaremos explorando cómo trabajar con la
estructura while y aprenderemos a utilizar la sentencia continue para
saltar a la siguiente iteración del bucle.

\hypertarget{conceptos-clave-40}{%
\section{Conceptos Clave:}\label{conceptos-clave-40}}

\hypertarget{sentencia-continue}{%
\subsection{Sentencia continue}\label{sentencia-continue}}

Se utiliza para saltar a la siguiente iteración del bucle sin ejecutar
el resto del código en esa iteración.

\hypertarget{saltar-iteraciones}{%
\subsection{Saltar Iteraciones}\label{saltar-iteraciones}}

La sentencia continue permite omitir ciertas iteraciones basadas en una
condición.

\hypertarget{ejemplo-40}{%
\section{Ejemplo}\label{ejemplo-40}}

\begin{Shaded}
\begin{Highlighting}[]
\NormalTok{contador }\OperatorTok{=} \DecValTok{0}

\ControlFlowTok{while}\NormalTok{ contador }\OperatorTok{\textless{}} \DecValTok{5}\NormalTok{:}
\NormalTok{    contador }\OperatorTok{+=} \DecValTok{1}
    \ControlFlowTok{if}\NormalTok{ contador }\OperatorTok{==} \DecValTok{3}\NormalTok{:}
        \ControlFlowTok{continue}
    \BuiltInTok{print}\NormalTok{(}\StringTok{"Contador:"}\NormalTok{, contador)}
\end{Highlighting}
\end{Shaded}

\hypertarget{explicaciuxf3n-40}{%
\section{Explicación:}\label{explicaciuxf3n-40}}

En este ejemplo, se utiliza un bucle while para imprimir el valor del
contador.

Se utiliza la sentencia continue para omitir la iteración cuando el
contador es igual a 3.

\begin{tcolorbox}[enhanced jigsaw, colbacktitle=quarto-callout-important-color!10!white, toprule=.15mm, leftrule=.75mm, titlerule=0mm, opacityback=0, rightrule=.15mm, opacitybacktitle=0.6, breakable, left=2mm, coltitle=black, title=\textcolor{quarto-callout-important-color}{\faExclamation}\hspace{0.5em}{Actividad Práctica:}, toptitle=1mm, bottomtitle=1mm, arc=.35mm, bottomrule=.15mm, colback=white, colframe=quarto-callout-important-color-frame]

Crea un bucle while que imprima los números del 1 al 10, pero omite la
impresión del número 5.

Utiliza la sentencia continue para lograr esto.

\end{tcolorbox}

\hypertarget{explicaciuxf3n-de-la-actividad-38}{%
\section{Explicación de la
Actividad}\label{explicaciuxf3n-de-la-actividad-38}}

Esta actividad permite a los participantes practicar el uso de la
sentencia continue para omitir iteraciones específicas en un bucle
while. Les ayuda a comprender cómo controlar la ejecución de un bucle y
realizar acciones selectivas en cada iteración.

======= \#\# While, break y continue

En esta lección, continuaremos explorando cómo trabajar con la
estructura while y aprenderemos a utilizar la sentencia continue para
saltar a la siguiente iteración del bucle.

\hypertarget{conceptos-clave-41}{%
\section{Conceptos Clave:}\label{conceptos-clave-41}}

\hypertarget{sentencia-continue-1}{%
\subsection{Sentencia continue}\label{sentencia-continue-1}}

Se utiliza para saltar a la siguiente iteración del bucle sin ejecutar
el resto del código en esa iteración.

\hypertarget{saltar-iteraciones-1}{%
\subsection{Saltar Iteraciones}\label{saltar-iteraciones-1}}

La sentencia continue permite omitir ciertas iteraciones basadas en una
condición.

\hypertarget{ejemplo-41}{%
\section{Ejemplo}\label{ejemplo-41}}

\begin{Shaded}
\begin{Highlighting}[]
\NormalTok{contador }\OperatorTok{=} \DecValTok{0}

\ControlFlowTok{while}\NormalTok{ contador }\OperatorTok{\textless{}} \DecValTok{5}\NormalTok{:}
\NormalTok{    contador }\OperatorTok{+=} \DecValTok{1}
    \ControlFlowTok{if}\NormalTok{ contador }\OperatorTok{==} \DecValTok{3}\NormalTok{:}
        \ControlFlowTok{continue}
    \BuiltInTok{print}\NormalTok{(}\StringTok{"Contador:"}\NormalTok{, contador)}
\end{Highlighting}
\end{Shaded}

\hypertarget{explicaciuxf3n-41}{%
\section{Explicación:}\label{explicaciuxf3n-41}}

En este ejemplo, se utiliza un bucle while para imprimir el valor del
contador.

Se utiliza la sentencia continue para omitir la iteración cuando el
contador es igual a 3.

\begin{tcolorbox}[enhanced jigsaw, colbacktitle=quarto-callout-important-color!10!white, toprule=.15mm, leftrule=.75mm, titlerule=0mm, opacityback=0, rightrule=.15mm, opacitybacktitle=0.6, breakable, left=2mm, coltitle=black, title=\textcolor{quarto-callout-important-color}{\faExclamation}\hspace{0.5em}{Actividad Práctica:}, toptitle=1mm, bottomtitle=1mm, arc=.35mm, bottomrule=.15mm, colback=white, colframe=quarto-callout-important-color-frame]

Crea un bucle while que imprima los números del 1 al 10, pero omite la
impresión del número 5.

Utiliza la sentencia continue para lograr esto.

\end{tcolorbox}

\hypertarget{explicaciuxf3n-de-la-actividad-39}{%
\section{Explicación de la
Actividad}\label{explicaciuxf3n-de-la-actividad-39}}

Esta actividad permite a los participantes practicar el uso de la
sentencia continue para omitir iteraciones específicas en un bucle
while. Les ayuda a comprender cómo controlar la ejecución de un bucle y
realizar acciones selectivas en cada iteración.

\begin{quote}
\begin{quote}
\begin{quote}
\begin{quote}
\begin{quote}
\begin{quote}
\begin{quote}
e8ed08b1a5bbe1e369719187cfc4de7f7e2a41a9
\end{quote}
\end{quote}
\end{quote}
\end{quote}
\end{quote}
\end{quote}
\end{quote}

\hypertarget{for-loop}{%
\chapter{For Loop}\label{for-loop}}

\textless\textless\textless\textless\textless\textless\textless{} HEAD

En esta lección, aprenderemos sobre el bucle for, que se utiliza para
iterar sobre secuencias como listas, tuplas o cadenas de texto.

\hypertarget{conceptos-clave-42}{%
\section{Conceptos Clave:}\label{conceptos-clave-42}}

\hypertarget{bucle-for}{%
\subsection{Bucle for}\label{bucle-for}}

Se utiliza para recorrer elementos en una secuencia uno por uno.

\hypertarget{iteraciuxf3n}{%
\subsection{Iteración}\label{iteraciuxf3n}}

En cada iteración del bucle, se ejecuta un bloque de código con un valor
diferente de la secuencia.

\hypertarget{range-con-bucles-for}{%
\subsection{range() con Bucles for}\label{range-con-bucles-for}}

Se puede utilizar la función range() para generar una secuencia de
números y recorrerla con un bucle for.

\hypertarget{ejemplo-42}{%
\section{Ejemplo:}\label{ejemplo-42}}

\begin{Shaded}
\begin{Highlighting}[]
\NormalTok{frutas }\OperatorTok{=}\NormalTok{ [}\StringTok{"manzana"}\NormalTok{, }\StringTok{"banana"}\NormalTok{, }\StringTok{"naranja"}\NormalTok{]}

\ControlFlowTok{for}\NormalTok{ fruta }\KeywordTok{in}\NormalTok{ frutas:}
    \BuiltInTok{print}\NormalTok{(}\StringTok{"Me gusta"}\NormalTok{, fruta)}
\end{Highlighting}
\end{Shaded}

\hypertarget{explicaciuxf3n-42}{%
\section{Explicación:}\label{explicaciuxf3n-42}}

En este ejemplo, se utiliza un bucle for para recorrer una lista de
frutas.

En cada iteración, se asigna el valor actual de la lista a la variable
fruta.

\begin{tcolorbox}[enhanced jigsaw, colbacktitle=quarto-callout-important-color!10!white, toprule=.15mm, leftrule=.75mm, titlerule=0mm, opacityback=0, rightrule=.15mm, opacitybacktitle=0.6, breakable, left=2mm, coltitle=black, title=\textcolor{quarto-callout-important-color}{\faExclamation}\hspace{0.5em}{Actividad Práctica:}, toptitle=1mm, bottomtitle=1mm, arc=.35mm, bottomrule=.15mm, colback=white, colframe=quarto-callout-important-color-frame]

Crea una lista de colores.

Utiliza un bucle for para imprimir cada color en la lista precedido por
la palabra ``Color:''.

\end{tcolorbox}

\hypertarget{explicaciuxf3n-de-la-actividad-40}{%
\section{Explicación de la
Actividad:}\label{explicaciuxf3n-de-la-actividad-40}}

======= \#\# For Loop

En esta lección, aprenderemos sobre el bucle for, que se utiliza para
iterar sobre secuencias como listas, tuplas o cadenas de texto.

\hypertarget{conceptos-clave-43}{%
\section{Conceptos Clave:}\label{conceptos-clave-43}}

\hypertarget{bucle-for-1}{%
\subsection{Bucle for}\label{bucle-for-1}}

Se utiliza para recorrer elementos en una secuencia uno por uno.

\hypertarget{iteraciuxf3n-1}{%
\subsection{Iteración}\label{iteraciuxf3n-1}}

En cada iteración del bucle, se ejecuta un bloque de código con un valor
diferente de la secuencia.

\hypertarget{range-con-bucles-for-1}{%
\subsection{range() con Bucles for}\label{range-con-bucles-for-1}}

Se puede utilizar la función range() para generar una secuencia de
números y recorrerla con un bucle for.

\hypertarget{ejemplo-43}{%
\section{Ejemplo:}\label{ejemplo-43}}

\begin{Shaded}
\begin{Highlighting}[]
\NormalTok{frutas }\OperatorTok{=}\NormalTok{ [}\StringTok{"manzana"}\NormalTok{, }\StringTok{"banana"}\NormalTok{, }\StringTok{"naranja"}\NormalTok{]}

\ControlFlowTok{for}\NormalTok{ fruta }\KeywordTok{in}\NormalTok{ frutas:}
    \BuiltInTok{print}\NormalTok{(}\StringTok{"Me gusta"}\NormalTok{, fruta)}
\end{Highlighting}
\end{Shaded}

\hypertarget{explicaciuxf3n-43}{%
\section{Explicación:}\label{explicaciuxf3n-43}}

En este ejemplo, se utiliza un bucle for para recorrer una lista de
frutas.

En cada iteración, se asigna el valor actual de la lista a la variable
fruta.

\begin{tcolorbox}[enhanced jigsaw, colbacktitle=quarto-callout-important-color!10!white, toprule=.15mm, leftrule=.75mm, titlerule=0mm, opacityback=0, rightrule=.15mm, opacitybacktitle=0.6, breakable, left=2mm, coltitle=black, title=\textcolor{quarto-callout-important-color}{\faExclamation}\hspace{0.5em}{Actividad Práctica:}, toptitle=1mm, bottomtitle=1mm, arc=.35mm, bottomrule=.15mm, colback=white, colframe=quarto-callout-important-color-frame]

Crea una lista de colores.

Utiliza un bucle for para imprimir cada color en la lista precedido por
la palabra ``Color:''.

\end{tcolorbox}

\hypertarget{explicaciuxf3n-de-la-actividad-41}{%
\section{Explicación de la
Actividad:}\label{explicaciuxf3n-de-la-actividad-41}}

\begin{quote}
\begin{quote}
\begin{quote}
\begin{quote}
\begin{quote}
\begin{quote}
\begin{quote}
e8ed08b1a5bbe1e369719187cfc4de7f7e2a41a9 Esta actividad permite a los
participantes practicar el uso del bucle for para iterar sobre elementos
en una secuencia. Les ayuda a comprender cómo trabajar con bucles para
recorrer listas y otros tipos de secuencias en sus programas.
\end{quote}
\end{quote}
\end{quote}
\end{quote}
\end{quote}
\end{quote}
\end{quote}

\part{Unidad 6: Funciones}

\hypertarget{introducciuxf3n-a-funciones}{%
\chapter{Introducción a Funciones}\label{introducciuxf3n-a-funciones}}

\textless\textless\textless\textless\textless\textless\textless{} HEAD

En esta lección, exploraremos el concepto de funciones en Python.
Aprenderemos cómo definir funciones, pasar argumentos y cómo retornar
valores.

\hypertarget{conceptos-clave-44}{%
\section{Conceptos Clave:}\label{conceptos-clave-44}}

\hypertarget{funciones.}{%
\subsection{Funciones.}\label{funciones.}}

Bloques de código reutilizables que realizan una tarea específica.

\hypertarget{definiciuxf3n-de-funciones}{%
\subsection{Definición de Funciones}\label{definiciuxf3n-de-funciones}}

Se utiliza la palabra clave def para definir una función.

\hypertarget{argumentos}{%
\subsection{Argumentos}\label{argumentos}}

Valores que se pasan a una función para que trabaje con ellos.

\hypertarget{retorno-de-valores}{%
\section{Retorno de Valores}\label{retorno-de-valores}}

Una función puede retornar un valor utilizando la palabra clave return.

\hypertarget{ejemplo-44}{%
\section{Ejemplo:}\label{ejemplo-44}}

\begin{Shaded}
\begin{Highlighting}[]
\KeywordTok{def}\NormalTok{ saludar(nombre):}
    \ControlFlowTok{return} \StringTok{"Hola, "} \OperatorTok{+}\NormalTok{ nombre}

\NormalTok{mensaje }\OperatorTok{=}\NormalTok{ saludar(}\StringTok{"Juan"}\NormalTok{)}
\BuiltInTok{print}\NormalTok{(mensaje)}
\end{Highlighting}
\end{Shaded}

\hypertarget{explicaciuxf3n-44}{%
\section{Explicación:}\label{explicaciuxf3n-44}}

En este ejemplo, se define una función llamada saludar que toma un
argumento nombre.

La función retorna un saludo personalizado utilizando el argumento.

El resultado se asigna a la variable mensaje y se muestra en la consola.

\begin{tcolorbox}[enhanced jigsaw, colbacktitle=quarto-callout-important-color!10!white, toprule=.15mm, leftrule=.75mm, titlerule=0mm, opacityback=0, rightrule=.15mm, opacitybacktitle=0.6, breakable, left=2mm, coltitle=black, title=\textcolor{quarto-callout-important-color}{\faExclamation}\hspace{0.5em}{Actividad Práctica:}, toptitle=1mm, bottomtitle=1mm, arc=.35mm, bottomrule=.15mm, colback=white, colframe=quarto-callout-important-color-frame]

Crea una función llamada calcular\_cuadrado que reciba un número como
argumento y retorne el cuadrado de ese número.

Utiliza la función para calcular el cuadrado de un número y mostrarlo en
la consola.

\end{tcolorbox}

\hypertarget{explicaciuxf3n-de-la-actividad-42}{%
\section{Explicación de la
Actividad:}\label{explicaciuxf3n-de-la-actividad-42}}

Esta actividad permite a los participantes practicar la creación y uso
de funciones en Python. Les ayuda a comprender cómo definir funciones,
pasar argumentos y cómo trabajar con valores retornados por las
funciones.

======= \#\# Introducción a Funciones

En esta lección, exploraremos el concepto de funciones en Python.
Aprenderemos cómo definir funciones, pasar argumentos y cómo retornar
valores.

\hypertarget{conceptos-clave-45}{%
\section{Conceptos Clave:}\label{conceptos-clave-45}}

\hypertarget{funciones.-1}{%
\subsection{Funciones.}\label{funciones.-1}}

Bloques de código reutilizables que realizan una tarea específica.

\hypertarget{definiciuxf3n-de-funciones-1}{%
\subsection{Definición de
Funciones}\label{definiciuxf3n-de-funciones-1}}

Se utiliza la palabra clave def para definir una función.

\hypertarget{argumentos-1}{%
\subsection{Argumentos}\label{argumentos-1}}

Valores que se pasan a una función para que trabaje con ellos.

\hypertarget{retorno-de-valores-1}{%
\section{Retorno de Valores}\label{retorno-de-valores-1}}

Una función puede retornar un valor utilizando la palabra clave return.

\hypertarget{ejemplo-45}{%
\section{Ejemplo:}\label{ejemplo-45}}

\begin{Shaded}
\begin{Highlighting}[]
\KeywordTok{def}\NormalTok{ saludar(nombre):}
    \ControlFlowTok{return} \StringTok{"Hola, "} \OperatorTok{+}\NormalTok{ nombre}

\NormalTok{mensaje }\OperatorTok{=}\NormalTok{ saludar(}\StringTok{"Juan"}\NormalTok{)}
\BuiltInTok{print}\NormalTok{(mensaje)}
\end{Highlighting}
\end{Shaded}

\hypertarget{explicaciuxf3n-45}{%
\section{Explicación:}\label{explicaciuxf3n-45}}

En este ejemplo, se define una función llamada saludar que toma un
argumento nombre.

La función retorna un saludo personalizado utilizando el argumento.

El resultado se asigna a la variable mensaje y se muestra en la consola.

\begin{tcolorbox}[enhanced jigsaw, colbacktitle=quarto-callout-important-color!10!white, toprule=.15mm, leftrule=.75mm, titlerule=0mm, opacityback=0, rightrule=.15mm, opacitybacktitle=0.6, breakable, left=2mm, coltitle=black, title=\textcolor{quarto-callout-important-color}{\faExclamation}\hspace{0.5em}{Actividad Práctica:}, toptitle=1mm, bottomtitle=1mm, arc=.35mm, bottomrule=.15mm, colback=white, colframe=quarto-callout-important-color-frame]

Crea una función llamada calcular\_cuadrado que reciba un número como
argumento y retorne el cuadrado de ese número.

Utiliza la función para calcular el cuadrado de un número y mostrarlo en
la consola.

\end{tcolorbox}

\hypertarget{explicaciuxf3n-de-la-actividad-43}{%
\section{Explicación de la
Actividad:}\label{explicaciuxf3n-de-la-actividad-43}}

Esta actividad permite a los participantes practicar la creación y uso
de funciones en Python. Les ayuda a comprender cómo definir funciones,
pasar argumentos y cómo trabajar con valores retornados por las
funciones.

\begin{quote}
\begin{quote}
\begin{quote}
\begin{quote}
\begin{quote}
\begin{quote}
\begin{quote}
e8ed08b1a5bbe1e369719187cfc4de7f7e2a41a9
\end{quote}
\end{quote}
\end{quote}
\end{quote}
\end{quote}
\end{quote}
\end{quote}

\hypertarget{recursividad}{%
\chapter{Recursividad}\label{recursividad}}

\textless\textless\textless\textless\textless\textless\textless{} HEAD

En esta lección, exploraremos el concepto de recursividad, que es cuando
una función se llama a sí misma para resolver un problema. Aprenderemos
cómo implementar funciones recursivas y cuándo es apropiado usarlas.

\hypertarget{conceptos-clave-46}{%
\section{Conceptos Clave:}\label{conceptos-clave-46}}

\hypertarget{recursividad-1}{%
\subsection{Recursividad}\label{recursividad-1}}

Técnica en la que una función se llama a sí misma para resolver un
problema.

\hypertarget{caso-base}{%
\subsection{Caso Base}\label{caso-base}}

Condición que indica cuándo la recursión debe detenerse.

\hypertarget{caso-recursivo}{%
\subsection{Caso Recursivo}\label{caso-recursivo}}

Cómo se divide el problema en partes más pequeñas en cada llamada
recursiva.

\hypertarget{ejemplo-46}{%
\section{Ejemplo:}\label{ejemplo-46}}

\begin{Shaded}
\begin{Highlighting}[]
\KeywordTok{def}\NormalTok{ factorial(n):}
    \ControlFlowTok{if}\NormalTok{ n }\OperatorTok{==} \DecValTok{0}\NormalTok{:}
        \ControlFlowTok{return} \DecValTok{1}
    \ControlFlowTok{else}\NormalTok{:}
        \ControlFlowTok{return}\NormalTok{ n }\OperatorTok{*}\NormalTok{ factorial(n }\OperatorTok{{-}} \DecValTok{1}\NormalTok{)}

\NormalTok{resultado }\OperatorTok{=}\NormalTok{ factorial(}\DecValTok{5}\NormalTok{)}
\BuiltInTok{print}\NormalTok{(resultado)}
\end{Highlighting}
\end{Shaded}

\hypertarget{explicaciuxf3n-46}{%
\section{Explicación:}\label{explicaciuxf3n-46}}

En este ejemplo, se define una función recursiva llamada factorial para
calcular el factorial de un número.

La función utiliza un caso base (cuando n es 0) y un caso recursivo
(llamando a la función con n - 1).

\begin{tcolorbox}[enhanced jigsaw, colbacktitle=quarto-callout-important-color!10!white, toprule=.15mm, leftrule=.75mm, titlerule=0mm, opacityback=0, rightrule=.15mm, opacitybacktitle=0.6, breakable, left=2mm, coltitle=black, title=\textcolor{quarto-callout-important-color}{\faExclamation}\hspace{0.5em}{Actividad Práctica:}, toptitle=1mm, bottomtitle=1mm, arc=.35mm, bottomrule=.15mm, colback=white, colframe=quarto-callout-important-color-frame]

Crea una función recursiva llamada potencia que calcule la potencia de
un número base elevado a un exponente.

Utiliza la función para calcular 2\^{}3 y muestra el resultado en la
consola.

\end{tcolorbox}

\hypertarget{explicaciuxf3n-de-la-actividad-44}{%
\section{Explicación de la
Actividad:}\label{explicaciuxf3n-de-la-actividad-44}}

======= \#\# Recursividad

En esta lección, exploraremos el concepto de recursividad, que es cuando
una función se llama a sí misma para resolver un problema. Aprenderemos
cómo implementar funciones recursivas y cuándo es apropiado usarlas.

\hypertarget{conceptos-clave-47}{%
\section{Conceptos Clave:}\label{conceptos-clave-47}}

\hypertarget{recursividad-2}{%
\subsection{Recursividad}\label{recursividad-2}}

Técnica en la que una función se llama a sí misma para resolver un
problema.

\hypertarget{caso-base-1}{%
\subsection{Caso Base}\label{caso-base-1}}

Condición que indica cuándo la recursión debe detenerse.

\hypertarget{caso-recursivo-1}{%
\subsection{Caso Recursivo}\label{caso-recursivo-1}}

Cómo se divide el problema en partes más pequeñas en cada llamada
recursiva.

\hypertarget{ejemplo-47}{%
\section{Ejemplo:}\label{ejemplo-47}}

\begin{Shaded}
\begin{Highlighting}[]
\KeywordTok{def}\NormalTok{ factorial(n):}
    \ControlFlowTok{if}\NormalTok{ n }\OperatorTok{==} \DecValTok{0}\NormalTok{:}
        \ControlFlowTok{return} \DecValTok{1}
    \ControlFlowTok{else}\NormalTok{:}
        \ControlFlowTok{return}\NormalTok{ n }\OperatorTok{*}\NormalTok{ factorial(n }\OperatorTok{{-}} \DecValTok{1}\NormalTok{)}

\NormalTok{resultado }\OperatorTok{=}\NormalTok{ factorial(}\DecValTok{5}\NormalTok{)}
\BuiltInTok{print}\NormalTok{(resultado)}
\end{Highlighting}
\end{Shaded}

\hypertarget{explicaciuxf3n-47}{%
\section{Explicación:}\label{explicaciuxf3n-47}}

En este ejemplo, se define una función recursiva llamada factorial para
calcular el factorial de un número.

La función utiliza un caso base (cuando n es 0) y un caso recursivo
(llamando a la función con n - 1).

\begin{tcolorbox}[enhanced jigsaw, colbacktitle=quarto-callout-important-color!10!white, toprule=.15mm, leftrule=.75mm, titlerule=0mm, opacityback=0, rightrule=.15mm, opacitybacktitle=0.6, breakable, left=2mm, coltitle=black, title=\textcolor{quarto-callout-important-color}{\faExclamation}\hspace{0.5em}{Actividad Práctica:}, toptitle=1mm, bottomtitle=1mm, arc=.35mm, bottomrule=.15mm, colback=white, colframe=quarto-callout-important-color-frame]

Crea una función recursiva llamada potencia que calcule la potencia de
un número base elevado a un exponente.

Utiliza la función para calcular 2\^{}3 y muestra el resultado en la
consola.

\end{tcolorbox}

\hypertarget{explicaciuxf3n-de-la-actividad-45}{%
\section{Explicación de la
Actividad:}\label{explicaciuxf3n-de-la-actividad-45}}

\begin{quote}
\begin{quote}
\begin{quote}
\begin{quote}
\begin{quote}
\begin{quote}
\begin{quote}
e8ed08b1a5bbe1e369719187cfc4de7f7e2a41a9 Esta actividad permite a los
participantes practicar la implementación de funciones recursivas y
comprender cómo se dividen los problemas en partes más pequeñas para
resolverlos. Les ayuda a comprender cómo aplicar la recursividad de
manera efectiva en la solución de problemas.
\end{quote}
\end{quote}
\end{quote}
\end{quote}
\end{quote}
\end{quote}
\end{quote}

\part{Unidad 7: Objetos, Clases y Herencia}

\hypertarget{introducciuxf3n}{%
\chapter{Introducción}\label{introducciuxf3n}}

\textless\textless\textless\textless\textless\textless\textless{} HEAD

En esta lección, exploraremos el concepto de programación orientada a
objetos (POO). Aprenderemos sobre objetos, clases y cómo la POO nos
permite organizar y estructurar nuestro código de manera más eficiente.

\hypertarget{conceptos-clave-48}{%
\section{Conceptos Clave:}\label{conceptos-clave-48}}

\hypertarget{programaciuxf3n-orientada-a-objetos-poo}{%
\subsection{Programación Orientada a Objetos
(POO)}\label{programaciuxf3n-orientada-a-objetos-poo}}

Paradigma de programación que se basa en el uso de objetos y clases.

\hypertarget{objetos}{%
\subsection{Objetos}\label{objetos}}

Instancias de clases que representan entidades en el mundo real.

\hypertarget{clases}{%
\subsection{Clases}\label{clases}}

Plantillas o moldes que definen la estructura y el comportamiento de los
objetos.

\hypertarget{atributos}{%
\subsection{Atributos}\label{atributos}}

Características o propiedades de un objeto.

\hypertarget{muxe9todos}{%
\subsection{Métodos}\label{muxe9todos}}

Funciones que definen el comportamiento de un objeto.

\hypertarget{ejemplo-48}{%
\section{Ejemplo:}\label{ejemplo-48}}

\begin{Shaded}
\begin{Highlighting}[]
\KeywordTok{class}\NormalTok{ Persona:}
    \KeywordTok{def} \FunctionTok{\_\_init\_\_}\NormalTok{(}\VariableTok{self}\NormalTok{, nombre, edad):}
        \VariableTok{self}\NormalTok{.nombre }\OperatorTok{=}\NormalTok{ nombre}
        \VariableTok{self}\NormalTok{.edad }\OperatorTok{=}\NormalTok{ edad}

    \KeywordTok{def}\NormalTok{ saludar(}\VariableTok{self}\NormalTok{):}
        \BuiltInTok{print}\NormalTok{(}\SpecialStringTok{f"Hola, mi nombre es }\SpecialCharTok{\{}\VariableTok{self}\SpecialCharTok{.}\NormalTok{nombre}\SpecialCharTok{\}}\SpecialStringTok{ y tengo }\SpecialCharTok{\{}\VariableTok{self}\SpecialCharTok{.}\NormalTok{edad}\SpecialCharTok{\}}\SpecialStringTok{ años."}\NormalTok{)}

\NormalTok{persona1 }\OperatorTok{=}\NormalTok{ Persona(}\StringTok{"Juan"}\NormalTok{, }\DecValTok{25}\NormalTok{)}
\NormalTok{persona1.saludar()}
\end{Highlighting}
\end{Shaded}

\hypertarget{explicaciuxf3n-48}{%
\section{Explicación:}\label{explicaciuxf3n-48}}

En este ejemplo, se define una clase llamada Persona con un constructor
(\textbf{init}) que inicializa atributos.

La clase tiene un método llamado saludar que muestra un mensaje con el
nombre y edad del objeto.

\begin{tcolorbox}[enhanced jigsaw, colbacktitle=quarto-callout-important-color!10!white, toprule=.15mm, leftrule=.75mm, titlerule=0mm, opacityback=0, rightrule=.15mm, opacitybacktitle=0.6, breakable, left=2mm, coltitle=black, title=\textcolor{quarto-callout-important-color}{\faExclamation}\hspace{0.5em}{Actividad Práctica:}, toptitle=1mm, bottomtitle=1mm, arc=.35mm, bottomrule=.15mm, colback=white, colframe=quarto-callout-important-color-frame]

Crea una clase llamada Libro con atributos titulo y autor, y un método
mostrar\_info que imprima los atributos.

Crea una instancia de la clase Libro y llama al método mostrar\_info.

\end{tcolorbox}

\hypertarget{explicaciuxf3n-de-la-actividad-46}{%
\section{Explicación de la
Actividad:}\label{explicaciuxf3n-de-la-actividad-46}}

Esta actividad permite a los participantes practicar la definición de
clases y la creación de objetos. Les ayuda a comprender cómo la POO nos
permite modelar entidades y organizar el código de manera más
estructurada y eficiente.

======= \#\# Introducción

En esta lección, exploraremos el concepto de programación orientada a
objetos (POO). Aprenderemos sobre objetos, clases y cómo la POO nos
permite organizar y estructurar nuestro código de manera más eficiente.

\hypertarget{conceptos-clave-49}{%
\section{Conceptos Clave:}\label{conceptos-clave-49}}

\hypertarget{programaciuxf3n-orientada-a-objetos-poo-1}{%
\subsection{Programación Orientada a Objetos
(POO)}\label{programaciuxf3n-orientada-a-objetos-poo-1}}

Paradigma de programación que se basa en el uso de objetos y clases.

\hypertarget{objetos-1}{%
\subsection{Objetos}\label{objetos-1}}

Instancias de clases que representan entidades en el mundo real.

\hypertarget{clases-1}{%
\subsection{Clases}\label{clases-1}}

Plantillas o moldes que definen la estructura y el comportamiento de los
objetos.

\hypertarget{atributos-1}{%
\subsection{Atributos}\label{atributos-1}}

Características o propiedades de un objeto.

\hypertarget{muxe9todos-1}{%
\subsection{Métodos}\label{muxe9todos-1}}

Funciones que definen el comportamiento de un objeto.

\hypertarget{ejemplo-49}{%
\section{Ejemplo:}\label{ejemplo-49}}

\begin{Shaded}
\begin{Highlighting}[]
\KeywordTok{class}\NormalTok{ Persona:}
    \KeywordTok{def} \FunctionTok{\_\_init\_\_}\NormalTok{(}\VariableTok{self}\NormalTok{, nombre, edad):}
        \VariableTok{self}\NormalTok{.nombre }\OperatorTok{=}\NormalTok{ nombre}
        \VariableTok{self}\NormalTok{.edad }\OperatorTok{=}\NormalTok{ edad}

    \KeywordTok{def}\NormalTok{ saludar(}\VariableTok{self}\NormalTok{):}
        \BuiltInTok{print}\NormalTok{(}\SpecialStringTok{f"Hola, mi nombre es }\SpecialCharTok{\{}\VariableTok{self}\SpecialCharTok{.}\NormalTok{nombre}\SpecialCharTok{\}}\SpecialStringTok{ y tengo }\SpecialCharTok{\{}\VariableTok{self}\SpecialCharTok{.}\NormalTok{edad}\SpecialCharTok{\}}\SpecialStringTok{ años."}\NormalTok{)}

\NormalTok{persona1 }\OperatorTok{=}\NormalTok{ Persona(}\StringTok{"Juan"}\NormalTok{, }\DecValTok{25}\NormalTok{)}
\NormalTok{persona1.saludar()}
\end{Highlighting}
\end{Shaded}

\hypertarget{explicaciuxf3n-49}{%
\section{Explicación:}\label{explicaciuxf3n-49}}

En este ejemplo, se define una clase llamada Persona con un constructor
(\textbf{init}) que inicializa atributos.

La clase tiene un método llamado saludar que muestra un mensaje con el
nombre y edad del objeto.

\begin{tcolorbox}[enhanced jigsaw, colbacktitle=quarto-callout-important-color!10!white, toprule=.15mm, leftrule=.75mm, titlerule=0mm, opacityback=0, rightrule=.15mm, opacitybacktitle=0.6, breakable, left=2mm, coltitle=black, title=\textcolor{quarto-callout-important-color}{\faExclamation}\hspace{0.5em}{Actividad Práctica:}, toptitle=1mm, bottomtitle=1mm, arc=.35mm, bottomrule=.15mm, colback=white, colframe=quarto-callout-important-color-frame]

Crea una clase llamada Libro con atributos titulo y autor, y un método
mostrar\_info que imprima los atributos.

Crea una instancia de la clase Libro y llama al método mostrar\_info.

\end{tcolorbox}

\hypertarget{explicaciuxf3n-de-la-actividad-47}{%
\section{Explicación de la
Actividad:}\label{explicaciuxf3n-de-la-actividad-47}}

Esta actividad permite a los participantes practicar la definición de
clases y la creación de objetos. Les ayuda a comprender cómo la POO nos
permite modelar entidades y organizar el código de manera más
estructurada y eficiente.

\begin{quote}
\begin{quote}
\begin{quote}
\begin{quote}
\begin{quote}
\begin{quote}
\begin{quote}
e8ed08b1a5bbe1e369719187cfc4de7f7e2a41a9
\end{quote}
\end{quote}
\end{quote}
\end{quote}
\end{quote}
\end{quote}
\end{quote}

\hypertarget{objetos-y-clases}{%
\chapter{Objetos y Clases}\label{objetos-y-clases}}

\textless\textless\textless\textless\textless\textless\textless{} HEAD

En esta lección, continuaremos explorando los conceptos de objetos y
clases en la programación orientada a objetos. Aprenderemos cómo crear
múltiples objetos a partir de una misma clase y cómo trabajar con sus
atributos y métodos.

\hypertarget{conceptos-clave-50}{%
\section{Conceptos Clave:}\label{conceptos-clave-50}}

\hypertarget{instancias-de-clase}{%
\subsection{Instancias de Clase}\label{instancias-de-clase}}

Cuando se crea un objeto a partir de una clase, se crea una instancia de
esa clase.

\hypertarget{atributos-de-instancia}{%
\subsection{Atributos de Instancia:}\label{atributos-de-instancia}}

Características específicas de un objeto que se almacenan como variables
en la instancia.

\hypertarget{muxe9todos-de-instancia}{%
\subsection{Métodos de Instancia}\label{muxe9todos-de-instancia}}

Funciones definidas en la clase que operan en los atributos de la
instancia.

\hypertarget{ejemplo-50}{%
\section{Ejemplo:}\label{ejemplo-50}}

\begin{Shaded}
\begin{Highlighting}[]
\KeywordTok{class}\NormalTok{ Perro:}
    \KeywordTok{def} \FunctionTok{\_\_init\_\_}\NormalTok{(}\VariableTok{self}\NormalTok{, nombre, raza):}
        \VariableTok{self}\NormalTok{.nombre }\OperatorTok{=}\NormalTok{ nombre}
        \VariableTok{self}\NormalTok{.raza }\OperatorTok{=}\NormalTok{ raza}

    \KeywordTok{def}\NormalTok{ ladrar(}\VariableTok{self}\NormalTok{):}
        \BuiltInTok{print}\NormalTok{(}\SpecialStringTok{f"}\SpecialCharTok{\{}\VariableTok{self}\SpecialCharTok{.}\NormalTok{nombre}\SpecialCharTok{\}}\SpecialStringTok{ está ladrando."}\NormalTok{)}

\NormalTok{perro1 }\OperatorTok{=}\NormalTok{ Perro(}\StringTok{"Max"}\NormalTok{, }\StringTok{"Labrador"}\NormalTok{)}
\NormalTok{perro2 }\OperatorTok{=}\NormalTok{ Perro(}\StringTok{"Buddy"}\NormalTok{, }\StringTok{"Chihuahua"}\NormalTok{)}

\NormalTok{perro1.ladrar()}
\NormalTok{perro2.ladrar()}
\end{Highlighting}
\end{Shaded}

\hypertarget{explicaciuxf3n-50}{%
\section{Explicación:}\label{explicaciuxf3n-50}}

En este ejemplo, se define una clase Perro con un constructor y un
método ladrar.

Se crean dos objetos (perro1 y perro2) a partir de la misma clase y se
les asignan diferentes valores para sus atributos.

Los métodos de instancia son llamados en cada objeto para realizar
acciones específicas.

\begin{tcolorbox}[enhanced jigsaw, colbacktitle=quarto-callout-important-color!10!white, toprule=.15mm, leftrule=.75mm, titlerule=0mm, opacityback=0, rightrule=.15mm, opacitybacktitle=0.6, breakable, left=2mm, coltitle=black, title=\textcolor{quarto-callout-important-color}{\faExclamation}\hspace{0.5em}{Actividad Práctica:}, toptitle=1mm, bottomtitle=1mm, arc=.35mm, bottomrule=.15mm, colback=white, colframe=quarto-callout-important-color-frame]

Crea una clase Rectangulo con atributos ancho y alto, y un método
calcular\_area que calcule y retorne el área del rectángulo.

Crea dos instancias de la clase Rectangulo con diferentes valores de
ancho y alto, y llama al método calcular\_area en cada una.

\end{tcolorbox}

\hypertarget{explicaciuxf3n-de-la-actividad-48}{%
\section{Explicación de la
Actividad:}\label{explicaciuxf3n-de-la-actividad-48}}

Esta actividad permite a los participantes practicar la creación de
instancias de clase y trabajar con sus atributos y métodos. Les ayuda a
comprender cómo cada objeto puede tener valores diferentes para sus
atributos y cómo ejecutar acciones específicas en cada objeto.

======= \#\# Objetos y Clases

En esta lección, continuaremos explorando los conceptos de objetos y
clases en la programación orientada a objetos. Aprenderemos cómo crear
múltiples objetos a partir de una misma clase y cómo trabajar con sus
atributos y métodos.

\hypertarget{conceptos-clave-51}{%
\section{Conceptos Clave:}\label{conceptos-clave-51}}

\hypertarget{instancias-de-clase-1}{%
\subsection{Instancias de Clase}\label{instancias-de-clase-1}}

Cuando se crea un objeto a partir de una clase, se crea una instancia de
esa clase.

\hypertarget{atributos-de-instancia-1}{%
\subsection{Atributos de Instancia:}\label{atributos-de-instancia-1}}

Características específicas de un objeto que se almacenan como variables
en la instancia.

\hypertarget{muxe9todos-de-instancia-1}{%
\subsection{Métodos de Instancia}\label{muxe9todos-de-instancia-1}}

Funciones definidas en la clase que operan en los atributos de la
instancia.

\hypertarget{ejemplo-51}{%
\section{Ejemplo:}\label{ejemplo-51}}

\begin{Shaded}
\begin{Highlighting}[]
\KeywordTok{class}\NormalTok{ Perro:}
    \KeywordTok{def} \FunctionTok{\_\_init\_\_}\NormalTok{(}\VariableTok{self}\NormalTok{, nombre, raza):}
        \VariableTok{self}\NormalTok{.nombre }\OperatorTok{=}\NormalTok{ nombre}
        \VariableTok{self}\NormalTok{.raza }\OperatorTok{=}\NormalTok{ raza}

    \KeywordTok{def}\NormalTok{ ladrar(}\VariableTok{self}\NormalTok{):}
        \BuiltInTok{print}\NormalTok{(}\SpecialStringTok{f"}\SpecialCharTok{\{}\VariableTok{self}\SpecialCharTok{.}\NormalTok{nombre}\SpecialCharTok{\}}\SpecialStringTok{ está ladrando."}\NormalTok{)}

\NormalTok{perro1 }\OperatorTok{=}\NormalTok{ Perro(}\StringTok{"Max"}\NormalTok{, }\StringTok{"Labrador"}\NormalTok{)}
\NormalTok{perro2 }\OperatorTok{=}\NormalTok{ Perro(}\StringTok{"Buddy"}\NormalTok{, }\StringTok{"Chihuahua"}\NormalTok{)}

\NormalTok{perro1.ladrar()}
\NormalTok{perro2.ladrar()}
\end{Highlighting}
\end{Shaded}

\hypertarget{explicaciuxf3n-51}{%
\section{Explicación:}\label{explicaciuxf3n-51}}

En este ejemplo, se define una clase Perro con un constructor y un
método ladrar.

Se crean dos objetos (perro1 y perro2) a partir de la misma clase y se
les asignan diferentes valores para sus atributos.

Los métodos de instancia son llamados en cada objeto para realizar
acciones específicas.

\begin{tcolorbox}[enhanced jigsaw, colbacktitle=quarto-callout-important-color!10!white, toprule=.15mm, leftrule=.75mm, titlerule=0mm, opacityback=0, rightrule=.15mm, opacitybacktitle=0.6, breakable, left=2mm, coltitle=black, title=\textcolor{quarto-callout-important-color}{\faExclamation}\hspace{0.5em}{Actividad Práctica:}, toptitle=1mm, bottomtitle=1mm, arc=.35mm, bottomrule=.15mm, colback=white, colframe=quarto-callout-important-color-frame]

Crea una clase Rectangulo con atributos ancho y alto, y un método
calcular\_area que calcule y retorne el área del rectángulo.

Crea dos instancias de la clase Rectangulo con diferentes valores de
ancho y alto, y llama al método calcular\_area en cada una.

\end{tcolorbox}

\hypertarget{explicaciuxf3n-de-la-actividad-49}{%
\section{Explicación de la
Actividad:}\label{explicaciuxf3n-de-la-actividad-49}}

Esta actividad permite a los participantes practicar la creación de
instancias de clase y trabajar con sus atributos y métodos. Les ayuda a
comprender cómo cada objeto puede tener valores diferentes para sus
atributos y cómo ejecutar acciones específicas en cada objeto.

\begin{quote}
\begin{quote}
\begin{quote}
\begin{quote}
\begin{quote}
\begin{quote}
\begin{quote}
e8ed08b1a5bbe1e369719187cfc4de7f7e2a41a9
\end{quote}
\end{quote}
\end{quote}
\end{quote}
\end{quote}
\end{quote}
\end{quote}

\hypertarget{muxe9todos-2}{%
\chapter{Métodos}\label{muxe9todos-2}}

\textless\textless\textless\textless\textless\textless\textless{} HEAD

En esta lección, profundizaremos en el concepto de métodos en la
programación orientada a objetos. Aprenderemos cómo definir y utilizar
métodos en una clase, y cómo acceder a los atributos de instancia dentro
de los métodos.

\hypertarget{conceptos-clave-52}{%
\section{Conceptos Clave:}\label{conceptos-clave-52}}

\hypertarget{muxe9todos-de-clase}{%
\subsection{Métodos de Clase}\label{muxe9todos-de-clase}}

Funciones definidas dentro de una clase que operan en los atributos de
instancia.

\hypertarget{acceso-a-atributos}{%
\subsection{Acceso a Atributos}\label{acceso-a-atributos}}

Dentro de un método, se puede acceder a los atributos de instancia
utilizando self.atributo.

\hypertarget{ejemplo-52}{%
\section{Ejemplo:}\label{ejemplo-52}}

\begin{Shaded}
\begin{Highlighting}[]
\KeywordTok{class}\NormalTok{ Circulo:}
    \KeywordTok{def} \FunctionTok{\_\_init\_\_}\NormalTok{(}\VariableTok{self}\NormalTok{, radio):}
        \VariableTok{self}\NormalTok{.radio }\OperatorTok{=}\NormalTok{ radio}

    \KeywordTok{def}\NormalTok{ calcular\_area(}\VariableTok{self}\NormalTok{):}
\NormalTok{        area }\OperatorTok{=} \FloatTok{3.14} \OperatorTok{*} \VariableTok{self}\NormalTok{.radio }\OperatorTok{**} \DecValTok{2}
        \ControlFlowTok{return}\NormalTok{ area}

\NormalTok{circulo1 }\OperatorTok{=}\NormalTok{ Circulo(}\DecValTok{5}\NormalTok{)}
\NormalTok{area\_circulo }\OperatorTok{=}\NormalTok{ circulo1.calcular\_area()}
\BuiltInTok{print}\NormalTok{(}\StringTok{"Área del círculo:"}\NormalTok{, area\_circulo)}
\end{Highlighting}
\end{Shaded}

\hypertarget{explicaciuxf3n-52}{%
\section{Explicación:}\label{explicaciuxf3n-52}}

En este ejemplo, se define una clase Circulo con un constructor y un
método calcular\_area.

Dentro del método, se accede al atributo de instancia radio utilizando
self.radio para calcular el área.

\begin{tcolorbox}[enhanced jigsaw, colbacktitle=quarto-callout-important-color!10!white, toprule=.15mm, leftrule=.75mm, titlerule=0mm, opacityback=0, rightrule=.15mm, opacitybacktitle=0.6, breakable, left=2mm, coltitle=black, title=\textcolor{quarto-callout-important-color}{\faExclamation}\hspace{0.5em}{Actividad Práctica:}, toptitle=1mm, bottomtitle=1mm, arc=.35mm, bottomrule=.15mm, colback=white, colframe=quarto-callout-important-color-frame]

Crea una clase Triangulo con atributos base y altura, y un método
calcular\_area que calcule y retorne el área del triángulo.

Crea una instancia de la clase Triangulo y llama al método
calcular\_area para calcular el área.

\end{tcolorbox}

\hypertarget{explicaciuxf3n-de-la-actividad-50}{%
\section{Explicación de la
Actividad:}\label{explicaciuxf3n-de-la-actividad-50}}

Esta actividad permite a los participantes practicar la definición y uso
de métodos en una clase. Les ayuda a comprender cómo trabajar con
atributos de instancia dentro de los métodos y cómo implementar lógica
específica para cada objeto.

======= \#\# Métodos

En esta lección, profundizaremos en el concepto de métodos en la
programación orientada a objetos. Aprenderemos cómo definir y utilizar
métodos en una clase, y cómo acceder a los atributos de instancia dentro
de los métodos.

\hypertarget{conceptos-clave-53}{%
\section{Conceptos Clave:}\label{conceptos-clave-53}}

\hypertarget{muxe9todos-de-clase-1}{%
\subsection{Métodos de Clase}\label{muxe9todos-de-clase-1}}

Funciones definidas dentro de una clase que operan en los atributos de
instancia.

\hypertarget{acceso-a-atributos-1}{%
\subsection{Acceso a Atributos}\label{acceso-a-atributos-1}}

Dentro de un método, se puede acceder a los atributos de instancia
utilizando self.atributo.

\hypertarget{ejemplo-53}{%
\section{Ejemplo:}\label{ejemplo-53}}

\begin{Shaded}
\begin{Highlighting}[]
\KeywordTok{class}\NormalTok{ Circulo:}
    \KeywordTok{def} \FunctionTok{\_\_init\_\_}\NormalTok{(}\VariableTok{self}\NormalTok{, radio):}
        \VariableTok{self}\NormalTok{.radio }\OperatorTok{=}\NormalTok{ radio}

    \KeywordTok{def}\NormalTok{ calcular\_area(}\VariableTok{self}\NormalTok{):}
\NormalTok{        area }\OperatorTok{=} \FloatTok{3.14} \OperatorTok{*} \VariableTok{self}\NormalTok{.radio }\OperatorTok{**} \DecValTok{2}
        \ControlFlowTok{return}\NormalTok{ area}

\NormalTok{circulo1 }\OperatorTok{=}\NormalTok{ Circulo(}\DecValTok{5}\NormalTok{)}
\NormalTok{area\_circulo }\OperatorTok{=}\NormalTok{ circulo1.calcular\_area()}
\BuiltInTok{print}\NormalTok{(}\StringTok{"Área del círculo:"}\NormalTok{, area\_circulo)}
\end{Highlighting}
\end{Shaded}

\hypertarget{explicaciuxf3n-53}{%
\section{Explicación:}\label{explicaciuxf3n-53}}

En este ejemplo, se define una clase Circulo con un constructor y un
método calcular\_area.

Dentro del método, se accede al atributo de instancia radio utilizando
self.radio para calcular el área.

\begin{tcolorbox}[enhanced jigsaw, colbacktitle=quarto-callout-important-color!10!white, toprule=.15mm, leftrule=.75mm, titlerule=0mm, opacityback=0, rightrule=.15mm, opacitybacktitle=0.6, breakable, left=2mm, coltitle=black, title=\textcolor{quarto-callout-important-color}{\faExclamation}\hspace{0.5em}{Actividad Práctica:}, toptitle=1mm, bottomtitle=1mm, arc=.35mm, bottomrule=.15mm, colback=white, colframe=quarto-callout-important-color-frame]

Crea una clase Triangulo con atributos base y altura, y un método
calcular\_area que calcule y retorne el área del triángulo.

Crea una instancia de la clase Triangulo y llama al método
calcular\_area para calcular el área.

\end{tcolorbox}

\hypertarget{explicaciuxf3n-de-la-actividad-51}{%
\section{Explicación de la
Actividad:}\label{explicaciuxf3n-de-la-actividad-51}}

Esta actividad permite a los participantes practicar la definición y uso
de métodos en una clase. Les ayuda a comprender cómo trabajar con
atributos de instancia dentro de los métodos y cómo implementar lógica
específica para cada objeto.

\begin{quote}
\begin{quote}
\begin{quote}
\begin{quote}
\begin{quote}
\begin{quote}
\begin{quote}
e8ed08b1a5bbe1e369719187cfc4de7f7e2a41a9
\end{quote}
\end{quote}
\end{quote}
\end{quote}
\end{quote}
\end{quote}
\end{quote}

\hypertarget{self-eliminar-propiedades-y-objetos}{%
\chapter{Self, Eliminar Propiedades y
Objetos}\label{self-eliminar-propiedades-y-objetos}}

\textless\textless\textless\textless\textless\textless\textless{} HEAD

En esta lección, aprenderemos más sobre el uso de self en los métodos de
clase. También exploraremos cómo eliminar atributos de instancia y
objetos en Python.

\hypertarget{conceptos-clave-54}{%
\section{Conceptos Clave:}\label{conceptos-clave-54}}

\hypertarget{self}{%
\subsection{self}\label{self}}

Referencia al objeto actual en un método de clase.

\hypertarget{eliminar-atributos}{%
\subsection{Eliminar Atributos}\label{eliminar-atributos}}

Se puede usar la palabra clave del para eliminar un atributo de
instancia.

\hypertarget{eliminar-objetos}{%
\subsection{Eliminar Objetos}\label{eliminar-objetos}}

Se utiliza la función del para eliminar un objeto y liberar memoria.

\hypertarget{ejemplo-54}{%
\section{Ejemplo:}\label{ejemplo-54}}

\begin{Shaded}
\begin{Highlighting}[]
\KeywordTok{class}\NormalTok{ Coche:}
    \KeywordTok{def} \FunctionTok{\_\_init\_\_}\NormalTok{(}\VariableTok{self}\NormalTok{, marca, modelo):}
        \VariableTok{self}\NormalTok{.marca }\OperatorTok{=}\NormalTok{ marca}
        \VariableTok{self}\NormalTok{.modelo }\OperatorTok{=}\NormalTok{ modelo}

    \KeywordTok{def}\NormalTok{ mostrar\_info(}\VariableTok{self}\NormalTok{):}
        \BuiltInTok{print}\NormalTok{(}\SpecialStringTok{f"Coche }\SpecialCharTok{\{}\VariableTok{self}\SpecialCharTok{.}\NormalTok{marca}\SpecialCharTok{\}}\SpecialStringTok{ }\SpecialCharTok{\{}\VariableTok{self}\SpecialCharTok{.}\NormalTok{modelo}\SpecialCharTok{\}}\SpecialStringTok{"}\NormalTok{)}

\NormalTok{coche1 }\OperatorTok{=}\NormalTok{ Coche(}\StringTok{"Toyota"}\NormalTok{, }\StringTok{"Corolla"}\NormalTok{)}
\NormalTok{coche1.mostrar\_info()}

\CommentTok{\# Eliminar el atributo \textquotesingle{}modelo\textquotesingle{}}
\KeywordTok{del}\NormalTok{ coche1.modelo}

\CommentTok{\# Intentar acceder al atributo eliminado generará un error}
\CommentTok{\# print(coche1.modelo)}
\end{Highlighting}
\end{Shaded}

\hypertarget{explicaciuxf3n-54}{%
\section{Explicación:}\label{explicaciuxf3n-54}}

En este ejemplo, se define una clase Coche con un constructor y un
método mostrar\_info.

Se crea una instancia coche1 y se muestra su información. Luego, se
elimina el atributo modelo utilizando del.

\begin{tcolorbox}[enhanced jigsaw, colbacktitle=quarto-callout-important-color!10!white, toprule=.15mm, leftrule=.75mm, titlerule=0mm, opacityback=0, rightrule=.15mm, opacitybacktitle=0.6, breakable, left=2mm, coltitle=black, title=\textcolor{quarto-callout-important-color}{\faExclamation}\hspace{0.5em}{Actividad Práctica:}, toptitle=1mm, bottomtitle=1mm, arc=.35mm, bottomrule=.15mm, colback=white, colframe=quarto-callout-important-color-frame]

Crea una clase Estudiante con atributos nombre y edad, y un método
mostrar\_info para mostrar la información del estudiante.

Crea una instancia de la clase Estudiante y llama al método
mostrar\_info.

Utiliza del para eliminar el atributo nombre de la instancia y verifica
el resultado.

\end{tcolorbox}

\hypertarget{explicaciuxf3n-de-la-actividad-52}{%
\section{Explicación de la
Actividad:}\label{explicaciuxf3n-de-la-actividad-52}}

Esta actividad permite a los participantes practicar el uso de self en
los métodos de clase y cómo eliminar atributos de instancia. Les ayuda a
comprender cómo trabajar con objetos y atributos, y cómo gestionar la
memoria en Python.

======= \#\# Self, Eliminar Propiedades y Objetos

En esta lección, aprenderemos más sobre el uso de self en los métodos de
clase. También exploraremos cómo eliminar atributos de instancia y
objetos en Python.

\hypertarget{conceptos-clave-55}{%
\section{Conceptos Clave:}\label{conceptos-clave-55}}

\hypertarget{self-1}{%
\subsection{self}\label{self-1}}

Referencia al objeto actual en un método de clase.

\hypertarget{eliminar-atributos-1}{%
\subsection{Eliminar Atributos}\label{eliminar-atributos-1}}

Se puede usar la palabra clave del para eliminar un atributo de
instancia.

\hypertarget{eliminar-objetos-1}{%
\subsection{Eliminar Objetos}\label{eliminar-objetos-1}}

Se utiliza la función del para eliminar un objeto y liberar memoria.

\hypertarget{ejemplo-55}{%
\section{Ejemplo:}\label{ejemplo-55}}

\begin{Shaded}
\begin{Highlighting}[]
\KeywordTok{class}\NormalTok{ Coche:}
    \KeywordTok{def} \FunctionTok{\_\_init\_\_}\NormalTok{(}\VariableTok{self}\NormalTok{, marca, modelo):}
        \VariableTok{self}\NormalTok{.marca }\OperatorTok{=}\NormalTok{ marca}
        \VariableTok{self}\NormalTok{.modelo }\OperatorTok{=}\NormalTok{ modelo}

    \KeywordTok{def}\NormalTok{ mostrar\_info(}\VariableTok{self}\NormalTok{):}
        \BuiltInTok{print}\NormalTok{(}\SpecialStringTok{f"Coche }\SpecialCharTok{\{}\VariableTok{self}\SpecialCharTok{.}\NormalTok{marca}\SpecialCharTok{\}}\SpecialStringTok{ }\SpecialCharTok{\{}\VariableTok{self}\SpecialCharTok{.}\NormalTok{modelo}\SpecialCharTok{\}}\SpecialStringTok{"}\NormalTok{)}

\NormalTok{coche1 }\OperatorTok{=}\NormalTok{ Coche(}\StringTok{"Toyota"}\NormalTok{, }\StringTok{"Corolla"}\NormalTok{)}
\NormalTok{coche1.mostrar\_info()}

\CommentTok{\# Eliminar el atributo \textquotesingle{}modelo\textquotesingle{}}
\KeywordTok{del}\NormalTok{ coche1.modelo}

\CommentTok{\# Intentar acceder al atributo eliminado generará un error}
\CommentTok{\# print(coche1.modelo)}
\end{Highlighting}
\end{Shaded}

\hypertarget{explicaciuxf3n-55}{%
\section{Explicación:}\label{explicaciuxf3n-55}}

En este ejemplo, se define una clase Coche con un constructor y un
método mostrar\_info.

Se crea una instancia coche1 y se muestra su información. Luego, se
elimina el atributo modelo utilizando del.

\begin{tcolorbox}[enhanced jigsaw, colbacktitle=quarto-callout-important-color!10!white, toprule=.15mm, leftrule=.75mm, titlerule=0mm, opacityback=0, rightrule=.15mm, opacitybacktitle=0.6, breakable, left=2mm, coltitle=black, title=\textcolor{quarto-callout-important-color}{\faExclamation}\hspace{0.5em}{Actividad Práctica:}, toptitle=1mm, bottomtitle=1mm, arc=.35mm, bottomrule=.15mm, colback=white, colframe=quarto-callout-important-color-frame]

Crea una clase Estudiante con atributos nombre y edad, y un método
mostrar\_info para mostrar la información del estudiante.

Crea una instancia de la clase Estudiante y llama al método
mostrar\_info.

Utiliza del para eliminar el atributo nombre de la instancia y verifica
el resultado.

\end{tcolorbox}

\hypertarget{explicaciuxf3n-de-la-actividad-53}{%
\section{Explicación de la
Actividad:}\label{explicaciuxf3n-de-la-actividad-53}}

Esta actividad permite a los participantes practicar el uso de self en
los métodos de clase y cómo eliminar atributos de instancia. Les ayuda a
comprender cómo trabajar con objetos y atributos, y cómo gestionar la
memoria en Python.

\begin{quote}
\begin{quote}
\begin{quote}
\begin{quote}
\begin{quote}
\begin{quote}
\begin{quote}
e8ed08b1a5bbe1e369719187cfc4de7f7e2a41a9
\end{quote}
\end{quote}
\end{quote}
\end{quote}
\end{quote}
\end{quote}
\end{quote}

\hypertarget{herencia}{%
\chapter{Herencia}\label{herencia}}

\textless\textless\textless\textless\textless\textless\textless{} HEAD

En esta lección, exploraremos el concepto de herencia en la programación
orientada a objetos. Aprenderemos cómo crear clases que heredan
atributos y métodos de una clase base.

\hypertarget{conceptos-clave-56}{%
\section{Conceptos Clave:}\label{conceptos-clave-56}}

\hypertarget{herencia-1}{%
\subsection{Herencia}\label{herencia-1}}

Mecanismo que permite que una clase herede atributos y métodos de otra
clase base.

\hypertarget{clase-padre-o-base}{%
\subsection{Clase Padre (o Base)}\label{clase-padre-o-base}}

La clase de la que se heredan atributos y métodos.

\hypertarget{clase-hija-o-derivada}{%
\subsection{Clase Hija (o Derivada)}\label{clase-hija-o-derivada}}

La clase que hereda de la clase base.

\hypertarget{ejemplo-56}{%
\section{Ejemplo:}\label{ejemplo-56}}

\begin{Shaded}
\begin{Highlighting}[]
\KeywordTok{class}\NormalTok{ Animal:}
    \KeywordTok{def} \FunctionTok{\_\_init\_\_}\NormalTok{(}\VariableTok{self}\NormalTok{, nombre):}
        \VariableTok{self}\NormalTok{.nombre }\OperatorTok{=}\NormalTok{ nombre}

    \KeywordTok{def}\NormalTok{ saludar(}\VariableTok{self}\NormalTok{):}
        \BuiltInTok{print}\NormalTok{(}\SpecialStringTok{f"}\SpecialCharTok{\{}\VariableTok{self}\SpecialCharTok{.}\NormalTok{nombre}\SpecialCharTok{\}}\SpecialStringTok{ saluda"}\NormalTok{)}

\KeywordTok{class}\NormalTok{ Perro(Animal):}
    \KeywordTok{def}\NormalTok{ ladrar(}\VariableTok{self}\NormalTok{):}
        \BuiltInTok{print}\NormalTok{(}\SpecialStringTok{f"}\SpecialCharTok{\{}\VariableTok{self}\SpecialCharTok{.}\NormalTok{nombre}\SpecialCharTok{\}}\SpecialStringTok{ está ladrando"}\NormalTok{)}

\NormalTok{perro1 }\OperatorTok{=}\NormalTok{ Perro(}\StringTok{"Buddy"}\NormalTok{)}
\NormalTok{perro1.saludar()}
\NormalTok{perro1.ladrar()}
\end{Highlighting}
\end{Shaded}

\hypertarget{explicaciuxf3n-56}{%
\subsection{Explicación:}\label{explicaciuxf3n-56}}

En este ejemplo, se define una clase base Animal con un constructor y un
método saludar.

Se define una clase derivada Perro que hereda de Animal y agrega un
método adicional ladrar.

Se crea una instancia perro1 de la clase Perro y se llama a sus métodos.

\begin{tcolorbox}[enhanced jigsaw, colbacktitle=quarto-callout-important-color!10!white, toprule=.15mm, leftrule=.75mm, titlerule=0mm, opacityback=0, rightrule=.15mm, opacitybacktitle=0.6, breakable, left=2mm, coltitle=black, title=\textcolor{quarto-callout-important-color}{\faExclamation}\hspace{0.5em}{Actividad Práctica:}, toptitle=1mm, bottomtitle=1mm, arc=.35mm, bottomrule=.15mm, colback=white, colframe=quarto-callout-important-color-frame]

Crea una clase Figura con un atributo color y un método mostrar\_color
para mostrar el color.

Crea una clase derivada Circulo que herede de Figura y agrega un
atributo radio y un método calcular\_area para calcular el área del
círculo.

Crea una instancia de la clase Circulo, establece su color y calcula el
área.

\end{tcolorbox}

\hypertarget{explicaciuxf3n-de-la-actividad-54}{%
\section{Explicación de la
Actividad:}\label{explicaciuxf3n-de-la-actividad-54}}

======= \#\# Herencia

En esta lección, exploraremos el concepto de herencia en la programación
orientada a objetos. Aprenderemos cómo crear clases que heredan
atributos y métodos de una clase base.

\hypertarget{conceptos-clave-57}{%
\section{Conceptos Clave:}\label{conceptos-clave-57}}

\hypertarget{herencia-2}{%
\subsection{Herencia}\label{herencia-2}}

Mecanismo que permite que una clase herede atributos y métodos de otra
clase base.

\hypertarget{clase-padre-o-base-1}{%
\subsection{Clase Padre (o Base)}\label{clase-padre-o-base-1}}

La clase de la que se heredan atributos y métodos.

\hypertarget{clase-hija-o-derivada-1}{%
\subsection{Clase Hija (o Derivada)}\label{clase-hija-o-derivada-1}}

La clase que hereda de la clase base.

\hypertarget{ejemplo-57}{%
\section{Ejemplo:}\label{ejemplo-57}}

\begin{Shaded}
\begin{Highlighting}[]
\KeywordTok{class}\NormalTok{ Animal:}
    \KeywordTok{def} \FunctionTok{\_\_init\_\_}\NormalTok{(}\VariableTok{self}\NormalTok{, nombre):}
        \VariableTok{self}\NormalTok{.nombre }\OperatorTok{=}\NormalTok{ nombre}

    \KeywordTok{def}\NormalTok{ saludar(}\VariableTok{self}\NormalTok{):}
        \BuiltInTok{print}\NormalTok{(}\SpecialStringTok{f"}\SpecialCharTok{\{}\VariableTok{self}\SpecialCharTok{.}\NormalTok{nombre}\SpecialCharTok{\}}\SpecialStringTok{ saluda"}\NormalTok{)}

\KeywordTok{class}\NormalTok{ Perro(Animal):}
    \KeywordTok{def}\NormalTok{ ladrar(}\VariableTok{self}\NormalTok{):}
        \BuiltInTok{print}\NormalTok{(}\SpecialStringTok{f"}\SpecialCharTok{\{}\VariableTok{self}\SpecialCharTok{.}\NormalTok{nombre}\SpecialCharTok{\}}\SpecialStringTok{ está ladrando"}\NormalTok{)}

\NormalTok{perro1 }\OperatorTok{=}\NormalTok{ Perro(}\StringTok{"Buddy"}\NormalTok{)}
\NormalTok{perro1.saludar()}
\NormalTok{perro1.ladrar()}
\end{Highlighting}
\end{Shaded}

\hypertarget{explicaciuxf3n-57}{%
\subsection{Explicación:}\label{explicaciuxf3n-57}}

En este ejemplo, se define una clase base Animal con un constructor y un
método saludar.

Se define una clase derivada Perro que hereda de Animal y agrega un
método adicional ladrar.

Se crea una instancia perro1 de la clase Perro y se llama a sus métodos.

\begin{tcolorbox}[enhanced jigsaw, colbacktitle=quarto-callout-important-color!10!white, toprule=.15mm, leftrule=.75mm, titlerule=0mm, opacityback=0, rightrule=.15mm, opacitybacktitle=0.6, breakable, left=2mm, coltitle=black, title=\textcolor{quarto-callout-important-color}{\faExclamation}\hspace{0.5em}{Actividad Práctica:}, toptitle=1mm, bottomtitle=1mm, arc=.35mm, bottomrule=.15mm, colback=white, colframe=quarto-callout-important-color-frame]

Crea una clase Figura con un atributo color y un método mostrar\_color
para mostrar el color.

Crea una clase derivada Circulo que herede de Figura y agrega un
atributo radio y un método calcular\_area para calcular el área del
círculo.

Crea una instancia de la clase Circulo, establece su color y calcula el
área.

\end{tcolorbox}

\hypertarget{explicaciuxf3n-de-la-actividad-55}{%
\section{Explicación de la
Actividad:}\label{explicaciuxf3n-de-la-actividad-55}}

\begin{quote}
\begin{quote}
\begin{quote}
\begin{quote}
\begin{quote}
\begin{quote}
\begin{quote}
e8ed08b1a5bbe1e369719187cfc4de7f7e2a41a9 Esta actividad permite a los
participantes practicar la creación de clases derivadas y la herencia de
atributos y métodos. Les ayuda a comprender cómo utilizar la herencia
para reutilizar código y extender funcionalidades en las clases
derivadas.
\end{quote}
\end{quote}
\end{quote}
\end{quote}
\end{quote}
\end{quote}
\end{quote}

\part{Unidad 8: Módulos}

\hypertarget{introducciuxf3n-1}{%
\chapter{Introducción}\label{introducciuxf3n-1}}

\textless\textless\textless\textless\textless\textless\textless{} HEAD

En esta lección, exploraremos cómo trabajar con módulos en Python.
Aprenderemos cómo dividir nuestro código en módulos reutilizables y cómo
importarlos en otros programas.

\hypertarget{conceptos-clave-58}{%
\section{Conceptos Clave:}\label{conceptos-clave-58}}

\hypertarget{muxf3dulos}{%
\subsection{Módulos}\label{muxf3dulos}}

Archivos que contienen código Python y se utilizan para organizar y
reutilizar funciones, clases y variables.

\hypertarget{importar-muxf3dulos}{%
\subsection{Importar Módulos}\label{importar-muxf3dulos}}

Se utiliza la palabra clave import para cargar un módulo en un programa.

\hypertarget{usar-funciones-y-clases}{%
\subsection{Usar Funciones y Clases}\label{usar-funciones-y-clases}}

Después de importar un módulo, sus funciones y clases pueden ser
utilizadas como si estuvieran definidas en el mismo archivo.

\hypertarget{ejemplo-58}{%
\section{Ejemplo:}\label{ejemplo-58}}

\begin{Shaded}
\begin{Highlighting}[]
\CommentTok{\# En el archivo calculadora.py}
\KeywordTok{def}\NormalTok{ suma(a, b):}
    \ControlFlowTok{return}\NormalTok{ a }\OperatorTok{+}\NormalTok{ b}

\CommentTok{\# En otro archivo}
\ImportTok{import}\NormalTok{ calculadora}

\NormalTok{resultado }\OperatorTok{=}\NormalTok{ calculadora.suma(}\DecValTok{3}\NormalTok{, }\DecValTok{5}\NormalTok{)}
\BuiltInTok{print}\NormalTok{(}\StringTok{"Resultado:"}\NormalTok{, resultado)}
\end{Highlighting}
\end{Shaded}

\hypertarget{explicaciuxf3n-58}{%
\section{Explicación:}\label{explicaciuxf3n-58}}

En este ejemplo, se define una función suma en el módulo calculadora.py.

En otro archivo, se importa el módulo calculadora utilizando import y se
utiliza la función suma del módulo.

\begin{tcolorbox}[enhanced jigsaw, colbacktitle=quarto-callout-important-color!10!white, toprule=.15mm, leftrule=.75mm, titlerule=0mm, opacityback=0, rightrule=.15mm, opacitybacktitle=0.6, breakable, left=2mm, coltitle=black, title=\textcolor{quarto-callout-important-color}{\faExclamation}\hspace{0.5em}{Actividad Práctica:}, toptitle=1mm, bottomtitle=1mm, arc=.35mm, bottomrule=.15mm, colback=white, colframe=quarto-callout-important-color-frame]

Crea un módulo llamado matematicas con una función multiplicacion que
multiplique dos números.

Importa el módulo en otro archivo y utiliza la función multiplicacion
para calcular el producto de dos números.

\end{tcolorbox}

\hypertarget{explicaciuxf3n-de-la-actividad-56}{%
\section{Explicación de la
Actividad:}\label{explicaciuxf3n-de-la-actividad-56}}

Esta actividad permite a los participantes practicar la creación y uso
de módulos en Python. Les ayuda a comprender cómo organizar su código en
módulos reutilizables y cómo importar funciones y clases desde otros
archivos.

======= \#\# Introducción

En esta lección, exploraremos cómo trabajar con módulos en Python.
Aprenderemos cómo dividir nuestro código en módulos reutilizables y cómo
importarlos en otros programas.

\hypertarget{conceptos-clave-59}{%
\section{Conceptos Clave:}\label{conceptos-clave-59}}

\hypertarget{muxf3dulos-1}{%
\subsection{Módulos}\label{muxf3dulos-1}}

Archivos que contienen código Python y se utilizan para organizar y
reutilizar funciones, clases y variables.

\hypertarget{importar-muxf3dulos-1}{%
\subsection{Importar Módulos}\label{importar-muxf3dulos-1}}

Se utiliza la palabra clave import para cargar un módulo en un programa.

\hypertarget{usar-funciones-y-clases-1}{%
\subsection{Usar Funciones y Clases}\label{usar-funciones-y-clases-1}}

Después de importar un módulo, sus funciones y clases pueden ser
utilizadas como si estuvieran definidas en el mismo archivo.

\hypertarget{ejemplo-59}{%
\section{Ejemplo:}\label{ejemplo-59}}

\begin{Shaded}
\begin{Highlighting}[]
\CommentTok{\# En el archivo calculadora.py}
\KeywordTok{def}\NormalTok{ suma(a, b):}
    \ControlFlowTok{return}\NormalTok{ a }\OperatorTok{+}\NormalTok{ b}

\CommentTok{\# En otro archivo}
\ImportTok{import}\NormalTok{ calculadora}

\NormalTok{resultado }\OperatorTok{=}\NormalTok{ calculadora.suma(}\DecValTok{3}\NormalTok{, }\DecValTok{5}\NormalTok{)}
\BuiltInTok{print}\NormalTok{(}\StringTok{"Resultado:"}\NormalTok{, resultado)}
\end{Highlighting}
\end{Shaded}

\hypertarget{explicaciuxf3n-59}{%
\section{Explicación:}\label{explicaciuxf3n-59}}

En este ejemplo, se define una función suma en el módulo calculadora.py.

En otro archivo, se importa el módulo calculadora utilizando import y se
utiliza la función suma del módulo.

\begin{tcolorbox}[enhanced jigsaw, colbacktitle=quarto-callout-important-color!10!white, toprule=.15mm, leftrule=.75mm, titlerule=0mm, opacityback=0, rightrule=.15mm, opacitybacktitle=0.6, breakable, left=2mm, coltitle=black, title=\textcolor{quarto-callout-important-color}{\faExclamation}\hspace{0.5em}{Actividad Práctica:}, toptitle=1mm, bottomtitle=1mm, arc=.35mm, bottomrule=.15mm, colback=white, colframe=quarto-callout-important-color-frame]

Crea un módulo llamado matematicas con una función multiplicacion que
multiplique dos números.

Importa el módulo en otro archivo y utiliza la función multiplicacion
para calcular el producto de dos números.

\end{tcolorbox}

\hypertarget{explicaciuxf3n-de-la-actividad-57}{%
\section{Explicación de la
Actividad:}\label{explicaciuxf3n-de-la-actividad-57}}

Esta actividad permite a los participantes practicar la creación y uso
de módulos en Python. Les ayuda a comprender cómo organizar su código en
módulos reutilizables y cómo importar funciones y clases desde otros
archivos.

\begin{quote}
\begin{quote}
\begin{quote}
\begin{quote}
\begin{quote}
\begin{quote}
\begin{quote}
e8ed08b1a5bbe1e369719187cfc4de7f7e2a41a9
\end{quote}
\end{quote}
\end{quote}
\end{quote}
\end{quote}
\end{quote}
\end{quote}

\hypertarget{creando-nuestro-primer-muxf3dulo}{%
\chapter{Creando Nuestro Primer
Módulo}\label{creando-nuestro-primer-muxf3dulo}}

\textless\textless\textless\textless\textless\textless\textless{} HEAD

En esta lección, aprenderemos a crear nuestro propio módulo en Python.
Crearemos un módulo que contenga funciones y clases para realizar
operaciones matemáticas básicas.

\hypertarget{pasos-para-crear-un-muxf3dulo}{%
\section{Pasos para Crear un
Módulo:}\label{pasos-para-crear-un-muxf3dulo}}

Crea un archivo de Python con la extensión .py.

Define funciones y clases en el archivo.

Guarda el archivo en una ubicación accesible.

\hypertarget{ejemplo-60}{%
\section{Ejemplo:}\label{ejemplo-60}}

\begin{Shaded}
\begin{Highlighting}[]
\CommentTok{\# En el archivo operaciones.py}
\KeywordTok{def}\NormalTok{ suma(a, b):}
    \ControlFlowTok{return}\NormalTok{ a }\OperatorTok{+}\NormalTok{ b}

\KeywordTok{def}\NormalTok{ resta(a, b):}
    \ControlFlowTok{return}\NormalTok{ a }\OperatorTok{{-}}\NormalTok{ b}

\KeywordTok{class}\NormalTok{ Calculadora:}
    \KeywordTok{def}\NormalTok{ multiplicacion(}\VariableTok{self}\NormalTok{, a, b):}
        \ControlFlowTok{return}\NormalTok{ a }\OperatorTok{*}\NormalTok{ b}
\end{Highlighting}
\end{Shaded}

\hypertarget{explicaciuxf3n-60}{%
\section{Explicación:}\label{explicaciuxf3n-60}}

En este ejemplo, se crea un módulo llamado operaciones.py.

Se define una función suma y una función resta, junto con una clase
Calculadora que tiene un método multiplicacion.

\begin{tcolorbox}[enhanced jigsaw, colbacktitle=quarto-callout-important-color!10!white, toprule=.15mm, leftrule=.75mm, titlerule=0mm, opacityback=0, rightrule=.15mm, opacitybacktitle=0.6, breakable, left=2mm, coltitle=black, title=\textcolor{quarto-callout-important-color}{\faExclamation}\hspace{0.5em}{Actividad Práctica:}, toptitle=1mm, bottomtitle=1mm, arc=.35mm, bottomrule=.15mm, colback=white, colframe=quarto-callout-important-color-frame]

Crea un módulo llamado geometria con funciones para calcular el área de
un círculo y el perímetro de un cuadrado.

En otro archivo, importa el módulo geometria y utiliza las funciones
para realizar cálculos geométricos.

\end{tcolorbox}

\hypertarget{explicaciuxf3n-de-la-actividad-58}{%
\section{Explicación de la
Actividad:}\label{explicaciuxf3n-de-la-actividad-58}}

Esta actividad permite a los participantes practicar la creación de
módulos con funciones y clases. Les ayuda a comprender cómo organizar
diferentes funcionalidades en módulos separados y cómo importar esas
funcionalidades en otros archivos.

======= \#\# Creando Nuestro Primer Módulo

En esta lección, aprenderemos a crear nuestro propio módulo en Python.
Crearemos un módulo que contenga funciones y clases para realizar
operaciones matemáticas básicas.

\hypertarget{pasos-para-crear-un-muxf3dulo-1}{%
\section{Pasos para Crear un
Módulo:}\label{pasos-para-crear-un-muxf3dulo-1}}

Crea un archivo de Python con la extensión .py.

Define funciones y clases en el archivo.

Guarda el archivo en una ubicación accesible.

\hypertarget{ejemplo-61}{%
\section{Ejemplo:}\label{ejemplo-61}}

\begin{Shaded}
\begin{Highlighting}[]
\CommentTok{\# En el archivo operaciones.py}
\KeywordTok{def}\NormalTok{ suma(a, b):}
    \ControlFlowTok{return}\NormalTok{ a }\OperatorTok{+}\NormalTok{ b}

\KeywordTok{def}\NormalTok{ resta(a, b):}
    \ControlFlowTok{return}\NormalTok{ a }\OperatorTok{{-}}\NormalTok{ b}

\KeywordTok{class}\NormalTok{ Calculadora:}
    \KeywordTok{def}\NormalTok{ multiplicacion(}\VariableTok{self}\NormalTok{, a, b):}
        \ControlFlowTok{return}\NormalTok{ a }\OperatorTok{*}\NormalTok{ b}
\end{Highlighting}
\end{Shaded}

\hypertarget{explicaciuxf3n-61}{%
\section{Explicación:}\label{explicaciuxf3n-61}}

En este ejemplo, se crea un módulo llamado operaciones.py.

Se define una función suma y una función resta, junto con una clase
Calculadora que tiene un método multiplicacion.

\begin{tcolorbox}[enhanced jigsaw, colbacktitle=quarto-callout-important-color!10!white, toprule=.15mm, leftrule=.75mm, titlerule=0mm, opacityback=0, rightrule=.15mm, opacitybacktitle=0.6, breakable, left=2mm, coltitle=black, title=\textcolor{quarto-callout-important-color}{\faExclamation}\hspace{0.5em}{Actividad Práctica:}, toptitle=1mm, bottomtitle=1mm, arc=.35mm, bottomrule=.15mm, colback=white, colframe=quarto-callout-important-color-frame]

Crea un módulo llamado geometria con funciones para calcular el área de
un círculo y el perímetro de un cuadrado.

En otro archivo, importa el módulo geometria y utiliza las funciones
para realizar cálculos geométricos.

\end{tcolorbox}

\hypertarget{explicaciuxf3n-de-la-actividad-59}{%
\section{Explicación de la
Actividad:}\label{explicaciuxf3n-de-la-actividad-59}}

Esta actividad permite a los participantes practicar la creación de
módulos con funciones y clases. Les ayuda a comprender cómo organizar
diferentes funcionalidades en módulos separados y cómo importar esas
funcionalidades en otros archivos.

\begin{quote}
\begin{quote}
\begin{quote}
\begin{quote}
\begin{quote}
\begin{quote}
\begin{quote}
e8ed08b1a5bbe1e369719187cfc4de7f7e2a41a9
\end{quote}
\end{quote}
\end{quote}
\end{quote}
\end{quote}
\end{quote}
\end{quote}

\hypertarget{renombrando-muxf3dulos}{%
\chapter{Renombrando Módulos}\label{renombrando-muxf3dulos}}

\textless\textless\textless\textless\textless\textless\textless{} HEAD

En esta lección, aprenderemos cómo renombrar módulos al importarlos y
cómo seleccionar elementos específicos para importar. Esto nos permitirá
tener un mayor control sobre los nombres y las funcionalidades que
utilizamos en nuestro código.

\hypertarget{renombrando-muxf3dulos-al-importar}{%
\section{Renombrando Módulos al
Importar:}\label{renombrando-muxf3dulos-al-importar}}

\begin{Shaded}
\begin{Highlighting}[]
\ImportTok{import}\NormalTok{ modulo\_largo }\ImportTok{as}\NormalTok{ ml}
\end{Highlighting}
\end{Shaded}

\hypertarget{seleccionando-elementos-especuxedficos-para-importar}{%
\section{Seleccionando Elementos Específicos para
Importar:}\label{seleccionando-elementos-especuxedficos-para-importar}}

\begin{Shaded}
\begin{Highlighting}[]
\ImportTok{from}\NormalTok{ modulo }\ImportTok{import}\NormalTok{ funcion1, funcion2}
\end{Highlighting}
\end{Shaded}

\hypertarget{ejemplo---renombrando-muxf3dulos}{%
\section{Ejemplo - Renombrando
Módulos:}\label{ejemplo---renombrando-muxf3dulos}}

\begin{Shaded}
\begin{Highlighting}[]
\ImportTok{import}\NormalTok{ calculadora }\ImportTok{as}\NormalTok{ calc}

\NormalTok{resultado }\OperatorTok{=}\NormalTok{ calc.suma(}\DecValTok{3}\NormalTok{, }\DecValTok{4}\NormalTok{)}
\end{Highlighting}
\end{Shaded}

\hypertarget{ejemplo---seleccionando-elementos-especuxedficos}{%
\section{Ejemplo - Seleccionando Elementos
Específicos:}\label{ejemplo---seleccionando-elementos-especuxedficos}}

\begin{Shaded}
\begin{Highlighting}[]
\ImportTok{from}\NormalTok{ operaciones }\ImportTok{import}\NormalTok{ resta, Calculadora}

\NormalTok{resultado }\OperatorTok{=}\NormalTok{ resta(}\DecValTok{10}\NormalTok{, }\DecValTok{5}\NormalTok{)}
\end{Highlighting}
\end{Shaded}

:::\{.callout-important\} \#\#\# Actividad Práctica:

Renombra el módulo geometria como geo al importarlo en otro archivo.

Importa solo la función para calcular el área de un círculo y calcula el
área de un círculo con radio 5.

\hypertarget{explicaciuxf3n-de-la-actividad-60}{%
\section{Explicación de la
Actividad:}\label{explicaciuxf3n-de-la-actividad-60}}

\hypertarget{esta-actividad-permite-a-los-participantes-practicar-cuxf3mo-renombrar-muxf3dulos-al-importarlos-y-cuxf3mo-seleccionar-funciones-especuxedficas-para-importar.-les-ayuda-a-comprender-cuxf3mo-personalizar-los-nombres-de-los-muxf3dulos-y-cuxf3mo-importar-solo-las-funcionalidades-que-necesitan-en-su-cuxf3digo.}{%
\chapter{Esta actividad permite a los participantes practicar cómo
renombrar módulos al importarlos y cómo seleccionar funciones
específicas para importar. Les ayuda a comprender cómo personalizar los
nombres de los módulos y cómo importar solo las funcionalidades que
necesitan en su
código.}\label{esta-actividad-permite-a-los-participantes-practicar-cuxf3mo-renombrar-muxf3dulos-al-importarlos-y-cuxf3mo-seleccionar-funciones-especuxedficas-para-importar.-les-ayuda-a-comprender-cuxf3mo-personalizar-los-nombres-de-los-muxf3dulos-y-cuxf3mo-importar-solo-las-funcionalidades-que-necesitan-en-su-cuxf3digo.}}

\hypertarget{renombrando-muxf3dulos-1}{%
\section{Renombrando Módulos}\label{renombrando-muxf3dulos-1}}

En esta lección, aprenderemos cómo renombrar módulos al importarlos y
cómo seleccionar elementos específicos para importar. Esto nos permitirá
tener un mayor control sobre los nombres y las funcionalidades que
utilizamos en nuestro código.

\hypertarget{renombrando-muxf3dulos-al-importar-1}{%
\section{Renombrando Módulos al
Importar:}\label{renombrando-muxf3dulos-al-importar-1}}

\begin{Shaded}
\begin{Highlighting}[]
\ImportTok{import}\NormalTok{ modulo\_largo }\ImportTok{as}\NormalTok{ ml}
\end{Highlighting}
\end{Shaded}

\hypertarget{seleccionando-elementos-especuxedficos-para-importar-1}{%
\section{Seleccionando Elementos Específicos para
Importar:}\label{seleccionando-elementos-especuxedficos-para-importar-1}}

\begin{Shaded}
\begin{Highlighting}[]
\ImportTok{from}\NormalTok{ modulo }\ImportTok{import}\NormalTok{ funcion1, funcion2}
\end{Highlighting}
\end{Shaded}

\hypertarget{ejemplo---renombrando-muxf3dulos-1}{%
\section{Ejemplo - Renombrando
Módulos:}\label{ejemplo---renombrando-muxf3dulos-1}}

\begin{Shaded}
\begin{Highlighting}[]
\ImportTok{import}\NormalTok{ calculadora }\ImportTok{as}\NormalTok{ calc}

\NormalTok{resultado }\OperatorTok{=}\NormalTok{ calc.suma(}\DecValTok{3}\NormalTok{, }\DecValTok{4}\NormalTok{)}
\end{Highlighting}
\end{Shaded}

\hypertarget{ejemplo---seleccionando-elementos-especuxedficos-1}{%
\section{Ejemplo - Seleccionando Elementos
Específicos:}\label{ejemplo---seleccionando-elementos-especuxedficos-1}}

\begin{Shaded}
\begin{Highlighting}[]
\ImportTok{from}\NormalTok{ operaciones }\ImportTok{import}\NormalTok{ resta, Calculadora}

\NormalTok{resultado }\OperatorTok{=}\NormalTok{ resta(}\DecValTok{10}\NormalTok{, }\DecValTok{5}\NormalTok{)}
\end{Highlighting}
\end{Shaded}

:::\{.callout-important\} \#\#\# Actividad Práctica:

Renombra el módulo geometria como geo al importarlo en otro archivo.

Importa solo la función para calcular el área de un círculo y calcula el
área de un círculo con radio 5.

\hypertarget{explicaciuxf3n-de-la-actividad-61}{%
\section{Explicación de la
Actividad:}\label{explicaciuxf3n-de-la-actividad-61}}

Esta actividad permite a los participantes practicar cómo renombrar
módulos al importarlos y cómo seleccionar funciones específicas para
importar. Les ayuda a comprender cómo personalizar los nombres de los
módulos y cómo importar solo las funcionalidades que necesitan en su
código.
\textgreater\textgreater\textgreater\textgreater\textgreater\textgreater\textgreater{}
e8ed08b1a5bbe1e369719187cfc4de7f7e2a41a9

\hypertarget{seleccionando-lo-importado-y-pip}{%
\chapter{Seleccionando lo Importado y
Pip}\label{seleccionando-lo-importado-y-pip}}

\textless\textless\textless\textless\textless\textless\textless{} HEAD

En esta lección, continuaremos explorando cómo seleccionar elementos
específicos para importar y aprenderemos sobre pip, la herramienta de
gestión de paquetes de Python. pip nos permite instalar y gestionar
paquetes externos que contienen funcionalidades adicionales para
nuestros programas.

\hypertarget{seleccionando-elementos-especuxedficos-para-importar-2}{%
\section{Seleccionando Elementos Específicos para
Importar:}\label{seleccionando-elementos-especuxedficos-para-importar-2}}

\begin{Shaded}
\begin{Highlighting}[]
\ImportTok{from}\NormalTok{ modulo }\ImportTok{import}\NormalTok{ funcion1, funcion2}
\end{Highlighting}
\end{Shaded}

\hypertarget{usando-pip}{%
\subsection{Usando Pip:}\label{usando-pip}}

\begin{Shaded}
\begin{Highlighting}[]
\ExtensionTok{pip}\NormalTok{ install nombre\_del\_paquete: }\CommentTok{\#Instalar un paquete.}
\ExtensionTok{pip}\NormalTok{ uninstall nombre\_del\_paquete: }\CommentTok{\#Desinstalar un paquete.}
\end{Highlighting}
\end{Shaded}

\hypertarget{ejemplo---instalando-un-paquete-con-pip}{%
\section{Ejemplo - Instalando un Paquete con
Pip:}\label{ejemplo---instalando-un-paquete-con-pip}}

\begin{Shaded}
\begin{Highlighting}[]
\ExtensionTok{pip}\NormalTok{ install requests}
\end{Highlighting}
\end{Shaded}

\begin{tcolorbox}[enhanced jigsaw, colbacktitle=quarto-callout-important-color!10!white, toprule=.15mm, leftrule=.75mm, titlerule=0mm, opacityback=0, rightrule=.15mm, opacitybacktitle=0.6, breakable, left=2mm, coltitle=black, title=\textcolor{quarto-callout-important-color}{\faExclamation}\hspace{0.5em}{Actividad Práctica:}, toptitle=1mm, bottomtitle=1mm, arc=.35mm, bottomrule=.15mm, colback=white, colframe=quarto-callout-important-color-frame]

Utiliza pip para instalar el paquete matplotlib, que se utiliza para
trazar gráficos en Python.

En tu archivo de código, importa la función plot de matplotlib.pyplot y
crea un gráfico simple.

\end{tcolorbox}

\hypertarget{explicaciuxf3n-de-la-actividad-62}{%
\section{Explicación de la
Actividad:}\label{explicaciuxf3n-de-la-actividad-62}}

======= \#\# Seleccionando lo Importado y Pip

En esta lección, continuaremos explorando cómo seleccionar elementos
específicos para importar y aprenderemos sobre pip, la herramienta de
gestión de paquetes de Python. pip nos permite instalar y gestionar
paquetes externos que contienen funcionalidades adicionales para
nuestros programas.

\hypertarget{seleccionando-elementos-especuxedficos-para-importar-3}{%
\section{Seleccionando Elementos Específicos para
Importar:}\label{seleccionando-elementos-especuxedficos-para-importar-3}}

\begin{Shaded}
\begin{Highlighting}[]
\ImportTok{from}\NormalTok{ modulo }\ImportTok{import}\NormalTok{ funcion1, funcion2}
\end{Highlighting}
\end{Shaded}

\hypertarget{usando-pip-1}{%
\subsection{Usando Pip:}\label{usando-pip-1}}

\begin{Shaded}
\begin{Highlighting}[]
\ExtensionTok{pip}\NormalTok{ install nombre\_del\_paquete: }\CommentTok{\#Instalar un paquete.}
\ExtensionTok{pip}\NormalTok{ uninstall nombre\_del\_paquete: }\CommentTok{\#Desinstalar un paquete.}
\end{Highlighting}
\end{Shaded}

\hypertarget{ejemplo---instalando-un-paquete-con-pip-1}{%
\section{Ejemplo - Instalando un Paquete con
Pip:}\label{ejemplo---instalando-un-paquete-con-pip-1}}

\begin{Shaded}
\begin{Highlighting}[]
\ExtensionTok{pip}\NormalTok{ install requests}
\end{Highlighting}
\end{Shaded}

\begin{tcolorbox}[enhanced jigsaw, colbacktitle=quarto-callout-important-color!10!white, toprule=.15mm, leftrule=.75mm, titlerule=0mm, opacityback=0, rightrule=.15mm, opacitybacktitle=0.6, breakable, left=2mm, coltitle=black, title=\textcolor{quarto-callout-important-color}{\faExclamation}\hspace{0.5em}{Actividad Práctica:}, toptitle=1mm, bottomtitle=1mm, arc=.35mm, bottomrule=.15mm, colback=white, colframe=quarto-callout-important-color-frame]

Utiliza pip para instalar el paquete matplotlib, que se utiliza para
trazar gráficos en Python.

En tu archivo de código, importa la función plot de matplotlib.pyplot y
crea un gráfico simple.

\end{tcolorbox}

\hypertarget{explicaciuxf3n-de-la-actividad-63}{%
\section{Explicación de la
Actividad:}\label{explicaciuxf3n-de-la-actividad-63}}

\begin{quote}
\begin{quote}
\begin{quote}
\begin{quote}
\begin{quote}
\begin{quote}
\begin{quote}
e8ed08b1a5bbe1e369719187cfc4de7f7e2a41a9 Esta actividad permite a los
participantes practicar cómo utilizar pip para instalar paquetes
externos y cómo importar funcionalidades específicas de esos paquetes en
su código. Les ayuda a comprender cómo expandir las capacidades de
Python utilizando bibliotecas externas.
\end{quote}
\end{quote}
\end{quote}
\end{quote}
\end{quote}
\end{quote}
\end{quote}

\part{Unidad 9: Introducción a Bases de Datos}

\hypertarget{introducciuxf3n-a-bases-de-datos}{%
\chapter{Introducción a Bases de
Datos}\label{introducciuxf3n-a-bases-de-datos}}

\textless\textless\textless\textless\textless\textless\textless{} HEAD

En esta lección, exploraremos el concepto de bases de datos y su
importancia en el desarrollo de aplicaciones. Aprenderemos cómo las
bases de datos nos permiten almacenar y recuperar información de manera
eficiente.

\hypertarget{conceptos-clave-60}{%
\section{Conceptos Clave:}\label{conceptos-clave-60}}

\hypertarget{base-de-datos}{%
\subsection{Base de Datos}\label{base-de-datos}}

Colección organizada de datos almacenados en formato estructurado.

\hypertarget{sistemas-de-gestiuxf3n-de-bases-de-datos-dbms}{%
\subsection{Sistemas de Gestión de Bases de Datos
(DBMS)}\label{sistemas-de-gestiuxf3n-de-bases-de-datos-dbms}}

Software que administra y gestiona una base de datos.

\hypertarget{beneficios-de-las-bases-de-datos}{%
\subsection{Beneficios de las Bases de
Datos}\label{beneficios-de-las-bases-de-datos}}

Almacenamiento eficiente, acceso rápido y seguridad de datos.

\hypertarget{ejemplo-62}{%
\section{Ejemplo:}\label{ejemplo-62}}

\begin{Shaded}
\begin{Highlighting}[]
\CommentTok{\# Ejemplo de una tabla \textquotesingle{}usuarios\textquotesingle{} en una base de datos}
\OperatorTok{|} \BuiltInTok{id} \OperatorTok{|}\NormalTok{ nombre   }\OperatorTok{|}\NormalTok{ edad }\OperatorTok{|}\NormalTok{ email           }\OperatorTok{|}
\OperatorTok{|{-}{-}{-}{-}|{-}{-}{-}{-}{-}{-}{-}{-}{-}{-}|{-}{-}{-}{-}{-}{-}|{-}{-}{-}{-}{-}{-}{-}{-}{-}{-}{-}{-}{-}{-}{-}{-}{-}|}
\OperatorTok{|} \DecValTok{1}  \OperatorTok{|}\NormalTok{ Juan     }\OperatorTok{|} \DecValTok{25}   \OperatorTok{|}\NormalTok{ juan}\OperatorTok{@}\NormalTok{email.com }\OperatorTok{|}
\OperatorTok{|} \DecValTok{2}  \OperatorTok{|}\NormalTok{ María    }\OperatorTok{|} \DecValTok{30}   \OperatorTok{|}\NormalTok{ maria}\OperatorTok{@}\NormalTok{email.com}\OperatorTok{|}
\end{Highlighting}
\end{Shaded}

\hypertarget{explicaciuxf3n-62}{%
\section{Explicación:}\label{explicaciuxf3n-62}}

En este ejemplo, se muestra una tabla ficticia de una base de datos
llamada `usuarios'.

La tabla contiene filas que representan registros de usuarios con
diferentes atributos.

\begin{tcolorbox}[enhanced jigsaw, colbacktitle=quarto-callout-important-color!10!white, toprule=.15mm, leftrule=.75mm, titlerule=0mm, opacityback=0, rightrule=.15mm, opacitybacktitle=0.6, breakable, left=2mm, coltitle=black, title=\textcolor{quarto-callout-important-color}{\faExclamation}\hspace{0.5em}{Actividad Práctica:}, toptitle=1mm, bottomtitle=1mm, arc=.35mm, bottomrule=.15mm, colback=white, colframe=quarto-callout-important-color-frame]

Investiga y elige un Sistema de Gestión de Bases de Datos (DBMS) para
utilizar en el curso.

Explica por qué es importante utilizar bases de datos en el desarrollo
de aplicaciones.

\end{tcolorbox}

\hypertarget{explicaciuxf3n-de-la-actividad-64}{%
\section{Explicación de la
Actividad:}\label{explicaciuxf3n-de-la-actividad-64}}

Esta actividad permite a los participantes comprender la importancia de
las bases de datos en el desarrollo de aplicaciones y seleccionar una
opción adecuada de DBMS para usar en el curso. Les ayuda a
familiarizarse con el concepto de bases de datos y sus beneficios.

======= \#\# Introducción a Bases de Datos

En esta lección, exploraremos el concepto de bases de datos y su
importancia en el desarrollo de aplicaciones. Aprenderemos cómo las
bases de datos nos permiten almacenar y recuperar información de manera
eficiente.

\hypertarget{conceptos-clave-61}{%
\section{Conceptos Clave:}\label{conceptos-clave-61}}

\hypertarget{base-de-datos-1}{%
\subsection{Base de Datos}\label{base-de-datos-1}}

Colección organizada de datos almacenados en formato estructurado.

\hypertarget{sistemas-de-gestiuxf3n-de-bases-de-datos-dbms-1}{%
\subsection{Sistemas de Gestión de Bases de Datos
(DBMS)}\label{sistemas-de-gestiuxf3n-de-bases-de-datos-dbms-1}}

Software que administra y gestiona una base de datos.

\hypertarget{beneficios-de-las-bases-de-datos-1}{%
\subsection{Beneficios de las Bases de
Datos}\label{beneficios-de-las-bases-de-datos-1}}

Almacenamiento eficiente, acceso rápido y seguridad de datos.

\hypertarget{ejemplo-63}{%
\section{Ejemplo:}\label{ejemplo-63}}

\begin{Shaded}
\begin{Highlighting}[]
\CommentTok{\# Ejemplo de una tabla \textquotesingle{}usuarios\textquotesingle{} en una base de datos}
\OperatorTok{|} \BuiltInTok{id} \OperatorTok{|}\NormalTok{ nombre   }\OperatorTok{|}\NormalTok{ edad }\OperatorTok{|}\NormalTok{ email           }\OperatorTok{|}
\OperatorTok{|{-}{-}{-}{-}|{-}{-}{-}{-}{-}{-}{-}{-}{-}{-}|{-}{-}{-}{-}{-}{-}|{-}{-}{-}{-}{-}{-}{-}{-}{-}{-}{-}{-}{-}{-}{-}{-}{-}|}
\OperatorTok{|} \DecValTok{1}  \OperatorTok{|}\NormalTok{ Juan     }\OperatorTok{|} \DecValTok{25}   \OperatorTok{|}\NormalTok{ juan}\OperatorTok{@}\NormalTok{email.com }\OperatorTok{|}
\OperatorTok{|} \DecValTok{2}  \OperatorTok{|}\NormalTok{ María    }\OperatorTok{|} \DecValTok{30}   \OperatorTok{|}\NormalTok{ maria}\OperatorTok{@}\NormalTok{email.com}\OperatorTok{|}
\end{Highlighting}
\end{Shaded}

\hypertarget{explicaciuxf3n-63}{%
\section{Explicación:}\label{explicaciuxf3n-63}}

En este ejemplo, se muestra una tabla ficticia de una base de datos
llamada `usuarios'.

La tabla contiene filas que representan registros de usuarios con
diferentes atributos.

\begin{tcolorbox}[enhanced jigsaw, colbacktitle=quarto-callout-important-color!10!white, toprule=.15mm, leftrule=.75mm, titlerule=0mm, opacityback=0, rightrule=.15mm, opacitybacktitle=0.6, breakable, left=2mm, coltitle=black, title=\textcolor{quarto-callout-important-color}{\faExclamation}\hspace{0.5em}{Actividad Práctica:}, toptitle=1mm, bottomtitle=1mm, arc=.35mm, bottomrule=.15mm, colback=white, colframe=quarto-callout-important-color-frame]

Investiga y elige un Sistema de Gestión de Bases de Datos (DBMS) para
utilizar en el curso.

Explica por qué es importante utilizar bases de datos en el desarrollo
de aplicaciones.

\end{tcolorbox}

\hypertarget{explicaciuxf3n-de-la-actividad-65}{%
\section{Explicación de la
Actividad:}\label{explicaciuxf3n-de-la-actividad-65}}

Esta actividad permite a los participantes comprender la importancia de
las bases de datos en el desarrollo de aplicaciones y seleccionar una
opción adecuada de DBMS para usar en el curso. Les ayuda a
familiarizarse con el concepto de bases de datos y sus beneficios.

\begin{quote}
\begin{quote}
\begin{quote}
\begin{quote}
\begin{quote}
\begin{quote}
\begin{quote}
e8ed08b1a5bbe1e369719187cfc4de7f7e2a41a9
\end{quote}
\end{quote}
\end{quote}
\end{quote}
\end{quote}
\end{quote}
\end{quote}

\hypertarget{introducciuxf3n-a-postgresql}{%
\chapter{Introducción a PostgreSQL}\label{introducciuxf3n-a-postgresql}}

\textless\textless\textless\textless\textless\textless\textless{} HEAD

En esta lección, nos centraremos en PostgreSQL, un Sistema de Gestión de
Bases de Datos Relacionales (RDBMS) de código abierto. Aprenderemos cómo
instalar PostgreSQL y cómo realizar operaciones básicas en una base de
datos.

\hypertarget{instalaciuxf3n-de-postgresql}{%
\subsection{Instalación de
PostgreSQL:}\label{instalaciuxf3n-de-postgresql}}

Descargar e instalar PostgreSQL desde el sitio oficial.

Configurar contraseña para el usuario `postgres'.

\hypertarget{operaciones-buxe1sicas-en-postgresql}{%
\subsection{Operaciones Básicas en
PostgreSQL:}\label{operaciones-buxe1sicas-en-postgresql}}

\hypertarget{crear-una-base-de-datos}{%
\subsection{Crear una base de datos:}\label{crear-una-base-de-datos}}

\begin{Shaded}
\begin{Highlighting}[]
\KeywordTok{CREATE} \KeywordTok{DATABASE}\NormalTok{ nombre;}
\end{Highlighting}
\end{Shaded}

\hypertarget{conectar-a-una-base-de-datos}{%
\subsection{Conectar a una base de
datos:}\label{conectar-a-una-base-de-datos}}

\begin{Shaded}
\begin{Highlighting}[]
\NormalTok{\textbackslash{}c nombre;}
\end{Highlighting}
\end{Shaded}

\hypertarget{crear-una-tabla}{%
\subsection{Crear una tabla:}\label{crear-una-tabla}}

\begin{Shaded}
\begin{Highlighting}[]
\KeywordTok{CREATE} \KeywordTok{TABLE}\NormalTok{ tabla (columna1 tipo, columna2 tipo);}
\end{Highlighting}
\end{Shaded}

\hypertarget{insertar-registros}{%
\subsection{Insertar registros:}\label{insertar-registros}}

\begin{Shaded}
\begin{Highlighting}[]
\KeywordTok{INSERT} \KeywordTok{INTO}\NormalTok{ tabla (columna1, columna2) }\KeywordTok{VALUES}\NormalTok{ (valor1, valor2);}
\end{Highlighting}
\end{Shaded}

\hypertarget{consultar-registros}{%
\subsection{Consultar registros:}\label{consultar-registros}}

\begin{Shaded}
\begin{Highlighting}[]
\KeywordTok{SELECT} \OperatorTok{*} \KeywordTok{FROM}\NormalTok{ tabla;}
\end{Highlighting}
\end{Shaded}

\hypertarget{actualizar-registros}{%
\subsection{Actualizar registros:}\label{actualizar-registros}}

\begin{Shaded}
\begin{Highlighting}[]
\KeywordTok{UPDATE}\NormalTok{ tabla }\KeywordTok{SET}\NormalTok{ columna1 }\OperatorTok{=}\NormalTok{ valor }\KeywordTok{WHERE}\NormalTok{ condicion;}\OperatorTok{**}
\end{Highlighting}
\end{Shaded}

\hypertarget{eliminar-registros}{%
\subsection{Eliminar registros:}\label{eliminar-registros}}

\begin{Shaded}
\begin{Highlighting}[]
\KeywordTok{DELETE} \KeywordTok{FROM}\NormalTok{ tabla }\KeywordTok{WHERE}\NormalTok{ condicion;}
\end{Highlighting}
\end{Shaded}

\hypertarget{ejemplo---creaciuxf3n-de-una-tabla-en-postgresql}{%
\section{Ejemplo - Creación de una Tabla en
PostgreSQL:}\label{ejemplo---creaciuxf3n-de-una-tabla-en-postgresql}}

\begin{Shaded}
\begin{Highlighting}[]
\KeywordTok{CREATE} \KeywordTok{TABLE}\NormalTok{ estudiantes (}
    \KeywordTok{id}\NormalTok{ SERIAL }\KeywordTok{PRIMARY} \KeywordTok{KEY}\NormalTok{,}
\NormalTok{    nombre }\DataTypeTok{VARCHAR}\NormalTok{(}\DecValTok{100}\NormalTok{),}
\NormalTok{    edad }\DataTypeTok{INTEGER}
\NormalTok{);}
\end{Highlighting}
\end{Shaded}

\hypertarget{actividad-pruxe1ctica-60}{%
\subsection{Actividad Práctica:}\label{actividad-pruxe1ctica-60}}

Instala PostgreSQL en tu entorno.

Crea una base de datos llamada `universidad'.

Crea una tabla `alumnos' con las columnas `id', `nombre' y `edad'.

Inserta al menos dos registros en la tabla.

Realiza una consulta para obtener todos los registros de la tabla
`alumnos'.

\hypertarget{explicaciuxf3n-de-la-actividad-66}{%
\section{Explicación de la
Actividad:}\label{explicaciuxf3n-de-la-actividad-66}}

======= \#\# Introducción a PostgreSQL

En esta lección, nos centraremos en PostgreSQL, un Sistema de Gestión de
Bases de Datos Relacionales (RDBMS) de código abierto. Aprenderemos cómo
instalar PostgreSQL y cómo realizar operaciones básicas en una base de
datos.

\hypertarget{instalaciuxf3n-de-postgresql-1}{%
\subsection{Instalación de
PostgreSQL:}\label{instalaciuxf3n-de-postgresql-1}}

Descargar e instalar PostgreSQL desde el sitio oficial.

Configurar contraseña para el usuario `postgres'.

\hypertarget{operaciones-buxe1sicas-en-postgresql-1}{%
\subsection{Operaciones Básicas en
PostgreSQL:}\label{operaciones-buxe1sicas-en-postgresql-1}}

\hypertarget{crear-una-base-de-datos-1}{%
\subsection{Crear una base de datos:}\label{crear-una-base-de-datos-1}}

\begin{Shaded}
\begin{Highlighting}[]
\KeywordTok{CREATE} \KeywordTok{DATABASE}\NormalTok{ nombre;}
\end{Highlighting}
\end{Shaded}

\hypertarget{conectar-a-una-base-de-datos-1}{%
\subsection{Conectar a una base de
datos:}\label{conectar-a-una-base-de-datos-1}}

\begin{Shaded}
\begin{Highlighting}[]
\NormalTok{\textbackslash{}c nombre;}
\end{Highlighting}
\end{Shaded}

\hypertarget{crear-una-tabla-1}{%
\subsection{Crear una tabla:}\label{crear-una-tabla-1}}

\begin{Shaded}
\begin{Highlighting}[]
\KeywordTok{CREATE} \KeywordTok{TABLE}\NormalTok{ tabla (columna1 tipo, columna2 tipo);}
\end{Highlighting}
\end{Shaded}

\hypertarget{insertar-registros-1}{%
\subsection{Insertar registros:}\label{insertar-registros-1}}

\begin{Shaded}
\begin{Highlighting}[]
\KeywordTok{INSERT} \KeywordTok{INTO}\NormalTok{ tabla (columna1, columna2) }\KeywordTok{VALUES}\NormalTok{ (valor1, valor2);}
\end{Highlighting}
\end{Shaded}

\hypertarget{consultar-registros-1}{%
\subsection{Consultar registros:}\label{consultar-registros-1}}

\begin{Shaded}
\begin{Highlighting}[]
\KeywordTok{SELECT} \OperatorTok{*} \KeywordTok{FROM}\NormalTok{ tabla;}
\end{Highlighting}
\end{Shaded}

\hypertarget{actualizar-registros-1}{%
\subsection{Actualizar registros:}\label{actualizar-registros-1}}

\begin{Shaded}
\begin{Highlighting}[]
\KeywordTok{UPDATE}\NormalTok{ tabla }\KeywordTok{SET}\NormalTok{ columna1 }\OperatorTok{=}\NormalTok{ valor }\KeywordTok{WHERE}\NormalTok{ condicion;}\OperatorTok{**}
\end{Highlighting}
\end{Shaded}

\hypertarget{eliminar-registros-1}{%
\subsection{Eliminar registros:}\label{eliminar-registros-1}}

\begin{Shaded}
\begin{Highlighting}[]
\KeywordTok{DELETE} \KeywordTok{FROM}\NormalTok{ tabla }\KeywordTok{WHERE}\NormalTok{ condicion;}
\end{Highlighting}
\end{Shaded}

\hypertarget{ejemplo---creaciuxf3n-de-una-tabla-en-postgresql-1}{%
\section{Ejemplo - Creación de una Tabla en
PostgreSQL:}\label{ejemplo---creaciuxf3n-de-una-tabla-en-postgresql-1}}

\begin{Shaded}
\begin{Highlighting}[]
\KeywordTok{CREATE} \KeywordTok{TABLE}\NormalTok{ estudiantes (}
    \KeywordTok{id}\NormalTok{ SERIAL }\KeywordTok{PRIMARY} \KeywordTok{KEY}\NormalTok{,}
\NormalTok{    nombre }\DataTypeTok{VARCHAR}\NormalTok{(}\DecValTok{100}\NormalTok{),}
\NormalTok{    edad }\DataTypeTok{INTEGER}
\NormalTok{);}
\end{Highlighting}
\end{Shaded}

\hypertarget{actividad-pruxe1ctica-61}{%
\subsection{Actividad Práctica:}\label{actividad-pruxe1ctica-61}}

Instala PostgreSQL en tu entorno.

Crea una base de datos llamada `universidad'.

Crea una tabla `alumnos' con las columnas `id', `nombre' y `edad'.

Inserta al menos dos registros en la tabla.

Realiza una consulta para obtener todos los registros de la tabla
`alumnos'.

\hypertarget{explicaciuxf3n-de-la-actividad-67}{%
\section{Explicación de la
Actividad:}\label{explicaciuxf3n-de-la-actividad-67}}

\begin{quote}
\begin{quote}
\begin{quote}
\begin{quote}
\begin{quote}
\begin{quote}
\begin{quote}
e8ed08b1a5bbe1e369719187cfc4de7f7e2a41a9 Esta actividad permite a los
participantes familiarizarse con la instalación de PostgreSQL y realizar
operaciones básicas de creación de base de datos, creación de tablas,
inserción y consulta de registros. Les ayuda a adquirir experiencia
práctica en la gestión de bases de datos utilizando PostgreSQL.
\end{quote}
\end{quote}
\end{quote}
\end{quote}
\end{quote}
\end{quote}
\end{quote}

\hypertarget{introducciuxf3n-a-mongodb}{%
\chapter{Introducción a MongoDB}\label{introducciuxf3n-a-mongodb}}

\textless\textless\textless\textless\textless\textless\textless{} HEAD

En esta lección, nos centraremos en MongoDB, una base de datos NoSQL de
código abierto. Aprenderemos cómo instalar MongoDB y cómo realizar
operaciones básicas en una base de datos NoSQL.

\hypertarget{instalaciuxf3n-de-mongodb}{%
\section{Instalación de MongoDB:}\label{instalaciuxf3n-de-mongodb}}

Descargar e instalar MongoDB desde el sitio oficial.

Configurar directorio de datos y logs.

\hypertarget{operaciones-buxe1sicas-en-mongodb}{%
\section{Operaciones Básicas en
MongoDB:}\label{operaciones-buxe1sicas-en-mongodb}}

\hypertarget{crear-una-base-de-datos-2}{%
\subsection{Crear una base de datos:}\label{crear-una-base-de-datos-2}}

\begin{Shaded}
\begin{Highlighting}[]
\NormalTok{use nombre;}
\end{Highlighting}
\end{Shaded}

\hypertarget{crear-una-colecciuxf3n-tabla}{%
\subsection{Crear una colección
(tabla):}\label{crear-una-colecciuxf3n-tabla}}

\begin{Shaded}
\begin{Highlighting}[]
\NormalTok{db.createCollection("coleccion");}
\end{Highlighting}
\end{Shaded}

\hypertarget{insertar-documentos-registros}{%
\subsection{Insertar documentos
(registros):}\label{insertar-documentos-registros}}

\begin{Shaded}
\begin{Highlighting}[]
\NormalTok{db.coleccion.insert(\{ campo1: valor1, campo2: valor2 \});}
\end{Highlighting}
\end{Shaded}

\hypertarget{consultar-documentos}{%
\subsection{Consultar documentos:}\label{consultar-documentos}}

\begin{Shaded}
\begin{Highlighting}[]
\NormalTok{db.coleccion.find();}
\end{Highlighting}
\end{Shaded}

\hypertarget{actualizar-documentos}{%
\subsection{Actualizar documentos:}\label{actualizar-documentos}}

\begin{Shaded}
\begin{Highlighting}[]
\NormalTok{db.coleccion.update(\{ campo: valor \}, \{ $set: \{ campo\_actualizado: nuevo\_valor \} \});}
\end{Highlighting}
\end{Shaded}

\hypertarget{eliminar-documentos}{%
\subsection{Eliminar documentos:}\label{eliminar-documentos}}

\begin{Shaded}
\begin{Highlighting}[]
\NormalTok{db.coleccion.remove(\{ campo: valor \});}
\end{Highlighting}
\end{Shaded}

\hypertarget{ejemplo---creaciuxf3n-de-una-colecciuxf3n-en-mongodb}{%
\section{Ejemplo - Creación de una Colección en
MongoDB:}\label{ejemplo---creaciuxf3n-de-una-colecciuxf3n-en-mongodb}}

\begin{Shaded}
\begin{Highlighting}[]
\NormalTok{use tienda;}
\NormalTok{db.createCollection("productos");}
\end{Highlighting}
\end{Shaded}

\begin{tcolorbox}[enhanced jigsaw, colbacktitle=quarto-callout-important-color!10!white, toprule=.15mm, leftrule=.75mm, titlerule=0mm, opacityback=0, rightrule=.15mm, opacitybacktitle=0.6, breakable, left=2mm, coltitle=black, title=\textcolor{quarto-callout-important-color}{\faExclamation}\hspace{0.5em}{Actividad Práctica:}, toptitle=1mm, bottomtitle=1mm, arc=.35mm, bottomrule=.15mm, colback=white, colframe=quarto-callout-important-color-frame]

Instala MongoDB en tu entorno.

Crea una base de datos llamada `blog'.

Crea una colección `articulos'.

Inserta al menos dos documentos (artículos) en la colección.

Realiza una consulta para obtener todos los documentos de la colección
`articulos'.

\end{tcolorbox}

\hypertarget{explicaciuxf3n-de-la-actividad-68}{%
\section{Explicación de la
Actividad:}\label{explicaciuxf3n-de-la-actividad-68}}

======= \#\# Introducción a MongoDB

En esta lección, nos centraremos en MongoDB, una base de datos NoSQL de
código abierto. Aprenderemos cómo instalar MongoDB y cómo realizar
operaciones básicas en una base de datos NoSQL.

\hypertarget{instalaciuxf3n-de-mongodb-1}{%
\section{Instalación de MongoDB:}\label{instalaciuxf3n-de-mongodb-1}}

Descargar e instalar MongoDB desde el sitio oficial.

Configurar directorio de datos y logs.

\hypertarget{operaciones-buxe1sicas-en-mongodb-1}{%
\section{Operaciones Básicas en
MongoDB:}\label{operaciones-buxe1sicas-en-mongodb-1}}

\hypertarget{crear-una-base-de-datos-3}{%
\subsection{Crear una base de datos:}\label{crear-una-base-de-datos-3}}

\begin{Shaded}
\begin{Highlighting}[]
\NormalTok{use nombre;}
\end{Highlighting}
\end{Shaded}

\hypertarget{crear-una-colecciuxf3n-tabla-1}{%
\subsection{Crear una colección
(tabla):}\label{crear-una-colecciuxf3n-tabla-1}}

\begin{Shaded}
\begin{Highlighting}[]
\NormalTok{db.createCollection("coleccion");}
\end{Highlighting}
\end{Shaded}

\hypertarget{insertar-documentos-registros-1}{%
\subsection{Insertar documentos
(registros):}\label{insertar-documentos-registros-1}}

\begin{Shaded}
\begin{Highlighting}[]
\NormalTok{db.coleccion.insert(\{ campo1: valor1, campo2: valor2 \});}
\end{Highlighting}
\end{Shaded}

\hypertarget{consultar-documentos-1}{%
\subsection{Consultar documentos:}\label{consultar-documentos-1}}

\begin{Shaded}
\begin{Highlighting}[]
\NormalTok{db.coleccion.find();}
\end{Highlighting}
\end{Shaded}

\hypertarget{actualizar-documentos-1}{%
\subsection{Actualizar documentos:}\label{actualizar-documentos-1}}

\begin{Shaded}
\begin{Highlighting}[]
\NormalTok{db.coleccion.update(\{ campo: valor \}, \{ $set: \{ campo\_actualizado: nuevo\_valor \} \});}
\end{Highlighting}
\end{Shaded}

\hypertarget{eliminar-documentos-1}{%
\subsection{Eliminar documentos:}\label{eliminar-documentos-1}}

\begin{Shaded}
\begin{Highlighting}[]
\NormalTok{db.coleccion.remove(\{ campo: valor \});}
\end{Highlighting}
\end{Shaded}

\hypertarget{ejemplo---creaciuxf3n-de-una-colecciuxf3n-en-mongodb-1}{%
\section{Ejemplo - Creación de una Colección en
MongoDB:}\label{ejemplo---creaciuxf3n-de-una-colecciuxf3n-en-mongodb-1}}

\begin{Shaded}
\begin{Highlighting}[]
\NormalTok{use tienda;}
\NormalTok{db.createCollection("productos");}
\end{Highlighting}
\end{Shaded}

\begin{tcolorbox}[enhanced jigsaw, colbacktitle=quarto-callout-important-color!10!white, toprule=.15mm, leftrule=.75mm, titlerule=0mm, opacityback=0, rightrule=.15mm, opacitybacktitle=0.6, breakable, left=2mm, coltitle=black, title=\textcolor{quarto-callout-important-color}{\faExclamation}\hspace{0.5em}{Actividad Práctica:}, toptitle=1mm, bottomtitle=1mm, arc=.35mm, bottomrule=.15mm, colback=white, colframe=quarto-callout-important-color-frame]

Instala MongoDB en tu entorno.

Crea una base de datos llamada `blog'.

Crea una colección `articulos'.

Inserta al menos dos documentos (artículos) en la colección.

Realiza una consulta para obtener todos los documentos de la colección
`articulos'.

\end{tcolorbox}

\hypertarget{explicaciuxf3n-de-la-actividad-69}{%
\section{Explicación de la
Actividad:}\label{explicaciuxf3n-de-la-actividad-69}}

\begin{quote}
\begin{quote}
\begin{quote}
\begin{quote}
\begin{quote}
\begin{quote}
\begin{quote}
e8ed08b1a5bbe1e369719187cfc4de7f7e2a41a9 Esta actividad permite a los
participantes familiarizarse con la instalación de MongoDB y realizar
operaciones básicas en una base de datos NoSQL. Les ayuda a adquirir
experiencia práctica en la gestión de datos en MongoDB y a comprender
las diferencias entre bases de datos SQL y NoSQL.
\end{quote}
\end{quote}
\end{quote}
\end{quote}
\end{quote}
\end{quote}
\end{quote}

\part{Unidad 10: Operaciones Básicas en Bases de Datos}

\hypertarget{introducciuxf3n-e-instalaciuxf3n}{%
\chapter{Introducción e
Instalación}\label{introducciuxf3n-e-instalaciuxf3n}}

\textless\textless\textless\textless\textless\textless\textless{} HEAD

En esta lección, nos centraremos en realizar operaciones básicas en
bases de datos utilizando diferentes sistemas de gestión: MySQL,
PostgreSQL y MongoDB. Aprenderemos cómo realizar la instalación de estos
sistemas y cómo conectarnos a las bases de datos.

\hypertarget{instalaciuxf3n-de-mysql}{%
\section{Instalación de MySQL:}\label{instalaciuxf3n-de-mysql}}

Descargar e instalar MySQL desde el sitio oficial.

Configurar contraseña para el usuario `root'.

\hypertarget{instalaciuxf3n-de-postgresql-2}{%
\section{Instalación de
PostgreSQL:}\label{instalaciuxf3n-de-postgresql-2}}

Descargar e instalar PostgreSQL desde el sitio oficial.

Configurar contraseña para el usuario `postgres'.

\hypertarget{instalaciuxf3n-de-mongodb-2}{%
\section{Instalación de MongoDB:}\label{instalaciuxf3n-de-mongodb-2}}

Descargar e instalar MongoDB desde el sitio oficial.

Configurar directorio de datos y logs.

\hypertarget{conexiuxf3n-a-la-base-de-datos}{%
\section{Conexión a la Base de
Datos:}\label{conexiuxf3n-a-la-base-de-datos}}

\hypertarget{mysql-y-postgresql}{%
\subsection{MySQL y PostgreSQL:}\label{mysql-y-postgresql}}

Usar bibliotecas como mysql-connector-python o psycopg2 para conectarse
y realizar operaciones.

\hypertarget{mongodb}{%
\subsection{MongoDB:}\label{mongodb}}

Usar la biblioteca pymongo para conectarse y realizar operaciones.

\hypertarget{ejemplo---conexiuxf3n-a-mysql}{%
\section{Ejemplo - Conexión a
MySQL:}\label{ejemplo---conexiuxf3n-a-mysql}}

\begin{Shaded}
\begin{Highlighting}[]
\ImportTok{import}\NormalTok{ mysql.connector}

\CommentTok{\# Conexión a la base de datos}
\NormalTok{conn }\OperatorTok{=}\NormalTok{ mysql.connector.}\ExtensionTok{connect}\NormalTok{(}
\NormalTok{    host}\OperatorTok{=}\StringTok{"localhost"}\NormalTok{,}
\NormalTok{    user}\OperatorTok{=}\StringTok{"root"}\NormalTok{,}
\NormalTok{    password}\OperatorTok{=}\StringTok{"contraseña"}\NormalTok{,}
\NormalTok{    database}\OperatorTok{=}\StringTok{"basededatos"}
\NormalTok{)}
\end{Highlighting}
\end{Shaded}

\hypertarget{ejemplo---conexiuxf3n-a-mongodb}{%
\section{Ejemplo - Conexión a
MongoDB:}\label{ejemplo---conexiuxf3n-a-mongodb}}

\begin{Shaded}
\begin{Highlighting}[]
\ImportTok{import}\NormalTok{ pymongo}

\CommentTok{\# Conexión al servidor MongoDB}
\NormalTok{client }\OperatorTok{=}\NormalTok{ pymongo.MongoClient(}\StringTok{"mongodb://localhost:27017/"}\NormalTok{)}
\end{Highlighting}
\end{Shaded}

\begin{tcolorbox}[enhanced jigsaw, colbacktitle=quarto-callout-important-color!10!white, toprule=.15mm, leftrule=.75mm, titlerule=0mm, opacityback=0, rightrule=.15mm, opacitybacktitle=0.6, breakable, left=2mm, coltitle=black, title=\textcolor{quarto-callout-important-color}{\faExclamation}\hspace{0.5em}{Actividad Práctica:}, toptitle=1mm, bottomtitle=1mm, arc=.35mm, bottomrule=.15mm, colback=white, colframe=quarto-callout-important-color-frame]

Instala MySQL, PostgreSQL y MongoDB en tu entorno. Crea una base de
datos en cada uno de los sistemas. Conéctate a cada una de las bases de
datos utilizando las bibliotecas adecuadas. Realiza una consulta de
prueba en cada sistema para verificar la conexión.

\end{tcolorbox}

\hypertarget{explicaciuxf3n-de-la-actividad-70}{%
\section{Explicación de la
Actividad:}\label{explicaciuxf3n-de-la-actividad-70}}

======= \#\# Introducción e Instalación

En esta lección, nos centraremos en realizar operaciones básicas en
bases de datos utilizando diferentes sistemas de gestión: MySQL,
PostgreSQL y MongoDB. Aprenderemos cómo realizar la instalación de estos
sistemas y cómo conectarnos a las bases de datos.

\hypertarget{instalaciuxf3n-de-mysql-1}{%
\section{Instalación de MySQL:}\label{instalaciuxf3n-de-mysql-1}}

Descargar e instalar MySQL desde el sitio oficial.

Configurar contraseña para el usuario `root'.

\hypertarget{instalaciuxf3n-de-postgresql-3}{%
\section{Instalación de
PostgreSQL:}\label{instalaciuxf3n-de-postgresql-3}}

Descargar e instalar PostgreSQL desde el sitio oficial.

Configurar contraseña para el usuario `postgres'.

\hypertarget{instalaciuxf3n-de-mongodb-3}{%
\section{Instalación de MongoDB:}\label{instalaciuxf3n-de-mongodb-3}}

Descargar e instalar MongoDB desde el sitio oficial.

Configurar directorio de datos y logs.

\hypertarget{conexiuxf3n-a-la-base-de-datos-1}{%
\section{Conexión a la Base de
Datos:}\label{conexiuxf3n-a-la-base-de-datos-1}}

\hypertarget{mysql-y-postgresql-1}{%
\subsection{MySQL y PostgreSQL:}\label{mysql-y-postgresql-1}}

Usar bibliotecas como mysql-connector-python o psycopg2 para conectarse
y realizar operaciones.

\hypertarget{mongodb-1}{%
\subsection{MongoDB:}\label{mongodb-1}}

Usar la biblioteca pymongo para conectarse y realizar operaciones.

\hypertarget{ejemplo---conexiuxf3n-a-mysql-1}{%
\section{Ejemplo - Conexión a
MySQL:}\label{ejemplo---conexiuxf3n-a-mysql-1}}

\begin{Shaded}
\begin{Highlighting}[]
\ImportTok{import}\NormalTok{ mysql.connector}

\CommentTok{\# Conexión a la base de datos}
\NormalTok{conn }\OperatorTok{=}\NormalTok{ mysql.connector.}\ExtensionTok{connect}\NormalTok{(}
\NormalTok{    host}\OperatorTok{=}\StringTok{"localhost"}\NormalTok{,}
\NormalTok{    user}\OperatorTok{=}\StringTok{"root"}\NormalTok{,}
\NormalTok{    password}\OperatorTok{=}\StringTok{"contraseña"}\NormalTok{,}
\NormalTok{    database}\OperatorTok{=}\StringTok{"basededatos"}
\NormalTok{)}
\end{Highlighting}
\end{Shaded}

\hypertarget{ejemplo---conexiuxf3n-a-mongodb-1}{%
\section{Ejemplo - Conexión a
MongoDB:}\label{ejemplo---conexiuxf3n-a-mongodb-1}}

\begin{Shaded}
\begin{Highlighting}[]
\ImportTok{import}\NormalTok{ pymongo}

\CommentTok{\# Conexión al servidor MongoDB}
\NormalTok{client }\OperatorTok{=}\NormalTok{ pymongo.MongoClient(}\StringTok{"mongodb://localhost:27017/"}\NormalTok{)}
\end{Highlighting}
\end{Shaded}

\begin{tcolorbox}[enhanced jigsaw, colbacktitle=quarto-callout-important-color!10!white, toprule=.15mm, leftrule=.75mm, titlerule=0mm, opacityback=0, rightrule=.15mm, opacitybacktitle=0.6, breakable, left=2mm, coltitle=black, title=\textcolor{quarto-callout-important-color}{\faExclamation}\hspace{0.5em}{Actividad Práctica:}, toptitle=1mm, bottomtitle=1mm, arc=.35mm, bottomrule=.15mm, colback=white, colframe=quarto-callout-important-color-frame]

Instala MySQL, PostgreSQL y MongoDB en tu entorno. Crea una base de
datos en cada uno de los sistemas. Conéctate a cada una de las bases de
datos utilizando las bibliotecas adecuadas. Realiza una consulta de
prueba en cada sistema para verificar la conexión.

\end{tcolorbox}

\hypertarget{explicaciuxf3n-de-la-actividad-71}{%
\section{Explicación de la
Actividad:}\label{explicaciuxf3n-de-la-actividad-71}}

\begin{quote}
\begin{quote}
\begin{quote}
\begin{quote}
\begin{quote}
\begin{quote}
\begin{quote}
e8ed08b1a5bbe1e369719187cfc4de7f7e2a41a9 Esta actividad permite a los
participantes adquirir experiencia práctica en la instalación de
diferentes sistemas de bases de datos y en la conexión a estas bases de
datos utilizando las bibliotecas correspondientes. Les ayuda a
comprender cómo establecer una conexión exitosa y cómo preparar el
entorno para las operaciones futuras en bases de datos.
\end{quote}
\end{quote}
\end{quote}
\end{quote}
\end{quote}
\end{quote}
\end{quote}

\hypertarget{bases-de-datos-en-mysql}{%
\chapter{Bases de Datos en MySQL}\label{bases-de-datos-en-mysql}}

\textless\textless\textless\textless\textless\textless\textless{} HEAD

En esta lección, aprenderemos a realizar operaciones básicas en una base
de datos MySQL, como crear y eliminar tablas, insertar registros y
realizar consultas.

\hypertarget{operaciones-en-mysql}{%
\section{Operaciones en MySQL:}\label{operaciones-en-mysql}}

\hypertarget{crear-una-tabla-2}{%
\subsection{Crear una tabla:}\label{crear-una-tabla-2}}

\begin{Shaded}
\begin{Highlighting}[]
\KeywordTok{CREATE} \KeywordTok{TABLE}\NormalTok{ nombre (columna1 tipo, columna2 tipo);}
\end{Highlighting}
\end{Shaded}

\hypertarget{insertar-registros-2}{%
\subsection{Insertar registros:}\label{insertar-registros-2}}

\begin{Shaded}
\begin{Highlighting}[]
\KeywordTok{INSERT} \KeywordTok{INTO}\NormalTok{ nombre (columna1, columna2) }\KeywordTok{VALUES}\NormalTok{ (valor1, valor2);}
\end{Highlighting}
\end{Shaded}

\hypertarget{consultar-registros-2}{%
\subsection{Consultar registros:}\label{consultar-registros-2}}

\begin{Shaded}
\begin{Highlighting}[]
\KeywordTok{SELECT} \OperatorTok{*} \KeywordTok{FROM}\NormalTok{ nombre;}
\end{Highlighting}
\end{Shaded}

\hypertarget{actualizar-registros-2}{%
\subsection{Actualizar registros:}\label{actualizar-registros-2}}

\begin{Shaded}
\begin{Highlighting}[]
\KeywordTok{UPDATE}\NormalTok{ nombre }\KeywordTok{SET}\NormalTok{ columna }\OperatorTok{=}\NormalTok{ valor }\KeywordTok{WHERE}\NormalTok{ condicion;}
\end{Highlighting}
\end{Shaded}

\hypertarget{eliminar-registros-2}{%
\subsection{Eliminar registros:}\label{eliminar-registros-2}}

\begin{Shaded}
\begin{Highlighting}[]
\KeywordTok{DELETE} \KeywordTok{FROM}\NormalTok{ nombre }\KeywordTok{WHERE}\NormalTok{ condicion;}
\end{Highlighting}
\end{Shaded}

\hypertarget{eliminar-tabla}{%
\subsection{Eliminar tabla:}\label{eliminar-tabla}}

\begin{Shaded}
\begin{Highlighting}[]
\KeywordTok{DROP} \KeywordTok{TABLE}\NormalTok{ nombre;}
\end{Highlighting}
\end{Shaded}

\hypertarget{ejemplo---creaciuxf3n-de-una-tabla-en-mysql}{%
\section{Ejemplo - Creación de una Tabla en
MySQL:}\label{ejemplo---creaciuxf3n-de-una-tabla-en-mysql}}

\begin{Shaded}
\begin{Highlighting}[]
\KeywordTok{CREATE} \KeywordTok{TABLE}\NormalTok{ empleados (}
    \KeywordTok{id} \DataTypeTok{INT} \KeywordTok{PRIMARY} \KeywordTok{KEY}\NormalTok{,}
\NormalTok{    nombre }\DataTypeTok{VARCHAR}\NormalTok{(}\DecValTok{100}\NormalTok{),}
\NormalTok{    salario }\DataTypeTok{DECIMAL}\NormalTok{(}\DecValTok{10}\NormalTok{, }\DecValTok{2}\NormalTok{)}
\NormalTok{);}
\end{Highlighting}
\end{Shaded}

\begin{tcolorbox}[enhanced jigsaw, colbacktitle=quarto-callout-important-color!10!white, toprule=.15mm, leftrule=.75mm, titlerule=0mm, opacityback=0, rightrule=.15mm, opacitybacktitle=0.6, breakable, left=2mm, coltitle=black, title=\textcolor{quarto-callout-important-color}{\faExclamation}\hspace{0.5em}{Actividad Práctica:}, toptitle=1mm, bottomtitle=1mm, arc=.35mm, bottomrule=.15mm, colback=white, colframe=quarto-callout-important-color-frame]

Conéctate a la base de datos MySQL.

Crea una tabla `productos' con las columnas `id', `nombre' y `precio'.

Inserta al menos dos registros en la tabla `productos'.

Realiza una consulta para obtener todos los registros de la tabla
`productos'.

\end{tcolorbox}

\hypertarget{explicaciuxf3n-de-la-actividad-72}{%
\section{Explicación de la
Actividad:}\label{explicaciuxf3n-de-la-actividad-72}}

Esta actividad permite a los participantes aplicar los conocimientos
adquiridos en la creación de tablas, inserción de registros y consultas
en una base de datos MySQL. Les ayuda a ganar experiencia práctica en la
manipulación de datos utilizando SQL en MySQL.

======= \#\# Bases de Datos en MySQL

En esta lección, aprenderemos a realizar operaciones básicas en una base
de datos MySQL, como crear y eliminar tablas, insertar registros y
realizar consultas.

\hypertarget{operaciones-en-mysql-1}{%
\section{Operaciones en MySQL:}\label{operaciones-en-mysql-1}}

\hypertarget{crear-una-tabla-3}{%
\subsection{Crear una tabla:}\label{crear-una-tabla-3}}

\begin{Shaded}
\begin{Highlighting}[]
\KeywordTok{CREATE} \KeywordTok{TABLE}\NormalTok{ nombre (columna1 tipo, columna2 tipo);}
\end{Highlighting}
\end{Shaded}

\hypertarget{insertar-registros-3}{%
\subsection{Insertar registros:}\label{insertar-registros-3}}

\begin{Shaded}
\begin{Highlighting}[]
\KeywordTok{INSERT} \KeywordTok{INTO}\NormalTok{ nombre (columna1, columna2) }\KeywordTok{VALUES}\NormalTok{ (valor1, valor2);}
\end{Highlighting}
\end{Shaded}

\hypertarget{consultar-registros-3}{%
\subsection{Consultar registros:}\label{consultar-registros-3}}

\begin{Shaded}
\begin{Highlighting}[]
\KeywordTok{SELECT} \OperatorTok{*} \KeywordTok{FROM}\NormalTok{ nombre;}
\end{Highlighting}
\end{Shaded}

\hypertarget{actualizar-registros-3}{%
\subsection{Actualizar registros:}\label{actualizar-registros-3}}

\begin{Shaded}
\begin{Highlighting}[]
\KeywordTok{UPDATE}\NormalTok{ nombre }\KeywordTok{SET}\NormalTok{ columna }\OperatorTok{=}\NormalTok{ valor }\KeywordTok{WHERE}\NormalTok{ condicion;}
\end{Highlighting}
\end{Shaded}

\hypertarget{eliminar-registros-3}{%
\subsection{Eliminar registros:}\label{eliminar-registros-3}}

\begin{Shaded}
\begin{Highlighting}[]
\KeywordTok{DELETE} \KeywordTok{FROM}\NormalTok{ nombre }\KeywordTok{WHERE}\NormalTok{ condicion;}
\end{Highlighting}
\end{Shaded}

\hypertarget{eliminar-tabla-1}{%
\subsection{Eliminar tabla:}\label{eliminar-tabla-1}}

\begin{Shaded}
\begin{Highlighting}[]
\KeywordTok{DROP} \KeywordTok{TABLE}\NormalTok{ nombre;}
\end{Highlighting}
\end{Shaded}

\hypertarget{ejemplo---creaciuxf3n-de-una-tabla-en-mysql-1}{%
\section{Ejemplo - Creación de una Tabla en
MySQL:}\label{ejemplo---creaciuxf3n-de-una-tabla-en-mysql-1}}

\begin{Shaded}
\begin{Highlighting}[]
\KeywordTok{CREATE} \KeywordTok{TABLE}\NormalTok{ empleados (}
    \KeywordTok{id} \DataTypeTok{INT} \KeywordTok{PRIMARY} \KeywordTok{KEY}\NormalTok{,}
\NormalTok{    nombre }\DataTypeTok{VARCHAR}\NormalTok{(}\DecValTok{100}\NormalTok{),}
\NormalTok{    salario }\DataTypeTok{DECIMAL}\NormalTok{(}\DecValTok{10}\NormalTok{, }\DecValTok{2}\NormalTok{)}
\NormalTok{);}
\end{Highlighting}
\end{Shaded}

\begin{tcolorbox}[enhanced jigsaw, colbacktitle=quarto-callout-important-color!10!white, toprule=.15mm, leftrule=.75mm, titlerule=0mm, opacityback=0, rightrule=.15mm, opacitybacktitle=0.6, breakable, left=2mm, coltitle=black, title=\textcolor{quarto-callout-important-color}{\faExclamation}\hspace{0.5em}{Actividad Práctica:}, toptitle=1mm, bottomtitle=1mm, arc=.35mm, bottomrule=.15mm, colback=white, colframe=quarto-callout-important-color-frame]

Conéctate a la base de datos MySQL.

Crea una tabla `productos' con las columnas `id', `nombre' y `precio'.

Inserta al menos dos registros en la tabla `productos'.

Realiza una consulta para obtener todos los registros de la tabla
`productos'.

\end{tcolorbox}

\hypertarget{explicaciuxf3n-de-la-actividad-73}{%
\section{Explicación de la
Actividad:}\label{explicaciuxf3n-de-la-actividad-73}}

Esta actividad permite a los participantes aplicar los conocimientos
adquiridos en la creación de tablas, inserción de registros y consultas
en una base de datos MySQL. Les ayuda a ganar experiencia práctica en la
manipulación de datos utilizando SQL en MySQL.

\begin{quote}
\begin{quote}
\begin{quote}
\begin{quote}
\begin{quote}
\begin{quote}
\begin{quote}
e8ed08b1a5bbe1e369719187cfc4de7f7e2a41a9
\end{quote}
\end{quote}
\end{quote}
\end{quote}
\end{quote}
\end{quote}
\end{quote}

\hypertarget{crear-y-eliminar-tablas-en-postgresql}{%
\chapter{Crear y Eliminar Tablas en
PostgreSQL}\label{crear-y-eliminar-tablas-en-postgresql}}

\textless\textless\textless\textless\textless\textless\textless{} HEAD

En esta lección, aprenderemos a realizar operaciones básicas en una base
de datos PostgreSQL, como crear y eliminar tablas, insertar registros y
realizar consultas.

\hypertarget{operaciones-en-postgresql}{%
\section{Operaciones en PostgreSQL:}\label{operaciones-en-postgresql}}

\hypertarget{crear-una-tabla-4}{%
\subsection{Crear una tabla:}\label{crear-una-tabla-4}}

\begin{Shaded}
\begin{Highlighting}[]
\KeywordTok{CREATE} \KeywordTok{TABLE}\NormalTok{ nombre (columna1 tipo, columna2 tipo);}
\end{Highlighting}
\end{Shaded}

\hypertarget{insertar-registros-4}{%
\subsection{Insertar registros:}\label{insertar-registros-4}}

\begin{Shaded}
\begin{Highlighting}[]
\KeywordTok{INSERT} \KeywordTok{INTO}\NormalTok{ nombre (columna1, columna2) }\KeywordTok{VALUES}\NormalTok{ (valor1, valor2);}
\end{Highlighting}
\end{Shaded}

\hypertarget{consultar-registros-4}{%
\subsection{Consultar registros:}\label{consultar-registros-4}}

\begin{Shaded}
\begin{Highlighting}[]
\KeywordTok{SELECT} \OperatorTok{*} \KeywordTok{FROM}\NormalTok{ nombre;}
\end{Highlighting}
\end{Shaded}

\hypertarget{actualizar-registros-4}{%
\subsection{Actualizar registros:}\label{actualizar-registros-4}}

\begin{Shaded}
\begin{Highlighting}[]
\KeywordTok{UPDATE}\NormalTok{ nombre }\KeywordTok{SET}\NormalTok{ columna }\OperatorTok{=}\NormalTok{ valor }\KeywordTok{WHERE}\NormalTok{ condicion;}
\end{Highlighting}
\end{Shaded}

\hypertarget{eliminar-registros-4}{%
\subsection{Eliminar registros:}\label{eliminar-registros-4}}

\begin{Shaded}
\begin{Highlighting}[]
\KeywordTok{DELETE} \KeywordTok{FROM}\NormalTok{ nombre }\KeywordTok{WHERE}\NormalTok{ condicion;}
\end{Highlighting}
\end{Shaded}

\hypertarget{eliminar-tabla-2}{%
\subsection{Eliminar tabla:}\label{eliminar-tabla-2}}

\begin{Shaded}
\begin{Highlighting}[]
\KeywordTok{DROP} \KeywordTok{TABLE}\NormalTok{ nombre;}
\end{Highlighting}
\end{Shaded}

\hypertarget{ejemplo---creaciuxf3n-de-una-tabla-en-postgresql-2}{%
\section{Ejemplo - Creación de una Tabla en
PostgreSQL:}\label{ejemplo---creaciuxf3n-de-una-tabla-en-postgresql-2}}

\begin{Shaded}
\begin{Highlighting}[]
\KeywordTok{CREATE} \KeywordTok{TABLE}\NormalTok{ empleados (}
    \KeywordTok{id}\NormalTok{ SERIAL }\KeywordTok{PRIMARY} \KeywordTok{KEY}\NormalTok{,}
\NormalTok{    nombre }\DataTypeTok{VARCHAR}\NormalTok{(}\DecValTok{100}\NormalTok{),}
\NormalTok{    salario }\DataTypeTok{DECIMAL}\NormalTok{(}\DecValTok{10}\NormalTok{, }\DecValTok{2}\NormalTok{)}
\NormalTok{);}
\end{Highlighting}
\end{Shaded}

\begin{tcolorbox}[enhanced jigsaw, colbacktitle=quarto-callout-important-color!10!white, toprule=.15mm, leftrule=.75mm, titlerule=0mm, opacityback=0, rightrule=.15mm, opacitybacktitle=0.6, breakable, left=2mm, coltitle=black, title=\textcolor{quarto-callout-important-color}{\faExclamation}\hspace{0.5em}{Actividad Práctica}, toptitle=1mm, bottomtitle=1mm, arc=.35mm, bottomrule=.15mm, colback=white, colframe=quarto-callout-important-color-frame]

\hypertarget{conuxe9ctate-a-la-base-de-datos-postgresql.}{%
\section{Conéctate a la base de datos
PostgreSQL.}\label{conuxe9ctate-a-la-base-de-datos-postgresql.}}

Crea una tabla `clientes' con las columnas `id', `nombre' y `email'.

Inserta al menos dos registros en la tabla `clientes'.

Realiza una consulta para obtener todos los registros de la tabla
`clientes'.

\end{tcolorbox}

\hypertarget{explicaciuxf3n-de-la-actividad-74}{%
\section{Explicación de la
Actividad:}\label{explicaciuxf3n-de-la-actividad-74}}

======= \#\# Crear y Eliminar Tablas en PostgreSQL

En esta lección, aprenderemos a realizar operaciones básicas en una base
de datos PostgreSQL, como crear y eliminar tablas, insertar registros y
realizar consultas.

\hypertarget{operaciones-en-postgresql-1}{%
\section{Operaciones en PostgreSQL:}\label{operaciones-en-postgresql-1}}

\hypertarget{crear-una-tabla-5}{%
\subsection{Crear una tabla:}\label{crear-una-tabla-5}}

\begin{Shaded}
\begin{Highlighting}[]
\KeywordTok{CREATE} \KeywordTok{TABLE}\NormalTok{ nombre (columna1 tipo, columna2 tipo);}
\end{Highlighting}
\end{Shaded}

\hypertarget{insertar-registros-5}{%
\subsection{Insertar registros:}\label{insertar-registros-5}}

\begin{Shaded}
\begin{Highlighting}[]
\KeywordTok{INSERT} \KeywordTok{INTO}\NormalTok{ nombre (columna1, columna2) }\KeywordTok{VALUES}\NormalTok{ (valor1, valor2);}
\end{Highlighting}
\end{Shaded}

\hypertarget{consultar-registros-5}{%
\subsection{Consultar registros:}\label{consultar-registros-5}}

\begin{Shaded}
\begin{Highlighting}[]
\KeywordTok{SELECT} \OperatorTok{*} \KeywordTok{FROM}\NormalTok{ nombre;}
\end{Highlighting}
\end{Shaded}

\hypertarget{actualizar-registros-5}{%
\subsection{Actualizar registros:}\label{actualizar-registros-5}}

\begin{Shaded}
\begin{Highlighting}[]
\KeywordTok{UPDATE}\NormalTok{ nombre }\KeywordTok{SET}\NormalTok{ columna }\OperatorTok{=}\NormalTok{ valor }\KeywordTok{WHERE}\NormalTok{ condicion;}
\end{Highlighting}
\end{Shaded}

\hypertarget{eliminar-registros-5}{%
\subsection{Eliminar registros:}\label{eliminar-registros-5}}

\begin{Shaded}
\begin{Highlighting}[]
\KeywordTok{DELETE} \KeywordTok{FROM}\NormalTok{ nombre }\KeywordTok{WHERE}\NormalTok{ condicion;}
\end{Highlighting}
\end{Shaded}

\hypertarget{eliminar-tabla-3}{%
\subsection{Eliminar tabla:}\label{eliminar-tabla-3}}

\begin{Shaded}
\begin{Highlighting}[]
\KeywordTok{DROP} \KeywordTok{TABLE}\NormalTok{ nombre;}
\end{Highlighting}
\end{Shaded}

\hypertarget{ejemplo---creaciuxf3n-de-una-tabla-en-postgresql-3}{%
\section{Ejemplo - Creación de una Tabla en
PostgreSQL:}\label{ejemplo---creaciuxf3n-de-una-tabla-en-postgresql-3}}

\begin{Shaded}
\begin{Highlighting}[]
\KeywordTok{CREATE} \KeywordTok{TABLE}\NormalTok{ empleados (}
    \KeywordTok{id}\NormalTok{ SERIAL }\KeywordTok{PRIMARY} \KeywordTok{KEY}\NormalTok{,}
\NormalTok{    nombre }\DataTypeTok{VARCHAR}\NormalTok{(}\DecValTok{100}\NormalTok{),}
\NormalTok{    salario }\DataTypeTok{DECIMAL}\NormalTok{(}\DecValTok{10}\NormalTok{, }\DecValTok{2}\NormalTok{)}
\NormalTok{);}
\end{Highlighting}
\end{Shaded}

\begin{tcolorbox}[enhanced jigsaw, colbacktitle=quarto-callout-important-color!10!white, toprule=.15mm, leftrule=.75mm, titlerule=0mm, opacityback=0, rightrule=.15mm, opacitybacktitle=0.6, breakable, left=2mm, coltitle=black, title=\textcolor{quarto-callout-important-color}{\faExclamation}\hspace{0.5em}{Actividad Práctica}, toptitle=1mm, bottomtitle=1mm, arc=.35mm, bottomrule=.15mm, colback=white, colframe=quarto-callout-important-color-frame]

\hypertarget{conuxe9ctate-a-la-base-de-datos-postgresql.-1}{%
\section{Conéctate a la base de datos
PostgreSQL.}\label{conuxe9ctate-a-la-base-de-datos-postgresql.-1}}

Crea una tabla `clientes' con las columnas `id', `nombre' y `email'.

Inserta al menos dos registros en la tabla `clientes'.

Realiza una consulta para obtener todos los registros de la tabla
`clientes'.

\end{tcolorbox}

\hypertarget{explicaciuxf3n-de-la-actividad-75}{%
\section{Explicación de la
Actividad:}\label{explicaciuxf3n-de-la-actividad-75}}

\begin{quote}
\begin{quote}
\begin{quote}
\begin{quote}
\begin{quote}
\begin{quote}
\begin{quote}
e8ed08b1a5bbe1e369719187cfc4de7f7e2a41a9 Esta actividad permite a los
participantes aplicar los conocimientos adquiridos en la creación de
tablas, inserción de registros y consultas en una base de datos
PostgreSQL. Les ayuda a ganar experiencia práctica en la manipulación de
datos utilizando SQL en PostgreSQL.
\end{quote}
\end{quote}
\end{quote}
\end{quote}
\end{quote}
\end{quote}
\end{quote}

\hypertarget{operaciones-buxe1sicas-en-mongodb-2}{%
\chapter{Operaciones Básicas en
MongoDB}\label{operaciones-buxe1sicas-en-mongodb-2}}

\textless\textless\textless\textless\textless\textless\textless{} HEAD

En esta lección, aprenderemos a realizar operaciones básicas en una base
de datos MongoDB, como insertar documentos, consultar documentos y
actualizar documentos.

\hypertarget{operaciones-en-mongodb}{%
\section{Operaciones en MongoDB:}\label{operaciones-en-mongodb}}

\hypertarget{insertar-documentos}{%
\subsection{Insertar documentos:}\label{insertar-documentos}}

\begin{Shaded}
\begin{Highlighting}[]
\NormalTok{db.coleccion.insert(\{ campo1: valor1, campo2: valor2 \});}
\end{Highlighting}
\end{Shaded}

\hypertarget{consultar-documentos-2}{%
\subsection{Consultar documentos:}\label{consultar-documentos-2}}

\begin{Shaded}
\begin{Highlighting}[]
\NormalTok{db.coleccion.find();}
\end{Highlighting}
\end{Shaded}

\hypertarget{actualizar-documentos-2}{%
\subsection{Actualizar documentos:}\label{actualizar-documentos-2}}

\begin{Shaded}
\begin{Highlighting}[]
\NormalTok{db.coleccion.update(\{ campo: valor \}, \{ $set: \{ campo\_actualizado: nuevo\_valor \} \});}
\end{Highlighting}
\end{Shaded}

\hypertarget{eliminar-documentos-2}{%
\subsection{Eliminar documentos:}\label{eliminar-documentos-2}}

\begin{Shaded}
\begin{Highlighting}[]
\NormalTok{db.coleccion.remove(\{ campo: valor \});}
\end{Highlighting}
\end{Shaded}

\hypertarget{ejemplo---inserciuxf3n-de-un-documento-en-mongodb}{%
\section{Ejemplo - Inserción de un Documento en
MongoDB:}\label{ejemplo---inserciuxf3n-de-un-documento-en-mongodb}}

\begin{Shaded}
\begin{Highlighting}[]
\NormalTok{// Insertar un documento en la colección \textquotesingle{}productos\textquotesingle{}}
\NormalTok{db.productos.insert(\{ nombre: "Camiseta", precio: 20 \});}
\end{Highlighting}
\end{Shaded}

\begin{tcolorbox}[enhanced jigsaw, colbacktitle=quarto-callout-important-color!10!white, toprule=.15mm, leftrule=.75mm, titlerule=0mm, opacityback=0, rightrule=.15mm, opacitybacktitle=0.6, breakable, left=2mm, coltitle=black, title=\textcolor{quarto-callout-important-color}{\faExclamation}\hspace{0.5em}{Actividad Práctica:}, toptitle=1mm, bottomtitle=1mm, arc=.35mm, bottomrule=.15mm, colback=white, colframe=quarto-callout-important-color-frame]

Conéctate a la base de datos MongoDB.

Inserta al menos dos documentos en la colección `productos'.

Realiza una consulta para obtener todos los documentos de la colección
`productos'.

Actualiza el precio de uno de los documentos en la colección.

Elimina un documento de la colección.

\end{tcolorbox}

\hypertarget{explicaciuxf3n-de-la-actividad-76}{%
\section{Explicación de la
Actividad:}\label{explicaciuxf3n-de-la-actividad-76}}

======= \#\# Operaciones Básicas en MongoDB

En esta lección, aprenderemos a realizar operaciones básicas en una base
de datos MongoDB, como insertar documentos, consultar documentos y
actualizar documentos.

\hypertarget{operaciones-en-mongodb-1}{%
\section{Operaciones en MongoDB:}\label{operaciones-en-mongodb-1}}

\hypertarget{insertar-documentos-1}{%
\subsection{Insertar documentos:}\label{insertar-documentos-1}}

\begin{Shaded}
\begin{Highlighting}[]
\NormalTok{db.coleccion.insert(\{ campo1: valor1, campo2: valor2 \});}
\end{Highlighting}
\end{Shaded}

\hypertarget{consultar-documentos-3}{%
\subsection{Consultar documentos:}\label{consultar-documentos-3}}

\begin{Shaded}
\begin{Highlighting}[]
\NormalTok{db.coleccion.find();}
\end{Highlighting}
\end{Shaded}

\hypertarget{actualizar-documentos-3}{%
\subsection{Actualizar documentos:}\label{actualizar-documentos-3}}

\begin{Shaded}
\begin{Highlighting}[]
\NormalTok{db.coleccion.update(\{ campo: valor \}, \{ $set: \{ campo\_actualizado: nuevo\_valor \} \});}
\end{Highlighting}
\end{Shaded}

\hypertarget{eliminar-documentos-3}{%
\subsection{Eliminar documentos:}\label{eliminar-documentos-3}}

\begin{Shaded}
\begin{Highlighting}[]
\NormalTok{db.coleccion.remove(\{ campo: valor \});}
\end{Highlighting}
\end{Shaded}

\hypertarget{ejemplo---inserciuxf3n-de-un-documento-en-mongodb-1}{%
\section{Ejemplo - Inserción de un Documento en
MongoDB:}\label{ejemplo---inserciuxf3n-de-un-documento-en-mongodb-1}}

\begin{Shaded}
\begin{Highlighting}[]
\NormalTok{// Insertar un documento en la colección \textquotesingle{}productos\textquotesingle{}}
\NormalTok{db.productos.insert(\{ nombre: "Camiseta", precio: 20 \});}
\end{Highlighting}
\end{Shaded}

\begin{tcolorbox}[enhanced jigsaw, colbacktitle=quarto-callout-important-color!10!white, toprule=.15mm, leftrule=.75mm, titlerule=0mm, opacityback=0, rightrule=.15mm, opacitybacktitle=0.6, breakable, left=2mm, coltitle=black, title=\textcolor{quarto-callout-important-color}{\faExclamation}\hspace{0.5em}{Actividad Práctica:}, toptitle=1mm, bottomtitle=1mm, arc=.35mm, bottomrule=.15mm, colback=white, colframe=quarto-callout-important-color-frame]

Conéctate a la base de datos MongoDB.

Inserta al menos dos documentos en la colección `productos'.

Realiza una consulta para obtener todos los documentos de la colección
`productos'.

Actualiza el precio de uno de los documentos en la colección.

Elimina un documento de la colección.

\end{tcolorbox}

\hypertarget{explicaciuxf3n-de-la-actividad-77}{%
\section{Explicación de la
Actividad:}\label{explicaciuxf3n-de-la-actividad-77}}

\begin{quote}
\begin{quote}
\begin{quote}
\begin{quote}
\begin{quote}
\begin{quote}
\begin{quote}
e8ed08b1a5bbe1e369719187cfc4de7f7e2a41a9 Esta actividad permite a los
participantes aplicar los conocimientos adquiridos en la inserción,
consulta, actualización y eliminación de documentos en una base de datos
MongoDB. Les ayuda a ganar experiencia práctica en la manipulación de
datos en una base de datos NoSQL.
\end{quote}
\end{quote}
\end{quote}
\end{quote}
\end{quote}
\end{quote}
\end{quote}

\part{Unidad 11: ¿Cómo me amplío con Python?}

\hypertarget{introducciuxf3n-a-data-science}{%
\chapter{Introducción a Data
Science}\label{introducciuxf3n-a-data-science}}

\textless\textless\textless\textless\textless\textless\textless{} HEAD

En esta lección, exploraremos el emocionante campo de la Ciencia de
Datos y cómo Python se ha convertido en una herramienta esencial en este
ámbito. Aprenderemos qué es la Ciencia de Datos, su importancia y cómo
Python se utiliza para analizar y visualizar datos.

\hypertarget{conceptos-clave-62}{%
\section{Conceptos Clave:}\label{conceptos-clave-62}}

\hypertarget{ciencia-de-datos}{%
\subsection{Ciencia de Datos:}\label{ciencia-de-datos}}

Proceso de extracción de conocimiento y perspectivas a partir de datos.

\hypertarget{uso-de-python-en-data-science}{%
\subsection{Uso de Python en Data
Science:}\label{uso-de-python-en-data-science}}

Bibliotecas como NumPy, Pandas y Matplotlib.

\hypertarget{ejemplos-de-aplicaciuxf3n}{%
\subsection{Ejemplos de Aplicación:}\label{ejemplos-de-aplicaciuxf3n}}

Análisis de datos,

Visualización,

Aprendizaje Automático,

etc.

\hypertarget{ejemplo---uso-de-pandas-para-anuxe1lisis-de-datos}{%
\section{Ejemplo - Uso de Pandas para Análisis de
Datos:}\label{ejemplo---uso-de-pandas-para-anuxe1lisis-de-datos}}

\begin{Shaded}
\begin{Highlighting}[]
\ImportTok{import}\NormalTok{ pandas }\ImportTok{as}\NormalTok{ pd}

\NormalTok{data }\OperatorTok{=}\NormalTok{ \{}
    \StringTok{\textquotesingle{}nombre\textquotesingle{}}\NormalTok{: [}\StringTok{\textquotesingle{}Juan\textquotesingle{}}\NormalTok{, }\StringTok{\textquotesingle{}María\textquotesingle{}}\NormalTok{, }\StringTok{\textquotesingle{}Pedro\textquotesingle{}}\NormalTok{],}
    \StringTok{\textquotesingle{}edad\textquotesingle{}}\NormalTok{: [}\DecValTok{25}\NormalTok{, }\DecValTok{30}\NormalTok{, }\DecValTok{28}\NormalTok{]}
\NormalTok{\}}

\NormalTok{df }\OperatorTok{=}\NormalTok{ pd.DataFrame(data)}
\BuiltInTok{print}\NormalTok{(df)}
\end{Highlighting}
\end{Shaded}

\begin{tcolorbox}[enhanced jigsaw, colbacktitle=quarto-callout-important-color!10!white, toprule=.15mm, leftrule=.75mm, titlerule=0mm, opacityback=0, rightrule=.15mm, opacitybacktitle=0.6, breakable, left=2mm, coltitle=black, title=\textcolor{quarto-callout-important-color}{\faExclamation}\hspace{0.5em}{Actividad Práctica:}, toptitle=1mm, bottomtitle=1mm, arc=.35mm, bottomrule=.15mm, colback=white, colframe=quarto-callout-important-color-frame]

Investiga y elige un conjunto de datos disponible en línea.

Utiliza la biblioteca Pandas para cargar y analizar los datos.

Realiza un análisis simple, como calcular estadísticas descriptivas, en
el conjunto de datos.

\end{tcolorbox}

\hypertarget{explicaciuxf3n-de-la-actividad-78}{%
\section{Explicación de la
Actividad:}\label{explicaciuxf3n-de-la-actividad-78}}

======= \#\# Introducción a Data Science

En esta lección, exploraremos el emocionante campo de la Ciencia de
Datos y cómo Python se ha convertido en una herramienta esencial en este
ámbito. Aprenderemos qué es la Ciencia de Datos, su importancia y cómo
Python se utiliza para analizar y visualizar datos.

\hypertarget{conceptos-clave-63}{%
\section{Conceptos Clave:}\label{conceptos-clave-63}}

\hypertarget{ciencia-de-datos-1}{%
\subsection{Ciencia de Datos:}\label{ciencia-de-datos-1}}

Proceso de extracción de conocimiento y perspectivas a partir de datos.

\hypertarget{uso-de-python-en-data-science-1}{%
\subsection{Uso de Python en Data
Science:}\label{uso-de-python-en-data-science-1}}

Bibliotecas como NumPy, Pandas y Matplotlib.

\hypertarget{ejemplos-de-aplicaciuxf3n-1}{%
\subsection{Ejemplos de Aplicación:}\label{ejemplos-de-aplicaciuxf3n-1}}

Análisis de datos,

Visualización,

Aprendizaje Automático,

etc.

\hypertarget{ejemplo---uso-de-pandas-para-anuxe1lisis-de-datos-1}{%
\section{Ejemplo - Uso de Pandas para Análisis de
Datos:}\label{ejemplo---uso-de-pandas-para-anuxe1lisis-de-datos-1}}

\begin{Shaded}
\begin{Highlighting}[]
\ImportTok{import}\NormalTok{ pandas }\ImportTok{as}\NormalTok{ pd}

\NormalTok{data }\OperatorTok{=}\NormalTok{ \{}
    \StringTok{\textquotesingle{}nombre\textquotesingle{}}\NormalTok{: [}\StringTok{\textquotesingle{}Juan\textquotesingle{}}\NormalTok{, }\StringTok{\textquotesingle{}María\textquotesingle{}}\NormalTok{, }\StringTok{\textquotesingle{}Pedro\textquotesingle{}}\NormalTok{],}
    \StringTok{\textquotesingle{}edad\textquotesingle{}}\NormalTok{: [}\DecValTok{25}\NormalTok{, }\DecValTok{30}\NormalTok{, }\DecValTok{28}\NormalTok{]}
\NormalTok{\}}

\NormalTok{df }\OperatorTok{=}\NormalTok{ pd.DataFrame(data)}
\BuiltInTok{print}\NormalTok{(df)}
\end{Highlighting}
\end{Shaded}

\begin{tcolorbox}[enhanced jigsaw, colbacktitle=quarto-callout-important-color!10!white, toprule=.15mm, leftrule=.75mm, titlerule=0mm, opacityback=0, rightrule=.15mm, opacitybacktitle=0.6, breakable, left=2mm, coltitle=black, title=\textcolor{quarto-callout-important-color}{\faExclamation}\hspace{0.5em}{Actividad Práctica:}, toptitle=1mm, bottomtitle=1mm, arc=.35mm, bottomrule=.15mm, colback=white, colframe=quarto-callout-important-color-frame]

Investiga y elige un conjunto de datos disponible en línea.

Utiliza la biblioteca Pandas para cargar y analizar los datos.

Realiza un análisis simple, como calcular estadísticas descriptivas, en
el conjunto de datos.

\end{tcolorbox}

\hypertarget{explicaciuxf3n-de-la-actividad-79}{%
\section{Explicación de la
Actividad:}\label{explicaciuxf3n-de-la-actividad-79}}

\begin{quote}
\begin{quote}
\begin{quote}
\begin{quote}
\begin{quote}
\begin{quote}
\begin{quote}
e8ed08b1a5bbe1e369719187cfc4de7f7e2a41a9 Esta actividad permite a los
participantes explorar la aplicación de Python en el campo de la Ciencia
de Datos. Les ayuda a comprender cómo utilizar bibliotecas como Pandas
para analizar datos y extraer información útil.
\end{quote}
\end{quote}
\end{quote}
\end{quote}
\end{quote}
\end{quote}
\end{quote}

\hypertarget{introducciuxf3n-a-django-framework}{%
\chapter{Introducción a Django
Framework}\label{introducciuxf3n-a-django-framework}}

\textless\textless\textless\textless\textless\textless\textless{} HEAD

En esta lección, nos adentraremos en el mundo de Django, un popular
framework de desarrollo web en Python. Aprenderemos qué es Django, cómo
instalarlo y cómo crear una aplicación web básica utilizando este
framework.

\hypertarget{quuxe9-es-django}{%
\section{Qué es Django:}\label{quuxe9-es-django}}

Django es un framework de desarrollo web de alto nivel y de código
abierto.

Proporciona una estructura organizada para crear aplicaciones web de
manera eficiente.

\hypertarget{instalaciuxf3n-de-django}{%
\section{Instalación de Django:}\label{instalaciuxf3n-de-django}}

\hypertarget{instalar-django-utilizando-pip}{%
\subsection{Instalar Django utilizando
pip:}\label{instalar-django-utilizando-pip}}

\begin{Shaded}
\begin{Highlighting}[]
\ExtensionTok{pip}\NormalTok{ install django}
\end{Highlighting}
\end{Shaded}

\hypertarget{verificar-la-instalaciuxf3n}{%
\subsection{Verificar la
instalación:}\label{verificar-la-instalaciuxf3n}}

\begin{Shaded}
\begin{Highlighting}[]
\ExtensionTok{django{-}admin} \AttributeTok{{-}{-}version}
\end{Highlighting}
\end{Shaded}

\hypertarget{creaciuxf3n-de-una-aplicaciuxf3n-web-buxe1sica}{%
\section{Creación de una Aplicación Web
Básica:}\label{creaciuxf3n-de-una-aplicaciuxf3n-web-buxe1sica}}

\hypertarget{crear-un-nuevo-proyecto}{%
\subsection{Crear un nuevo proyecto:}\label{crear-un-nuevo-proyecto}}

\begin{Shaded}
\begin{Highlighting}[]
\ExtensionTok{django{-}admin}\NormalTok{ startproject proyecto .}
\end{Highlighting}
\end{Shaded}

\hypertarget{crear-una-nueva-aplicaciuxf3n-dentro-del-proyecto}{%
\subsection{Crear una nueva aplicación dentro del
proyecto:}\label{crear-una-nueva-aplicaciuxf3n-dentro-del-proyecto}}

\begin{Shaded}
\begin{Highlighting}[]
\ExtensionTok{python}\NormalTok{ manage.py startapp app}
\end{Highlighting}
\end{Shaded}

\hypertarget{ejemplo---creaciuxf3n-de-una-puxe1gina-web-con-django}{%
\section{Ejemplo - Creación de una Página Web con
Django:}\label{ejemplo---creaciuxf3n-de-una-puxe1gina-web-con-django}}

\begin{Shaded}
\begin{Highlighting}[]
\CommentTok{\# views.py}
\ImportTok{from}\NormalTok{ django.http }\ImportTok{import}\NormalTok{ HttpResponse}

\KeywordTok{def}\NormalTok{ hola\_mundo(request):}
    \ControlFlowTok{return}\NormalTok{ HttpResponse(}\StringTok{"¡Hola, mundo!"}\NormalTok{)}
\end{Highlighting}
\end{Shaded}

\begin{Shaded}
\begin{Highlighting}[]
\CommentTok{\# urls.py}
\ImportTok{from}\NormalTok{ django.urls }\ImportTok{import}\NormalTok{ path}
\ImportTok{from}\NormalTok{ . }\ImportTok{import}\NormalTok{ views}

\NormalTok{urlpatterns }\OperatorTok{=}\NormalTok{ [}
\NormalTok{    path(}\StringTok{\textquotesingle{}hola/\textquotesingle{}}\NormalTok{, views.hola\_mundo, name}\OperatorTok{=}\StringTok{\textquotesingle{}hola\_mundo\textquotesingle{}}\NormalTok{),}
\NormalTok{]}
\end{Highlighting}
\end{Shaded}

\begin{tcolorbox}[enhanced jigsaw, colbacktitle=quarto-callout-important-color!10!white, toprule=.15mm, leftrule=.75mm, titlerule=0mm, opacityback=0, rightrule=.15mm, opacitybacktitle=0.6, breakable, left=2mm, coltitle=black, title=\textcolor{quarto-callout-important-color}{\faExclamation}\hspace{0.5em}{Actividad Práctica:}, toptitle=1mm, bottomtitle=1mm, arc=.35mm, bottomrule=.15mm, colback=white, colframe=quarto-callout-important-color-frame]

Instala Django en tu entorno.

Crea un proyecto llamado `blog' y una aplicación llamada `articulos'.

Crea una vista que muestre un mensaje de bienvenida en la página
principal.

Configura una URL para acceder a la vista creada.

\end{tcolorbox}

\hypertarget{explicaciuxf3n-de-la-actividad-80}{%
\section{Explicación de la
Actividad:}\label{explicaciuxf3n-de-la-actividad-80}}

Esta actividad permite a los participantes experimentar con la creación
de proyectos y aplicaciones utilizando Django. Les ayuda a comprender
cómo estructurar una aplicación web utilizando este framework y cómo
definir rutas y vistas para mostrar contenido en el navegador.

======= \#\# Introducción a Django Framework

En esta lección, nos adentraremos en el mundo de Django, un popular
framework de desarrollo web en Python. Aprenderemos qué es Django, cómo
instalarlo y cómo crear una aplicación web básica utilizando este
framework.

\hypertarget{quuxe9-es-django-1}{%
\section{Qué es Django:}\label{quuxe9-es-django-1}}

Django es un framework de desarrollo web de alto nivel y de código
abierto.

Proporciona una estructura organizada para crear aplicaciones web de
manera eficiente.

\hypertarget{instalaciuxf3n-de-django-1}{%
\section{Instalación de Django:}\label{instalaciuxf3n-de-django-1}}

\hypertarget{instalar-django-utilizando-pip-1}{%
\subsection{Instalar Django utilizando
pip:}\label{instalar-django-utilizando-pip-1}}

\begin{Shaded}
\begin{Highlighting}[]
\ExtensionTok{pip}\NormalTok{ install django}
\end{Highlighting}
\end{Shaded}

\hypertarget{verificar-la-instalaciuxf3n-1}{%
\subsection{Verificar la
instalación:}\label{verificar-la-instalaciuxf3n-1}}

\begin{Shaded}
\begin{Highlighting}[]
\ExtensionTok{django{-}admin} \AttributeTok{{-}{-}version}
\end{Highlighting}
\end{Shaded}

\hypertarget{creaciuxf3n-de-una-aplicaciuxf3n-web-buxe1sica-1}{%
\section{Creación de una Aplicación Web
Básica:}\label{creaciuxf3n-de-una-aplicaciuxf3n-web-buxe1sica-1}}

\hypertarget{crear-un-nuevo-proyecto-1}{%
\subsection{Crear un nuevo proyecto:}\label{crear-un-nuevo-proyecto-1}}

\begin{Shaded}
\begin{Highlighting}[]
\ExtensionTok{django{-}admin}\NormalTok{ startproject proyecto .}
\end{Highlighting}
\end{Shaded}

\hypertarget{crear-una-nueva-aplicaciuxf3n-dentro-del-proyecto-1}{%
\subsection{Crear una nueva aplicación dentro del
proyecto:}\label{crear-una-nueva-aplicaciuxf3n-dentro-del-proyecto-1}}

\begin{Shaded}
\begin{Highlighting}[]
\ExtensionTok{python}\NormalTok{ manage.py startapp app}
\end{Highlighting}
\end{Shaded}

\hypertarget{ejemplo---creaciuxf3n-de-una-puxe1gina-web-con-django-1}{%
\section{Ejemplo - Creación de una Página Web con
Django:}\label{ejemplo---creaciuxf3n-de-una-puxe1gina-web-con-django-1}}

\begin{Shaded}
\begin{Highlighting}[]
\CommentTok{\# views.py}
\ImportTok{from}\NormalTok{ django.http }\ImportTok{import}\NormalTok{ HttpResponse}

\KeywordTok{def}\NormalTok{ hola\_mundo(request):}
    \ControlFlowTok{return}\NormalTok{ HttpResponse(}\StringTok{"¡Hola, mundo!"}\NormalTok{)}
\end{Highlighting}
\end{Shaded}

\begin{Shaded}
\begin{Highlighting}[]
\CommentTok{\# urls.py}
\ImportTok{from}\NormalTok{ django.urls }\ImportTok{import}\NormalTok{ path}
\ImportTok{from}\NormalTok{ . }\ImportTok{import}\NormalTok{ views}

\NormalTok{urlpatterns }\OperatorTok{=}\NormalTok{ [}
\NormalTok{    path(}\StringTok{\textquotesingle{}hola/\textquotesingle{}}\NormalTok{, views.hola\_mundo, name}\OperatorTok{=}\StringTok{\textquotesingle{}hola\_mundo\textquotesingle{}}\NormalTok{),}
\NormalTok{]}
\end{Highlighting}
\end{Shaded}

\begin{tcolorbox}[enhanced jigsaw, colbacktitle=quarto-callout-important-color!10!white, toprule=.15mm, leftrule=.75mm, titlerule=0mm, opacityback=0, rightrule=.15mm, opacitybacktitle=0.6, breakable, left=2mm, coltitle=black, title=\textcolor{quarto-callout-important-color}{\faExclamation}\hspace{0.5em}{Actividad Práctica:}, toptitle=1mm, bottomtitle=1mm, arc=.35mm, bottomrule=.15mm, colback=white, colframe=quarto-callout-important-color-frame]

Instala Django en tu entorno.

Crea un proyecto llamado `blog' y una aplicación llamada `articulos'.

Crea una vista que muestre un mensaje de bienvenida en la página
principal.

Configura una URL para acceder a la vista creada.

\end{tcolorbox}

\hypertarget{explicaciuxf3n-de-la-actividad-81}{%
\section{Explicación de la
Actividad:}\label{explicaciuxf3n-de-la-actividad-81}}

Esta actividad permite a los participantes experimentar con la creación
de proyectos y aplicaciones utilizando Django. Les ayuda a comprender
cómo estructurar una aplicación web utilizando este framework y cómo
definir rutas y vistas para mostrar contenido en el navegador.

\begin{quote}
\begin{quote}
\begin{quote}
\begin{quote}
\begin{quote}
\begin{quote}
\begin{quote}
e8ed08b1a5bbe1e369719187cfc4de7f7e2a41a9
\end{quote}
\end{quote}
\end{quote}
\end{quote}
\end{quote}
\end{quote}
\end{quote}

\hypertarget{introducciuxf3n-a-fastapi-y-pyscript}{%
\chapter{Introducción a FastAPI y
PyScript}\label{introducciuxf3n-a-fastapi-y-pyscript}}

\textless\textless\textless\textless\textless\textless\textless{} HEAD

En esta lección, exploraremos FastAPI, un moderno framework de
desarrollo web en Python, y PyScript, una herramienta que permite crear
scripts de Python en un entorno interactivo. Aprenderemos cómo utilizar
FastAPI para construir APIs rápidas y cómo utilizar PyScript para
escribir y ejecutar scripts de manera interactiva.

\hypertarget{quuxe9-es-fastapi}{%
\section{Qué es FastAPI:}\label{quuxe9-es-fastapi}}

FastAPI es un framework de desarrollo web rápido (high-performance)
basado en Python.

Permite construir APIs rápidas y seguras de manera sencilla.

\hypertarget{instalaciuxf3n-de-fastapi}{%
\section{Instalación de FastAPI:}\label{instalaciuxf3n-de-fastapi}}

\hypertarget{instalar-fastapi-utilizando-pip}{%
\subsection{Instalar FastAPI utilizando
pip:}\label{instalar-fastapi-utilizando-pip}}

\begin{Shaded}
\begin{Highlighting}[]
\ExtensionTok{pip}\NormalTok{ install fastapi}
\end{Highlighting}
\end{Shaded}

\hypertarget{instalar-el-servidor-asgi-por-ejemplo-uvicorn}{%
\subsection{Instalar el servidor ASGI (por ejemplo,
Uvicorn):}\label{instalar-el-servidor-asgi-por-ejemplo-uvicorn}}

\begin{Shaded}
\begin{Highlighting}[]
\ExtensionTok{pip}\NormalTok{ install uvicorn}
\end{Highlighting}
\end{Shaded}

\hypertarget{creaciuxf3n-de-una-api-buxe1sica-con-fastapi}{%
\section{Creación de una API Básica con
FastAPI:}\label{creaciuxf3n-de-una-api-buxe1sica-con-fastapi}}

\begin{Shaded}
\begin{Highlighting}[]
\ImportTok{from}\NormalTok{ fastapi }\ImportTok{import}\NormalTok{ FastAPI}

\NormalTok{app }\OperatorTok{=}\NormalTok{ FastAPI()}

\AttributeTok{@app.get}\NormalTok{(}\StringTok{"/"}\NormalTok{)}
\KeywordTok{def}\NormalTok{ leer\_raiz():}
    \ControlFlowTok{return}\NormalTok{ \{}\StringTok{"mensaje"}\NormalTok{: }\StringTok{"¡Hola desde FastAPI!"}\NormalTok{\}}
\end{Highlighting}
\end{Shaded}

\hypertarget{uso-de-pyscript}{%
\section{Uso de PyScript:}\label{uso-de-pyscript}}

\hypertarget{instalar-pyscript-utilizando-pip}{%
\subsection{Instalar PyScript utilizando
pip:}\label{instalar-pyscript-utilizando-pip}}

\begin{Shaded}
\begin{Highlighting}[]
\ExtensionTok{pip}\NormalTok{ install pyscript}
\end{Highlighting}
\end{Shaded}

\hypertarget{ejecutar-pyscript-en-un-entorno-interactivo}{%
\subsection{Ejecutar PyScript en un entorno
interactivo:}\label{ejecutar-pyscript-en-un-entorno-interactivo}}

\begin{Shaded}
\begin{Highlighting}[]
\ExtensionTok{pyscript}
\end{Highlighting}
\end{Shaded}

\hypertarget{ejemplo---ejecuciuxf3n-de-pyscript}{%
\section{Ejemplo - Ejecución de
PyScript:}\label{ejemplo---ejecuciuxf3n-de-pyscript}}

\begin{Shaded}
\begin{Highlighting}[]
\NormalTok{a }\OperatorTok{=} \DecValTok{5}
\NormalTok{b }\OperatorTok{=} \DecValTok{10}
\NormalTok{a }\OperatorTok{+}\NormalTok{ b}
\DecValTok{15}
\end{Highlighting}
\end{Shaded}

\begin{tcolorbox}[enhanced jigsaw, colbacktitle=quarto-callout-important-color!10!white, toprule=.15mm, leftrule=.75mm, titlerule=0mm, opacityback=0, rightrule=.15mm, opacitybacktitle=0.6, breakable, left=2mm, coltitle=black, title=\textcolor{quarto-callout-important-color}{\faExclamation}\hspace{0.5em}{Actividad Práctica:}, toptitle=1mm, bottomtitle=1mm, arc=.35mm, bottomrule=.15mm, colback=white, colframe=quarto-callout-important-color-frame]

Instala FastAPI en tu entorno.

Crea una API con FastAPI que tenga al menos un endpoint para obtener
información.

Instala PyScript en tu entorno y realiza algunos cálculos y experimentos
interactivos.

\end{tcolorbox}

\hypertarget{explicaciuxf3n-de-la-actividad-82}{%
\section{Explicación de la
Actividad:}\label{explicaciuxf3n-de-la-actividad-82}}

======= \#\# Introducción a FastAPI y PyScript

En esta lección, exploraremos FastAPI, un moderno framework de
desarrollo web en Python, y PyScript, una herramienta que permite crear
scripts de Python en un entorno interactivo. Aprenderemos cómo utilizar
FastAPI para construir APIs rápidas y cómo utilizar PyScript para
escribir y ejecutar scripts de manera interactiva.

\hypertarget{quuxe9-es-fastapi-1}{%
\section{Qué es FastAPI:}\label{quuxe9-es-fastapi-1}}

FastAPI es un framework de desarrollo web rápido (high-performance)
basado en Python.

Permite construir APIs rápidas y seguras de manera sencilla.

\hypertarget{instalaciuxf3n-de-fastapi-1}{%
\section{Instalación de FastAPI:}\label{instalaciuxf3n-de-fastapi-1}}

\hypertarget{instalar-fastapi-utilizando-pip-1}{%
\subsection{Instalar FastAPI utilizando
pip:}\label{instalar-fastapi-utilizando-pip-1}}

\begin{Shaded}
\begin{Highlighting}[]
\ExtensionTok{pip}\NormalTok{ install fastapi}
\end{Highlighting}
\end{Shaded}

\hypertarget{instalar-el-servidor-asgi-por-ejemplo-uvicorn-1}{%
\subsection{Instalar el servidor ASGI (por ejemplo,
Uvicorn):}\label{instalar-el-servidor-asgi-por-ejemplo-uvicorn-1}}

\begin{Shaded}
\begin{Highlighting}[]
\ExtensionTok{pip}\NormalTok{ install uvicorn}
\end{Highlighting}
\end{Shaded}

\hypertarget{creaciuxf3n-de-una-api-buxe1sica-con-fastapi-1}{%
\section{Creación de una API Básica con
FastAPI:}\label{creaciuxf3n-de-una-api-buxe1sica-con-fastapi-1}}

\begin{Shaded}
\begin{Highlighting}[]
\ImportTok{from}\NormalTok{ fastapi }\ImportTok{import}\NormalTok{ FastAPI}

\NormalTok{app }\OperatorTok{=}\NormalTok{ FastAPI()}

\AttributeTok{@app.get}\NormalTok{(}\StringTok{"/"}\NormalTok{)}
\KeywordTok{def}\NormalTok{ leer\_raiz():}
    \ControlFlowTok{return}\NormalTok{ \{}\StringTok{"mensaje"}\NormalTok{: }\StringTok{"¡Hola desde FastAPI!"}\NormalTok{\}}
\end{Highlighting}
\end{Shaded}

\hypertarget{uso-de-pyscript-1}{%
\section{Uso de PyScript:}\label{uso-de-pyscript-1}}

\hypertarget{instalar-pyscript-utilizando-pip-1}{%
\subsection{Instalar PyScript utilizando
pip:}\label{instalar-pyscript-utilizando-pip-1}}

\begin{Shaded}
\begin{Highlighting}[]
\ExtensionTok{pip}\NormalTok{ install pyscript}
\end{Highlighting}
\end{Shaded}

\hypertarget{ejecutar-pyscript-en-un-entorno-interactivo-1}{%
\subsection{Ejecutar PyScript en un entorno
interactivo:}\label{ejecutar-pyscript-en-un-entorno-interactivo-1}}

\begin{Shaded}
\begin{Highlighting}[]
\ExtensionTok{pyscript}
\end{Highlighting}
\end{Shaded}

\hypertarget{ejemplo---ejecuciuxf3n-de-pyscript-1}{%
\section{Ejemplo - Ejecución de
PyScript:}\label{ejemplo---ejecuciuxf3n-de-pyscript-1}}

\begin{Shaded}
\begin{Highlighting}[]
\NormalTok{a }\OperatorTok{=} \DecValTok{5}
\NormalTok{b }\OperatorTok{=} \DecValTok{10}
\NormalTok{a }\OperatorTok{+}\NormalTok{ b}
\DecValTok{15}
\end{Highlighting}
\end{Shaded}

\begin{tcolorbox}[enhanced jigsaw, colbacktitle=quarto-callout-important-color!10!white, toprule=.15mm, leftrule=.75mm, titlerule=0mm, opacityback=0, rightrule=.15mm, opacitybacktitle=0.6, breakable, left=2mm, coltitle=black, title=\textcolor{quarto-callout-important-color}{\faExclamation}\hspace{0.5em}{Actividad Práctica:}, toptitle=1mm, bottomtitle=1mm, arc=.35mm, bottomrule=.15mm, colback=white, colframe=quarto-callout-important-color-frame]

Instala FastAPI en tu entorno.

Crea una API con FastAPI que tenga al menos un endpoint para obtener
información.

Instala PyScript en tu entorno y realiza algunos cálculos y experimentos
interactivos.

\end{tcolorbox}

\hypertarget{explicaciuxf3n-de-la-actividad-83}{%
\section{Explicación de la
Actividad:}\label{explicaciuxf3n-de-la-actividad-83}}

\begin{quote}
\begin{quote}
\begin{quote}
\begin{quote}
\begin{quote}
\begin{quote}
\begin{quote}
e8ed08b1a5bbe1e369719187cfc4de7f7e2a41a9 Esta actividad permite a los
participantes experimentar con FastAPI y PyScript para crear una API
básica y ejecutar scripts interactivos. Les ayuda a comprender cómo
utilizar FastAPI para construir APIs de manera rápida y cómo utilizar
PyScript para escribir y ejecutar código Python de manera interactiva en
la consola.
\end{quote}
\end{quote}
\end{quote}
\end{quote}
\end{quote}
\end{quote}
\end{quote}

\part{Unidad 12: Proyecto: API de Tareas con Django Rest Framework}

\hypertarget{explicaciuxf3n-del-proyecto}{%
\chapter{Explicación del Proyecto}\label{explicaciuxf3n-del-proyecto}}

\textless\textless\textless\textless\textless\textless\textless{} HEAD

En este proyecto, construiremos una API utilizando Django Rest Framework
para gestionar tareas. La API permitirá a los usuarios crear,
actualizar, listar y eliminar tareas. Utilizaremos Django Rest Framework
para definir los modelos, las vistas y las URL necesarias para
interactuar con la API.

\hypertarget{quuxe9-se-necesita-conocer}{%
\section{Qué se necesita conocer:}\label{quuxe9-se-necesita-conocer}}

\begin{itemize}
\tightlist
\item
  Conocimientos básicos de Python.
\item
  Familiaridad con Django y Django Rest Framework.
\item
  Entorno de desarrollo configurado con Django y Django Rest Framework.
\end{itemize}

\hypertarget{estructura-del-proyecto}{%
\section{Estructura del Proyecto:}\label{estructura-del-proyecto}}

\begin{Shaded}
\begin{Highlighting}[]
\NormalTok{proyecto\_api\_tareas/}
\NormalTok{├── api\_tareas/}
\NormalTok{│   ├── migrations/}
\NormalTok{│   ├── templates/}
\NormalTok{│   ├── \_\_init\_\_.py}
\NormalTok{│   ├── admin.py}
\NormalTok{│   ├── apps.py}
\NormalTok{│   ├── models.py}
\NormalTok{│   ├── serializers.py}
\NormalTok{│   ├── tests.py}
\NormalTok{│   └── views.py}
\NormalTok{├── proyecto\_api\_tareas/}
\NormalTok{│   ├── \_\_init\_\_.py}
\NormalTok{│   ├── asgi.py}
\NormalTok{│   ├── settings.py}
\NormalTok{│   ├── urls.py}
\NormalTok{│   └── wsgi.py}
\NormalTok{├── db.sqlite3}
\NormalTok{└── manage.py}
\end{Highlighting}
\end{Shaded}

\hypertarget{cuxf3digo}{%
\section{Código:}\label{cuxf3digo}}

\begin{Shaded}
\begin{Highlighting}[]
\CommentTok{\#models.py:}
\ImportTok{from}\NormalTok{ django.db }\ImportTok{import}\NormalTok{ models}

\KeywordTok{class}\NormalTok{ Tarea(models.Model):}
\NormalTok{    titulo }\OperatorTok{=}\NormalTok{ models.CharField(max\_length}\OperatorTok{=}\DecValTok{100}\NormalTok{)}
\NormalTok{    descripcion }\OperatorTok{=}\NormalTok{ models.TextField()}
\NormalTok{    fecha\_creacion }\OperatorTok{=}\NormalTok{ models.DateTimeField(auto\_now\_add}\OperatorTok{=}\VariableTok{True}\NormalTok{)}
\NormalTok{    completada }\OperatorTok{=}\NormalTok{ models.BooleanField(default}\OperatorTok{=}\VariableTok{False}\NormalTok{)}

    \KeywordTok{def} \FunctionTok{\_\_str\_\_}\NormalTok{(}\VariableTok{self}\NormalTok{):}
        \ControlFlowTok{return} \VariableTok{self}\NormalTok{.titulo}
\end{Highlighting}
\end{Shaded}

\begin{Shaded}
\begin{Highlighting}[]
\CommentTok{\#serializers.py:}

\ImportTok{from}\NormalTok{ rest\_framework }\ImportTok{import}\NormalTok{ serializers}
\ImportTok{from}\NormalTok{ .models }\ImportTok{import}\NormalTok{ Tarea}

\KeywordTok{class}\NormalTok{ TareaSerializer(serializers.ModelSerializer):}
    \KeywordTok{class}\NormalTok{ Meta:}
\NormalTok{        model }\OperatorTok{=}\NormalTok{ Tarea}
\NormalTok{        fields }\OperatorTok{=} \StringTok{\textquotesingle{}\_\_all\_\_\textquotesingle{}}
\end{Highlighting}
\end{Shaded}

\begin{Shaded}
\begin{Highlighting}[]
\CommentTok{\#views.py:}
\ImportTok{from}\NormalTok{ rest\_framework }\ImportTok{import}\NormalTok{ viewsets}
\ImportTok{from}\NormalTok{ .models }\ImportTok{import}\NormalTok{ Tarea}
\ImportTok{from}\NormalTok{ .serializers }\ImportTok{import}\NormalTok{ TareaSerializer}

\KeywordTok{class}\NormalTok{ TareaViewSet(viewsets.ModelViewSet):}
\NormalTok{    queryset }\OperatorTok{=}\NormalTok{ Tarea.objects.}\BuiltInTok{all}\NormalTok{()}
\NormalTok{    serializer\_class }\OperatorTok{=}\NormalTok{ TareaSerializer}

\NormalTok{    urls.py (api\_tareas):}
\end{Highlighting}
\end{Shaded}

\begin{Shaded}
\begin{Highlighting}[]
\ImportTok{from}\NormalTok{ rest\_framework.routers }\ImportTok{import}\NormalTok{ DefaultRouter}
\ImportTok{from}\NormalTok{ .views }\ImportTok{import}\NormalTok{ TareaViewSet}

\NormalTok{router }\OperatorTok{=}\NormalTok{ DefaultRouter()}
\NormalTok{router.register(}\VerbatimStringTok{r\textquotesingle{}tareas\textquotesingle{}}\NormalTok{, TareaViewSet)}

\NormalTok{urlpatterns }\OperatorTok{=}\NormalTok{ router.urls}
\end{Highlighting}
\end{Shaded}

A continuación en el archivo \textbf{settings.py} agregar
\textbf{`rest\_framework'} y \textbf{`api\_tareas'} en
\textbf{INSTALLED\_APPS}.

\begin{tcolorbox}[enhanced jigsaw, colbacktitle=quarto-callout-important-color!10!white, toprule=.15mm, leftrule=.75mm, titlerule=0mm, opacityback=0, rightrule=.15mm, opacitybacktitle=0.6, breakable, left=2mm, coltitle=black, title=\textcolor{quarto-callout-important-color}{\faExclamation}\hspace{0.5em}{Actividad Práctica:}, toptitle=1mm, bottomtitle=1mm, arc=.35mm, bottomrule=.15mm, colback=white, colframe=quarto-callout-important-color-frame]

Configura un proyecto Django y una aplicación llamada `api\_tareas'.

Define el modelo Tarea en models.py con los campos necesarios.

Crea un serializador en serializers.py para el modelo Tarea.

Implementa las vistas en views.py utilizando Django Rest Framework.

Configura las URLs en urls.py para las vistas de la API.

Migrar y ejecutar el servidor para probar la API utilizando el navegador
o herramientas como Postman.

\end{tcolorbox}

\hypertarget{explicaciuxf3n-de-la-actividad-84}{%
\section{Explicación de la
Actividad:}\label{explicaciuxf3n-de-la-actividad-84}}

======= \#\# Explicación del Proyecto

En este proyecto, construiremos una API utilizando Django Rest Framework
para gestionar tareas. La API permitirá a los usuarios crear,
actualizar, listar y eliminar tareas. Utilizaremos Django Rest Framework
para definir los modelos, las vistas y las URL necesarias para
interactuar con la API.

\hypertarget{quuxe9-se-necesita-conocer-1}{%
\section{Qué se necesita conocer:}\label{quuxe9-se-necesita-conocer-1}}

\begin{itemize}
\tightlist
\item
  Conocimientos básicos de Python.
\item
  Familiaridad con Django y Django Rest Framework.
\item
  Entorno de desarrollo configurado con Django y Django Rest Framework.
\end{itemize}

\hypertarget{estructura-del-proyecto-1}{%
\section{Estructura del Proyecto:}\label{estructura-del-proyecto-1}}

\begin{Shaded}
\begin{Highlighting}[]
\NormalTok{proyecto\_api\_tareas/}
\NormalTok{├── api\_tareas/}
\NormalTok{│   ├── migrations/}
\NormalTok{│   ├── templates/}
\NormalTok{│   ├── \_\_init\_\_.py}
\NormalTok{│   ├── admin.py}
\NormalTok{│   ├── apps.py}
\NormalTok{│   ├── models.py}
\NormalTok{│   ├── serializers.py}
\NormalTok{│   ├── tests.py}
\NormalTok{│   └── views.py}
\NormalTok{├── proyecto\_api\_tareas/}
\NormalTok{│   ├── \_\_init\_\_.py}
\NormalTok{│   ├── asgi.py}
\NormalTok{│   ├── settings.py}
\NormalTok{│   ├── urls.py}
\NormalTok{│   └── wsgi.py}
\NormalTok{├── db.sqlite3}
\NormalTok{└── manage.py}
\end{Highlighting}
\end{Shaded}

\hypertarget{cuxf3digo-1}{%
\section{Código:}\label{cuxf3digo-1}}

\begin{Shaded}
\begin{Highlighting}[]
\CommentTok{\#models.py:}
\ImportTok{from}\NormalTok{ django.db }\ImportTok{import}\NormalTok{ models}

\KeywordTok{class}\NormalTok{ Tarea(models.Model):}
\NormalTok{    titulo }\OperatorTok{=}\NormalTok{ models.CharField(max\_length}\OperatorTok{=}\DecValTok{100}\NormalTok{)}
\NormalTok{    descripcion }\OperatorTok{=}\NormalTok{ models.TextField()}
\NormalTok{    fecha\_creacion }\OperatorTok{=}\NormalTok{ models.DateTimeField(auto\_now\_add}\OperatorTok{=}\VariableTok{True}\NormalTok{)}
\NormalTok{    completada }\OperatorTok{=}\NormalTok{ models.BooleanField(default}\OperatorTok{=}\VariableTok{False}\NormalTok{)}

    \KeywordTok{def} \FunctionTok{\_\_str\_\_}\NormalTok{(}\VariableTok{self}\NormalTok{):}
        \ControlFlowTok{return} \VariableTok{self}\NormalTok{.titulo}
\end{Highlighting}
\end{Shaded}

\begin{Shaded}
\begin{Highlighting}[]
\CommentTok{\#serializers.py:}

\ImportTok{from}\NormalTok{ rest\_framework }\ImportTok{import}\NormalTok{ serializers}
\ImportTok{from}\NormalTok{ .models }\ImportTok{import}\NormalTok{ Tarea}

\KeywordTok{class}\NormalTok{ TareaSerializer(serializers.ModelSerializer):}
    \KeywordTok{class}\NormalTok{ Meta:}
\NormalTok{        model }\OperatorTok{=}\NormalTok{ Tarea}
\NormalTok{        fields }\OperatorTok{=} \StringTok{\textquotesingle{}\_\_all\_\_\textquotesingle{}}
\end{Highlighting}
\end{Shaded}

\begin{Shaded}
\begin{Highlighting}[]
\CommentTok{\#views.py:}
\ImportTok{from}\NormalTok{ rest\_framework }\ImportTok{import}\NormalTok{ viewsets}
\ImportTok{from}\NormalTok{ .models }\ImportTok{import}\NormalTok{ Tarea}
\ImportTok{from}\NormalTok{ .serializers }\ImportTok{import}\NormalTok{ TareaSerializer}

\KeywordTok{class}\NormalTok{ TareaViewSet(viewsets.ModelViewSet):}
\NormalTok{    queryset }\OperatorTok{=}\NormalTok{ Tarea.objects.}\BuiltInTok{all}\NormalTok{()}
\NormalTok{    serializer\_class }\OperatorTok{=}\NormalTok{ TareaSerializer}

\NormalTok{    urls.py (api\_tareas):}
\end{Highlighting}
\end{Shaded}

\begin{Shaded}
\begin{Highlighting}[]
\ImportTok{from}\NormalTok{ rest\_framework.routers }\ImportTok{import}\NormalTok{ DefaultRouter}
\ImportTok{from}\NormalTok{ .views }\ImportTok{import}\NormalTok{ TareaViewSet}

\NormalTok{router }\OperatorTok{=}\NormalTok{ DefaultRouter()}
\NormalTok{router.register(}\VerbatimStringTok{r\textquotesingle{}tareas\textquotesingle{}}\NormalTok{, TareaViewSet)}

\NormalTok{urlpatterns }\OperatorTok{=}\NormalTok{ router.urls}
\end{Highlighting}
\end{Shaded}

A continuación en el archivo \textbf{settings.py} agregar
\textbf{`rest\_framework'} y \textbf{`api\_tareas'} en
\textbf{INSTALLED\_APPS}.

\begin{tcolorbox}[enhanced jigsaw, colbacktitle=quarto-callout-important-color!10!white, toprule=.15mm, leftrule=.75mm, titlerule=0mm, opacityback=0, rightrule=.15mm, opacitybacktitle=0.6, breakable, left=2mm, coltitle=black, title=\textcolor{quarto-callout-important-color}{\faExclamation}\hspace{0.5em}{Actividad Práctica:}, toptitle=1mm, bottomtitle=1mm, arc=.35mm, bottomrule=.15mm, colback=white, colframe=quarto-callout-important-color-frame]

Configura un proyecto Django y una aplicación llamada `api\_tareas'.

Define el modelo Tarea en models.py con los campos necesarios.

Crea un serializador en serializers.py para el modelo Tarea.

Implementa las vistas en views.py utilizando Django Rest Framework.

Configura las URLs en urls.py para las vistas de la API.

Migrar y ejecutar el servidor para probar la API utilizando el navegador
o herramientas como Postman.

\end{tcolorbox}

\hypertarget{explicaciuxf3n-de-la-actividad-85}{%
\section{Explicación de la
Actividad:}\label{explicaciuxf3n-de-la-actividad-85}}

\begin{quote}
\begin{quote}
\begin{quote}
\begin{quote}
\begin{quote}
\begin{quote}
\begin{quote}
e8ed08b1a5bbe1e369719187cfc4de7f7e2a41a9 Este proyecto permite a los
participantes aplicar los conocimientos adquiridos en Django y Django
Rest Framework para crear una API de gestión de tareas. Aprenden cómo
definir modelos, serializadores, vistas y URLs en Django Rest Framework
para construir una API completa. Les ayuda a comprender cómo desarrollar
aplicaciones web con APIs utilizando tecnologías modernas.
\end{quote}
\end{quote}
\end{quote}
\end{quote}
\end{quote}
\end{quote}
\end{quote}

\part{Ejercicios}

\hypertarget{ejercicio-1}{%
\chapter{Ejercicio 1:}\label{ejercicio-1}}

\textless\textless\textless\textless\textless\textless\textless{} HEAD

¿Cómo se define una variable en Python?

Respuesta:

Se define una variable en Python asignándole un nombre y un valor. Por
ejemplo:

\begin{Shaded}
\begin{Highlighting}[]
\NormalTok{nombre }\OperatorTok{=} \StringTok{"Juan"}
\end{Highlighting}
\end{Shaded}

======= \#\# Ejercicio 1:

¿Cómo se define una variable en Python?

Respuesta:

Se define una variable en Python asignándole un nombre y un valor. Por
ejemplo:

\begin{Shaded}
\begin{Highlighting}[]
\NormalTok{nombre }\OperatorTok{=} \StringTok{"Juan"}
\end{Highlighting}
\end{Shaded}

\begin{quote}
\begin{quote}
\begin{quote}
\begin{quote}
\begin{quote}
\begin{quote}
\begin{quote}
e8ed08b1a5bbe1e369719187cfc4de7f7e2a41a9
\end{quote}
\end{quote}
\end{quote}
\end{quote}
\end{quote}
\end{quote}
\end{quote}

\hypertarget{ejercicio-2}{%
\chapter{Ejercicio 2:}\label{ejercicio-2}}

\textless\textless\textless\textless\textless\textless\textless{} HEAD
¿Cuál es el resultado de la siguiente expresión?

\begin{Shaded}
\begin{Highlighting}[]
\NormalTok{x }\OperatorTok{=} \DecValTok{10}
\NormalTok{y }\OperatorTok{=} \DecValTok{5}
\NormalTok{resultado }\OperatorTok{=}\NormalTok{ x }\OperatorTok{+}\NormalTok{ y}
\BuiltInTok{print}\NormalTok{(resultado)}
\end{Highlighting}
\end{Shaded}

Respuesta:

\hypertarget{el-resultado-de-la-expresiuxf3n-es-15-ya-que-se-suman-los-valores-de-las-variables-x-10-y-y-5.}{%
\chapter{\texorpdfstring{El resultado de la expresión es 15, ya que se
suman los valores de las variables \texttt{x} (10) y \texttt{y}
(5).}{El resultado de la expresión es 15, ya que se suman los valores de las variables x (10) y y (5).}}\label{el-resultado-de-la-expresiuxf3n-es-15-ya-que-se-suman-los-valores-de-las-variables-x-10-y-y-5.}}

\hypertarget{ejercicio-2-1}{%
\section{Ejercicio 2:}\label{ejercicio-2-1}}

¿Cuál es el resultado de la siguiente expresión?

\begin{Shaded}
\begin{Highlighting}[]
\NormalTok{x }\OperatorTok{=} \DecValTok{10}
\NormalTok{y }\OperatorTok{=} \DecValTok{5}
\NormalTok{resultado }\OperatorTok{=}\NormalTok{ x }\OperatorTok{+}\NormalTok{ y}
\BuiltInTok{print}\NormalTok{(resultado)}
\end{Highlighting}
\end{Shaded}

Respuesta:

El resultado de la expresión es 15, ya que se suman los valores de las
variables \texttt{x} (10) y \texttt{y} (5).
\textgreater\textgreater\textgreater\textgreater\textgreater\textgreater\textgreater{}
e8ed08b1a5bbe1e369719187cfc4de7f7e2a41a9

\hypertarget{ejercicio-3}{%
\chapter{Ejercicio 3:}\label{ejercicio-3}}

\textless\textless\textless\textless\textless\textless\textless{} HEAD

¿Qué hace el siguiente fragmento de código?

\begin{Shaded}
\begin{Highlighting}[]
\NormalTok{frutas }\OperatorTok{=}\NormalTok{ [}\StringTok{"manzana"}\NormalTok{, }\StringTok{"banana"}\NormalTok{, }\StringTok{"naranja"}\NormalTok{]}
\ControlFlowTok{for}\NormalTok{ fruta }\KeywordTok{in}\NormalTok{ frutas:}
    \BuiltInTok{print}\NormalTok{(fruta)}
\end{Highlighting}
\end{Shaded}

Respuesta:

El código recorre la lista \texttt{frutas} e imprime cada elemento en
una línea separada:

\begin{Shaded}
\begin{Highlighting}[]
\NormalTok{manzana}
\NormalTok{banana}
\NormalTok{naranja}
\end{Highlighting}
\end{Shaded}

======= \#\# Ejercicio 3:

¿Qué hace el siguiente fragmento de código?

\begin{Shaded}
\begin{Highlighting}[]
\NormalTok{frutas }\OperatorTok{=}\NormalTok{ [}\StringTok{"manzana"}\NormalTok{, }\StringTok{"banana"}\NormalTok{, }\StringTok{"naranja"}\NormalTok{]}
\ControlFlowTok{for}\NormalTok{ fruta }\KeywordTok{in}\NormalTok{ frutas:}
    \BuiltInTok{print}\NormalTok{(fruta)}
\end{Highlighting}
\end{Shaded}

Respuesta:

El código recorre la lista \texttt{frutas} e imprime cada elemento en
una línea separada:

\begin{Shaded}
\begin{Highlighting}[]
\NormalTok{manzana}
\NormalTok{banana}
\NormalTok{naranja}
\end{Highlighting}
\end{Shaded}

\begin{quote}
\begin{quote}
\begin{quote}
\begin{quote}
\begin{quote}
\begin{quote}
\begin{quote}
e8ed08b1a5bbe1e369719187cfc4de7f7e2a41a9
\end{quote}
\end{quote}
\end{quote}
\end{quote}
\end{quote}
\end{quote}
\end{quote}

\hypertarget{ejercicio-4}{%
\chapter{Ejercicio 4:}\label{ejercicio-4}}

\textless\textless\textless\textless\textless\textless\textless{} HEAD
¿Cuál es el valor de la variable resultado después de ejecutar el
siguiente código?

\begin{Shaded}
\begin{Highlighting}[]
\NormalTok{numero }\OperatorTok{=} \DecValTok{7}
\NormalTok{resultado }\OperatorTok{=}\NormalTok{ numero }\OperatorTok{*} \DecValTok{2}
\NormalTok{resultado }\OperatorTok{=}\NormalTok{ resultado }\OperatorTok{+} \DecValTok{3}
\end{Highlighting}
\end{Shaded}

Respuesta:

\hypertarget{el-valor-de-la-variable-resultado-es-17-ya-que-se-multiplica-numero-por-2-14-y-luego-se-le-suma-3.}{%
\chapter{\texorpdfstring{El valor de la variable \texttt{resultado} es
17, ya que se multiplica \texttt{numero} por 2 (14) y luego se le suma
3.}{El valor de la variable resultado es 17, ya que se multiplica numero por 2 (14) y luego se le suma 3.}}\label{el-valor-de-la-variable-resultado-es-17-ya-que-se-multiplica-numero-por-2-14-y-luego-se-le-suma-3.}}

\hypertarget{ejercicio-4-1}{%
\section{Ejercicio 4:}\label{ejercicio-4-1}}

¿Cuál es el valor de la variable resultado después de ejecutar el
siguiente código?

\begin{Shaded}
\begin{Highlighting}[]
\NormalTok{numero }\OperatorTok{=} \DecValTok{7}
\NormalTok{resultado }\OperatorTok{=}\NormalTok{ numero }\OperatorTok{*} \DecValTok{2}
\NormalTok{resultado }\OperatorTok{=}\NormalTok{ resultado }\OperatorTok{+} \DecValTok{3}
\end{Highlighting}
\end{Shaded}

Respuesta:

El valor de la variable \texttt{resultado} es 17, ya que se multiplica
\texttt{numero} por 2 (14) y luego se le suma 3.
\textgreater\textgreater\textgreater\textgreater\textgreater\textgreater\textgreater{}
e8ed08b1a5bbe1e369719187cfc4de7f7e2a41a9

\hypertarget{ejercicio-5}{%
\chapter{Ejercicio 5:}\label{ejercicio-5}}

\textless\textless\textless\textless\textless\textless\textless{} HEAD

¿Qué tipo de dato es el resultado de la siguiente expresión?

\begin{Shaded}
\begin{Highlighting}[]
\NormalTok{resultado }\OperatorTok{=} \DecValTok{10} \OperatorTok{/} \DecValTok{2}
\end{Highlighting}
\end{Shaded}

Respuesta:

\hypertarget{el-resultado-es-de-tipo-float-nuxfamero-de-punto-flotante-ya-que-la-divisiuxf3n-produce-un-valor-decimal.}{%
\chapter{\texorpdfstring{El resultado es de tipo \texttt{float} (número
de punto flotante), ya que la división produce un valor
decimal.}{El resultado es de tipo float (número de punto flotante), ya que la división produce un valor decimal.}}\label{el-resultado-es-de-tipo-float-nuxfamero-de-punto-flotante-ya-que-la-divisiuxf3n-produce-un-valor-decimal.}}

\hypertarget{ejercicio-5-1}{%
\section{Ejercicio 5:}\label{ejercicio-5-1}}

¿Qué tipo de dato es el resultado de la siguiente expresión?

\begin{Shaded}
\begin{Highlighting}[]
\NormalTok{resultado }\OperatorTok{=} \DecValTok{10} \OperatorTok{/} \DecValTok{2}
\end{Highlighting}
\end{Shaded}

Respuesta:

El resultado es de tipo \texttt{float} (número de punto flotante), ya
que la división produce un valor decimal.
\textgreater\textgreater\textgreater\textgreater\textgreater\textgreater\textgreater{}
e8ed08b1a5bbe1e369719187cfc4de7f7e2a41a9

\hypertarget{ejercicio-6}{%
\chapter{Ejercicio 6:}\label{ejercicio-6}}

\textless\textless\textless\textless\textless\textless\textless{} HEAD

¿Cómo se define una función en Python?

Respuesta:

Una función en Python se define utilizando la palabra clave
\texttt{def}, seguida del nombre de la función y los parámetros entre
paréntesis. Por ejemplo:

\begin{Shaded}
\begin{Highlighting}[]
\KeywordTok{def}\NormalTok{ saludar(nombre):}
    \BuiltInTok{print}\NormalTok{(}\StringTok{"Hola,"}\NormalTok{, nombre)}
\end{Highlighting}
\end{Shaded}

======= \#\# Ejercicio 6:

¿Cómo se define una función en Python?

Respuesta:

Una función en Python se define utilizando la palabra clave
\texttt{def}, seguida del nombre de la función y los parámetros entre
paréntesis. Por ejemplo:

\begin{Shaded}
\begin{Highlighting}[]
\KeywordTok{def}\NormalTok{ saludar(nombre):}
    \BuiltInTok{print}\NormalTok{(}\StringTok{"Hola,"}\NormalTok{, nombre)}
\end{Highlighting}
\end{Shaded}

\begin{quote}
\begin{quote}
\begin{quote}
\begin{quote}
\begin{quote}
\begin{quote}
\begin{quote}
e8ed08b1a5bbe1e369719187cfc4de7f7e2a41a9
\end{quote}
\end{quote}
\end{quote}
\end{quote}
\end{quote}
\end{quote}
\end{quote}

\hypertarget{ejercicio-7}{%
\chapter{Ejercicio 7:}\label{ejercicio-7}}

\textless\textless\textless\textless\textless\textless\textless{} HEAD
¿Cuál es la salida de este código?

\begin{Shaded}
\begin{Highlighting}[]
\NormalTok{numero }\OperatorTok{=} \DecValTok{5}
\ControlFlowTok{if}\NormalTok{ numero }\OperatorTok{\textgreater{}} \DecValTok{0}\NormalTok{:}
    \BuiltInTok{print}\NormalTok{(}\StringTok{"El número es positivo"}\NormalTok{)}
\ControlFlowTok{else}\NormalTok{:}
    \BuiltInTok{print}\NormalTok{(}\StringTok{"El número no es positivo"}\NormalTok{)}
\end{Highlighting}
\end{Shaded}

Respuesta:

La salida es:

\begin{Shaded}
\begin{Highlighting}[]
\ExtensionTok{El}\NormalTok{ número es positivo}
\end{Highlighting}
\end{Shaded}

ya que el valor de \texttt{numero} (5) es mayor que 0.

======= \#\# Ejercicio 7: ¿Cuál es la salida de este código?

\begin{Shaded}
\begin{Highlighting}[]
\NormalTok{numero }\OperatorTok{=} \DecValTok{5}
\ControlFlowTok{if}\NormalTok{ numero }\OperatorTok{\textgreater{}} \DecValTok{0}\NormalTok{:}
    \BuiltInTok{print}\NormalTok{(}\StringTok{"El número es positivo"}\NormalTok{)}
\ControlFlowTok{else}\NormalTok{:}
    \BuiltInTok{print}\NormalTok{(}\StringTok{"El número no es positivo"}\NormalTok{)}
\end{Highlighting}
\end{Shaded}

Respuesta:

La salida es:

\begin{Shaded}
\begin{Highlighting}[]
\ExtensionTok{El}\NormalTok{ número es positivo}
\end{Highlighting}
\end{Shaded}

ya que el valor de \texttt{numero} (5) es mayor que 0.

\begin{quote}
\begin{quote}
\begin{quote}
\begin{quote}
\begin{quote}
\begin{quote}
\begin{quote}
e8ed08b1a5bbe1e369719187cfc4de7f7e2a41a9
\end{quote}
\end{quote}
\end{quote}
\end{quote}
\end{quote}
\end{quote}
\end{quote}

\hypertarget{ejercicio-8}{%
\chapter{Ejercicio 8:}\label{ejercicio-8}}

\textless\textless\textless\textless\textless\textless\textless{} HEAD
¿Qué hace el siguiente código?

\begin{Shaded}
\begin{Highlighting}[]
\ControlFlowTok{for}\NormalTok{ i }\KeywordTok{in} \BuiltInTok{range}\NormalTok{(}\DecValTok{3}\NormalTok{):}
    \BuiltInTok{print}\NormalTok{(i)}
\end{Highlighting}
\end{Shaded}

Respuesta:

El código imprime los números del 0 al 2 en líneas separadas:

\begin{Shaded}
\begin{Highlighting}[]
\ExtensionTok{0}
\ExtensionTok{1}
\ExtensionTok{2}
\end{Highlighting}
\end{Shaded}

======= \#\# Ejercicio 8: ¿Qué hace el siguiente código?

\begin{Shaded}
\begin{Highlighting}[]
\ControlFlowTok{for}\NormalTok{ i }\KeywordTok{in} \BuiltInTok{range}\NormalTok{(}\DecValTok{3}\NormalTok{):}
    \BuiltInTok{print}\NormalTok{(i)}
\end{Highlighting}
\end{Shaded}

Respuesta:

El código imprime los números del 0 al 2 en líneas separadas:

\begin{Shaded}
\begin{Highlighting}[]
\ExtensionTok{0}
\ExtensionTok{1}
\ExtensionTok{2}
\end{Highlighting}
\end{Shaded}

\begin{quote}
\begin{quote}
\begin{quote}
\begin{quote}
\begin{quote}
\begin{quote}
\begin{quote}
e8ed08b1a5bbe1e369719187cfc4de7f7e2a41a9
\end{quote}
\end{quote}
\end{quote}
\end{quote}
\end{quote}
\end{quote}
\end{quote}

\hypertarget{ejercicio-9}{%
\chapter{Ejercicio 9:}\label{ejercicio-9}}

\textless\textless\textless\textless\textless\textless\textless{} HEAD

¿Cuál es el valor de la variable longitud después de ejecutar este
código?

\begin{Shaded}
\begin{Highlighting}[]
\NormalTok{frase }\OperatorTok{=} \StringTok{"Hola, mundo"}
\NormalTok{longitud }\OperatorTok{=} \BuiltInTok{len}\NormalTok{(frase)}
\end{Highlighting}
\end{Shaded}

Respuesta:

\hypertarget{el-valor-de-la-variable-longitud-seruxe1-11-ya-que-la-funciuxf3n-len-retorna-la-cantidad-de-caracteres-en-la-cadena.}{%
\chapter{\texorpdfstring{El valor de la variable \texttt{longitud} será
11, ya que la función \texttt{len()} retorna la cantidad de caracteres
en la
cadena.}{El valor de la variable longitud será 11, ya que la función len() retorna la cantidad de caracteres en la cadena.}}\label{el-valor-de-la-variable-longitud-seruxe1-11-ya-que-la-funciuxf3n-len-retorna-la-cantidad-de-caracteres-en-la-cadena.}}

\hypertarget{ejercicio-9-1}{%
\section{Ejercicio 9:}\label{ejercicio-9-1}}

¿Cuál es el valor de la variable longitud después de ejecutar este
código?

\begin{Shaded}
\begin{Highlighting}[]
\NormalTok{frase }\OperatorTok{=} \StringTok{"Hola, mundo"}
\NormalTok{longitud }\OperatorTok{=} \BuiltInTok{len}\NormalTok{(frase)}
\end{Highlighting}
\end{Shaded}

Respuesta:

El valor de la variable \texttt{longitud} será 11, ya que la función
\texttt{len()} retorna la cantidad de caracteres en la cadena.
\textgreater\textgreater\textgreater\textgreater\textgreater\textgreater\textgreater{}
e8ed08b1a5bbe1e369719187cfc4de7f7e2a41a9

\hypertarget{ejercicio-10}{%
\chapter{Ejercicio 10:}\label{ejercicio-10}}

\textless\textless\textless\textless\textless\textless\textless{} HEAD

¿Cuál es la sintaxis correcta para importar la biblioteca math en
Python?

Respuesta:

La sintaxis correcta es:

\begin{Shaded}
\begin{Highlighting}[]
\ImportTok{import}\NormalTok{ math}
\end{Highlighting}
\end{Shaded}

======= \#\# Ejercicio 10:

¿Cuál es la sintaxis correcta para importar la biblioteca math en
Python?

Respuesta:

La sintaxis correcta es:

\begin{Shaded}
\begin{Highlighting}[]
\ImportTok{import}\NormalTok{ math}
\end{Highlighting}
\end{Shaded}

\begin{quote}
\begin{quote}
\begin{quote}
\begin{quote}
\begin{quote}
\begin{quote}
\begin{quote}
e8ed08b1a5bbe1e369719187cfc4de7f7e2a41a9
\end{quote}
\end{quote}
\end{quote}
\end{quote}
\end{quote}
\end{quote}
\end{quote}

\hypertarget{ejercicio-11}{%
\chapter{Ejercicio 11:}\label{ejercicio-11}}

\textless\textless\textless\textless\textless\textless\textless{} HEAD

¿Qué método se utiliza para agregar un elemento al final de una lista?

Respuesta:

El método utilizado para agregar un elemento al final de una lista es
\texttt{append()}. Por ejemplo:

\begin{Shaded}
\begin{Highlighting}[]
\NormalTok{mi\_lista }\OperatorTok{=}\NormalTok{ [}\DecValTok{1}\NormalTok{, }\DecValTok{2}\NormalTok{, }\DecValTok{3}\NormalTok{]}
\NormalTok{mi\_lista.append(}\DecValTok{4}\NormalTok{)}
\end{Highlighting}
\end{Shaded}

======= \#\# Ejercicio 11:

¿Qué método se utiliza para agregar un elemento al final de una lista?

Respuesta:

El método utilizado para agregar un elemento al final de una lista es
\texttt{append()}. Por ejemplo:

\begin{Shaded}
\begin{Highlighting}[]
\NormalTok{mi\_lista }\OperatorTok{=}\NormalTok{ [}\DecValTok{1}\NormalTok{, }\DecValTok{2}\NormalTok{, }\DecValTok{3}\NormalTok{]}
\NormalTok{mi\_lista.append(}\DecValTok{4}\NormalTok{)}
\end{Highlighting}
\end{Shaded}

\begin{quote}
\begin{quote}
\begin{quote}
\begin{quote}
\begin{quote}
\begin{quote}
\begin{quote}
e8ed08b1a5bbe1e369719187cfc4de7f7e2a41a9
\end{quote}
\end{quote}
\end{quote}
\end{quote}
\end{quote}
\end{quote}
\end{quote}

\hypertarget{ejercicio-12}{%
\chapter{Ejercicio 12:}\label{ejercicio-12}}

\textless\textless\textless\textless\textless\textless\textless{} HEAD

¿Cuál es el resultado de la siguiente expresión?

\begin{Shaded}
\begin{Highlighting}[]
\NormalTok{resultado }\OperatorTok{=} \DecValTok{2} \OperatorTok{**} \DecValTok{3}
\end{Highlighting}
\end{Shaded}

Respuesta:

El resultado de la expresión es 8, ya que \texttt{2\ **\ 3} representa
la potencia de 2 elevado a la 3, que es 8.

======= \#\# Ejercicio 12:

¿Cuál es el resultado de la siguiente expresión?

\begin{Shaded}
\begin{Highlighting}[]
\NormalTok{resultado }\OperatorTok{=} \DecValTok{2} \OperatorTok{**} \DecValTok{3}
\end{Highlighting}
\end{Shaded}

Respuesta:

El resultado de la expresión es 8, ya que \texttt{2\ **\ 3} representa
la potencia de 2 elevado a la 3, que es 8.

\begin{quote}
\begin{quote}
\begin{quote}
\begin{quote}
\begin{quote}
\begin{quote}
\begin{quote}
e8ed08b1a5bbe1e369719187cfc4de7f7e2a41a9
\end{quote}
\end{quote}
\end{quote}
\end{quote}
\end{quote}
\end{quote}
\end{quote}

\hypertarget{ejercicio-13}{%
\chapter{Ejercicio 13:}\label{ejercicio-13}}

\textless\textless\textless\textless\textless\textless\textless{} HEAD

¿Qué función se utiliza para convertir un valor a tipo int en Python?

Respuesta:

La función utilizada para convertir un valor a tipo \texttt{int} es
\texttt{int()}. Por ejemplo:

\begin{Shaded}
\begin{Highlighting}[]
\NormalTok{numero }\OperatorTok{=} \BuiltInTok{int}\NormalTok{(}\StringTok{"10"}\NormalTok{)}
\end{Highlighting}
\end{Shaded}

======= \#\# Ejercicio 13:

¿Qué función se utiliza para convertir un valor a tipo int en Python?

Respuesta:

La función utilizada para convertir un valor a tipo \texttt{int} es
\texttt{int()}. Por ejemplo:

\begin{Shaded}
\begin{Highlighting}[]
\NormalTok{numero }\OperatorTok{=} \BuiltInTok{int}\NormalTok{(}\StringTok{"10"}\NormalTok{)}
\end{Highlighting}
\end{Shaded}

\begin{quote}
\begin{quote}
\begin{quote}
\begin{quote}
\begin{quote}
\begin{quote}
\begin{quote}
e8ed08b1a5bbe1e369719187cfc4de7f7e2a41a9
\end{quote}
\end{quote}
\end{quote}
\end{quote}
\end{quote}
\end{quote}
\end{quote}

\hypertarget{ejercicio-14}{%
\chapter{Ejercicio 14:}\label{ejercicio-14}}

\textless\textless\textless\textless\textless\textless\textless{} HEAD

¿Qué método se utiliza para unir elementos de una lista en una cadena?

Respuesta:

El método utilizado para unir elementos de una lista en una cadena es
\texttt{join()}. Por ejemplo:

\begin{Shaded}
\begin{Highlighting}[]
\NormalTok{elementos }\OperatorTok{=}\NormalTok{ [}\StringTok{"a"}\NormalTok{, }\StringTok{"b"}\NormalTok{, }\StringTok{"c"}\NormalTok{]}
\NormalTok{cadena }\OperatorTok{=} \StringTok{"{-}"}\NormalTok{.join(elementos)}
\end{Highlighting}
\end{Shaded}

======= \#\# Ejercicio 14:

¿Qué método se utiliza para unir elementos de una lista en una cadena?

Respuesta:

El método utilizado para unir elementos de una lista en una cadena es
\texttt{join()}. Por ejemplo:

\begin{Shaded}
\begin{Highlighting}[]
\NormalTok{elementos }\OperatorTok{=}\NormalTok{ [}\StringTok{"a"}\NormalTok{, }\StringTok{"b"}\NormalTok{, }\StringTok{"c"}\NormalTok{]}
\NormalTok{cadena }\OperatorTok{=} \StringTok{"{-}"}\NormalTok{.join(elementos)}
\end{Highlighting}
\end{Shaded}

\begin{quote}
\begin{quote}
\begin{quote}
\begin{quote}
\begin{quote}
\begin{quote}
\begin{quote}
e8ed08b1a5bbe1e369719187cfc4de7f7e2a41a9
\end{quote}
\end{quote}
\end{quote}
\end{quote}
\end{quote}
\end{quote}
\end{quote}

\hypertarget{ejercicio-15}{%
\chapter{Ejercicio 15:}\label{ejercicio-15}}

\textless\textless\textless\textless\textless\textless\textless{} HEAD

¿Cuál es la salida de este código?

\begin{Shaded}
\begin{Highlighting}[]
\ControlFlowTok{for}\NormalTok{ i }\KeywordTok{in} \BuiltInTok{range}\NormalTok{(}\DecValTok{1}\NormalTok{, }\DecValTok{6}\NormalTok{):}
    \ControlFlowTok{if}\NormalTok{ i }\OperatorTok{==} \DecValTok{3}\NormalTok{:}
        \ControlFlowTok{continue}
    \BuiltInTok{print}\NormalTok{(i)}
\end{Highlighting}
\end{Shaded}

Respuesta:

La salida es:

\begin{Shaded}
\begin{Highlighting}[]
\ExtensionTok{1}
\ExtensionTok{2}
\ExtensionTok{4}
\ExtensionTok{5}
\end{Highlighting}
\end{Shaded}

\hypertarget{ya-que-el-valor-3-es-omitido-debido-al-uso-de-continue.}{%
\chapter{\texorpdfstring{ya que el valor \texttt{3} es omitido debido al
uso de
\texttt{continue}.}{ya que el valor 3 es omitido debido al uso de continue.}}\label{ya-que-el-valor-3-es-omitido-debido-al-uso-de-continue.}}

\hypertarget{ejercicio-15-1}{%
\section{Ejercicio 15:}\label{ejercicio-15-1}}

¿Cuál es la salida de este código?

\begin{Shaded}
\begin{Highlighting}[]
\ControlFlowTok{for}\NormalTok{ i }\KeywordTok{in} \BuiltInTok{range}\NormalTok{(}\DecValTok{1}\NormalTok{, }\DecValTok{6}\NormalTok{):}
    \ControlFlowTok{if}\NormalTok{ i }\OperatorTok{==} \DecValTok{3}\NormalTok{:}
        \ControlFlowTok{continue}
    \BuiltInTok{print}\NormalTok{(i)}
\end{Highlighting}
\end{Shaded}

Respuesta:

La salida es:

\begin{Shaded}
\begin{Highlighting}[]
\ExtensionTok{1}
\ExtensionTok{2}
\ExtensionTok{4}
\ExtensionTok{5}
\end{Highlighting}
\end{Shaded}

ya que el valor \texttt{3} es omitido debido al uso de
\texttt{continue}.
\textgreater\textgreater\textgreater\textgreater\textgreater\textgreater\textgreater{}
e8ed08b1a5bbe1e369719187cfc4de7f7e2a41a9

\hypertarget{ejercicio-16}{%
\chapter{Ejercicio 16:}\label{ejercicio-16}}

\textless\textless\textless\textless\textless\textless\textless{} HEAD

¿Qué método se utiliza para eliminar un elemento específico de una
lista?

Respuesta:

El método utilizado para eliminar un elemento específico de una lista es
\texttt{remove()}. Por ejemplo:

\begin{Shaded}
\begin{Highlighting}[]
\NormalTok{mi\_lista }\OperatorTok{=}\NormalTok{ [}\DecValTok{1}\NormalTok{, }\DecValTok{2}\NormalTok{, }\DecValTok{3}\NormalTok{]}
\NormalTok{mi\_lista.remove(}\DecValTok{2}\NormalTok{)}
\end{Highlighting}
\end{Shaded}

======= \#\# Ejercicio 16:

¿Qué método se utiliza para eliminar un elemento específico de una
lista?

Respuesta:

El método utilizado para eliminar un elemento específico de una lista es
\texttt{remove()}. Por ejemplo:

\begin{Shaded}
\begin{Highlighting}[]
\NormalTok{mi\_lista }\OperatorTok{=}\NormalTok{ [}\DecValTok{1}\NormalTok{, }\DecValTok{2}\NormalTok{, }\DecValTok{3}\NormalTok{]}
\NormalTok{mi\_lista.remove(}\DecValTok{2}\NormalTok{)}
\end{Highlighting}
\end{Shaded}

\begin{quote}
\begin{quote}
\begin{quote}
\begin{quote}
\begin{quote}
\begin{quote}
\begin{quote}
e8ed08b1a5bbe1e369719187cfc4de7f7e2a41a9
\end{quote}
\end{quote}
\end{quote}
\end{quote}
\end{quote}
\end{quote}
\end{quote}

\hypertarget{ejercicio-17}{%
\chapter{Ejercicio 17:}\label{ejercicio-17}}

\textless\textless\textless\textless\textless\textless\textless{} HEAD

¿Cómo se define una clase en Python?

Respuesta:

Una clase en Python se define utilizando la palabra clave
\texttt{class}, seguida del nombre de la clase y los métodos y atributos
definidos dentro de la clase. Por ejemplo:

\begin{Shaded}
\begin{Highlighting}[]
\KeywordTok{class}\NormalTok{ Persona:}
    \KeywordTok{def} \FunctionTok{\_\_init\_\_}\NormalTok{(}\VariableTok{self}\NormalTok{, nombre, edad):}
        \VariableTok{self}\NormalTok{.nombre }\OperatorTok{=}\NormalTok{ nombre}
        \VariableTok{self}\NormalTok{.edad }\OperatorTok{=}\NormalTok{ edad}
\end{Highlighting}
\end{Shaded}

=======

\hypertarget{ejercicio-17-1}{%
\section{Ejercicio 17:}\label{ejercicio-17-1}}

¿Cómo se define una clase en Python?

Respuesta:

Una clase en Python se define utilizando la palabra clave
\texttt{class}, seguida del nombre de la clase y los métodos y atributos
definidos dentro de la clase. Por ejemplo:

\begin{Shaded}
\begin{Highlighting}[]
\KeywordTok{class}\NormalTok{ Persona:}
    \KeywordTok{def} \FunctionTok{\_\_init\_\_}\NormalTok{(}\VariableTok{self}\NormalTok{, nombre, edad):}
        \VariableTok{self}\NormalTok{.nombre }\OperatorTok{=}\NormalTok{ nombre}
        \VariableTok{self}\NormalTok{.edad }\OperatorTok{=}\NormalTok{ edad}
\end{Highlighting}
\end{Shaded}

\begin{quote}
\begin{quote}
\begin{quote}
\begin{quote}
\begin{quote}
\begin{quote}
\begin{quote}
e8ed08b1a5bbe1e369719187cfc4de7f7e2a41a9
\end{quote}
\end{quote}
\end{quote}
\end{quote}
\end{quote}
\end{quote}
\end{quote}

\hypertarget{ejercicio-18}{%
\chapter{Ejercicio 18:}\label{ejercicio-18}}

\textless\textless\textless\textless\textless\textless\textless{} HEAD

¿Cuál es el resultado de la siguiente expresión?

\begin{Shaded}
\begin{Highlighting}[]
\NormalTok{x }\OperatorTok{=} \StringTok{"Hola"}
\NormalTok{y }\OperatorTok{=} \StringTok{"Mundo"}
\NormalTok{resultado }\OperatorTok{=}\NormalTok{ x }\OperatorTok{+} \StringTok{" "} \OperatorTok{+}\NormalTok{ y}
\end{Highlighting}
\end{Shaded}

Respuesta:

El resultado de la expresión es la cadena ``Hola Mundo'', ya que se
concatenan las cadenas \texttt{x} y \texttt{y} junto con un espacio.

======= \#\# Ejercicio 18:

¿Cuál es el resultado de la siguiente expresión?

\begin{Shaded}
\begin{Highlighting}[]
\NormalTok{x }\OperatorTok{=} \StringTok{"Hola"}
\NormalTok{y }\OperatorTok{=} \StringTok{"Mundo"}
\NormalTok{resultado }\OperatorTok{=}\NormalTok{ x }\OperatorTok{+} \StringTok{" "} \OperatorTok{+}\NormalTok{ y}
\end{Highlighting}
\end{Shaded}

Respuesta:

El resultado de la expresión es la cadena ``Hola Mundo'', ya que se
concatenan las cadenas \texttt{x} y \texttt{y} junto con un espacio.

\begin{quote}
\begin{quote}
\begin{quote}
\begin{quote}
\begin{quote}
\begin{quote}
\begin{quote}
e8ed08b1a5bbe1e369719187cfc4de7f7e2a41a9
\end{quote}
\end{quote}
\end{quote}
\end{quote}
\end{quote}
\end{quote}
\end{quote}

\hypertarget{ejercicio-19}{%
\chapter{Ejercicio 19:}\label{ejercicio-19}}

\textless\textless\textless\textless\textless\textless\textless{} HEAD

¿Cómo se crea una nueva base de datos en PostgreSQL utilizando SQL?

Respuesta:

Para crear una nueva base de datos en PostgreSQL utilizando SQL, se
utiliza la siguiente consulta:

\begin{Shaded}
\begin{Highlighting}[]
\KeywordTok{CREATE} \KeywordTok{DATABASE}\NormalTok{ nombre\_basededatos;}
\end{Highlighting}
\end{Shaded}

======= \#\# Ejercicio 19:

¿Cómo se crea una nueva base de datos en PostgreSQL utilizando SQL?

Respuesta:

Para crear una nueva base de datos en PostgreSQL utilizando SQL, se
utiliza la siguiente consulta:

\begin{Shaded}
\begin{Highlighting}[]
\KeywordTok{CREATE} \KeywordTok{DATABASE}\NormalTok{ nombre\_basededatos;}
\end{Highlighting}
\end{Shaded}

\begin{quote}
\begin{quote}
\begin{quote}
\begin{quote}
\begin{quote}
\begin{quote}
\begin{quote}
e8ed08b1a5bbe1e369719187cfc4de7f7e2a41a9
\end{quote}
\end{quote}
\end{quote}
\end{quote}
\end{quote}
\end{quote}
\end{quote}

\hypertarget{ejercicio-20}{%
\chapter{Ejercicio 20:}\label{ejercicio-20}}

\textless\textless\textless\textless\textless\textless\textless{} HEAD

¿Cuál es la forma correcta de realizar una consulta a una colección en
MongoDB?

Respuesta:

La forma correcta de realizar una consulta a una colección en MongoDB es
utilizando el método \texttt{find()}. Por ejemplo:

\begin{Shaded}
\begin{Highlighting}[]
\NormalTok{resultados }\OperatorTok{=}\NormalTok{ db.coleccion.find(\{}\StringTok{"campo"}\NormalTok{: valor\})}
\end{Highlighting}
\end{Shaded}

======= \#\# Ejercicio 20:

¿Cuál es la forma correcta de realizar una consulta a una colección en
MongoDB?

Respuesta:

La forma correcta de realizar una consulta a una colección en MongoDB es
utilizando el método \texttt{find()}. Por ejemplo:

\begin{Shaded}
\begin{Highlighting}[]
\NormalTok{resultados }\OperatorTok{=}\NormalTok{ db.coleccion.find(\{}\StringTok{"campo"}\NormalTok{: valor\})}
\end{Highlighting}
\end{Shaded}

\begin{quote}
\begin{quote}
\begin{quote}
\begin{quote}
\begin{quote}
\begin{quote}
\begin{quote}
e8ed08b1a5bbe1e369719187cfc4de7f7e2a41a9
\end{quote}
\end{quote}
\end{quote}
\end{quote}
\end{quote}
\end{quote}
\end{quote}

\hypertarget{unidad-i-introducciuxf3n-a-la-programaciuxf3n}{%
\chapter{UNIDAD I: Introducción a la
programación}\label{unidad-i-introducciuxf3n-a-la-programaciuxf3n}}

\textless\textless\textless\textless\textless\textless\textless{} HEAD

Ejercicio 1: ¿Cuál es el objetivo principal de la programación?

Respuesta:

El objetivo principal de la programación es resolver problemas y
automatizar tareas utilizando un lenguaje de programación.

Ejercicio 2: ¿Qué es un algoritmo?

Respuesta:

Un algoritmo es un conjunto de instrucciones ordenadas y precisas que
describen cómo realizar una tarea o resolver un problema.

Ejercicio 3: ¿Cuál es la importancia de la indentación en Python?

Respuesta:

La indentación en Python es importante porque define el bloque de código
perteneciente a una estructura, como un bucle o una función. Python
utiliza la indentación en lugar de llaves u otros caracteres para
delimitar bloques de código.

Ejercicio 4: ¿Qué es un comentario en programación?

Respuesta:

Un comentario en programación es un texto explicativo que se agrega en
el código para hacerlo más comprensible. Los comentarios son ignorados
por el intérprete y son útiles para documentar el código.

Ejercicio 5: Escribe un programa en Python que imprima ``¡Hola,
mundo!''.

Respuesta:

\begin{Shaded}
\begin{Highlighting}[]
\BuiltInTok{print}\NormalTok{(}\StringTok{"¡Hola, mundo!"}\NormalTok{)}
\end{Highlighting}
\end{Shaded}

\hypertarget{unidad-ii-instalaciuxf3n-de-python-y-muxe1s-herramientas}{%
\section{UNIDAD II: Instalación de Python y más
herramientas}\label{unidad-ii-instalaciuxf3n-de-python-y-muxe1s-herramientas}}

Ejercicio 6: ¿Cuál es la forma de verificar la versión de Python
instalada en tu sistema?

Respuesta:

Ejecutando el comando \texttt{python\ -\/-version} en la línea de
comandos.

Ejercicio 7: ¿Cuál es el propósito de Git en el desarrollo de software?

Respuesta:

Git es un sistema de control de versiones que permite rastrear cambios
en el código, colaborar con otros desarrolladores y mantener un
historial completo de modificaciones en un proyecto.

Ejercicio 8: ¿Cómo se instala una extensión (extensión) en Visual Studio
Code?

Respuesta:

En Visual Studio Code, puedes instalar extensiones desde la barra
lateral izquierda, haciendo clic en el ícono de extensiones (cuatro
cuadros) y buscando la extensión que deseas instalar.

Ejercicio 9: ¿Cuál es el resultado del siguiente código?

\begin{Shaded}
\begin{Highlighting}[]
\BuiltInTok{print}\NormalTok{(}\StringTok{"Hola, "} \OperatorTok{+} \StringTok{"mundo"}\NormalTok{)}
\end{Highlighting}
\end{Shaded}

Respuesta:

El resultado es la cadena ``Hola, mundo'' al concatenar las dos cadenas.

Ejercicio 10: ¿Cuál es el propósito de un entorno virtual en Python?

Respuesta:

Un entorno virtual en Python permite aislar y gestionar las dependencias
y paquetes utilizados en un proyecto específico, evitando conflictos con
otros proyectos y asegurando un entorno limpio y controlado.

\hypertarget{unidad-iii-introducciuxf3n-a-python}{%
\section{UNIDAD III: Introducción a
Python}\label{unidad-iii-introducciuxf3n-a-python}}

Ejercicio 11: ¿Cuál es la diferencia entre una variable y una constante
en programación?

Respuesta:

Una variable puede cambiar su valor a lo largo del programa, mientras
que una constante mantiene su valor constante durante la ejecución.

Ejercicio 12: Escribe un programa que solicite al usuario su nombre y
luego imprima un mensaje de bienvenida con el nombre ingresado.

Respuesta:

\begin{Shaded}
\begin{Highlighting}[]
\NormalTok{nombre }\OperatorTok{=} \BuiltInTok{input}\NormalTok{(}\StringTok{"Ingresa tu nombre: "}\NormalTok{)}
\BuiltInTok{print}\NormalTok{(}\StringTok{"¡Bienvenido,"}\NormalTok{, nombre, }\StringTok{"!"}\NormalTok{)}
\end{Highlighting}
\end{Shaded}

Ejercicio 13: ¿Cuál es el valor de la variable resultado después de
ejecutar el siguiente código?

\begin{Shaded}
\begin{Highlighting}[]
\NormalTok{x }\OperatorTok{=} \DecValTok{5}
\NormalTok{y }\OperatorTok{=} \DecValTok{2}
\NormalTok{resultado }\OperatorTok{=}\NormalTok{ x }\OperatorTok{//}\NormalTok{ y}
\end{Highlighting}
\end{Shaded}

Respuesta:

El valor de la variable \texttt{resultado} será 2, ya que \texttt{//}
realiza la división entera de 5 entre 2.

Ejercicio 14: Escribe un programa en Python que determine si un número
ingresado por el usuario es par o impar.

Respuesta:

\begin{Shaded}
\begin{Highlighting}[]
\NormalTok{numero }\OperatorTok{=} \BuiltInTok{int}\NormalTok{(}\BuiltInTok{input}\NormalTok{(}\StringTok{"Ingresa un número: "}\NormalTok{))}
\ControlFlowTok{if}\NormalTok{ numero }\OperatorTok{\%} \DecValTok{2} \OperatorTok{==} \DecValTok{0}\NormalTok{:}
    \BuiltInTok{print}\NormalTok{(}\StringTok{"El número es par."}\NormalTok{)}
\ControlFlowTok{else}\NormalTok{:}
    \BuiltInTok{print}\NormalTok{(}\StringTok{"El número es impar."}\NormalTok{)}
\end{Highlighting}
\end{Shaded}

Ejercicio 15: ¿Cuál es la función del operador not en Python?

Respuesta:

El operador \texttt{not} se utiliza para negar una expresión booleana.
Si la expresión es verdadera, \texttt{not} la convierte en falsa, y
viceversa.

\hypertarget{unidad-iv-tipos-de-datos}{%
\section{UNIDAD IV: Tipos de Datos}\label{unidad-iv-tipos-de-datos}}

Ejercicio 16: ¿Cuál es la diferencia entre una lista y una tupla en
Python?

Respuesta:

La principal diferencia es que las listas son mutables (pueden cambiar)
y las tuplas son inmutables (no pueden cambiar). En otras palabras,
puedes agregar, eliminar y modificar elementos en una lista, pero no en
una tupla.

Ejercicio 17: Escribe un programa que ordene una lista de números en
orden ascendente.

Respuesta:

\begin{Shaded}
\begin{Highlighting}[]
\NormalTok{numeros }\OperatorTok{=}\NormalTok{ [}\DecValTok{4}\NormalTok{, }\DecValTok{1}\NormalTok{, }\DecValTok{6}\NormalTok{, }\DecValTok{3}\NormalTok{, }\DecValTok{2}\NormalTok{]}
\NormalTok{numeros.sort()}
\BuiltInTok{print}\NormalTok{(numeros)}
\end{Highlighting}
\end{Shaded}

Ejercicio 18: ¿Cómo se accede al tercer elemento de una lista en Python?

Respuesta:

Utilizando el índice \texttt{2}. Por ejemplo, si la lista se llama
\texttt{mi\_lista}, puedes acceder al tercer elemento con
\texttt{mi\_lista{[}2{]}}.

Ejercicio 19: ¿Qué método se utiliza para agregar un elemento al final
de una lista?

Respuesta:

El método utilizado es \texttt{append()}. Por ejemplo,
\texttt{mi\_lista.append(7)} agrega el número 7 al final de la lista.

Ejercicio 20: Escribe un programa que cuente cuántas veces aparece un
elemento específico en una lista.

Respuesta:

\begin{Shaded}
\begin{Highlighting}[]
\NormalTok{mi\_lista }\OperatorTok{=}\NormalTok{ [}\DecValTok{2}\NormalTok{, }\DecValTok{4}\NormalTok{, }\DecValTok{6}\NormalTok{, }\DecValTok{4}\NormalTok{, }\DecValTok{8}\NormalTok{, }\DecValTok{4}\NormalTok{, }\DecValTok{10}\NormalTok{]}
\NormalTok{elemento }\OperatorTok{=} \DecValTok{4}
\NormalTok{contador }\OperatorTok{=}\NormalTok{ mi\_lista.count(elemento)}
\BuiltInTok{print}\NormalTok{(}\StringTok{"El elemento"}\NormalTok{, elemento, }\StringTok{"aparece"}\NormalTok{, contador, }\StringTok{"veces."}\NormalTok{)}
\end{Highlighting}
\end{Shaded}

\hypertarget{unidad-v-control-de-flujo}{%
\section{UNIDAD V: Control de Flujo}\label{unidad-v-control-de-flujo}}

Ejercicio 21: Escribe un programa que determine si un número ingresado
por el usuario es positivo, negativo o cero.

Respuesta:

\begin{Shaded}
\begin{Highlighting}[]
\NormalTok{numero }\OperatorTok{=} \BuiltInTok{int}\NormalTok{(}\BuiltInTok{input}\NormalTok{(}\StringTok{"Ingresa un número: "}\NormalTok{))}
\ControlFlowTok{if}\NormalTok{ numero }\OperatorTok{\textgreater{}} \DecValTok{0}\NormalTok{:}
    \BuiltInTok{print}\NormalTok{(}\StringTok{"El número es positivo."}\NormalTok{)}
\ControlFlowTok{elif}\NormalTok{ numero }\OperatorTok{\textless{}} \DecValTok{0}\NormalTok{:}
    \BuiltInTok{print}\NormalTok{(}\StringTok{"El número es negativo."}\NormalTok{)}
\ControlFlowTok{else}\NormalTok{:}
    \BuiltInTok{print}\NormalTok{(}\StringTok{"El número es cero."}\NormalTok{)}
\end{Highlighting}
\end{Shaded}

Ejercicio 22: ¿Qué hace el siguiente código?

\begin{Shaded}
\begin{Highlighting}[]
\NormalTok{contador }\OperatorTok{=} \DecValTok{0}
\ControlFlowTok{while}\NormalTok{ contador }\OperatorTok{\textless{}} \DecValTok{5}\NormalTok{:}
    \BuiltInTok{print}\NormalTok{(contador)}
\NormalTok{    contador }\OperatorTok{+=} \DecValTok{1}
\end{Highlighting}
\end{Shaded}

Respuesta:

El código imprime los números del 0 al 4 en líneas separadas utilizando
un bucle \texttt{while}.

Ejercicio 23: ¿Cuál es el resultado de la siguiente expresión?

\begin{Shaded}
\begin{Highlighting}[]
\NormalTok{resultado }\OperatorTok{=} \DecValTok{0}
\ControlFlowTok{for}\NormalTok{ i }\KeywordTok{in} \BuiltInTok{range}\NormalTok{(}\DecValTok{1}\NormalTok{, }\DecValTok{6}\NormalTok{):}
\NormalTok{    resultado }\OperatorTok{+=}\NormalTok{ i}
\BuiltInTok{print}\NormalTok{(resultado)}
\end{Highlighting}
\end{Shaded}

Respuesta:

El resultado es 15, ya que se suma los números del 1 al 5 en el bucle
\texttt{for}.

Ejercicio 24: Escribe un programa que calcule la suma de todos los
números pares entre 1 y 100.

Respuesta:

\begin{Shaded}
\begin{Highlighting}[]
\NormalTok{suma }\OperatorTok{=} \DecValTok{0}
\ControlFlowTok{for}\NormalTok{ i }\KeywordTok{in} \BuiltInTok{range}\NormalTok{(}\DecValTok{2}\NormalTok{, }\DecValTok{101}\NormalTok{, }\DecValTok{2}\NormalTok{):}
\NormalTok{    suma }\OperatorTok{+=}\NormalTok{ i}
\BuiltInTok{print}\NormalTok{(}\StringTok{"La suma de los números pares entre 1 y 100 es:"}\NormalTok{, suma)}
\end{Highlighting}
\end{Shaded}

Ejercicio 25: ¿Cuál es el propósito de la instrucción break en un bucle?

Respuesta:

La instrucción \texttt{break} se utiliza para salir inmediatamente de un
bucle, interrumpiendo su ejecución antes de que termine naturalmente.

\hypertarget{unidad-vi-funciones}{%
\section{UNIDAD VI: Funciones}\label{unidad-vi-funciones}}

Ejercicio 26: ¿Qué es una función en programación?

Respuesta:

Una función es un bloque de código reutilizable que realiza una tarea
específica. Puede recibir argumentos, ejecutar instrucciones y devolver
un valor.

Ejercicio 27: Escribe una función en Python que calcule el área de un
círculo.

Respuesta:

\begin{Shaded}
\begin{Highlighting}[]
\ImportTok{import}\NormalTok{ math}

\KeywordTok{def}\NormalTok{ area\_circulo(radio):}
\ControlFlowTok{return}\NormalTok{ math.pi }\OperatorTok{*}\NormalTok{ radio }\OperatorTok{**} \DecValTok{2}
\end{Highlighting}
\end{Shaded}

\textbf{Ejercicio 28:} ¿Qué es la recursividad en programación?

Respuesta:

La recursividad es una técnica donde una función se llama a sí misma
para resolver un problema. Es útil para resolver problemas que se pueden
descomponer en subproblemas similares.

\textbf{Ejercicio 29:} Escribe una función recursiva en Python para
calcular el factorial de un número.

Respuesta:

\begin{Shaded}
\begin{Highlighting}[]
\KeywordTok{def}\NormalTok{ factorial(n):}
    \ControlFlowTok{if}\NormalTok{ n }\OperatorTok{==} \DecValTok{0}\NormalTok{:}
        \ControlFlowTok{return} \DecValTok{1}
    \ControlFlowTok{else}\NormalTok{:}
        \ControlFlowTok{return}\NormalTok{ n }\OperatorTok{*}\NormalTok{ factorial(n }\OperatorTok{{-}} \DecValTok{1}\NormalTok{)}
\end{Highlighting}
\end{Shaded}

Ejercicio 30: ¿Por qué es importante utilizar funciones en la
programación?

Respuesta:

Las funciones permiten dividir el código en bloques más pequeños y
manejables, lo que facilita la reutilización, la depuración y la
comprensión del código. Además, promueven la modularidad y el diseño
limpio.

\hypertarget{unidad-vii-objetos-clases-y-herencia}{%
\section{UNIDAD VII: Objetos, clases y
herencia}\label{unidad-vii-objetos-clases-y-herencia}}

Ejercicio 31: ¿Qué es una clase en programación orientada a objetos?

Respuesta:

Una clase es un plano o plantilla para crear objetos en programación
orientada a objetos. Define las propiedades (atributos) y
comportamientos (métodos) que tendrán los objetos creados a partir de
ella.

Ejercicio 32: Escribe una clase en Python llamada Persona con los
atributos nombre y edad, y un método saludar() que imprima un saludo con
el nombre de la persona.

Respuesta:

\begin{Shaded}
\begin{Highlighting}[]
\KeywordTok{class}\NormalTok{ Persona:}
    \KeywordTok{def} \FunctionTok{\_\_init\_\_}\NormalTok{(}\VariableTok{self}\NormalTok{, nombre, edad):}
        \VariableTok{self}\NormalTok{.nombre }\OperatorTok{=}\NormalTok{ nombre}
        \VariableTok{self}\NormalTok{.edad }\OperatorTok{=}\NormalTok{ edad}
\end{Highlighting}
\end{Shaded}

\begin{Shaded}
\begin{Highlighting}[]
\KeywordTok{def}\NormalTok{ saludar(}\VariableTok{self}\NormalTok{):}
    \BuiltInTok{print}\NormalTok{(}\StringTok{"¡Hola, soy"}\NormalTok{, }\VariableTok{self}\NormalTok{.nombre, }\StringTok{"y tengo"}\NormalTok{, }\VariableTok{self}\NormalTok{.edad, }\StringTok{"años!"}\NormalTok{)}
\end{Highlighting}
\end{Shaded}

\textbf{Ejercicio 33:} ¿Qué es la herencia en programación orientada a
objetos?

Respuesta:

La herencia es un concepto en el que una clase (subclase) puede heredar
atributos y métodos de otra clase (superclase). Permite reutilizar y
extender el código de una clase existente para crear una nueva clase.

\textbf{Ejercicio 34:} Escribe una clase en Python llamada
\texttt{Estudiante} que herede de la clase \texttt{Persona} y tenga un
atributo adicional \texttt{curso}.

Respuesta:

\begin{Shaded}
\begin{Highlighting}[]
\KeywordTok{class}\NormalTok{ Estudiante(Persona):}
    \KeywordTok{def} \FunctionTok{\_\_init\_\_}\NormalTok{(}\VariableTok{self}\NormalTok{, nombre, edad, curso):}
        \BuiltInTok{super}\NormalTok{().}\FunctionTok{\_\_init\_\_}\NormalTok{(nombre, edad)}
        \VariableTok{self}\NormalTok{.curso }\OperatorTok{=}\NormalTok{ curso}
\end{Highlighting}
\end{Shaded}

Ejercicio 35: ¿Por qué es beneficioso utilizar la herencia en
programación?

Respuesta:

La herencia permite reutilizar código, promover la coherencia y
facilitar la actualización y mantenimiento. También permite crear
jerarquías de clases para modelar relaciones entre objetos del mundo
real.

\hypertarget{unidad-viii-muxf3dulos}{%
\section{UNIDAD VIII: Módulos}\label{unidad-viii-muxf3dulos}}

Ejercicio 36: ¿Qué es un módulo en Python?

Respuesta:

Un módulo en Python es un archivo que contiene definiciones y
declaraciones de variables, funciones y clases. Permite organizar y
reutilizar el código en diferentes programas.

Ejercicio 37: Escribe un módulo en Python llamado operaciones que
contenga una función suma para sumar dos números.

Respuesta:

Archivo \texttt{operaciones.py}:

\begin{Shaded}
\begin{Highlighting}[]
\KeywordTok{def}\NormalTok{ suma(a, b):}
    \ControlFlowTok{return}\NormalTok{ a }\OperatorTok{+}\NormalTok{ b}
\end{Highlighting}
\end{Shaded}

Ejercicio 38: ¿Cómo se importa un módulo en Python?

Respuesta:

Se importa utilizando la palabra clave \texttt{import}, seguida del
nombre del módulo. Por ejemplo, \texttt{import\ operaciones} importaría
el módulo \texttt{operaciones}.

Ejercicio 39: Escribe un programa que utilice la función suma del módulo
operaciones para sumar dos números ingresados por el usuario.

Respuesta:

\begin{Shaded}
\begin{Highlighting}[]
\ImportTok{import}\NormalTok{ operaciones}

\NormalTok{num1 }\OperatorTok{=} \BuiltInTok{float}\NormalTok{(}\BuiltInTok{input}\NormalTok{(}\StringTok{"Ingresa el primer número: "}\NormalTok{))}
\NormalTok{num2 }\OperatorTok{=} \BuiltInTok{float}\NormalTok{(}\BuiltInTok{input}\NormalTok{(}\StringTok{"Ingresa el segundo número: "}\NormalTok{))}
\NormalTok{resultado }\OperatorTok{=}\NormalTok{ operaciones.suma(num1, num2)}
\BuiltInTok{print}\NormalTok{(}\StringTok{"La suma es:"}\NormalTok{, resultado)}
\end{Highlighting}
\end{Shaded}

\textbf{Ejercicio 40:} ¿Cuál es la ventaja de utilizar módulos en
Python?

Respuesta:

Los módulos permiten la modularidad, la reutilización de código y la
organización efectiva del código en componentes separados. También
facilitan la colaboración y la mantenibilidad.

\hypertarget{unidad-ix-introducciuxf3n-a-bases-de-datos}{%
\section{UNIDAD IX: Introducción a Bases de
Datos}\label{unidad-ix-introducciuxf3n-a-bases-de-datos}}

Ejercicio 41: ¿Qué es una base de datos en el contexto de la
programación?

Respuesta:

Una base de datos es un sistema organizado para almacenar, administrar y
recuperar información de manera eficiente. Se utiliza para almacenar
datos estructurados de manera persistente.

Ejercicio 42: ¿Qué es PostgreSQL?

Respuesta:

PostgreSQL es un sistema de gestión de bases de datos relacional de
código abierto y potente. Es conocido por su capacidad de manejar cargas
de trabajo complejas y por sus características avanzadas.

Ejercicio 43: ¿Qué es MongoDB?

Respuesta:

MongoDB es una base de datos NoSQL orientada a documentos. Almacena los
datos en documentos JSON flexibles en lugar de en tablas tradicionales,
lo que permite una gran flexibilidad y escalabilidad.

Ejercicio 44: ¿Cuál es la ventaja de utilizar bases de datos en
programas?

Respuesta:

Las bases de datos permiten almacenar y administrar grandes cantidades
de datos de manera estructurada y eficiente. Esto facilita el acceso y
la manipulación de datos en aplicaciones.

Ejercicio 45: ¿Cuál es el propósito de una clave primaria en una base de
datos?

Respuesta:

Una clave primaria es un campo único en una tabla que se utiliza para
identificar de manera única cada registro en la tabla. Se utiliza como
referencia para relacionar tablas y mantener la integridad de los datos.

UNIDAD X: MySQL, PostgreSQL y MongoDB: Operaciones básicas en bases de
datos

Ejercicio 46: ¿Cómo se realiza una consulta básica a una tabla en SQL?

Respuesta:

Utilizando la sentencia \texttt{SELECT}. Por ejemplo,
\texttt{SELECT\ *\ FROM\ tabla} recuperará todos los registros de la
tabla.

Ejercicio 47: ¿Qué comando se utiliza para insertar un nuevo registro en
una tabla en SQL?

Respuesta:

El comando utilizado es \texttt{INSERT\ INTO}. Por ejemplo,
\texttt{INSERT\ INTO\ tabla\ (columna1,\ columna2)\ VALUES\ (valor1,\ valor2)}
insertará un nuevo registro en la tabla.

Ejercicio 48: ¿Cómo se actualiza un registro en una tabla en SQL?

Respuesta:

Utilizando el comando \texttt{UPDATE}. Por ejemplo,
\texttt{UPDATE\ tabla\ SET\ columna\ =\ valor\ WHERE\ condicion}
actualizará los registros que cumplan con la condición.

Ejercicio 49: ¿Cuál es el propósito de la sentencia DELETE en SQL?

Respuesta:

La sentencia \texttt{DELETE} se utiliza para eliminar uno o varios
registros de una tabla. Por ejemplo,
\texttt{DELETE\ FROM\ tabla\ WHERE\ condicion} eliminará los registros
que cumplan con la condición.

Ejercicio 50: ¿Cuál es la ventaja de utilizar bases de datos NoSQL como
MongoDB?

Respuesta:

Las bases de datos NoSQL, como MongoDB, son flexibles y escalables, lo
que las hace ideales para manejar grandes cantidades de datos no
estructurados o semiestructurados. Son adecuadas para aplicaciones web y
móviles modernas.

\hypertarget{unidad-xi-cuxf3mo-me-ampluxedo-con-python}{%
\section{UNIDAD XI: ¿Cómo me amplío con
Python?}\label{unidad-xi-cuxf3mo-me-ampluxedo-con-python}}

Ejercicio 51: ¿Qué es la ciencia de datos y cómo se relaciona con
Python?

Respuesta:

La ciencia de datos es el proceso de extracción, transformación y
análisis de datos para obtener conocimientos y tomar decisiones
informadas. Python es ampliamente utilizado en la ciencia de datos
debido a su amplio ecosistema de bibliotecas y herramientas.

Ejercicio 52: ¿Qué es Django Framework y para qué se utiliza?

Respuesta:

Django es un framework web de alto nivel en Python que facilita la
creación de aplicaciones web robustas y escalables. Se utiliza para
construir sitios web y aplicaciones con características como
autenticación, seguridad y manejo de bases de datos.

Ejercicio 53: ¿Qué es FastAPI y cómo se diferencia de otros frameworks?

Respuesta:

FastAPI es un framework web moderno y de alto rendimiento para construir
APIs en Python. Se destaca por su velocidad, facilidad de uso y
generación automática de documentación interactiva. Utiliza anotaciones
de tipo para validar datos y reducir errores.

Ejercicio 54: ¿Cuál es el propósito de las APIs en el desarrollo web?

Respuesta:

Las APIs (Interfaces de Programación de Aplicaciones) se utilizan para
permitir la comunicación y la integración entre diferentes aplicaciones
y sistemas. Facilitan el intercambio de datos y funcionalidades entre
aplicaciones.

Ejercicio 55: ¿Por qué es importante ampliarse en Python más allá de los
conceptos básicos?

Respuesta:

\hypertarget{ampliarse-en-python-permite-abordar-proyectos-muxe1s-complejos-y-desafiantes-como-desarrollo-web-anuxe1lisis-de-datos-automatizaciuxf3n-inteligencia-artificial-y-muxe1s.-ademuxe1s-mejora-las-habilidades-y-la-versatilidad-como-programador.}{%
\chapter{Ampliarse en Python permite abordar proyectos más complejos y
desafiantes, como desarrollo web, análisis de datos, automatización,
inteligencia artificial y más. Además, mejora las habilidades y la
versatilidad como
programador.}\label{ampliarse-en-python-permite-abordar-proyectos-muxe1s-complejos-y-desafiantes-como-desarrollo-web-anuxe1lisis-de-datos-automatizaciuxf3n-inteligencia-artificial-y-muxe1s.-ademuxe1s-mejora-las-habilidades-y-la-versatilidad-como-programador.}}

\hypertarget{unidad-i-introducciuxf3n-a-la-programaciuxf3n-1}{%
\section{UNIDAD I: Introducción a la
programación}\label{unidad-i-introducciuxf3n-a-la-programaciuxf3n-1}}

Ejercicio 1: ¿Cuál es el objetivo principal de la programación?

Respuesta:

El objetivo principal de la programación es resolver problemas y
automatizar tareas utilizando un lenguaje de programación.

Ejercicio 2: ¿Qué es un algoritmo?

Respuesta:

Un algoritmo es un conjunto de instrucciones ordenadas y precisas que
describen cómo realizar una tarea o resolver un problema.

Ejercicio 3: ¿Cuál es la importancia de la indentación en Python?

Respuesta:

La indentación en Python es importante porque define el bloque de código
perteneciente a una estructura, como un bucle o una función. Python
utiliza la indentación en lugar de llaves u otros caracteres para
delimitar bloques de código.

Ejercicio 4: ¿Qué es un comentario en programación?

Respuesta:

Un comentario en programación es un texto explicativo que se agrega en
el código para hacerlo más comprensible. Los comentarios son ignorados
por el intérprete y son útiles para documentar el código.

Ejercicio 5: Escribe un programa en Python que imprima ``¡Hola,
mundo!''.

Respuesta:

\begin{Shaded}
\begin{Highlighting}[]
\BuiltInTok{print}\NormalTok{(}\StringTok{"¡Hola, mundo!"}\NormalTok{)}
\end{Highlighting}
\end{Shaded}

\hypertarget{unidad-ii-instalaciuxf3n-de-python-y-muxe1s-herramientas-1}{%
\section{UNIDAD II: Instalación de Python y más
herramientas}\label{unidad-ii-instalaciuxf3n-de-python-y-muxe1s-herramientas-1}}

Ejercicio 6: ¿Cuál es la forma de verificar la versión de Python
instalada en tu sistema?

Respuesta:

Ejecutando el comando \texttt{python\ -\/-version} en la línea de
comandos.

Ejercicio 7: ¿Cuál es el propósito de Git en el desarrollo de software?

Respuesta:

Git es un sistema de control de versiones que permite rastrear cambios
en el código, colaborar con otros desarrolladores y mantener un
historial completo de modificaciones en un proyecto.

Ejercicio 8: ¿Cómo se instala una extensión (extensión) en Visual Studio
Code?

Respuesta:

En Visual Studio Code, puedes instalar extensiones desde la barra
lateral izquierda, haciendo clic en el ícono de extensiones (cuatro
cuadros) y buscando la extensión que deseas instalar.

Ejercicio 9: ¿Cuál es el resultado del siguiente código?

\begin{Shaded}
\begin{Highlighting}[]
\BuiltInTok{print}\NormalTok{(}\StringTok{"Hola, "} \OperatorTok{+} \StringTok{"mundo"}\NormalTok{)}
\end{Highlighting}
\end{Shaded}

Respuesta:

El resultado es la cadena ``Hola, mundo'' al concatenar las dos cadenas.

Ejercicio 10: ¿Cuál es el propósito de un entorno virtual en Python?

Respuesta:

Un entorno virtual en Python permite aislar y gestionar las dependencias
y paquetes utilizados en un proyecto específico, evitando conflictos con
otros proyectos y asegurando un entorno limpio y controlado.

\hypertarget{unidad-iii-introducciuxf3n-a-python-1}{%
\section{UNIDAD III: Introducción a
Python}\label{unidad-iii-introducciuxf3n-a-python-1}}

Ejercicio 11: ¿Cuál es la diferencia entre una variable y una constante
en programación?

Respuesta:

Una variable puede cambiar su valor a lo largo del programa, mientras
que una constante mantiene su valor constante durante la ejecución.

Ejercicio 12: Escribe un programa que solicite al usuario su nombre y
luego imprima un mensaje de bienvenida con el nombre ingresado.

Respuesta:

\begin{Shaded}
\begin{Highlighting}[]
\NormalTok{nombre }\OperatorTok{=} \BuiltInTok{input}\NormalTok{(}\StringTok{"Ingresa tu nombre: "}\NormalTok{)}
\BuiltInTok{print}\NormalTok{(}\StringTok{"¡Bienvenido,"}\NormalTok{, nombre, }\StringTok{"!"}\NormalTok{)}
\end{Highlighting}
\end{Shaded}

Ejercicio 13: ¿Cuál es el valor de la variable resultado después de
ejecutar el siguiente código?

\begin{Shaded}
\begin{Highlighting}[]
\NormalTok{x }\OperatorTok{=} \DecValTok{5}
\NormalTok{y }\OperatorTok{=} \DecValTok{2}
\NormalTok{resultado }\OperatorTok{=}\NormalTok{ x }\OperatorTok{//}\NormalTok{ y}
\end{Highlighting}
\end{Shaded}

Respuesta:

El valor de la variable \texttt{resultado} será 2, ya que \texttt{//}
realiza la división entera de 5 entre 2.

Ejercicio 14: Escribe un programa en Python que determine si un número
ingresado por el usuario es par o impar.

Respuesta:

\begin{Shaded}
\begin{Highlighting}[]
\NormalTok{numero }\OperatorTok{=} \BuiltInTok{int}\NormalTok{(}\BuiltInTok{input}\NormalTok{(}\StringTok{"Ingresa un número: "}\NormalTok{))}
\ControlFlowTok{if}\NormalTok{ numero }\OperatorTok{\%} \DecValTok{2} \OperatorTok{==} \DecValTok{0}\NormalTok{:}
    \BuiltInTok{print}\NormalTok{(}\StringTok{"El número es par."}\NormalTok{)}
\ControlFlowTok{else}\NormalTok{:}
    \BuiltInTok{print}\NormalTok{(}\StringTok{"El número es impar."}\NormalTok{)}
\end{Highlighting}
\end{Shaded}

Ejercicio 15: ¿Cuál es la función del operador not en Python?

Respuesta:

El operador \texttt{not} se utiliza para negar una expresión booleana.
Si la expresión es verdadera, \texttt{not} la convierte en falsa, y
viceversa.

\hypertarget{unidad-iv-tipos-de-datos-1}{%
\section{UNIDAD IV: Tipos de Datos}\label{unidad-iv-tipos-de-datos-1}}

Ejercicio 16: ¿Cuál es la diferencia entre una lista y una tupla en
Python?

Respuesta:

La principal diferencia es que las listas son mutables (pueden cambiar)
y las tuplas son inmutables (no pueden cambiar). En otras palabras,
puedes agregar, eliminar y modificar elementos en una lista, pero no en
una tupla.

Ejercicio 17: Escribe un programa que ordene una lista de números en
orden ascendente.

Respuesta:

\begin{Shaded}
\begin{Highlighting}[]
\NormalTok{numeros }\OperatorTok{=}\NormalTok{ [}\DecValTok{4}\NormalTok{, }\DecValTok{1}\NormalTok{, }\DecValTok{6}\NormalTok{, }\DecValTok{3}\NormalTok{, }\DecValTok{2}\NormalTok{]}
\NormalTok{numeros.sort()}
\BuiltInTok{print}\NormalTok{(numeros)}
\end{Highlighting}
\end{Shaded}

Ejercicio 18: ¿Cómo se accede al tercer elemento de una lista en Python?

Respuesta:

Utilizando el índice \texttt{2}. Por ejemplo, si la lista se llama
\texttt{mi\_lista}, puedes acceder al tercer elemento con
\texttt{mi\_lista{[}2{]}}.

Ejercicio 19: ¿Qué método se utiliza para agregar un elemento al final
de una lista?

Respuesta:

El método utilizado es \texttt{append()}. Por ejemplo,
\texttt{mi\_lista.append(7)} agrega el número 7 al final de la lista.

Ejercicio 20: Escribe un programa que cuente cuántas veces aparece un
elemento específico en una lista.

Respuesta:

\begin{Shaded}
\begin{Highlighting}[]
\NormalTok{mi\_lista }\OperatorTok{=}\NormalTok{ [}\DecValTok{2}\NormalTok{, }\DecValTok{4}\NormalTok{, }\DecValTok{6}\NormalTok{, }\DecValTok{4}\NormalTok{, }\DecValTok{8}\NormalTok{, }\DecValTok{4}\NormalTok{, }\DecValTok{10}\NormalTok{]}
\NormalTok{elemento }\OperatorTok{=} \DecValTok{4}
\NormalTok{contador }\OperatorTok{=}\NormalTok{ mi\_lista.count(elemento)}
\BuiltInTok{print}\NormalTok{(}\StringTok{"El elemento"}\NormalTok{, elemento, }\StringTok{"aparece"}\NormalTok{, contador, }\StringTok{"veces."}\NormalTok{)}
\end{Highlighting}
\end{Shaded}

\hypertarget{unidad-v-control-de-flujo-1}{%
\section{UNIDAD V: Control de Flujo}\label{unidad-v-control-de-flujo-1}}

Ejercicio 21: Escribe un programa que determine si un número ingresado
por el usuario es positivo, negativo o cero.

Respuesta:

\begin{Shaded}
\begin{Highlighting}[]
\NormalTok{numero }\OperatorTok{=} \BuiltInTok{int}\NormalTok{(}\BuiltInTok{input}\NormalTok{(}\StringTok{"Ingresa un número: "}\NormalTok{))}
\ControlFlowTok{if}\NormalTok{ numero }\OperatorTok{\textgreater{}} \DecValTok{0}\NormalTok{:}
    \BuiltInTok{print}\NormalTok{(}\StringTok{"El número es positivo."}\NormalTok{)}
\ControlFlowTok{elif}\NormalTok{ numero }\OperatorTok{\textless{}} \DecValTok{0}\NormalTok{:}
    \BuiltInTok{print}\NormalTok{(}\StringTok{"El número es negativo."}\NormalTok{)}
\ControlFlowTok{else}\NormalTok{:}
    \BuiltInTok{print}\NormalTok{(}\StringTok{"El número es cero."}\NormalTok{)}
\end{Highlighting}
\end{Shaded}

Ejercicio 22: ¿Qué hace el siguiente código?

\begin{Shaded}
\begin{Highlighting}[]
\NormalTok{contador }\OperatorTok{=} \DecValTok{0}
\ControlFlowTok{while}\NormalTok{ contador }\OperatorTok{\textless{}} \DecValTok{5}\NormalTok{:}
    \BuiltInTok{print}\NormalTok{(contador)}
\NormalTok{    contador }\OperatorTok{+=} \DecValTok{1}
\end{Highlighting}
\end{Shaded}

Respuesta:

El código imprime los números del 0 al 4 en líneas separadas utilizando
un bucle \texttt{while}.

Ejercicio 23: ¿Cuál es el resultado de la siguiente expresión?

\begin{Shaded}
\begin{Highlighting}[]
\NormalTok{resultado }\OperatorTok{=} \DecValTok{0}
\ControlFlowTok{for}\NormalTok{ i }\KeywordTok{in} \BuiltInTok{range}\NormalTok{(}\DecValTok{1}\NormalTok{, }\DecValTok{6}\NormalTok{):}
\NormalTok{    resultado }\OperatorTok{+=}\NormalTok{ i}
\BuiltInTok{print}\NormalTok{(resultado)}
\end{Highlighting}
\end{Shaded}

Respuesta:

El resultado es 15, ya que se suma los números del 1 al 5 en el bucle
\texttt{for}.

Ejercicio 24: Escribe un programa que calcule la suma de todos los
números pares entre 1 y 100.

Respuesta:

\begin{Shaded}
\begin{Highlighting}[]
\NormalTok{suma }\OperatorTok{=} \DecValTok{0}
\ControlFlowTok{for}\NormalTok{ i }\KeywordTok{in} \BuiltInTok{range}\NormalTok{(}\DecValTok{2}\NormalTok{, }\DecValTok{101}\NormalTok{, }\DecValTok{2}\NormalTok{):}
\NormalTok{    suma }\OperatorTok{+=}\NormalTok{ i}
\BuiltInTok{print}\NormalTok{(}\StringTok{"La suma de los números pares entre 1 y 100 es:"}\NormalTok{, suma)}
\end{Highlighting}
\end{Shaded}

Ejercicio 25: ¿Cuál es el propósito de la instrucción break en un bucle?

Respuesta:

La instrucción \texttt{break} se utiliza para salir inmediatamente de un
bucle, interrumpiendo su ejecución antes de que termine naturalmente.

\hypertarget{unidad-vi-funciones-1}{%
\section{UNIDAD VI: Funciones}\label{unidad-vi-funciones-1}}

Ejercicio 26: ¿Qué es una función en programación?

Respuesta:

Una función es un bloque de código reutilizable que realiza una tarea
específica. Puede recibir argumentos, ejecutar instrucciones y devolver
un valor.

Ejercicio 27: Escribe una función en Python que calcule el área de un
círculo.

Respuesta:

\begin{Shaded}
\begin{Highlighting}[]
\ImportTok{import}\NormalTok{ math}

\KeywordTok{def}\NormalTok{ area\_circulo(radio):}
\ControlFlowTok{return}\NormalTok{ math.pi }\OperatorTok{*}\NormalTok{ radio }\OperatorTok{**} \DecValTok{2}
\end{Highlighting}
\end{Shaded}

\textbf{Ejercicio 28:} ¿Qué es la recursividad en programación?

Respuesta:

La recursividad es una técnica donde una función se llama a sí misma
para resolver un problema. Es útil para resolver problemas que se pueden
descomponer en subproblemas similares.

\textbf{Ejercicio 29:} Escribe una función recursiva en Python para
calcular el factorial de un número.

Respuesta:

\begin{Shaded}
\begin{Highlighting}[]
\KeywordTok{def}\NormalTok{ factorial(n):}
    \ControlFlowTok{if}\NormalTok{ n }\OperatorTok{==} \DecValTok{0}\NormalTok{:}
        \ControlFlowTok{return} \DecValTok{1}
    \ControlFlowTok{else}\NormalTok{:}
        \ControlFlowTok{return}\NormalTok{ n }\OperatorTok{*}\NormalTok{ factorial(n }\OperatorTok{{-}} \DecValTok{1}\NormalTok{)}
\end{Highlighting}
\end{Shaded}

Ejercicio 30: ¿Por qué es importante utilizar funciones en la
programación?

Respuesta:

Las funciones permiten dividir el código en bloques más pequeños y
manejables, lo que facilita la reutilización, la depuración y la
comprensión del código. Además, promueven la modularidad y el diseño
limpio.

\hypertarget{unidad-vii-objetos-clases-y-herencia-1}{%
\section{UNIDAD VII: Objetos, clases y
herencia}\label{unidad-vii-objetos-clases-y-herencia-1}}

Ejercicio 31: ¿Qué es una clase en programación orientada a objetos?

Respuesta:

Una clase es un plano o plantilla para crear objetos en programación
orientada a objetos. Define las propiedades (atributos) y
comportamientos (métodos) que tendrán los objetos creados a partir de
ella.

Ejercicio 32: Escribe una clase en Python llamada Persona con los
atributos nombre y edad, y un método saludar() que imprima un saludo con
el nombre de la persona.

Respuesta:

\begin{Shaded}
\begin{Highlighting}[]
\KeywordTok{class}\NormalTok{ Persona:}
    \KeywordTok{def} \FunctionTok{\_\_init\_\_}\NormalTok{(}\VariableTok{self}\NormalTok{, nombre, edad):}
        \VariableTok{self}\NormalTok{.nombre }\OperatorTok{=}\NormalTok{ nombre}
        \VariableTok{self}\NormalTok{.edad }\OperatorTok{=}\NormalTok{ edad}
\end{Highlighting}
\end{Shaded}

\begin{Shaded}
\begin{Highlighting}[]
\KeywordTok{def}\NormalTok{ saludar(}\VariableTok{self}\NormalTok{):}
    \BuiltInTok{print}\NormalTok{(}\StringTok{"¡Hola, soy"}\NormalTok{, }\VariableTok{self}\NormalTok{.nombre, }\StringTok{"y tengo"}\NormalTok{, }\VariableTok{self}\NormalTok{.edad, }\StringTok{"años!"}\NormalTok{)}
\end{Highlighting}
\end{Shaded}

\textbf{Ejercicio 33:} ¿Qué es la herencia en programación orientada a
objetos?

Respuesta:

La herencia es un concepto en el que una clase (subclase) puede heredar
atributos y métodos de otra clase (superclase). Permite reutilizar y
extender el código de una clase existente para crear una nueva clase.

\textbf{Ejercicio 34:} Escribe una clase en Python llamada
\texttt{Estudiante} que herede de la clase \texttt{Persona} y tenga un
atributo adicional \texttt{curso}.

Respuesta:

\begin{Shaded}
\begin{Highlighting}[]
\KeywordTok{class}\NormalTok{ Estudiante(Persona):}
    \KeywordTok{def} \FunctionTok{\_\_init\_\_}\NormalTok{(}\VariableTok{self}\NormalTok{, nombre, edad, curso):}
        \BuiltInTok{super}\NormalTok{().}\FunctionTok{\_\_init\_\_}\NormalTok{(nombre, edad)}
        \VariableTok{self}\NormalTok{.curso }\OperatorTok{=}\NormalTok{ curso}
\end{Highlighting}
\end{Shaded}

Ejercicio 35: ¿Por qué es beneficioso utilizar la herencia en
programación?

Respuesta:

La herencia permite reutilizar código, promover la coherencia y
facilitar la actualización y mantenimiento. También permite crear
jerarquías de clases para modelar relaciones entre objetos del mundo
real.

\hypertarget{unidad-viii-muxf3dulos-1}{%
\section{UNIDAD VIII: Módulos}\label{unidad-viii-muxf3dulos-1}}

Ejercicio 36: ¿Qué es un módulo en Python?

Respuesta:

Un módulo en Python es un archivo que contiene definiciones y
declaraciones de variables, funciones y clases. Permite organizar y
reutilizar el código en diferentes programas.

Ejercicio 37: Escribe un módulo en Python llamado operaciones que
contenga una función suma para sumar dos números.

Respuesta:

Archivo \texttt{operaciones.py}:

\begin{Shaded}
\begin{Highlighting}[]
\KeywordTok{def}\NormalTok{ suma(a, b):}
    \ControlFlowTok{return}\NormalTok{ a }\OperatorTok{+}\NormalTok{ b}
\end{Highlighting}
\end{Shaded}

Ejercicio 38: ¿Cómo se importa un módulo en Python?

Respuesta:

Se importa utilizando la palabra clave \texttt{import}, seguida del
nombre del módulo. Por ejemplo, \texttt{import\ operaciones} importaría
el módulo \texttt{operaciones}.

Ejercicio 39: Escribe un programa que utilice la función suma del módulo
operaciones para sumar dos números ingresados por el usuario.

Respuesta:

\begin{Shaded}
\begin{Highlighting}[]
\ImportTok{import}\NormalTok{ operaciones}

\NormalTok{num1 }\OperatorTok{=} \BuiltInTok{float}\NormalTok{(}\BuiltInTok{input}\NormalTok{(}\StringTok{"Ingresa el primer número: "}\NormalTok{))}
\NormalTok{num2 }\OperatorTok{=} \BuiltInTok{float}\NormalTok{(}\BuiltInTok{input}\NormalTok{(}\StringTok{"Ingresa el segundo número: "}\NormalTok{))}
\NormalTok{resultado }\OperatorTok{=}\NormalTok{ operaciones.suma(num1, num2)}
\BuiltInTok{print}\NormalTok{(}\StringTok{"La suma es:"}\NormalTok{, resultado)}
\end{Highlighting}
\end{Shaded}

\textbf{Ejercicio 40:} ¿Cuál es la ventaja de utilizar módulos en
Python?

Respuesta:

Los módulos permiten la modularidad, la reutilización de código y la
organización efectiva del código en componentes separados. También
facilitan la colaboración y la mantenibilidad.

\hypertarget{unidad-ix-introducciuxf3n-a-bases-de-datos-1}{%
\section{UNIDAD IX: Introducción a Bases de
Datos}\label{unidad-ix-introducciuxf3n-a-bases-de-datos-1}}

Ejercicio 41: ¿Qué es una base de datos en el contexto de la
programación?

Respuesta:

Una base de datos es un sistema organizado para almacenar, administrar y
recuperar información de manera eficiente. Se utiliza para almacenar
datos estructurados de manera persistente.

Ejercicio 42: ¿Qué es PostgreSQL?

Respuesta:

PostgreSQL es un sistema de gestión de bases de datos relacional de
código abierto y potente. Es conocido por su capacidad de manejar cargas
de trabajo complejas y por sus características avanzadas.

Ejercicio 43: ¿Qué es MongoDB?

Respuesta:

MongoDB es una base de datos NoSQL orientada a documentos. Almacena los
datos en documentos JSON flexibles en lugar de en tablas tradicionales,
lo que permite una gran flexibilidad y escalabilidad.

Ejercicio 44: ¿Cuál es la ventaja de utilizar bases de datos en
programas?

Respuesta:

Las bases de datos permiten almacenar y administrar grandes cantidades
de datos de manera estructurada y eficiente. Esto facilita el acceso y
la manipulación de datos en aplicaciones.

Ejercicio 45: ¿Cuál es el propósito de una clave primaria en una base de
datos?

Respuesta:

Una clave primaria es un campo único en una tabla que se utiliza para
identificar de manera única cada registro en la tabla. Se utiliza como
referencia para relacionar tablas y mantener la integridad de los datos.

UNIDAD X: MySQL, PostgreSQL y MongoDB: Operaciones básicas en bases de
datos

Ejercicio 46: ¿Cómo se realiza una consulta básica a una tabla en SQL?

Respuesta:

Utilizando la sentencia \texttt{SELECT}. Por ejemplo,
\texttt{SELECT\ *\ FROM\ tabla} recuperará todos los registros de la
tabla.

Ejercicio 47: ¿Qué comando se utiliza para insertar un nuevo registro en
una tabla en SQL?

Respuesta:

El comando utilizado es \texttt{INSERT\ INTO}. Por ejemplo,
\texttt{INSERT\ INTO\ tabla\ (columna1,\ columna2)\ VALUES\ (valor1,\ valor2)}
insertará un nuevo registro en la tabla.

Ejercicio 48: ¿Cómo se actualiza un registro en una tabla en SQL?

Respuesta:

Utilizando el comando \texttt{UPDATE}. Por ejemplo,
\texttt{UPDATE\ tabla\ SET\ columna\ =\ valor\ WHERE\ condicion}
actualizará los registros que cumplan con la condición.

Ejercicio 49: ¿Cuál es el propósito de la sentencia DELETE en SQL?

Respuesta:

La sentencia \texttt{DELETE} se utiliza para eliminar uno o varios
registros de una tabla. Por ejemplo,
\texttt{DELETE\ FROM\ tabla\ WHERE\ condicion} eliminará los registros
que cumplan con la condición.

Ejercicio 50: ¿Cuál es la ventaja de utilizar bases de datos NoSQL como
MongoDB?

Respuesta:

Las bases de datos NoSQL, como MongoDB, son flexibles y escalables, lo
que las hace ideales para manejar grandes cantidades de datos no
estructurados o semiestructurados. Son adecuadas para aplicaciones web y
móviles modernas.

\hypertarget{unidad-xi-cuxf3mo-me-ampluxedo-con-python-1}{%
\section{UNIDAD XI: ¿Cómo me amplío con
Python?}\label{unidad-xi-cuxf3mo-me-ampluxedo-con-python-1}}

Ejercicio 51: ¿Qué es la ciencia de datos y cómo se relaciona con
Python?

Respuesta:

La ciencia de datos es el proceso de extracción, transformación y
análisis de datos para obtener conocimientos y tomar decisiones
informadas. Python es ampliamente utilizado en la ciencia de datos
debido a su amplio ecosistema de bibliotecas y herramientas.

Ejercicio 52: ¿Qué es Django Framework y para qué se utiliza?

Respuesta:

Django es un framework web de alto nivel en Python que facilita la
creación de aplicaciones web robustas y escalables. Se utiliza para
construir sitios web y aplicaciones con características como
autenticación, seguridad y manejo de bases de datos.

Ejercicio 53: ¿Qué es FastAPI y cómo se diferencia de otros frameworks?

Respuesta:

FastAPI es un framework web moderno y de alto rendimiento para construir
APIs en Python. Se destaca por su velocidad, facilidad de uso y
generación automática de documentación interactiva. Utiliza anotaciones
de tipo para validar datos y reducir errores.

Ejercicio 54: ¿Cuál es el propósito de las APIs en el desarrollo web?

Respuesta:

Las APIs (Interfaces de Programación de Aplicaciones) se utilizan para
permitir la comunicación y la integración entre diferentes aplicaciones
y sistemas. Facilitan el intercambio de datos y funcionalidades entre
aplicaciones.

Ejercicio 55: ¿Por qué es importante ampliarse en Python más allá de los
conceptos básicos?

Respuesta:

Ampliarse en Python permite abordar proyectos más complejos y
desafiantes, como desarrollo web, análisis de datos, automatización,
inteligencia artificial y más. Además, mejora las habilidades y la
versatilidad como programador.
\textgreater\textgreater\textgreater\textgreater\textgreater\textgreater\textgreater{}
e8ed08b1a5bbe1e369719187cfc4de7f7e2a41a9



\end{document}
